
\documentclass[a4paper,12pt]{article}

\usepackage{fancyhdr}
\usepackage{amssymb}
%\usepackage{mathpazo}
\usepackage{mathtools}
\usepackage{amsmath}
\usepackage{slashed}
\usepackage{cancel}
\usepackage[mathscr]{euscript}
\usepackage{MaxPackage} %Note: You need MaxPackage installed or in the same folder as your .tex file or something.

\newcommand{\colorcomment}[2]{\textcolor{#1}{#2}} %First of these leaves in comments. Second one kills them.
%\newcommand{\colorcomment}[2]{}


\pagestyle{fancy}
\lhead{Linear Algebra}
\chead{MA265}
\rhead{Quiz 1 Key, 6/17/2015}

%Number of Problems		: 
%Clear					:
%Begun					:
%Not started			:
%Can complete via book	:
%Needs Polish			:

%Pomodoros logged		:

\begin{document}

{\Large{\bf Problem 1:}} Solve the following system of linear equations, by elimination:

\begin{align*}
x+y&=2\\
4x+y&=5
\end{align*}

Answer: $x=1$, $y=1$.

\shunt

\shunt

{\Large{\bf Problem 2:}} The following system of linear equations cannot be uniquely solved using elimination. Why not?

\begin{align*}
x+y&= 2\\
2x+2y &= 4
\end{align*}

Answer: Anything like ``There's more than one solution'', ``They're the same equation, fundamentally'', or ``They're the same line/parallel'' was taken as a valid answer. The key thing to notice is that when we try to eliminate, we get $0=0$;

\begin{align*}
x+y&=2\\
2x+2y&=4\\
-2x-2y&=-4 \text{ (Multiply first equation by -2)}\\
0&=0 \text{ (Add the above two equations.)}\\
\end{align*}

Coming up with two different solutions would also have worked.

\shunt

\shunt

\end{document}