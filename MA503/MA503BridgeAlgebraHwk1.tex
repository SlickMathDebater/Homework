
\documentclass[a4paper,12pt]{article}

\usepackage{fancyhdr}
\usepackage{amssymb}
%\usepackage{mathpazo}
\usepackage{mathtools}
\usepackage{amsmath}
\usepackage{slashed}
\usepackage[mathscr]{euscript}

\newcommand{\tab}{\hspace{4mm}} %Spacing aliases
\newcommand{\shunt}{\vspace{20mm}}

\newcommand{\sd}{\partial} %Squiggle d

\newcommand{\absval}[1]{\lvert #1 \rvert}
\newcommand{\norm}[1]{\|#1\|}


\newcommand{\al}{\alpha} %Steal ALL of Dr. Kable's Aliases! MWAHAHAHAHA!
\newcommand{\be}{\beta}
\newcommand{\ga}{\gamma}
\newcommand{\Ga}{\Gamma}
\newcommand{\de}{\delta}
\newcommand{\De}{\Delta}
\newcommand{\ep}{\epsilon}
\newcommand{\vep}{\varepsilon}
\newcommand{\ze}{\zeta}
\newcommand{\et}{\eta}
\newcommand{\tha}{\theta}
\newcommand{\vtha}{\vartheta}
\newcommand{\Tha}{\Theta}
\newcommand{\io}{\iota}
\newcommand{\ka}{\kappa}
\newcommand{\la}{\lambda}
\newcommand{\La}{\Lambda}
\newcommand{\rh}{\rho}
\newcommand{\si}{\sigma}
\newcommand{\Si}{\Sigma}
\newcommand{\ta}{\tau}
\newcommand{\ups}{\upsilon}
\newcommand{\Ups}{\Upsilon}
\newcommand{\ph}{\phi}
\newcommand{\Ph}{\Phi}
\newcommand{\vph}{\varphi}
\newcommand{\vpi}{\varpi}
\newcommand{\ch}{\chi}
\newcommand{\ps}{\psi}
\newcommand{\Ps}{\Psi}
\newcommand{\om}{\omega}
\newcommand{\Om}{\Omega}

\newcommand{\bbA}{\mathbb{A}}
\newcommand{\A}{\mathbb{A}}
\newcommand{\bbB}{\mathbb{B}}
\newcommand{\bbC}{\mathbb{C}}
\newcommand{\C}{\mathbb{C}}
\newcommand{\bbD}{\mathbb{D}}
\newcommand{\bbE}{\mathbb{E}}
\newcommand{\bbF}{\mathbb{F}}
\newcommand{\bbG}{\mathbb{G}}
\newcommand{\G}{\mathbb{G}}
\newcommand{\bbH}{\mathbb{H}}
\newcommand{\HH}{\mathbb{H}}
\newcommand{\bbI}{\mathbb{I}}
\newcommand{\I}{\mathbb{I}}
\newcommand{\bbJ}{\mathbb{J}}
\newcommand{\bbK}{\mathbb{K}}
\newcommand{\bbL}{\mathbb{L}}
\newcommand{\bbM}{\mathbb{M}}
\newcommand{\bbN}{\mathbb{N}}
\newcommand{\N}{\mathbb{N}}
\newcommand{\bbO}{\mathbb{O}}
\newcommand{\bbP}{\mathbb{P}}
\newcommand{\PP}{\mathbb{P}}
\newcommand{\bbQ}{\mathbb{Q}}
\newcommand{\Q}{\mathbb{Q}}
\newcommand{\bbR}{\mathbb{R}}
\newcommand{\R}{\mathbb{R}}
\newcommand{\bbS}{\mathbb{S}}
\newcommand{\bbT}{\mathbb{T}}
\newcommand{\bbU}{\mathbb{U}}
\newcommand{\bbV}{\mathbb{V}}
\newcommand{\bbW}{\mathbb{W}}
\newcommand{\bbX}{\mathbb{X}}
\newcommand{\bbY}{\mathbb{Y}}
\newcommand{\bbZ}{\mathbb{Z}}
\newcommand{\Z}{\mathbb{Z}}

\newcommand{\scrA}{\mathcal{A}}
\newcommand{\scrB}{\mathcal{B}}
\newcommand{\scrC}{\mathcal{C}}
\newcommand{\scrD}{\mathcal{D}}
\newcommand{\scrE}{\mathcal{E}}
\newcommand{\scrF}{\mathcal{F}}
\newcommand{\scrG}{\mathcal{G}}
\newcommand{\scrH}{\mathcal{H}}
\newcommand{\scrI}{\mathcal{I}}
\newcommand{\scrJ}{\mathcal{J}}
\newcommand{\scrK}{\mathcal{K}}
\newcommand{\scrL}{\mathcal{L}}
\newcommand{\scrM}{\mathcal{M}}
\newcommand{\scrN}{\mathcal{N}}
\newcommand{\scrO}{\mathcal{O}}
\newcommand{\scrP}{\mathcal{P}}
\newcommand{\scrQ}{\mathcal{Q}}
\newcommand{\scrR}{\mathcal{R}}
\newcommand{\scrS}{\mathcal{S}}
\newcommand{\scrT}{\mathcal{T}}
\newcommand{\scrU}{\mathcal{U}}
\newcommand{\scrV}{\mathcal{V}}
\newcommand{\scrW}{\mathcal{W}}
\newcommand{\scrX}{\mathcal{X}}
\newcommand{\scrY}{\mathcal{Y}}
\newcommand{\scrZ}{\mathcal{Z}}

\DeclarePairedDelimiter\ceil{\lceil}{\rceil}
\DeclarePairedDelimiter\floor{\lfloor}{\rfloor}

\newcommand{\colorcomment}[2]{\textcolor{#1}{#2}} %First of these leaves in comments. Second one kills them.
%\newcommand{\colorcomment}[2]{}


\pagestyle{fancy}
\lhead{Max Jeter}
\rhead{MA503, Assignment 1, Page \thepage}

\begin{document}

{\bf Problem 1:}

%Show: O(S_n) = n!

\shunt

{\bf Problem 2:}

Let $G$ be a group and $H \subset G$ be nonempty and finite.

If $H < G$, then if $a \in H$ and $b \in H$, then $ab \in H$. Else, $H$ does not inherit the group operation of $G$.

Next, assume that $a,b \in H$ implies that $ab \in H$.

\tab Then $H$ inherits the group operation of $G$.

\tab Because the group operation of $G$ was associative, the inherited group operation on $H$ is also associative.

\tab Further, $e \in H$:

\tab \tab %Prove that

\tab Next, for all $a \in H$, $a^{-1} \in H$:

\tab \tab %Proof

\tab Thus, $H$ is a group under the inherited group operation of $G$: $H<G$.

So $H<G$ if and only if for all $a, b \in H$, $ab \in H$.

\shunt

{\bf Problem 3:}

Let $G$ be a group such that for all $a,b \in G$ and for three given consectuve integers $i$, $(ab)^i = a^ib^i$.

Then there is an $n \in \Z$ such that for all $a, b \in G$:

\begin{align*}
&(ab)^n = a^nb^n \\
&(ab)^{n-1} = a^{n-1}b^{n-1} = a^{-1}(ab)^n b^{-1} \\
&(ab)^{n+1} = a^{n+1}b^{n+1} = a(ab)^nb \\
\end{align*}

Now, for all $a, b \in G$,

\begin{align*}
ab &= (ab)^n {((ab)^{n-1})}^{-1} \\
&=a^nb^n b ((ab)^{n})^{-1}a \\
&=a^nb^n b ((ab)^{n})^{-1}a \\
&=a^{-1}a^{n+1}b^{n+1}((ab)^{n})^{-1}a \\
&=a^{-1}(ab)^{n+1}(ab)^{-n}a \\
&=a^{-1}(ab)a \\
&=a^{-1}aba \\
&=ba
\end{align*}

To summarize, for all $a, b \in G$, $ab= ba$. That is, $G$ is abelian.

\shunt

{\bf Problem 4:}

Let $G = \left\langle\left[\begin{smallmatrix}0&1\\ -1&0 \end{smallmatrix}\right],\left[\begin{smallmatrix}0&1\\ 1&0 \end{smallmatrix}\right] \right\rangle < GL_2(\R)$.

$G$ has $8$ elements; they are:
\begin{align*}
&\left[\begin{smallmatrix}0&1\\ -1&0 \end{smallmatrix}\right],
\left[\begin{smallmatrix}-1&0\\ 0&-1 \end{smallmatrix}\right],
\left[\begin{smallmatrix}0&-1\\ 1&0 \end{smallmatrix}\right],
\left[\begin{smallmatrix}1&0\\ 0&1 \end{smallmatrix}\right],\\
&\left[\begin{smallmatrix}0&1\\ 1&0 \end{smallmatrix}\right],
\left[\begin{smallmatrix}-1&0\\ 0&1 \end{smallmatrix}\right],
\left[\begin{smallmatrix}0&-1\\ -1&0 \end{smallmatrix}\right],
\left[\begin{smallmatrix}1&0\\ 0&-1 \end{smallmatrix}\right].
\end{align*}

Now, consider $D_4 = \left\langle r, s \right\rangle$, where $r$ is a rotation by $90$ degrees and $s$ is a reflection.

Let $\phi : G \to D_4$ be given as follows:

\begin{align*}
&\phi(\left[\begin{smallmatrix}0&1\\ -1&0 \end{smallmatrix}\right]) = r,\\
&\phi(\left[\begin{smallmatrix}-1&0\\ 0&-1 \end{smallmatrix}\right]) = r^2,\\
&\phi(\left[\begin{smallmatrix}0&-1\\ 1&0 \end{smallmatrix}\right]) = r^3,\\
&\phi(\left[\begin{smallmatrix}1&0\\ 0&1 \end{smallmatrix}\right]) = e,\\
&\phi(\left[\begin{smallmatrix}0&1\\ 1&0 \end{smallmatrix}\right]) = s,\\
&\phi(\left[\begin{smallmatrix}-1&0\\ 0&1 \end{smallmatrix}\right]) = sr,\\
&\phi(\left[\begin{smallmatrix}0&-1\\ -1&0 \end{smallmatrix}\right]) = sr^2,\\
&\phi(\left[\begin{smallmatrix}1&0\\ 0&-1 \end{smallmatrix}\right]) = sr^3.
\end{align*}

Then it is readily checked (where ``readily'' means in a series of not more than $64$ matrix multiplications) that $\phi$ is a homomorphism. It is clear that $\phi$ is both one-to-one and onto, so that $\phi$ is a bijection. That is, $\phi$ is an isomorphism.

\shunt

{\bf Problem 5:}

\shunt

{\bf Problem 6:}

\shunt

\end{document}