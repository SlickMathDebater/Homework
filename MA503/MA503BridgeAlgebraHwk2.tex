
\documentclass[a4paper,12pt]{article}

\usepackage{fancyhdr}
\usepackage{amssymb}
%\usepackage{mathpazo}
\usepackage{mathtools}
\usepackage{amsmath}
\usepackage{slashed}
\usepackage[mathscr]{euscript}

\newcommand{\tab}{\hspace{4mm}} %Spacing aliases
\newcommand{\shunt}{\vspace{20mm}}

\newcommand{\sd}{\partial} %Squiggle d

\newcommand{\absval}[1]{\lvert #1 \rvert}
\newcommand{\anbrack}[1]{\left\langle #1 \right\rangle}
\newcommand{\norm}[1]{\|#1\|}


\newcommand{\al}{\alpha} %Steal ALL of Dr. Kable's Aliases! MWAHAHAHAHA!
\newcommand{\be}{\beta}
\newcommand{\ga}{\gamma}
\newcommand{\Ga}{\Gamma}
\newcommand{\de}{\delta}
\newcommand{\De}{\Delta}
\newcommand{\ep}{\epsilon}
\newcommand{\vep}{\varepsilon}
\newcommand{\ze}{\zeta}
\newcommand{\et}{\eta}
\newcommand{\tha}{\theta}
\newcommand{\vtha}{\vartheta}
\newcommand{\Tha}{\Theta}
\newcommand{\io}{\iota}
\newcommand{\ka}{\kappa}
\newcommand{\la}{\lambda}
\newcommand{\La}{\Lambda}
\newcommand{\rh}{\rho}
\newcommand{\si}{\sigma}
\newcommand{\Si}{\Sigma}
\newcommand{\ta}{\tau}
\newcommand{\ups}{\upsilon}
\newcommand{\Ups}{\Upsilon}
\newcommand{\ph}{\phi}
\newcommand{\Ph}{\Phi}
\newcommand{\vph}{\varphi}
\newcommand{\vpi}{\varpi}
\newcommand{\ch}{\chi}
\newcommand{\ps}{\psi}
\newcommand{\Ps}{\Psi}
\newcommand{\om}{\omega}
\newcommand{\Om}{\Omega}

\newcommand{\bbA}{\mathbb{A}}
\newcommand{\A}{\mathbb{A}}
\newcommand{\bbB}{\mathbb{B}}
\newcommand{\bbC}{\mathbb{C}}
\newcommand{\C}{\mathbb{C}}
\newcommand{\bbD}{\mathbb{D}}
\newcommand{\bbE}{\mathbb{E}}
\newcommand{\bbF}{\mathbb{F}}
\newcommand{\bbG}{\mathbb{G}}
\newcommand{\G}{\mathbb{G}}
\newcommand{\bbH}{\mathbb{H}}
\newcommand{\HH}{\mathbb{H}}
\newcommand{\bbI}{\mathbb{I}}
\newcommand{\I}{\mathbb{I}}
\newcommand{\bbJ}{\mathbb{J}}
\newcommand{\bbK}{\mathbb{K}}
\newcommand{\bbL}{\mathbb{L}}
\newcommand{\bbM}{\mathbb{M}}
\newcommand{\bbN}{\mathbb{N}}
\newcommand{\N}{\mathbb{N}}
\newcommand{\bbO}{\mathbb{O}}
\newcommand{\bbP}{\mathbb{P}}
\newcommand{\PP}{\mathbb{P}}
\newcommand{\bbQ}{\mathbb{Q}}
\newcommand{\Q}{\mathbb{Q}}
\newcommand{\bbR}{\mathbb{R}}
\newcommand{\R}{\mathbb{R}}
\newcommand{\bbS}{\mathbb{S}}
\newcommand{\bbT}{\mathbb{T}}
\newcommand{\bbU}{\mathbb{U}}
\newcommand{\bbV}{\mathbb{V}}
\newcommand{\bbW}{\mathbb{W}}
\newcommand{\bbX}{\mathbb{X}}
\newcommand{\bbY}{\mathbb{Y}}
\newcommand{\bbZ}{\mathbb{Z}}
\newcommand{\Z}{\mathbb{Z}}

\newcommand{\scrA}{\mathcal{A}}
\newcommand{\scrB}{\mathcal{B}}
\newcommand{\scrC}{\mathcal{C}}
\newcommand{\scrD}{\mathcal{D}}
\newcommand{\scrE}{\mathcal{E}}
\newcommand{\scrF}{\mathcal{F}}
\newcommand{\scrG}{\mathcal{G}}
\newcommand{\scrH}{\mathcal{H}}
\newcommand{\scrI}{\mathcal{I}}
\newcommand{\scrJ}{\mathcal{J}}
\newcommand{\scrK}{\mathcal{K}}
\newcommand{\scrL}{\mathcal{L}}
\newcommand{\scrM}{\mathcal{M}}
\newcommand{\scrN}{\mathcal{N}}
\newcommand{\scrO}{\mathcal{O}}
\newcommand{\scrP}{\mathcal{P}}
\newcommand{\scrQ}{\mathcal{Q}}
\newcommand{\scrR}{\mathcal{R}}
\newcommand{\scrS}{\mathcal{S}}
\newcommand{\scrT}{\mathcal{T}}
\newcommand{\scrU}{\mathcal{U}}
\newcommand{\scrV}{\mathcal{V}}
\newcommand{\scrW}{\mathcal{W}}
\newcommand{\scrX}{\mathcal{X}}
\newcommand{\scrY}{\mathcal{Y}}
\newcommand{\scrZ}{\mathcal{Z}}

\newcommand{\subgp}{\mathrel{\unlhd}}

\DeclarePairedDelimiter\ceil{\lceil}{\rceil}
\DeclarePairedDelimiter\floor{\lfloor}{\rfloor}

\newcommand{\colorcomment}[2]{\textcolor{#1}{#2}} %First of these leaves in comments. Second one kills them.
%\newcommand{\colorcomment}[2]{}


\pagestyle{fancy}
\lhead{Max Jeter}
\rhead{MA503, Assignment 2, Page \thepage}

\begin{document}

{\bf Problem 1:}

Let $G$ be a group, with $H<G$ and $K <G$.

Let $HK < G$.

\tab Then for all $x,y \in HK$, $xy^{-1} \in HK$.

\tab Let $h \in H$ and $k \in K$. Then $e,h,k,h^{-1},k^{-1} \in HK$. This means that $h^{-1}k^{-1} \in HK$. So $ekh = kh \in HK$. That is, if $h \in H$ and $k \in K$, $kh \in HK$. That is, any element of $KH$ is contained in $HK$. Similarly, any element of $HK$ is contained in $KH$

\tab So $HK = KH$ if $HK$ is a subgroup.

Now, let $HK=KH$.

\tab Then let $x,y \in HK$. There are $h_1,h_2 \in H$, $k_1,k_2 \in K$ such that $h_1k_1 = x$ and $h_2k_2 = y$.

\tab Now, $HK=KH$. So $xy^{-1} = h_1k_1k_2^{-1}h_2^{-1} \in HKKH=HKH=HHK=HK$

\tab So if $x,y \in HK$, then $xy^{-1} \in HK$. This means that $HK$ satisfies the subgroup criterion; $HK$ is a subgroup.

Thus, $HK < G$ if and only if $KH=HK$.

\shunt

{\bf Problem 2:}

Let $G$ be a group and $H \subgp G$ and $K \subgp G$, such that $H \cup K = \{e\}$.

Part a:

Let $h \in H$, $k \in K$.

Because $K$ is normal, $hkh^{-1} \in K$.

Thus, $hkh^{-1}k^{-1} \in K$.

But because $H$ is normal, $kh^{-1}k^{-1} \in H$.

So $hkh^{-1}k^{-1} \in H$.

So $hkh^{-1}k^{-1} \in H \cup K$, which means that $hkh^{-1}k^{-1} = e$.

So $hkh^{-1} = k$, which means that $hk = kh$.

So $hk = kh$ for all $h \in H$, $k \in K$.

\shunt

Part b:

From the above, it is clear that $HK = KH$. From this fact and problem 1, it follows that $HK$ is a subgroup of $G$.

Now, let $\phi : H \times K \to HK$ be given by $\phi((h,k)) = hk$.

We show that $\phi$ is an isomorphism:

First, $\phi$ is a homomorphism:

\tab Let $(a,b), (c,d) \in H \times K$.

\tab Then 

\begin{align*}
\phi((a,b)(c,d)) &= \phi((ac,bd)) \\
&= acbd \\
&= abcd \\
&= \phi((a,b))\phi((c,d))
\end{align*}

\tab (The third line follows from part a)

\tab To summarize the above, $\phi((a,b)(c,d)) = \phi((a,b))\phi((c,d))$ for all $(a,b), (c,d) \in H \times K$. That is, $\phi$ is a homomorphism.

Next, $\phi$ is one-to-one:

\tab Let $(a,b) \in \ker(\phi)$. Then $ab = e$. In other words, $a = b^{-1}$. This implies that $a \in K$, which would mean that $a = e$. This means that $b = e$.

\tab So $\ker(\phi) = \{e\}$. This means that $\phi$ is one-to-one. (If we don't know this implication yet...If $\phi$ is not one-to-one, then there are two distinct elements ($x,y$) that map to the same thing ($z$) so you can show that there is an element ($xy^{-1}$) that maps to $e$...So the kernel would have more than one thing in it. Take the contrapositive, and you get this result.)

Last $\phi$ is onto:

\tab Let $x \in HK$. Then $x = hk$ for some $h \in H$, $k \in K$. So $x = \phi((h,k))$ for some $h \in H$, $k \in K$.

Thus, there is an isomorphism from $H \times K$ to $HK$. That is, $H \times K \cong HK$.

\shunt

{\bf Problem 3:}

First, $Q_8$ is non-Abelian: 
\begin{align*}
\left[\begin{smallmatrix}0&1\\ -1&0 \end{smallmatrix}\right] \left[\begin{smallmatrix}0&i\\ i&0 \end{smallmatrix}\right] &= \left[\begin{smallmatrix}i&0\\ 0&-i \end{smallmatrix}\right] \\
\left[\begin{smallmatrix}0&i\\ i&0 \end{smallmatrix}\right] \left[\begin{smallmatrix}0&1\\ -1&0 \end{smallmatrix}\right] &= \left[\begin{smallmatrix}-i&0\\ 0&i \end{smallmatrix}\right]
\end{align*}


However, all of $Q_8$'s subgroups are normal.

Because $8 = 2*2*2$, any subgroup of $Q_8$ has order $1$,$2$,$4$, or $8$.

Any subgroup of order $1$,$4$, or $8$ is trivially normal, from the discussion in class. It only remains to show that the subgroups of order $2$ are normal.

Now, there is only one subgroup of order $2$ in $Q_8$; it is $\{I, -I\}$. This is clear because there is only one element of order $2$ in $Q_8$, and any subgroup of order $2$ has to have exactly one element of order $2$ (which is trivial from Cayley's theorem... elements of a group must have order dividing the group, and there can only be one element of order $1$ ($e$). So there has to be an element of an order other than $1$...there must be an element of order $2$. But $e$ has to be in the subgroup, so there's an element of order $1$. And because there's only two elements, one of them is $e$ and the other is the element of order $2$).

Now, $\{I, -I\}$ is normal:

Let $A \in Q_8$.

Recall that $I$ and $-I$ commute with every matrix.

$AIA^{-1} = IAA^{-1} = II=I$.

$A(-I)A^{-1} = (-I)AA^{-1} = (-I)I=-I$.

So for all $A \in Q_8$ and $B \in \{I, -I\}$, $ABA^{-1} \in \{I, -I\}$. That is, $\{I, -I\}$ is normal.

So all of $Q_8$'s subgroups of order $1$, $2$, $4$, and $8$ are normal. That is, all of $Q_8$'s subgroups are normal even though $Q_8$ isn't abelian.

\shunt

{\bf Problem 4:}

Consider $\anbrack{s} < \anbrack{s,r^2} < D_4$.

Now, $\anbrack{s,r^2} = \{e,s,r^2,sr^2\}$ has order $4$; it is normal in $D_4$.

Also, $\anbrack{s}$ has order $2$; it is normal in $\anbrack{blah}$.

However, $\anbrack{s}$ is not normal in $D_4$: $rsr^{-1} = sr^3r^{-1}=sr^2$, and $sr^2 \notin \anbrack{s}$.

So $\anbrack{s} \subgp \anbrack{s,r^2} \subgp D_4$, but $\anbrack{s}$ isn't a normal subgroup of $D_4$.

\shunt

{\bf Problem 5:}

Part a:

\shunt

Part b (i):

\shunt

Part b (ii):

\shunt

{\bf Problem 6:}

\shunt

\end{document}