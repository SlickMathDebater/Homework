
\documentclass[a4paper,12pt]{article}

\usepackage{fancyhdr}
\usepackage{amssymb}
%\usepackage{mathpazo}
\usepackage{mathtools}
\usepackage{amsmath}
\usepackage{slashed}
\usepackage[mathscr]{euscript}

\newcommand{\tab}{\hspace{4mm}} %Spacing aliases
\newcommand{\shunt}{\vspace{20mm}}

\newcommand{\sd}{\partial} %Squiggle d

\newcommand{\absval}[1]{\lvert #1 \rvert}
\newcommand{\anbrack}[1]{\left\langle #1 \right\rangle}
\newcommand{\norm}[1]{\|#1\|}


\newcommand{\al}{\alpha} %Steal ALL of Dr. Kable's Aliases! MWAHAHAHAHA!
\newcommand{\be}{\beta}
\newcommand{\ga}{\gamma}
\newcommand{\Ga}{\Gamma}
\newcommand{\de}{\delta}
\newcommand{\De}{\Delta}
\newcommand{\ep}{\epsilon}
\newcommand{\vep}{\varepsilon}
\newcommand{\ze}{\zeta}
\newcommand{\et}{\eta}
\newcommand{\tha}{\theta}
\newcommand{\vtha}{\vartheta}
\newcommand{\Tha}{\Theta}
\newcommand{\io}{\iota}
\newcommand{\ka}{\kappa}
\newcommand{\la}{\lambda}
\newcommand{\La}{\Lambda}
\newcommand{\rh}{\rho}
\newcommand{\si}{\sigma}
\newcommand{\Si}{\Sigma}
\newcommand{\ta}{\tau}
\newcommand{\ups}{\upsilon}
\newcommand{\Ups}{\Upsilon}
\newcommand{\ph}{\phi}
\newcommand{\Ph}{\Phi}
\newcommand{\vph}{\varphi}
\newcommand{\vpi}{\varpi}
\newcommand{\ch}{\chi}
\newcommand{\ps}{\psi}
\newcommand{\Ps}{\Psi}
\newcommand{\om}{\omega}
\newcommand{\Om}{\Omega}

\newcommand{\bbA}{\mathbb{A}}
\newcommand{\A}{\mathbb{A}}
\newcommand{\bbB}{\mathbb{B}}
\newcommand{\bbC}{\mathbb{C}}
\newcommand{\C}{\mathbb{C}}
\newcommand{\bbD}{\mathbb{D}}
\newcommand{\bbE}{\mathbb{E}}
\newcommand{\bbF}{\mathbb{F}}
\newcommand{\bbG}{\mathbb{G}}
\newcommand{\G}{\mathbb{G}}
\newcommand{\bbH}{\mathbb{H}}
\newcommand{\HH}{\mathbb{H}}
\newcommand{\bbI}{\mathbb{I}}
\newcommand{\I}{\mathbb{I}}
\newcommand{\bbJ}{\mathbb{J}}
\newcommand{\bbK}{\mathbb{K}}
\newcommand{\bbL}{\mathbb{L}}
\newcommand{\bbM}{\mathbb{M}}
\newcommand{\bbN}{\mathbb{N}}
\newcommand{\N}{\mathbb{N}}
\newcommand{\bbO}{\mathbb{O}}
\newcommand{\bbP}{\mathbb{P}}
\newcommand{\PP}{\mathbb{P}}
\newcommand{\bbQ}{\mathbb{Q}}
\newcommand{\Q}{\mathbb{Q}}
\newcommand{\bbR}{\mathbb{R}}
\newcommand{\R}{\mathbb{R}}
\newcommand{\bbS}{\mathbb{S}}
\newcommand{\bbT}{\mathbb{T}}
\newcommand{\bbU}{\mathbb{U}}
\newcommand{\bbV}{\mathbb{V}}
\newcommand{\bbW}{\mathbb{W}}
\newcommand{\bbX}{\mathbb{X}}
\newcommand{\bbY}{\mathbb{Y}}
\newcommand{\bbZ}{\mathbb{Z}}
\newcommand{\Z}{\mathbb{Z}}

\newcommand{\scrA}{\mathcal{A}}
\newcommand{\scrB}{\mathcal{B}}
\newcommand{\scrC}{\mathcal{C}}
\newcommand{\scrD}{\mathcal{D}}
\newcommand{\scrE}{\mathcal{E}}
\newcommand{\scrF}{\mathcal{F}}
\newcommand{\scrG}{\mathcal{G}}
\newcommand{\scrH}{\mathcal{H}}
\newcommand{\scrI}{\mathcal{I}}
\newcommand{\scrJ}{\mathcal{J}}
\newcommand{\scrK}{\mathcal{K}}
\newcommand{\scrL}{\mathcal{L}}
\newcommand{\scrM}{\mathcal{M}}
\newcommand{\scrN}{\mathcal{N}}
\newcommand{\scrO}{\mathcal{O}}
\newcommand{\scrP}{\mathcal{P}}
\newcommand{\scrQ}{\mathcal{Q}}
\newcommand{\scrR}{\mathcal{R}}
\newcommand{\scrS}{\mathcal{S}}
\newcommand{\scrT}{\mathcal{T}}
\newcommand{\scrU}{\mathcal{U}}
\newcommand{\scrV}{\mathcal{V}}
\newcommand{\scrW}{\mathcal{W}}
\newcommand{\scrX}{\mathcal{X}}
\newcommand{\scrY}{\mathcal{Y}}
\newcommand{\scrZ}{\mathcal{Z}}

\newcommand{\subgp}{\mathrel{\unlhd}}

\DeclarePairedDelimiter\ceil{\lceil}{\rceil}
\DeclarePairedDelimiter\floor{\lfloor}{\rfloor}

\newcommand{\colorcomment}[2]{\textcolor{#1}{#2}} %First of these leaves in comments. Second one kills them.
%\newcommand{\colorcomment}[2]{}


\pagestyle{fancy}
\lhead{Max Jeter}
\rhead{MA503, Assignment 2, Page \thepage}

\begin{document}

{\bf Problem 1:}

Let $G$ be a group, with $H<G$ and $K <G$.

Let $HK < G$.

\tab Then for all $x,y \in HK$, $xy^{-1} \in HK$.

\tab Let $h \in H$ and $k \in K$. Then $h^{-1} \in H$ and $k^{-1} \in K$. Also, $e \in H$ and $e \in K$. Then $h^{-1}e=h^{-1},ek^{-1}=k^{-1},ee=e \in HK$. Because $HK$ is a subgroup, this means that $h^{-1}k^{-1} \in HK$. So, by the subgroup criterion, $e(h^{-1}k^{-1})^{-1}=ekh = kh \in HK$.

\tab To summarize, if $h \in H$ and $k \in K$, $kh \in HK$. That is, any element of $KH$ is contained in $HK$. Similarly, any element of $HK$ is contained in $KH$

\tab So $HK = KH$ if $HK$ is a subgroup.

Now, let $HK=KH$.

\tab Then let $x,y \in HK$. There are $h_1,h_2 \in H$, $k_1,k_2 \in K$ such that $h_1k_1 = x$ and $h_2k_2 = y$.

\tab Now, $HK=KH$. So $xy^{-1} = h_1k_1k_2^{-1}h_2^{-1} \in HKKH=HKH=HHK=HK$

\tab So if $x,y \in HK$, then $xy^{-1} \in HK$. This means that $HK$ satisfies the subgroup criterion; $HK$ is a subgroup.

Thus, $HK < G$ if and only if $KH=HK$.

\shunt

{\bf Problem 2:}

Let $G$ be a group and $H \subgp G$ and $K \subgp G$, such that $H \cup K = \{e\}$.

Part a:

Let $h \in H$, $k \in K$.

Because $K$ is normal, $hkh^{-1} \in K$.

Thus, $hkh^{-1}k^{-1} \in K$.

But because $H$ is normal, $kh^{-1}k^{-1} \in H$.

So $hkh^{-1}k^{-1} \in H$.

So $hkh^{-1}k^{-1} \in H \cup K$, which means that $hkh^{-1}k^{-1} = e$.

So $hkh^{-1} = k$, which means that $hk = kh$.

So $hk = kh$ for all $h \in H$, $k \in K$.

\shunt

Part b:

From the above, it is clear that $HK = KH$.(If this is not clear: let $x \in HK$. Then $x = hk$ for some $h \in H$, $k \in K$. But this means that $x = kh$. So $x =kh$ for some $k \in K$, $h \in H$; that is, $x \in KH$. Similarly, if $x \in KH$, then $x \in HK$.)

Moving on, from this fact and problem 1, it follows that $HK$ is a subgroup of $G$.

Now, let $\phi : H \times K \to HK$ be given by $\phi((h,k)) = hk$.

We show that $\phi$ is an isomorphism:

First, $\phi$ is a homomorphism:

\tab Let $(a,b), (c,d) \in H \times K$.

\tab Then 

\begin{align*}
\phi((a,b)(c,d)) &= \phi((ac,bd)) \\
&= acbd \\
&= abcd \tab \text{ (this follows from part a) }\\
&= \phi((a,b))\phi((c,d))
\end{align*}

\tab To summarize the above, $\phi((a,b)(c,d)) = \phi((a,b))\phi((c,d))$ for all $(a,b), (c,d) \in H \times K$. That is, $\phi$ is a homomorphism.

Next, $\phi$ is one-to-one:

\tab Let $(a,b) \in \ker(\phi)$. Then $ab = e$. In other words, $a = b^{-1}$. This implies that $a \in K$, which would mean that $a = e$. This means that $b = e$.

\tab So $\ker(\phi) = \{e\}$. This means that $\phi$ is one-to-one.

\tab \tab (If a homomorphism, $\phi$, is not one-to-one, then there are two distinct elements ($x,y$) that $\phi$ maps to the same thing ($z$) so there is an element ($xy^{-1}$) that $\phi$ maps to $e$...So the kernel would have more than one thing in it. Take the contrapositive, and you get the above line. (I can't recall if we've done this in class yet...))

Last $\phi$ is onto:

\tab Let $x \in HK$. Then $x = hk$ for some $h \in H$, $k \in K$. So $x = \phi((h,k))$ for some $h \in H$, $k \in K$.

Thus, there is an isomorphism from $H \times K$ to $HK$. That is, $H \times K \cong HK$.

\shunt

{\bf Problem 3:}

First, $Q_8$ is non-Abelian. Here is a display of this fact: 
\begin{align*}
\left[\begin{smallmatrix}0&1\\ -1&0 \end{smallmatrix}\right] \left[\begin{smallmatrix}0&i\\ i&0 \end{smallmatrix}\right] &= \left[\begin{smallmatrix}i&0\\ 0&-i \end{smallmatrix}\right] \\
\left[\begin{smallmatrix}0&i\\ i&0 \end{smallmatrix}\right] \left[\begin{smallmatrix}0&1\\ -1&0 \end{smallmatrix}\right] &= \left[\begin{smallmatrix}-i&0\\ 0&i \end{smallmatrix}\right]
\end{align*}

However, all of $Q_8$'s subgroups are normal.

Because $8 = 2 \times 2 \times 2$, any subgroup of $Q_8$ has order $1$,$2$,$4$, or $8$.

Any subgroup of order $1$,$4$, or $8$ is trivially normal, from the discussion in class. (For order $1$, the subgroup is necessarily just $\{e\}$. For order $8$, the subgroup is necessarily just $Q_8$. For order $4$, we apply the fact that subgroups of order ``half the order of the original group'' are normal.) It only remains to show that the subgroups of order $2$ are normal.

Now, there is only one subgroup of order $2$ in $Q_8$; it is $\{I, -I\}$. This is clear because there is only one element of order $2$ in $Q_8$, and any subgroup of order $2$ has to have exactly one element of order $2$ (which is trivial from Cayley's theorem... elements of a group must have order dividing the group, and there can only be one element of order $1$ ($e$). So there has to be an element of an order other than $1$...there must be an element of order $2$. But $e$ has to be in the subgroup, so there's an element of order $1$. And because there's only two elements, one of them is $e$ and the other is the element of order $2$).

Now, $\{I, -I\}$ is normal:

Let $A \in Q_8$.

Recall that $I$ and $-I$ commute with every matrix.

This means that $AIA^{-1} = IAA^{-1} = II=I$.

And also that $A(-I)A^{-1} = (-I)AA^{-1} = (-I)I=-I$.

So for all $A \in Q_8$ and $B \in \{I, -I\}$, $ABA^{-1} \in \{I, -I\}$. That is, $\{I, -I\}$ is normal.

So all of $Q_8$'s subgroups of order $1$, $2$, $4$, and $8$ are normal. That is, all of $Q_8$'s subgroups are normal even though $Q_8$ isn't abelian.

\shunt

{\bf Problem 4:}

Consider $\anbrack{s} < \anbrack{s,r^2} < D_4$.

Now, $\anbrack{s,r^2} = \{e,s,r^2,sr^2\}$ has order $4$; it is normal in $D_4$.

Also, $\anbrack{s}$ has order $2$; it is normal in $\anbrack{s,r^2}$.

However, $\anbrack{s}$ is not normal in $D_4$: $rsr^{-1} = sr^3r^{-1}=sr^2$, and $sr^2 \notin \anbrack{s}$.

So $\anbrack{s} \subgp \anbrack{s,r^2} \subgp D_4$, but $\anbrack{s}$ isn't a normal subgroup of $D_4$.

\shunt

{\bf Problem 5:}

Note: I will suppress the overline notation in this problem. That is, for this problem consider $\overline{a/p^i}$ and $a/p^i$ to be the same thing. Moreover, I am using $+$ as the operation, because this makes the problem more natural.

Part a:

First, $\Z_{p^\infty}$ is infinite: 

\tab For each $x \in \N$, $1/p^x \in \Z_{p^\infty}$.

\tab That is, there is an injection of $\N$ into $\Z_{p^\infty}$. So there's an injection of some infinite set into $\Z_{p^\infty}$...that means that $\Z_{p^\infty}$ is infinite.

Second, $\Z_{p^\infty}$ is a subgroup of $\Q/\Z$:

\tab First, we point out that $\Q/\Z$ is a group; $\Q$ is a group, and $\Z$ is an abelian subgroup of $\Q$. This means that $\Z$ is normal in $\Q$.

\tab Now, we apply the subgroup criterion to show that $\Z_{p^\infty}$ is a subgroup of $\Q/\Z$; let $x, y \in \Z_{p^\infty}$. 

\tab Then $x = a/p^i$ for some $a \in \Z$, $i \geq 0$.

\tab Also, $y = b/p^j$ for some $b \in \Z$, $j \geq 0$.

\tab So $x+y^{-1} = a/p^i - b/p^j = \frac{ap^j - bp^i}{p^{i+j}}$. That is, $x+y^{-1} = c/p^k$ for some $c \in \Z$, $k \geq 0$. So $x+y^{-1} \in \Z_{p^\infty}$.

\tab So applying the subgroup criterion, $\Z_{p^\infty}$ is a subgroup of $\Q/\Z$. 

Last, $\Z_{p^\infty}$ is abelian:

\tab Let $x, y \in \Z_{p^\infty}$.

\tab Then $x = a/p^i$ for some $a \in \Z$, $i \geq 0$.

\tab Also, $y = b/p^j$ for some $b \in \Z$, $j \geq 0$.

\tab This means that 

\begin{align*}
x+y &= a/p^i + b/p^j \\
&= \frac{ap^j + bp^i}{p^{i+j}}\\
&= \frac{ bp^i+ ap^j}{p^{i+j}}\\
&= b/p^j+ a/p^i\\
&= y+x
\end{align*}

\tab So for all $x, y \in \Z_{p^\infty}$, $x+y=y+x$.

So $\Z_{p^\infty}$ is an infinite abelian subgroup of $\Q/\Z$. That means that $\Z_{p^\infty}$ is an infinite abelian group.

\shunt

Part b:

Let $H < \Z_{p^\infty}$ with $H \neq \Z_{p^\infty}$.

Part (i):

First, there is an element $1/p^n \in H$, for some $n \in \N$; the identity is of the form $1/p^0$.

Next, there is a largest $n \in \N$ such that $1/p^n \in H$:

\tab Assume not.

\tab Now, if $1/p^N \in H$, then $1/p^n \in H$ for all $n < N$; because $H$ is a subgroup, $1/p^N \in H$ implies that $1/p^{N-1} = p/p^N \in H$. Similarly, $1/p^{N-2}$,$1/p^{N-3}$...and $1/p^{N-N}$ are all in $H$.

\tab But there is no largest $n \in \N$ such that $1/p^n \in H$. So by the above line, for every $n \in \N$, $1/p^n \in H$ (because we can just pick a sufficiently large $n \in \N$ and apply the above line to it...).

\tab But we know that $H$ is a subgroup; this means that $a/p^n \in H$ for all $a in \Z$, $n \in \N$ (either add $1/p^n$ $a$ times if $a \geq 0$, or add $-1/p^n$ $-a$ times if $a < 0$). In other words, $a/p^i \in H$ for all $a \in \Z$ and $i \geq 0$.

\tab And the above line means that $H = \Z_{p^\infty}$, which we assumed to not be the case.

Last, $H = \anbrack{1/p^n}$:

\tab Let $x \in H$.

\tab Then $x = a/p^i$ for some $a \in \Z$, $i \in \N$.

\tab If $x = 0$, then $x \in \anbrack{1/p^n}$.

\tab Else, we can reduce the above fraction so that $a$ and $p$ are relatively prime (if we don't have it fully reduced, we can just reduce the fraction until it is in that form or until we wind up with an integer...in which case, $x=0$).

\tab Now, we know (from number theory) that there are $r,s \in \Z$ such that $ra+sp^i = 1$.

\tab Consider $rx$;

\begin{align*}
rx &= ra/p^i\\
&= ra/p^i + sp^i/p^i\\
&= (ra+sp^i)/p^i\\
&= 1/p^i
\end{align*}

\tab This means that if $x \in H$ is such that $x=a/p^i$, then $1/p^i \in H$. Because we know that there is a largest $n$ such that $1/p^n \in H$, this means that $i \leq n$.

\tab So if $x \in H$, then if we take $x=a/p^i$, we have that $x=ap^j/p^n$ for some $j \geq 0$. That means that $x \in \anbrack{1/p^n}$;

\tab So we have that $H \subset \anbrack{1/p^n}$. But because $1/p^n \in H$ and $H$ is a subgroup, this means that $H \supset \anbrack{1/p^n}$. So we have that $H = \anbrack{1/p^n}$.

So, $H = \anbrack{1/p^n}$ for some $n \geq 0$.

Moving on, $\anbrack{1/p^n} \cong \Z_{p^n}$ (Here, I guess that $\Z_{p^n} = \{a/p^n: a \in \Z\}$.):

\tab Consider the map $\phi: \anbrack{1/p^n} \to \Z_{p^n}$ given by $a/p^n \mapsto a/p^n$. (Note that this is just the identity map.)

First, $\phi$ is a homomorphism:

\tab Let $x,y \in \anbrack{1/p^n}$. Then

\begin{align*}
\phi(x+y) &= x+y\\
&=\phi(x)+\phi(y)
\end{align*}

Next, $\phi$ is one-to-one:

\tab Let $x,y \in \anbrack{1/p^n}$ be such that $\phi(x)=\phi(y)$.

\tab Then

\begin{align*}
\phi(x) &=\phi(y)\\
x&=y
\end{align*}

Last, $\phi$ is onto:

\tab Let $x \in \Z_{p^n}$. Then $\phi(x) = x$. So there is a $y \in \anbrack{1/p^n}$ such that $\phi(y) = y$.

So $\phi$ is a bijective homomorphism; it is an isomorphism. That means that $\anbrack{1/p^n} \cong \Z_{p^n}$.

To summarize the above, we have shown that $H = \anbrack{1/p^n} \cong \Z_{p^n}$

\shunt

Part (ii):

First, we point out that $H$ is normal, because this is an abelian group. That means that $\Z_{p^\infty}/H$ is defined.

Moving on, we exhibit an isomorphism from $\Z_{p^\infty}$ to $\Z_{p^\infty}/H$.

Consider $\phi: \Z_{p^\infty} \to \Z_{p^\infty}/H $ given by $x \mapsto \overline{x/p^n}$.

Now, $\phi$ is a homomorphism:

\tab Let $x,y \in \Z_{p^\infty}$. Then:

\begin{align*}
\phi(x+y) &= (x+y)/p^n\\
&= x/p^n + y/p^n\\
&= \phi(x)+\phi(y)
\end{align*}

\tab So $\phi(x+y) = \phi(x) + \phi(y)$; $\phi$ is a homomorphism.

Also, $\phi$ is one-to-one:

\tab Let $x \in \ker(\phi)$. Then $x/p^n \in H$. So $x/p^n = a/p^n$ for some $a \in \Z$. This means that $x \in \Z$, and we are working in $\Q/\Z$; this means that $x = 0$.

\tab We already know that $\ker(\phi)=\{0\}$ implies that $\phi$ is one-to-one if $\phi$ is a homomorphism. So because $\phi$ is a homomorphism with $\ker (\phi)=\{0\}$, $\phi$ is one-to-one.

Last $\phi$ is onto:

\tab Let $y \in \Z_{p^\infty}/H$. Then there is an $x \in \Z_{p^\infty}$ such that $\overline{x} = y$. Now, $\phi(xp^n) = \overline{x} = y$. So there is an $x \in \Z_{p^\infty}$ such that $\phi(x)=y$.

\tab So for all $y \in \Z_{p^\infty}/H$, there is an $x \in \Z_{p^\infty}$ such that $\phi(x)=y$.

\tab So $\phi$ is onto.

Thus, $\phi$ is a bijective homomorphism; it is an isomorphism. That means that $\Z_{p^\infty} \cong \Z_{p^\infty}/H$.

\end{document}