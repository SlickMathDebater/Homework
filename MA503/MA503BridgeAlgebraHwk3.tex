
\documentclass[a4paper,12pt]{article}

\usepackage{fancyhdr}
\usepackage{amssymb}
%\usepackage{mathpazo}
\usepackage{mathtools}
\usepackage{amsmath}
\usepackage{slashed}
\usepackage[mathscr]{euscript}

\newcommand{\tab}{\hspace{4mm}} %Spacing aliases
\newcommand{\shunt}{\vspace{20mm}}

\newcommand{\sd}{\partial} %Squiggle d

\newcommand{\absval}[1]{\lvert #1 \rvert}
\newcommand{\anbrack}[1]{\left\langle #1 \right\rangle}
\newcommand{\norm}[1]{\|#1\|}


\newcommand{\al}{\alpha} %Steal ALL of Dr. Kable's Aliases! MWAHAHAHAHA!
\newcommand{\be}{\beta}
\newcommand{\ga}{\gamma}
\newcommand{\Ga}{\Gamma}
\newcommand{\de}{\delta}
\newcommand{\De}{\Delta}
\newcommand{\ep}{\epsilon}
\newcommand{\vep}{\varepsilon}
\newcommand{\ze}{\zeta}
\newcommand{\et}{\eta}
\newcommand{\tha}{\theta}
\newcommand{\vtha}{\vartheta}
\newcommand{\Tha}{\Theta}
\newcommand{\io}{\iota}
\newcommand{\ka}{\kappa}
\newcommand{\la}{\lambda}
\newcommand{\La}{\Lambda}
\newcommand{\rh}{\rho}
\newcommand{\si}{\sigma}
\newcommand{\Si}{\Sigma}
\newcommand{\ta}{\tau}
\newcommand{\ups}{\upsilon}
\newcommand{\Ups}{\Upsilon}
\newcommand{\ph}{\phi}
\newcommand{\Ph}{\Phi}
\newcommand{\vph}{\varphi}
\newcommand{\vpi}{\varpi}
\newcommand{\ch}{\chi}
\newcommand{\ps}{\psi}
\newcommand{\Ps}{\Psi}
\newcommand{\om}{\omega}
\newcommand{\Om}{\Omega}

\newcommand{\bbA}{\mathbb{A}}
\newcommand{\A}{\mathbb{A}}
\newcommand{\bbB}{\mathbb{B}}
\newcommand{\bbC}{\mathbb{C}}
\newcommand{\C}{\mathbb{C}}
\newcommand{\bbD}{\mathbb{D}}
\newcommand{\bbE}{\mathbb{E}}
\newcommand{\bbF}{\mathbb{F}}
\newcommand{\bbG}{\mathbb{G}}
\newcommand{\G}{\mathbb{G}}
\newcommand{\bbH}{\mathbb{H}}
\newcommand{\HH}{\mathbb{H}}
\newcommand{\bbI}{\mathbb{I}}
\newcommand{\I}{\mathbb{I}}
\newcommand{\bbJ}{\mathbb{J}}
\newcommand{\bbK}{\mathbb{K}}
\newcommand{\bbL}{\mathbb{L}}
\newcommand{\bbM}{\mathbb{M}}
\newcommand{\bbN}{\mathbb{N}}
\newcommand{\N}{\mathbb{N}}
\newcommand{\bbO}{\mathbb{O}}
\newcommand{\bbP}{\mathbb{P}}
\newcommand{\PP}{\mathbb{P}}
\newcommand{\bbQ}{\mathbb{Q}}
\newcommand{\Q}{\mathbb{Q}}
\newcommand{\bbR}{\mathbb{R}}
\newcommand{\R}{\mathbb{R}}
\newcommand{\bbS}{\mathbb{S}}
\newcommand{\bbT}{\mathbb{T}}
\newcommand{\bbU}{\mathbb{U}}
\newcommand{\bbV}{\mathbb{V}}
\newcommand{\bbW}{\mathbb{W}}
\newcommand{\bbX}{\mathbb{X}}
\newcommand{\bbY}{\mathbb{Y}}
\newcommand{\bbZ}{\mathbb{Z}}
\newcommand{\Z}{\mathbb{Z}}

\newcommand{\scrA}{\mathcal{A}}
\newcommand{\scrB}{\mathcal{B}}
\newcommand{\scrC}{\mathcal{C}}
\newcommand{\scrD}{\mathcal{D}}
\newcommand{\scrE}{\mathcal{E}}
\newcommand{\scrF}{\mathcal{F}}
\newcommand{\scrG}{\mathcal{G}}
\newcommand{\scrH}{\mathcal{H}}
\newcommand{\scrI}{\mathcal{I}}
\newcommand{\scrJ}{\mathcal{J}}
\newcommand{\scrK}{\mathcal{K}}
\newcommand{\scrL}{\mathcal{L}}
\newcommand{\scrM}{\mathcal{M}}
\newcommand{\scrN}{\mathcal{N}}
\newcommand{\scrO}{\mathcal{O}}
\newcommand{\scrP}{\mathcal{P}}
\newcommand{\scrQ}{\mathcal{Q}}
\newcommand{\scrR}{\mathcal{R}}
\newcommand{\scrS}{\mathcal{S}}
\newcommand{\scrT}{\mathcal{T}}
\newcommand{\scrU}{\mathcal{U}}
\newcommand{\scrV}{\mathcal{V}}
\newcommand{\scrW}{\mathcal{W}}
\newcommand{\scrX}{\mathcal{X}}
\newcommand{\scrY}{\mathcal{Y}}
\newcommand{\scrZ}{\mathcal{Z}}

\newcommand{\subgp}{\mathrel{\unlhd}}

\DeclarePairedDelimiter\ceil{\lceil}{\rceil}
\DeclarePairedDelimiter\floor{\lfloor}{\rfloor}

\newcommand{\colorcomment}[2]{\textcolor{#1}{#2}} %First of these leaves in comments. Second one kills them.
%\newcommand{\colorcomment}[2]{}


\pagestyle{fancy}
\lhead{Max Jeter}
\rhead{MA503, Assignment 3, Page \thepage}

\begin{document}

Note: I am accustomed to writing ``The element $g \in G$ acting on the element $x \in S$'' as $g.x$ instead of $gx$. I use the stated notation, as I feel it is clearer.

{\bf Problem 1:}

Let $G$ be a finite abelian group, with $n \in \N$ and $n \mid \absval{G}$.

We know that for each $n \in \N$, $n$ has a unique prime factorization; that is, $n=p_1^{a_1}p_2^{a_2}p_3{a_3}\ldots p_k^{a_k}$ for some $p_1, p_2 \ldots p_k$ each prime, and $a_1, a_2 \ldots a_k$ each positive and nonzero.

Proceed as follows:

\tab For each $p_i$, there is an element with order $p_i$ in $G$:

\tab \tab We proceed by induction. If $\absval{G} =1$, then $\absval{G}$ has no prime divisors, so there is vacuously an element of order $p$ in $G$ if $p \mid \absval{G}$.

\tab \tab Now, assume that there is an element of order $p$ (with $p$ prime) in $G$ if $p \mid \absval{G}$ for all $G$ with order less than $N$. Let $G$ be a group of order $N$, and let $p \mid N$ be a prime number. Pick an element of $G$, call it $g$, other than the identity. It has some order, $m>1$. Either $p \mid m$ or not. If so, then take $g^{m/p}$; this element clearly has order $p$. If not, then consider $G/\anbrack{a}$ (this is well defined; $G$ is abelian, so $\anbrack{a}$ is normal, because all subgroups are normal in abelian groups). We know that $p \mid \absval{G /\anbrack{a}}$; $p \mid \absval{G} = \absval{G/\anbrack{a}}\absval{a}$, so because $p$ is prime, we know that $p \mid \absval{G/\anbrack{a}}$. So there's an element, $b$, of order $p$ in $G/\anbrack{a}$. So there is an element with $\overline{c} = b$. Now, $c^p \in \anbrack{a}$;

\begin{align*}
c &= ba^x \text{ for some $x \in \Z$ , because $\overline{c} = b$}\\
c^p &= b^pa^{px} \\
c^p &= a^y  \text{ for some $y \in \Z$ , because $\overline{b}^p = e$}\\
c^p &\in \anbrack{a}
\end{align*}

\tab \tab So, $c^p = a^l$ for some $l \in \N$. We know that $a^l$ has a finite order, $\absval{a^l}$. So clearly $c^\absval{a^l}$ has order $p$.

\tab \tab So in either case, there is an element of order $p$ in $G$ if $p$ is prime and $p \mid \absval{G}$.

\tab Thus, there is a subgroup, $H_1$, with order $p_1$ in $G$. This subgroup is normal, because $G$ is abelian. 

\tab Consider the new group $G_1=G/H_1$, along with $n_1 = n/p_1$. Note that $\absval{G_1} = \absval{G}/p_1$; this follows from the theorem that says $\absval{G/H} = \absval{G}/\absval{H}$ if $H$ is normal.

\tab We know that $G_1$ is abelian, by a theorem we probably discussed in class.

\tab So by similar logic as above, if $a_1 > 1$, then there is a subgroup of order $p_1$ in $G$. Otherwise, we know that there is a subgroup of order $p_2$ in $G$.

\tab Either way, there is a subgroup, $H_2' < G_1$, with order $p_1$ (or $p_2$) in $G_1$. This subgroup is normal in $G_1$, because $G_1$ is abelian. Now, there's a subgroup of $G$ containing $H_1$ that corresponds to $H_2'$: call this subgroup $H_2$. We know that $H_2$ is normal, because $G$ is abelian.

\tab Consider the group $G_2=G_1/H_2'$. Note that $G_2 \cong G/H_2$, by the third isomorphism theorem. Moreover, $\absval{G_2} = \absval{G}/p_1^2$ (or $\absval{G_2} = \absval{G}/p_1p_2$).

\tab We can proceed in the above manner $a_i$ times for each $p_i$. Proceed until each prime factor is exhausted.

Consider $G_{a_1+a_2 \ldots +a_k}$; it has order $\absval{G}/n$, by the construction above, and is isomorphic to $G/H_{a_1+a_2 \ldots +a_k}$ for some $H \subgp G$. This means that $\absval{H} = n$ (because $\absval{G/H} = \absval{G}/\absval{H}$...thus, $\absval{H} = \absval{G}/\absval{G/H}$, or in this case, $\absval{H} = \absval{G}/(\absval{G}/n) = n$.)

So $G$ has a normal subgroup of order $n$ if $G$ is a finite abelian group with $n \mid \absval{G}$.

\shunt

{\bf Problem 2:}

Let $H < G$ with $[G:H]$ finite.

\shunt

{\bf Problem 3:}

Let $G$ be a group acting transitively on a finite set, $S$, with $\absval{S} > 1$.

Now, the action has only one orbit; for all $x \in S$, $\overline{x} = S$. In other words, for every $x, y \in S$ there is a $g \in G$ such that $g.x = y$.

Before proceeding, I wish to point out that I use the following freely:

\tab If $g.x = x$, then $g^{-1}.x=x$: This is clear by applying $g^{-1}$ to both sides of the equation.

\tab If $g.x=y$, then $g^{-1}.y=x$: This is clear by applying $g^{-1}$ to both sides of the equation.

Assume that for all $g \in G$, there is an $x \in S$ such that $g.x=x$. We proceed by constructing an infinite set of points in $S$, by induction.

\tab Because $\absval{S} >1$, there are at least two distinct points of $S$: call them $x_0$ and $x_1$.

\tab There is an element, $g_2$, such that $g_2.x_0=x_1$, by transitivity of the action.

\tab There is an $x_2$ such that $g_2.x_2 = x_2$, by the assumption we made earlier.

\tab Now, $x_2 \neq x_0$, else:

\begin{align*}
g_2.x_0 &= g_2.x_2\\
x_1&= x_2=x_0
\end{align*}

\tab which is a contradiction.

\tab Also, $x_2 \neq x_1$, else:

\begin{align*}
g_2^{-1}.x_1 &= g_2^{-1}.x_2\\
x_0&= x_2=x_1
\end{align*} 

\tab which is also a contradiction.

\tab So $x_2$ is distinct from $x_0$ and $x_1$.

\tab Now, assume that we have the following: we have defined $x_n$ for each $n \in \N$ such that $n < N$, and $g_n$ for each $n \in \N$ such that $n < N-1$ and $n \geq 2$, with the following properties: $g_n.x_n = x_n$ and $g_n.x_0 = x_{n-1}$.

\tab Then there is a $g_N$ such that $g_N.x_0 = x_{N-1}$, because the action is transitive.

\tab Also, there is an $x_N$ such that $g_N.x_N = x_N$, by the assumption we made earlier.

\tab Now, $x_N \neq x_0$, else:

\begin{align*}
g_N.x_0 &= g_N.x_N\\
x_{N-1}&= x_N=x_0
\end{align*}

\tab which is a contradiction.

\tab Also, $x_N \neq x_{N-1}$, else:

\begin{align*}
g_N^{-1}.x_{N-1} &= g_N^{-1}.x_N\\
x_0&= x_N=x_{N-1}
\end{align*} 

\tab which is also a contradiction.

\tab Further, $x_N \neq x_{i}$ for any $i$ between $0$ and $N-1$ (exclusive), else:

\begin{align*}
g_Ng_i^{-1}.x_i &= g_N.x_i\\
g_nx_0 &= g_N.x_N\\
x_{N-1} = x_N
\end{align*} 

\tab which is also a contradiction.

\tab So $x_N$ is distinct from each $x_i$ with $i < N$.

\tab So we have two distinct points, and if we have $n$ distinct points in $S$, we can make $n+1$ distinct points in $S$; we can make infinitely many distinct points, thus $S$ is infinite.

So, if for all $g \in G$, $g$ has a fixed point, then $S$ is infinite.

Or, in other words, because $S$ is finite, there is a $g \in G$ that has no fixed point.

\shunt

{\bf Problem 4:}

Let $G$ be a group such that $G/Z(G)$ is cyclic.

Then there is an $a \in G$ such that for all $\overline{x} \in G/Z(G)$, $\overline{x} = \overline{a}^n$ for some $n \in \N$.

So for all $y \in G$, there is an $n \in \N$ and $b \in Z(G)$ such that $y = a^nb$; the left cosets of $Z(G)$ partition $G$, and the set of these cosets is $\{a^nZ(G): n \in \N\}$. So for all $y \in G$, $y \in a^nZ(G)$ for some $n \in \N$. 

So for all $y, z \in G$, we have $y = a^nb$ and $z=a^mc$ for some $n,m \in \N$ and $b,c \in Z(G)$.

Now we have:

\begin{align*}
yz &= a^nba^mc\\
&=a^na^mbc\\
&=a^ma^nbc\\
&=a^ma^ncb\\
&=a^mca^nb\\
&=zy
\end{align*}

Note that this fails if $G/Z(G)$ is only abelian:

\tab Consider $D_8=\anbrack{r,s}$. We note that the center of $D_8$ is $\{e,r^2\}$.

\tab (We freely use the identity $sr = r^3s$ in the following).

\tab \tab We know that $e$ commutes with every element, trivially.

\tab \tab Now, $r^2$ commutes with $e$, $r$,$r^2$, and $r^3$ trivially; %reasoning.

\tab \tab Also, $r^2$ commutes with $s$, $sr$, $sr^2$, and $sr^3$:

\begin{align*}
r^2s &= rsr^3 = sr^3r^3 = sr^2\\
r^2sr &= r^2r^3s = rs = sr^3 = srr^2 \\ 
r^2sr^2 &= r^2sr^2\\
r^2sr^3 &= r^2rs = sr = sr^5=sr^3r^2
\end{align*}

\tab \tab However, $r$ and $s$ do not commute with each other, and $r^3$ and $s$ do not commute with each other:

\begin{align*}
rs &= sr^3\\
\end{align*}

\tab \tab Also, $sr$, $sr^2$, and $sr^3$ do not commute with $r$:

\begin{align*}
rsr &= sr^3r = s \neq sr^2\\
rsr^2 &= sr^3r^2 = sr \neq sr^3 \\
rsr^3 &= sr^3r^3 = sr^2 \neq sr^3r = s
\end{align*}

\tab \tab So every element other than $e$ and $r^2$ fails to commute with something.
So 

\tab Also, $D_8/\anbrack{r^2}$ is abelian:

\tab \tab Observe that this group must have order $4$, and that $\{\overline{e},\overline{r}, \overline{s}, \overline{sr}\}$ are all distinct elements of this group (and thus this represents all elements in the quotient group).

\tab \tab Clearly, $\overline{e}$ commutes with everything. We proceed by exhaustion:

\begin{align*}
\overline{r}\overline{s} &= \overline{sr} = \overline{r^3s} = \overline{rs} = \overline{r}\overline{s}\\
\overline{r}\overline{sr} &= \overline{rsr} = \overline{sr^3r} = \overline{s} = \overline{sr^2} = \overline{sr}\overline{r}\\
\overline{s}\overline{sr} &= \overline{s^2r} = \overline{r} = \overline{rss} = \overline{sr^3s} = \overline{srs} = \overline{sr}\overline{s}\\
\end{align*}

\tab But we know from an earlier homework that $D_8$ is not abelian. So in general, $G/Z(G)$ being abelian does not imply that $G$ is abelian.

\shunt

{\bf Problem 5:}

Let $p$ be prime, and let $G$ be a group of order $p^2$. We know that $Z(G)$ is a subgroup of $G$; so $Z(G)$ has order $1$, $p$, or $p^2$.

We know that $Z(G)$ does not have order $1$, from the discussion in class.

If $Z(G)$ has order $p$, then $G/Z(G)$ is cyclic (as it is a group of order $p^2/p = p$); we know that $G$ is abelian, from problem 4. 

If $Z(G)$ has order $p^2$, then the center is the entire group; that is, $G$ is abelian.

So in all cases, $G$ is abelian.

\shunt

{\bf Problem 6:}

Let $p$ be prime, and let $G$ be a group of order $p^n$ for some $n \in \N$. Let $H \subgp G$ with $H \neq \{e\}$.

Now, $G$ acts on $H$ by conjugation. By the class equation,

\begin{align*}
\absval{H} &= \absval{\{h \in H: \forall g \in G, ghg^{-1} = h\}} + \sum\absval{\overline{x_i}} \\
&=\absval{H \cap Z(G)} + \sum\absval{[G:G_{x_i}]} \text{ Because the order of the orbit is the index of the stabilizer}
\end{align*}

We know that $p \mid \absval{H}$, because $H$ is a nontrivial subgroup of a $p$-group. Also, $p \mid [G:G_{x_i}]$ because each $G_{x_i}$ is a subgroup of $G$, and none are trivial (because none of those orbits contain only one element; if there was an element whose orbit had only one element, then it would be in the set $\{h \in H: \forall g \in G, ghg^{-1} = h\}$).

So we know that $p \mid \absval{H \cap Z(G)} + \sum\absval{[G:G_{x_i}]}$. So because $p \mid \sum\absval{[G:G_{x_i}]}$, we know that $p \mid \absval{H \cap Z(G)}$.

So because $p \mid \absval{H \cap Z(G)}$, we know that $\absval{H \cap Z(G)} \neq 1$; that is $H \cap Z(G)$ is not the trivial subgroup.

\shunt

\end{document}