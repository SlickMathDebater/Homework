
\documentclass[a4paper,12pt]{article}

\usepackage{fancyhdr}
\usepackage{amssymb}
%\usepackage{mathpazo}
\usepackage{mathtools}
\usepackage{amsmath}
\usepackage{slashed}
\usepackage[mathscr]{euscript}

\newcommand{\tab}{\hspace{4mm}} %Spacing aliases
\newcommand{\shunt}{\vspace{20mm}}

\newcommand{\sd}{\partial} %Squiggle d

\newcommand{\absval}[1]{\lvert #1 \rvert}
\newcommand{\anbrack}[1]{\left\langle #1 \right\rangle}
\newcommand{\norm}[1]{\|#1\|}


\newcommand{\al}{\alpha} %Steal ALL of Dr. Kable's Aliases! MWAHAHAHAHA!
\newcommand{\be}{\beta}
\newcommand{\ga}{\gamma}
\newcommand{\Ga}{\Gamma}
\newcommand{\de}{\delta}
\newcommand{\De}{\Delta}
\newcommand{\ep}{\epsilon}
\newcommand{\vep}{\varepsilon}
\newcommand{\ze}{\zeta}
\newcommand{\et}{\eta}
\newcommand{\tha}{\theta}
\newcommand{\vtha}{\vartheta}
\newcommand{\Tha}{\Theta}
\newcommand{\io}{\iota}
\newcommand{\ka}{\kappa}
\newcommand{\la}{\lambda}
\newcommand{\La}{\Lambda}
\newcommand{\rh}{\rho}
\newcommand{\si}{\sigma}
\newcommand{\Si}{\Sigma}
\newcommand{\ta}{\tau}
\newcommand{\ups}{\upsilon}
\newcommand{\Ups}{\Upsilon}
\newcommand{\ph}{\phi}
\newcommand{\Ph}{\Phi}
\newcommand{\vph}{\varphi}
\newcommand{\vpi}{\varpi}
\newcommand{\ch}{\chi}
\newcommand{\ps}{\psi}
\newcommand{\Ps}{\Psi}
\newcommand{\om}{\omega}
\newcommand{\Om}{\Omega}

\newcommand{\bbA}{\mathbb{A}}
\newcommand{\A}{\mathbb{A}}
\newcommand{\bbB}{\mathbb{B}}
\newcommand{\bbC}{\mathbb{C}}
\newcommand{\C}{\mathbb{C}}
\newcommand{\bbD}{\mathbb{D}}
\newcommand{\bbE}{\mathbb{E}}
\newcommand{\bbF}{\mathbb{F}}
\newcommand{\bbG}{\mathbb{G}}
\newcommand{\G}{\mathbb{G}}
\newcommand{\bbH}{\mathbb{H}}
\newcommand{\HH}{\mathbb{H}}
\newcommand{\bbI}{\mathbb{I}}
\newcommand{\I}{\mathbb{I}}
\newcommand{\bbJ}{\mathbb{J}}
\newcommand{\bbK}{\mathbb{K}}
\newcommand{\bbL}{\mathbb{L}}
\newcommand{\bbM}{\mathbb{M}}
\newcommand{\bbN}{\mathbb{N}}
\newcommand{\N}{\mathbb{N}}
\newcommand{\bbO}{\mathbb{O}}
\newcommand{\bbP}{\mathbb{P}}
\newcommand{\PP}{\mathbb{P}}
\newcommand{\bbQ}{\mathbb{Q}}
\newcommand{\Q}{\mathbb{Q}}
\newcommand{\bbR}{\mathbb{R}}
\newcommand{\R}{\mathbb{R}}
\newcommand{\bbS}{\mathbb{S}}
\newcommand{\bbT}{\mathbb{T}}
\newcommand{\bbU}{\mathbb{U}}
\newcommand{\bbV}{\mathbb{V}}
\newcommand{\bbW}{\mathbb{W}}
\newcommand{\bbX}{\mathbb{X}}
\newcommand{\bbY}{\mathbb{Y}}
\newcommand{\bbZ}{\mathbb{Z}}
\newcommand{\Z}{\mathbb{Z}}

\newcommand{\scrA}{\mathcal{A}}
\newcommand{\scrB}{\mathcal{B}}
\newcommand{\scrC}{\mathcal{C}}
\newcommand{\scrD}{\mathcal{D}}
\newcommand{\scrE}{\mathcal{E}}
\newcommand{\scrF}{\mathcal{F}}
\newcommand{\scrG}{\mathcal{G}}
\newcommand{\scrH}{\mathcal{H}}
\newcommand{\scrI}{\mathcal{I}}
\newcommand{\scrJ}{\mathcal{J}}
\newcommand{\scrK}{\mathcal{K}}
\newcommand{\scrL}{\mathcal{L}}
\newcommand{\scrM}{\mathcal{M}}
\newcommand{\scrN}{\mathcal{N}}
\newcommand{\scrO}{\mathcal{O}}
\newcommand{\scrP}{\mathcal{P}}
\newcommand{\scrQ}{\mathcal{Q}}
\newcommand{\scrR}{\mathcal{R}}
\newcommand{\scrS}{\mathcal{S}}
\newcommand{\scrT}{\mathcal{T}}
\newcommand{\scrU}{\mathcal{U}}
\newcommand{\scrV}{\mathcal{V}}
\newcommand{\scrW}{\mathcal{W}}
\newcommand{\scrX}{\mathcal{X}}
\newcommand{\scrY}{\mathcal{Y}}
\newcommand{\scrZ}{\mathcal{Z}}

\newcommand{\subgp}{\mathrel{\unlhd}}

\DeclarePairedDelimiter\ceil{\lceil}{\rceil}
\DeclarePairedDelimiter\floor{\lfloor}{\rfloor}

\newcommand{\colorcomment}[2]{\textcolor{#1}{#2}} %First of these leaves in comments. Second one kills them.
%\newcommand{\colorcomment}[2]{}


\pagestyle{fancy}
\lhead{Max Jeter}
\rhead{MA503, Assignment 3, Page \thepage}

\begin{document}

{\bf Problem 1:}

Let $G$ be a finite abelian group, with $n \in \N$ and $n \mid \absval{G}$.

We know that for each $n \in \N$, $n$ has a unique prime factorization; that is, $n=p_1^{a_1}p_2^{a_2}p_3{a_3}\ldots p_k^{a_k}$ for some $p_1, p_2 \ldots p_k$ each prime, and $a_1, a_2 \ldots a_k$ each positive and nonzero.

Proceed as follows:

\tab For each $p_i$, there is an element with order $p_i$ in $G$:

\tab \tab %Prove it.

\tab Thus, there is a subgroup, $H_1$, with order $p_1$ in $G$. This subgroup is normal, because $G$ is abelian. 

\tab Consider the new group $G_1=G/H_1$, along with $n_1 = n/p_1$. Note that $\absval{G_1} = \absval{G}/p_1$; this follows from the theorem that says $\absval{G/H} = \absval{G}/\absval{H}$ if $H$ is normal.

\tab We know that $G_1$ is abelian:

\tab \tab %Proof

\tab So by similar logic as above, if $a_1 > 1$, then there is a subgroup of order $p_1$ in $G$. Otherwise, we know that there is a subgroup of order $p_2$ in $G$.

\tab Either way, there is a subgroup, $H_2'$, with order $p_1$ (or $p_2$) in $G_1$, which is normal. %You also have to make H_2 a subgroup of G!

\tab Consider the group $G_2=G_1/H_2$. Note that $G_2 \cong G/H_2$, by the third isomorphism theorem. Moreover, $\absval{G_2} = \absval{G}/p_1^2$ (or $\absval{G_2} = \absval{G}/p_1p_2$).

\tab We can proceed in the above manner for $a_i$ times for each $p_i$. We end up with a group, $G_{a_1+a_2 \ldots + a_k}$.

Consider $G_{a_1+a_2 \ldots +a_k}$; it has order $\absval{G}/n$. It is isomorphic to $G/H_{a_1+a_2 \ldots +a_k}$ for some $H \subgp G$. This means that $\absval{H} = n$ (because $\absval{G/H} = \absval{G}/\absval{H}$...thus, $\absval{H} = \absval{G}/\absval{G/H}$, or in this case, $\absval{H} = \absval{G}/(\absval{G}/n) = n$.)

So $G$ has a normal subgroup of order $n$ if $G$ is a finite abelian group with $n \mid \absval{G}$.

\shunt

{\bf Problem 2:}

Let $H < G$ with $[G:H]$ finite.

%Try the center or normalizer of H: there's an associated action that makes the stabilizer of...something...equal to that. Moreover, the orbit of that...something...should have finite cardinality. If so, then it has a finite index in H. 

\shunt

{\bf Problem 3:}

Let $G$ be a group acting transitively on a finite set, $S$, with $\absval{S} > 1$.

Now, the action has only one orbit; for all $x \in S$, $\overline{x} = S$. In other words, for every $x, y \in S$ there is a $g \in G$ such that $g.x = y$.

Assume that for all $g \in G$, there is an $x \in S$ such that $g.x=x$.

\tab Pick a non-identity element, $g \in G$ (Note: we can do this because we know that $G \neq \{e\}$. %Proof)

\tab There is an element $x \in S$ such that $g.x=x$.

\tab Also, there is an element in $S$, $y$, distinct from $y$.

\tab So, there is an $h \in G$ such that $h.x = y$ (Moreover, $h \neq e$, else $x=y$).

\tab Now, consider the set $\{g^nh.x: n \in \N\}$.

\tab If this set is finite, then %Something bad happens with the fact that x is a fixed point of g.

\tab If this set is infinite, then $S$ was not finite.

\shunt

{\bf Problem 4:}

Let $G$ be a group such that $G/Z(G)$ is cyclic.

Then $G/Z(G) = \anbrack{\overline{a}}$ for some $a \in G$.

%Should be possible to prove that $a = e$...

Note that this fails if $G/Z(G)$ is only abelian:

\tab Consider $D_8=\anbrack{r,s}$. We note that the center of $D_8$ is $\{e,r^2\}$;

\tab \tab %Piece by piece proof.

\tab Also, $D_8/\anbrack{r^2}$ is abelian:

\tab \tab %Piece by piece proof.

\tab But we know from an earlier homework that $D_8$ is not abelian. So in general, $G/Z(G)$ being abelian does not imply that $G$ is abelian.

\shunt

{\bf Problem 5:}

Let $p$ be prime, and let $G$ be a group of order $p^2$.

We know that $G$ has an element of order $p$; any element has order $1$, $p$, or $p^2$. There are at least two elements in $G$ (because $1$ is not prime...). Pick an element other than $e$: call it $a$. We know that $a$ has order $p$ or $p^2$. If it has order $p^2$, then consider $a^p$. We know that $(a^p)^p = a^{p^2}$. So $a^p$ is an element of order $p$.

The upshot is that there is an element, $x$, of order $p$ in $G$. So there is a subgroup, $H=\anbrack{x}$, of order $p$ in $G$. Moreover, $H$ is cyclic (I think we proved this in class...any group of order $p$ is cyclic. If we haven't proved it, then %proof). 

Also, $H$ is normal in $G$: %I think you have to slug this one out...

Now, consider $G/H$; this is a group of order $p$, it is cyclic.

%I think that's the upshot. I'm not entirely sure, though.

\shunt

{\bf Problem 6:}

Let $p$ be prime, and let $G$ be a group of order $p^n$ for some $n \in \N$. Let $H \subgp G$ with $H \neq \{e\}$.

\shunt

\end{document}