
\documentclass[a4paper,12pt]{article}

\usepackage{fancyhdr}
\usepackage{amssymb}
%\usepackage{mathpazo}
\usepackage{mathtools}
\usepackage{amsmath}
\usepackage{slashed}
\usepackage[mathscr]{euscript}

\newcommand{\tab}{\hspace{4mm}} %Spacing aliases
\newcommand{\shunt}{\vspace{20mm}}

\newcommand{\sd}{\partial} %Squiggle d

\newcommand{\absval}[1]{\lvert #1 \rvert}
\newcommand{\anbrack}[1]{\left\langle #1 \right\rangle}
\newcommand{\norm}[1]{\|#1\|}


\newcommand{\al}{\alpha} %Steal ALL of Dr. Kable's Aliases! MWAHAHAHAHA!
\newcommand{\be}{\beta}
\newcommand{\ga}{\gamma}
\newcommand{\Ga}{\Gamma}
\newcommand{\de}{\delta}
\newcommand{\De}{\Delta}
\newcommand{\ep}{\epsilon}
\newcommand{\vep}{\varepsilon}
\newcommand{\ze}{\zeta}
\newcommand{\et}{\eta}
\newcommand{\tha}{\theta}
\newcommand{\vtha}{\vartheta}
\newcommand{\Tha}{\Theta}
\newcommand{\io}{\iota}
\newcommand{\ka}{\kappa}
\newcommand{\la}{\lambda}
\newcommand{\La}{\Lambda}
\newcommand{\rh}{\rho}
\newcommand{\si}{\sigma}
\newcommand{\Si}{\Sigma}
\newcommand{\ta}{\tau}
\newcommand{\ups}{\upsilon}
\newcommand{\Ups}{\Upsilon}
\newcommand{\ph}{\phi}
\newcommand{\Ph}{\Phi}
\newcommand{\vph}{\varphi}
\newcommand{\vpi}{\varpi}
\newcommand{\ch}{\chi}
\newcommand{\ps}{\psi}
\newcommand{\Ps}{\Psi}
\newcommand{\om}{\omega}
\newcommand{\Om}{\Omega}

\newcommand{\bbA}{\mathbb{A}}
\newcommand{\A}{\mathbb{A}}
\newcommand{\bbB}{\mathbb{B}}
\newcommand{\bbC}{\mathbb{C}}
\newcommand{\C}{\mathbb{C}}
\newcommand{\bbD}{\mathbb{D}}
\newcommand{\bbE}{\mathbb{E}}
\newcommand{\bbF}{\mathbb{F}}
\newcommand{\bbG}{\mathbb{G}}
\newcommand{\G}{\mathbb{G}}
\newcommand{\bbH}{\mathbb{H}}
\newcommand{\HH}{\mathbb{H}}
\newcommand{\bbI}{\mathbb{I}}
\newcommand{\I}{\mathbb{I}}
\newcommand{\bbJ}{\mathbb{J}}
\newcommand{\bbK}{\mathbb{K}}
\newcommand{\bbL}{\mathbb{L}}
\newcommand{\bbM}{\mathbb{M}}
\newcommand{\bbN}{\mathbb{N}}
\newcommand{\N}{\mathbb{N}}
\newcommand{\bbO}{\mathbb{O}}
\newcommand{\bbP}{\mathbb{P}}
\newcommand{\PP}{\mathbb{P}}
\newcommand{\bbQ}{\mathbb{Q}}
\newcommand{\Q}{\mathbb{Q}}
\newcommand{\bbR}{\mathbb{R}}
\newcommand{\R}{\mathbb{R}}
\newcommand{\bbS}{\mathbb{S}}
\newcommand{\bbT}{\mathbb{T}}
\newcommand{\bbU}{\mathbb{U}}
\newcommand{\bbV}{\mathbb{V}}
\newcommand{\bbW}{\mathbb{W}}
\newcommand{\bbX}{\mathbb{X}}
\newcommand{\bbY}{\mathbb{Y}}
\newcommand{\bbZ}{\mathbb{Z}}
\newcommand{\Z}{\mathbb{Z}}

\newcommand{\scrA}{\mathcal{A}}
\newcommand{\scrB}{\mathcal{B}}
\newcommand{\scrC}{\mathcal{C}}
\newcommand{\scrD}{\mathcal{D}}
\newcommand{\scrE}{\mathcal{E}}
\newcommand{\scrF}{\mathcal{F}}
\newcommand{\scrG}{\mathcal{G}}
\newcommand{\scrH}{\mathcal{H}}
\newcommand{\scrI}{\mathcal{I}}
\newcommand{\scrJ}{\mathcal{J}}
\newcommand{\scrK}{\mathcal{K}}
\newcommand{\scrL}{\mathcal{L}}
\newcommand{\scrM}{\mathcal{M}}
\newcommand{\scrN}{\mathcal{N}}
\newcommand{\scrO}{\mathcal{O}}
\newcommand{\scrP}{\mathcal{P}}
\newcommand{\scrQ}{\mathcal{Q}}
\newcommand{\scrR}{\mathcal{R}}
\newcommand{\scrS}{\mathcal{S}}
\newcommand{\scrT}{\mathcal{T}}
\newcommand{\scrU}{\mathcal{U}}
\newcommand{\scrV}{\mathcal{V}}
\newcommand{\scrW}{\mathcal{W}}
\newcommand{\scrX}{\mathcal{X}}
\newcommand{\scrY}{\mathcal{Y}}
\newcommand{\scrZ}{\mathcal{Z}}

\newcommand{\subgp}{\mathrel{\unlhd}}

\DeclarePairedDelimiter\ceil{\lceil}{\rceil}
\DeclarePairedDelimiter\floor{\lfloor}{\rfloor}

\newcommand{\colorcomment}[2]{\textcolor{#1}{#2}} %First of these leaves in comments. Second one kills them.
%\newcommand{\colorcomment}[2]{}


\pagestyle{fancy}
\lhead{Max Jeter}
\rhead{MA503, Assignment 5, Page \thepage}

\begin{document}

{\bf Problem 1:}

Note: Seriously, who says that $D_6 = \anbrack{a,b}$? Hungerford's the first time I've seen that, and doing that completely strips away all hope of understanding it intuitively. Here, $D_6 = \anbrack{r,s}$, with $r$ being ``rotation'' and $s$ being ``reflection''.

First, we point out that $D_6$ and $\{e\}$ are normal subgroups of $D_6$. Also, $\anbrack{r}$, $\anbrack{s,r^2}$, and $\anbrack{sr,r^2}$ are normal, as it they are subgroups of index $2$ (We know that subgroups of index $2$ are normal, by a theorem in class.) Also, $\anbrack{r^3}$ is normal; it is the center of $D_6$ (as was discussed in class), and is thus normal. In addition, $\anbrack{r^2}$ is normal:

\begin{align*}
srr^2(sr)^{-1} &= sr^2s=ssr^4=r^4 \\
sr^2r^2(sr^2)^{-1} &= sr^2s=ssr^4=r^4 \\
sr^3r^2(sr^3)^{-1} &= sr^2s=ssr^4=r^4 \\
sr^4r^2(sr^4)^{-1} &= sr^2s=ssr^4=r^4 \\
sr^5r^2(sr^5)^{-1} &= sr^2s=ssr^4=r^4 \\
sr^2(s)^{-1} &=ssr^4=r^4 \\
srr^4(sr)^{-1} &= sr^4s=ssr^2=r^2 \\
sr^2r^4(sr^2)^{-1} &= sr^4s=ssr^2=r^2 \\
sr^3r^4(sr^3)^{-1} &= sr^4s=ssr^2=r^2 \\
sr^4r^4(sr^4)^{-1} &= sr^4s=ssr^2=r^2 \\
sr^5r^4(sr^5)^{-1} &= sr^4s=ssr^2=r^2 \\
sr^4(s)^{-1} &=ssr^2=r^2 \\
\end{align*}

We can exclude conjugation by elements of the form $r^n$ in the above, because such elements commute with $r^2$ and $r^4$, and thus conjugation of either by such an element leaves $r^2$ and $r^4$ fixed.

However, this is all; let $H$ be a normal subgroup. We have shown that every subgroup containing only powers of $r$ is normal in $H$. 

If $sr^n \in H$ where $n \neq 3$, then $rsr^nr^{-1} = rsr^{n-1}=sr^5r^{n-1}=sr^{n+4}$. So $sr^{n+4} \in H$.

Because $H$ is a subgroup, this means that $sr^nsr^{n+4} = r^{-n}ssr^{n+4}=r^{-n}r^{n+4}=r^4 \in H$. By taking an inverse, we also get that $r^2 \in H$.

In other words, if $H$ is normal, then $H$ contains both $r^2$ and $sr^n$ for some $n \in \N$. If $n$ is even, we can multiply $sr^n$ on the right a number of times to get the result $s$. Else, we can do the same to get the result $sr$. So either $s$ or $sr$ is in $H$. Thus, $H$ is one of the subgroups already determined to be normal ($\anbrack{r^2,sr}$, $\anbrack{r^2,s}$, or $D_6$.)

So, we can determine that we have ``got 'em all''.

\shunt

{\bf Problem 2:}

Let $G$, $H$, $K$ be finite abelian groups, with $G \oplus H \cong G \oplus K$.

Now, because $G \oplus H$ is a finite abelian group, it is isomorphic to a direct sum of the form $\bigoplus\limits_{i=1}^m \Z/(p_i^{\al_i}\Z)$, with each $p_i$ prime. Moreover, $G \oplus K$ is isomorphic to the same direct sum, by a theorem discussed in class.

Moreover, there's an injective homomorphism from $G$ to $G \oplus H$ given by $\phi: G \to G \oplus H$ where $\phi(g) = (g,0)$. Thus, $G$ is isomorphic to a subgroup of $G \oplus H$. Thus, $G$ is isomorphic to a direct sum of the form $\bigoplus\limits_{k=1}^n \Z/(p_{i_k}^{\al_{i_k}}\Z)$ (as any subgroup of $\bigoplus\limits_{i=1}^m \Z/(p_i^{\al_i}\Z)$ is of this form...we probably discussed this some time.)

Now, by rearranging terms, we can see that $G \oplus H \cong \bigoplus\limits_{k=1}^n \Z/(p_{i_k}^{\al_{i_k}}\Z) \oplus \bigoplus\limits_{\{i: i \neq i_k \text{ for any } k \in \N\}}^{n-m} \Z/(p_i^{\al_i}\Z)$. Now, $H \cong \bigoplus\limits_{\{i: i \neq i_k \text{ for any } k \in \N\}}^{n-m} \Z/(p_i^{\al_i}\Z)$; else, $G \oplus H$ can be ``factored'' as a product of two distinct sums of the form $\bigoplus\limits_{k=1}^n \Z/(p_{i_k}^{\al_{i_k}}\Z)$. Similarly, $K \cong \bigoplus\limits_{\{i: i \neq i_k \text{ for any } k \in \N\}}^{n-m} \Z/(p_i^{\al_i}\Z)$. So $K \cong H$. 

\shunt

{\bf Problem 3:}

Let $G$ be a finite abelian group.

First, if $G$ is cyclic, then for each $n \in \N$ with $n \neq 1$, then either $n$ is a multiple of $\absval{G}$, in which case there are $\absval{G}$ elements, $a$, satisfying $na = 0$, or $n$ is not a multiple of $\absval{G}$, in which case there are none (because every element in $G$ has the same order, except for $0$). And of course, if $n = 1$, then there's only $1$ element satisfying $1a=0$, that is, $0$.

So if $G$ is cyclic, then for each $n \in \N$ there are at most $n$ elements $a \in G$ satisfying $na =0$.

Now, let $G$ not be cyclic. Then $G \cong \bigoplus\limits_{i=1}^n \Z/(m_i^{\al_i}\Z)$, with each $m_i$ dividing $m_{i+1}$, each $m_i >1$, and $n \geq 2$ (else, $G$ is obviously cyclic...). Consider $m_n$. Every element, $a$, of $G$ has the property $m_na=0$; let $a \in G$. Then $\phi(a)= (a_1,a_2 \ldots a_n) \in \bigoplus\limits_{i=1}^n \Z/(m_i^{\al_i}\Z)$, with an isomorphism, $\phi$ determined by the FTFAG. Moving on $m_n(a_1,a_2 \ldots a_n)=(m_na_1,m_na_2 \ldots m_na_n)=(0,0,\ldots 0)$ (the last of these is because each of the component groups has order dividing $m_n$). Thus, $a^{m_n} = 0$, for all $a \in G$. However, $m_n < \absval{G}$, because $\absval{G} = \prod m_i \geq m_{n-1}m_n > m_n$.

So if $G$ is not cyclic, then there is an $n \in \N$ with more than $n$ elements $a \in G$ satisfying $na=0$.

So $G$ is cyclic if and only if for each $n \in \N$ there are at most $n$ elements $a \in G$ satisfying $na =0$.

\shunt

\end{document}