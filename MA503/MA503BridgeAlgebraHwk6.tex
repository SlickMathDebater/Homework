
\documentclass[a4paper,12pt]{article}

\usepackage{fancyhdr}
\usepackage{amssymb}
%\usepackage{mathpazo}
\usepackage{mathtools}
\usepackage{amsmath}
\usepackage{slashed}
\usepackage[mathscr]{euscript}

\newcommand{\tab}{\hspace{4mm}} %Spacing aliases
\newcommand{\shunt}{\vspace{20mm}}

\newcommand{\sd}{\partial} %Squiggle d

\newcommand{\absval}[1]{\lvert #1 \rvert}
\newcommand{\anbrack}[1]{\left\langle #1 \right\rangle}
\newcommand{\norm}[1]{\|#1\|}


\newcommand{\al}{\alpha} %Steal ALL of Dr. Kable's Aliases! MWAHAHAHAHA!
\newcommand{\be}{\beta}
\newcommand{\ga}{\gamma}
\newcommand{\Ga}{\Gamma}
\newcommand{\de}{\delta}
\newcommand{\De}{\Delta}
\newcommand{\ep}{\epsilon}
\newcommand{\vep}{\varepsilon}
\newcommand{\ze}{\zeta}
\newcommand{\et}{\eta}
\newcommand{\tha}{\theta}
\newcommand{\vtha}{\vartheta}
\newcommand{\Tha}{\Theta}
\newcommand{\io}{\iota}
\newcommand{\ka}{\kappa}
\newcommand{\la}{\lambda}
\newcommand{\La}{\Lambda}
\newcommand{\rh}{\rho}
\newcommand{\si}{\sigma}
\newcommand{\Si}{\Sigma}
\newcommand{\ta}{\tau}
\newcommand{\ups}{\upsilon}
\newcommand{\Ups}{\Upsilon}
\newcommand{\ph}{\phi}
\newcommand{\Ph}{\Phi}
\newcommand{\vph}{\varphi}
\newcommand{\vpi}{\varpi}
\newcommand{\ch}{\chi}
\newcommand{\ps}{\psi}
\newcommand{\Ps}{\Psi}
\newcommand{\om}{\omega}
\newcommand{\Om}{\Omega}

\newcommand{\bbA}{\mathbb{A}}
\newcommand{\A}{\mathbb{A}}
\newcommand{\bbB}{\mathbb{B}}
\newcommand{\bbC}{\mathbb{C}}
\newcommand{\C}{\mathbb{C}}
\newcommand{\bbD}{\mathbb{D}}
\newcommand{\bbE}{\mathbb{E}}
\newcommand{\bbF}{\mathbb{F}}
\newcommand{\bbG}{\mathbb{G}}
\newcommand{\G}{\mathbb{G}}
\newcommand{\bbH}{\mathbb{H}}
\newcommand{\HH}{\mathbb{H}}
\newcommand{\bbI}{\mathbb{I}}
\newcommand{\I}{\mathbb{I}}
\newcommand{\bbJ}{\mathbb{J}}
\newcommand{\bbK}{\mathbb{K}}
\newcommand{\bbL}{\mathbb{L}}
\newcommand{\bbM}{\mathbb{M}}
\newcommand{\bbN}{\mathbb{N}}
\newcommand{\N}{\mathbb{N}}
\newcommand{\bbO}{\mathbb{O}}
\newcommand{\bbP}{\mathbb{P}}
\newcommand{\PP}{\mathbb{P}}
\newcommand{\bbQ}{\mathbb{Q}}
\newcommand{\Q}{\mathbb{Q}}
\newcommand{\bbR}{\mathbb{R}}
\newcommand{\R}{\mathbb{R}}
\newcommand{\bbS}{\mathbb{S}}
\newcommand{\bbT}{\mathbb{T}}
\newcommand{\bbU}{\mathbb{U}}
\newcommand{\bbV}{\mathbb{V}}
\newcommand{\bbW}{\mathbb{W}}
\newcommand{\bbX}{\mathbb{X}}
\newcommand{\bbY}{\mathbb{Y}}
\newcommand{\bbZ}{\mathbb{Z}}
\newcommand{\Z}{\mathbb{Z}}

\newcommand{\scrA}{\mathcal{A}}
\newcommand{\scrB}{\mathcal{B}}
\newcommand{\scrC}{\mathcal{C}}
\newcommand{\scrD}{\mathcal{D}}
\newcommand{\scrE}{\mathcal{E}}
\newcommand{\scrF}{\mathcal{F}}
\newcommand{\scrG}{\mathcal{G}}
\newcommand{\scrH}{\mathcal{H}}
\newcommand{\scrI}{\mathcal{I}}
\newcommand{\scrJ}{\mathcal{J}}
\newcommand{\scrK}{\mathcal{K}}
\newcommand{\scrL}{\mathcal{L}}
\newcommand{\scrM}{\mathcal{M}}
\newcommand{\scrN}{\mathcal{N}}
\newcommand{\scrO}{\mathcal{O}}
\newcommand{\scrP}{\mathcal{P}}
\newcommand{\scrQ}{\mathcal{Q}}
\newcommand{\scrR}{\mathcal{R}}
\newcommand{\scrS}{\mathcal{S}}
\newcommand{\scrT}{\mathcal{T}}
\newcommand{\scrU}{\mathcal{U}}
\newcommand{\scrV}{\mathcal{V}}
\newcommand{\scrW}{\mathcal{W}}
\newcommand{\scrX}{\mathcal{X}}
\newcommand{\scrY}{\mathcal{Y}}
\newcommand{\scrZ}{\mathcal{Z}}

\newcommand{\subgp}{\mathrel{\unlhd}}

\DeclarePairedDelimiter\ceil{\lceil}{\rceil}
\DeclarePairedDelimiter\floor{\lfloor}{\rfloor}

\newcommand{\colorcomment}[2]{\textcolor{#1}{#2}} %First of these leaves in comments. Second one kills them.
%\newcommand{\colorcomment}[2]{}


\pagestyle{fancy}
\lhead{Max Jeter}
\rhead{MA503, Assignment 6, Page \thepage}

\begin{document}

{\bf Problem 1:}

Let $p$ be a prime number, and $G$ be an abelian group of order $p^2$.

Then $G$ is isomorphic to a group of the form $\bigoplus\limits_{i=1}^n \Z/p_i^{\al_i}$, for some $p_i$, $\al_i$.

For that representation to make sense, $p_i=p$ or $p_i=p^2$ for all $i$, because $G$ is a group of order $p^2$; if $p_i \slashed{\mid} p$ for any $i$, then the order of $G$ would not be divisible by $p$. So $p_i \mid p$ for all $i$. Also, $p_i \leq p^2$ for all $i$, else the group has order bigger than $p^2$.

The only two ways to make that work are if $p_1 = p^2$ or if $p_1=p_2=p$, and this is clear.

So $\Z/p^2$ and $\Z/p \oplus \Z/p$ are the only two abelian groups of order $p^2$.

Note: Didn't we also have a homework problem that said that any group of order $p^2$ was abelian? You can throw out ``abelian'' in the problem and it works the same as long as you've given that problem previously, can't you?

\shunt

{\bf Problem 2:}

Note: For the sake of transparency, I am obliged to state that I found a chunk of this proof in Dummit and Foote.

Let $R$ be a finite, nontrivial ring (the one ring is not a field nor an integral domain, so we can get away with this). 

If $R$ is an integral domain, then $R$ is commutative. Also, $R$ has no zero divisors. Thus, $R \setminus \{0_R\}$ is closed under multiplication.

Before continuing, we show that for all $a,b,c \in R$, $ab=ac$ implies that $a = 0$ or $b=c$;

\tab If $ab=ac$, then $ab-ac=0$, so $a(b-c)=0$. This means that $a=0$ or $b-c=0$, so we have our result.

Now, $R \setminus \{0_R\}$ is a group with respect to multiplication:

\tab First, note that multiplication is associative.

\tab Next, note that $1_R \neq 0_R$, so $R \setminus \{0_R\}$ contains an identity element.

\tab Last, each element has an inverse: let $a \in R \setminus \{0_R\}$. The map $\phi: R \setminus \{0\} \to R \setminus \{0\}$ given by $x \mapsto ax$ is injective, by the above cancellation law. Because $R \setminus \{0\}$ is finite, this means that $\phi$ is a bijection. In particular, $\phi(x)= 1$ for some $x \in R \setminus \{0\}$. In other words, $ax=1$ for some $x \in R \setminus \{0\}$. Hence, $a$ has a multiplicative inverse.

If $R$ is a field, then $R$ is commutative. Also, $R$ is a division ring. So, $R \setminus \{0_R\} = R^*$ is a group (with the operation multiplication). That means that $R$ has no zero divisors (otherwise, $R \setminus \{0_R\}$ wouldn't be closed under multiplication). So $R$ is a commutative ring with no zero divisors, $R$ is an integral domain.

\shunt

{\bf Problem 3:} %a incomplete

Let $R$ be a ring and $S=M_n(R)$.

Part a:

Let $\phi: \scrI \to \scrJ$ be given by $I \mapsto J=\{(a_{ij}) : a_{ij} \in I\}$. Then $\phi$ is a bijection:

First, $\phi$ is well defined: if $I$ is an ideal, then $\phi(I) = \{(a_{ij}) : a_{ij} \in I\}$. Now, $\phi(I)$ is an ideal of $S$; if $M \in S$ and $N \in \phi(I)$, then each entry of $MN$ (or $NM$) is a linear combination of elements of the form $ma_{ij}$ with $m \in R$ and $a_{ij} \in I$. This means that each entry of $MN$ (or $NM$) is in $I$, so that $MN$ (and $NM$) is in $\phi(I)$. Also, if $M,N \in \phi(I)$, then each entry of $M+N$ is a sum of two elements in $I$, so that each entry of $M+N$ is an element of $I$, so that $M+N \in \phi(I)$.

Second, $\phi$ is injective: let $I_1,I_2$ be $R$-ideals, and $J=\phi(I_1)=\phi(I_2)$. For each $i \in I_1$, the matrix $\left[\begin{smallmatrix} i&0\\ 0&0 \end{smallmatrix}\right] \in J$. This implies that for each $i \in I_1$, $i \in I_2$. Similarly, for each $i \in I_2$ we have that $i \in \I_1$. So $I_1 = I_2$. 

Last, $\phi$ is surjective: let $J$ be an $S$-ideal. %This is the one I don't know how to do. The idea is that the condition for $J$ to be an S-ideal can be stripped to the index level, which gets you what you need.

\shunt

Part b:

If $R$ is a division ring then $(0)$ and $R$ are the only $R$-ideals; we discussed this in class. (Make sure we did).

So by the bijection above, there can only be two distinct $S$-ideals. We know that $(0)$ and $S$ are distinct $S$-ideals. This satisfies the problem.

\shunt

{\bf Problem 4:} %one direction incomplete

Let $R$ be a ring, and $I_1$, $I_2$, $\ldots$ $I_n$ be $R$-ideals.

Let $R=I_1 + I_2 \ldots +I_n$, with $I_j \cap \sum\limits_{i \neq j} I_i = (0)$ for all $j$.

\tab First, we know that $1 \in I_1 + I_2 \ldots +I_n$. So, there are $e_1, e_2 \ldots e_n$ such that $1 = e_1 + e_2 \ldots + e_n$. Pick any such set of $e_i$s.

\tab Next, we show that $I_i = Re_i$:

\tab \tab First, let $r \in I_i$. Then $r=r1=re_1+re_2 \ldots re_i + \ldots +re_n$. But each $re_k$ with $k \neq i$ is $0$, because each is in $I_i$ and $I_k$ (we know this because we know that $I_i \cap \sum\limits_{i \neq j} I_j = (0)$). So $r = re_i$, so $r \in Re_i$. So $I_i \subset Re_i$.

\tab \tab Next, let $r \in Re_i$. Then $r = r'e_i$ for some $r' \in R$. So $r \in I_i$. So $Re_i \subset I_i$.

\tab \tab So $Re_i = I_i$.

\tab Next, $e_ie_j = 0$ if $i \neq j$; $e_ie_j \in I_i \cap I_j$, so $e_ie_j = 0$ (we know this because we know that $I_i \cap \sum\limits_{i \neq j} I_j = (0)$).

\tab Also, $e_i^2 = e_i$ for all $i$; $e_i = e_i1= e_ie_1+e_ie_2 \ldots e_ie_i +\ldots + e_ie_n = 0+0+0\ldots+e_i^2+\ldots+0 = e_i^2$. 

\tab Last, $e_i \in Z(R)$ for all $i$; let $r \in R$. Then:

\begin{align*}
r1&=1r\\
re_1+re_2+\ldots re_n &=e_1r+e_2r+\ldots e_nr
\end{align*}

\tab This means that $re_i=e_ir$ for all $i$: if not, then there is an $i$ such that there is a nonzero $r'$ such that $re_i = e_ir + r'$. Moreover, because $Re_i=I_i$, this means that $r' \in I_i$. So, we have that

\begin{align*}
re_1+re_2+\ldots re_n -re_i &=e_1r+e_2r+\ldots e_nr - re_i\\
re_1+re_2+\ldots re_{i-1} + re_{i+1} \ldots +re_n &= -r' + e_1r+e_2r+\ldots e_{i-1}r + e_{i+1}r \ldots +e_nr\\
r' = e_1r + re_1 + e_2r + re_2 \ldots + e_{i-1}r + re_{i-1} + &e_{i+1}r + re_{i+1} + \ldots e_nr + re_n\\
r' \in I_i \cap \sum\limits_{i \neq j} I_j &\text{ If I hadn't formatted it like that, the text would run off the page}
\end{align*}

\tab But this means that $r' =0$. This is a contradiction. 

Now, let there be $e_1, e_2 \ldots e_n$ such that $1 = e_1 + e_2 \ldots +e_n$ with $I_i = Re_i$, $e_i \in Z(R)$, $e_i^2 = e_i$, and $e_ie_j = 0$ for every $i \neq j$.

\tab First, note that $Re_i = I_i$ for each $i$. Let $r \in R$. Because $1 = e_1 + e_2 \ldots +e_n$, we can take $r =re_1 + re_2 \ldots + re_n$ by multiplying on the left by $r$. But because $re_i \in I_i$ for each $i$, this means that $r \in I_1 + I_2 \ldots I_n$. Thus, we have $r \in I_1 + I_2 \ldots I_n$ for each $r \in R$: we have that $R = I_1 + I_2 \ldots +I_n$.

\tab Next, let $r \in I_i \cup I_j$ for any $i \neq j$. Then $r = r'e_i=r''e_j$ for some $r',r'' \in R$. Also, $r'e_ie_i=r''e_je_i$, so $r=r'e_i = r''0=0$. That is, $r=0$ for all $R \in I_i \cup I_j$ if $i \neq j$. So, $I_i \cup I_j = (0)$ for all $i \neq j$, so we have that $I_j \cup \sum\limits_{i \neq j} I_i = (0)$ as well.

Thus, we have that $R = I_1 +I_2 \ldots + I_n$ with $I_j \cup \sum\limits_{i \neq j} I_i = (0)$ if and only if there are $e_1, e_2 \ldots e_n$ such that $1 = e_1 + e_2 \ldots +e_n$ with $I_i = Re_i$, $e_i \in Z(R)$, $e_i^2 = e_i$, and $e_ie_j = 0$ for every $i \neq j$.

\shunt

\end{document}