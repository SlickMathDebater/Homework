
\documentclass[a4paper,12pt]{article}

\usepackage{fancyhdr}
\usepackage{amssymb}
%\usepackage{mathpazo}
\usepackage{mathtools}
\usepackage{amsmath}
\usepackage{slashed}
\usepackage[mathscr]{euscript}

\newcommand{\tab}{\hspace{4mm}} %Spacing aliases
\newcommand{\shunt}{\vspace{20mm}}

\newcommand{\sd}{\partial} %Squiggle d

\newcommand{\absval}[1]{\lvert #1 \rvert}
\newcommand{\anbrack}[1]{\left\langle #1 \right\rangle}
\newcommand{\norm}[1]{\|#1\|}


\newcommand{\al}{\alpha} %Steal ALL of Dr. Kable's Aliases! MWAHAHAHAHA!
\newcommand{\be}{\beta}
\newcommand{\ga}{\gamma}
\newcommand{\Ga}{\Gamma}
\newcommand{\de}{\delta}
\newcommand{\De}{\Delta}
\newcommand{\ep}{\epsilon}
\newcommand{\vep}{\varepsilon}
\newcommand{\ze}{\zeta}
\newcommand{\et}{\eta}
\newcommand{\tha}{\theta}
\newcommand{\vtha}{\vartheta}
\newcommand{\Tha}{\Theta}
\newcommand{\io}{\iota}
\newcommand{\ka}{\kappa}
\newcommand{\la}{\lambda}
\newcommand{\La}{\Lambda}
\newcommand{\rh}{\rho}
\newcommand{\si}{\sigma}
\newcommand{\Si}{\Sigma}
\newcommand{\ta}{\tau}
\newcommand{\ups}{\upsilon}
\newcommand{\Ups}{\Upsilon}
\newcommand{\ph}{\phi}
\newcommand{\Ph}{\Phi}
\newcommand{\vph}{\varphi}
\newcommand{\vpi}{\varpi}
\newcommand{\ch}{\chi}
\newcommand{\ps}{\psi}
\newcommand{\Ps}{\Psi}
\newcommand{\om}{\omega}
\newcommand{\Om}{\Omega}

\newcommand{\bbA}{\mathbb{A}}
\newcommand{\A}{\mathbb{A}}
\newcommand{\bbB}{\mathbb{B}}
\newcommand{\bbC}{\mathbb{C}}
\newcommand{\C}{\mathbb{C}}
\newcommand{\bbD}{\mathbb{D}}
\newcommand{\bbE}{\mathbb{E}}
\newcommand{\bbF}{\mathbb{F}}
\newcommand{\bbG}{\mathbb{G}}
\newcommand{\G}{\mathbb{G}}
\newcommand{\bbH}{\mathbb{H}}
\newcommand{\HH}{\mathbb{H}}
\newcommand{\bbI}{\mathbb{I}}
\newcommand{\I}{\mathbb{I}}
\newcommand{\bbJ}{\mathbb{J}}
\newcommand{\bbK}{\mathbb{K}}
\newcommand{\bbL}{\mathbb{L}}
\newcommand{\bbM}{\mathbb{M}}
\newcommand{\bbN}{\mathbb{N}}
\newcommand{\N}{\mathbb{N}}
\newcommand{\bbO}{\mathbb{O}}
\newcommand{\bbP}{\mathbb{P}}
\newcommand{\PP}{\mathbb{P}}
\newcommand{\bbQ}{\mathbb{Q}}
\newcommand{\Q}{\mathbb{Q}}
\newcommand{\bbR}{\mathbb{R}}
\newcommand{\R}{\mathbb{R}}
\newcommand{\bbS}{\mathbb{S}}
\newcommand{\bbT}{\mathbb{T}}
\newcommand{\bbU}{\mathbb{U}}
\newcommand{\bbV}{\mathbb{V}}
\newcommand{\bbW}{\mathbb{W}}
\newcommand{\bbX}{\mathbb{X}}
\newcommand{\bbY}{\mathbb{Y}}
\newcommand{\bbZ}{\mathbb{Z}}
\newcommand{\Z}{\mathbb{Z}}

\newcommand{\scrA}{\mathcal{A}}
\newcommand{\scrB}{\mathcal{B}}
\newcommand{\scrC}{\mathcal{C}}
\newcommand{\scrD}{\mathcal{D}}
\newcommand{\scrE}{\mathcal{E}}
\newcommand{\scrF}{\mathcal{F}}
\newcommand{\scrG}{\mathcal{G}}
\newcommand{\scrH}{\mathcal{H}}
\newcommand{\scrI}{\mathcal{I}}
\newcommand{\scrJ}{\mathcal{J}}
\newcommand{\scrK}{\mathcal{K}}
\newcommand{\scrL}{\mathcal{L}}
\newcommand{\scrM}{\mathcal{M}}
\newcommand{\scrN}{\mathcal{N}}
\newcommand{\scrO}{\mathcal{O}}
\newcommand{\scrP}{\mathcal{P}}
\newcommand{\scrQ}{\mathcal{Q}}
\newcommand{\scrR}{\mathcal{R}}
\newcommand{\scrS}{\mathcal{S}}
\newcommand{\scrT}{\mathcal{T}}
\newcommand{\scrU}{\mathcal{U}}
\newcommand{\scrV}{\mathcal{V}}
\newcommand{\scrW}{\mathcal{W}}
\newcommand{\scrX}{\mathcal{X}}
\newcommand{\scrY}{\mathcal{Y}}
\newcommand{\scrZ}{\mathcal{Z}}

\newcommand{\subgp}{\mathrel{\unlhd}}

\DeclarePairedDelimiter\ceil{\lceil}{\rceil}
\DeclarePairedDelimiter\floor{\lfloor}{\rfloor}

\newcommand{\colorcomment}[2]{\textcolor{#1}{#2}} %First of these leaves in comments. Second one kills them.
%\newcommand{\colorcomment}[2]{}


\pagestyle{fancy}
\lhead{Max Jeter}
\rhead{MA503, Assignment 7, Page \thepage}

\begin{document}

Note: I've used questionable typesetting here, but I finished the assignment too late to fix it.

{\bf Problem 1:}

Part a:

Let $r \in R^*$.

\tab Then there is an $r^{-1} \in R$ such that $r^{-1}r = 1$. If $r \in M$ for any maximal ideal, $M$, then $r^{-1}r \in M$, so $1 \in M$, so $M = R$. This means that $M$ is not a maximal ideal. That is,$r \notin \bigcup\limits_{M \in m-Spec(R)} M$, so $r \in R \setminus \bigcup\limits_{M \in m-Spec(R)} M$.

Now, let $r \notin \bigcup\limits_{M \in m-Spec(R)} M$. That is, $r \in R \setminus \bigcup\limits_{M \in m-Spec(R)} M$ isn't in any maximal ideal.

\tab Then $r$ is not in any ideal other than $R$; all ideals other than $R$ are contained in some maximal ideal.

\tab So $(r) = R$. This means that $1 \in (r)$. So there's an element, $r^{-1} \in R$, such that $r^{-1}r = 1$. So $r \in R^*$.

So $R^* \subset R \setminus \bigcup\limits_{M \in m-Spec(R)} M$ and $R^* \supset R \setminus \bigcup\limits_{M \in m-Spec(R)} M$.

So $R^* = R \setminus \bigcup\limits_{M \in m-Spec(R)} M$.

\shunt

Part b:

We freely use the fact that $R \setminus R^* = \bigcup\limits_{M \in m-Spec(R)} M$. This follows from the above, and is clear with proper notation:

\begin{align*}
R^* &= \left(\bigcup\limits_{M \in m-Spec(R)} M\right)^c \\
(R^*)^c &= \bigcup\limits_{M \in m-Spec(R)} M
\end{align*}

If $R \setminus R^* = \bigcup\limits_{M \in m-Spec(R)} M$ is an ideal, then $\bigcup\limits_{M \in m-Spec(R)} M$ is a maximal ideal or $R$; it contains every maximal ideal, so it contains every ideal other than $R$. But this ideal is not $R$, otherwise $R^*$ is empty (and we know that $R^*$ contains $1$.) So $\bigcup\limits_{M \in m-Spec(R)} M$ is a maximal ideal that contains every maximal ideal. That is, it is the unique maximal ideal. So $R$ is local.

If $R$ is local, then say that $M'$ is $R$'s unique maximal ideal. Then we have that $R \setminus R^* = \bigcup\limits_{M \in m-Spec(R)} M = M'$ is an ideal. 

\shunt

{\bf Problem 2:}

Note: This is more or less taken from Hungerford.

Let $P \in Spec(R)$.

Let $M$ be a maximal ideal of $R_P$. Then $M$ is prime. So $M = QR_P$ for some prime ideal $Q$ of $R$, with $Q \subset P$. But $Q \subset P$ implies that $QR_P \subset PR_P$. So $QR_P = PR_P$ So $PR_P$ is the unique maximal ideal in $R_P$. In particular, $R_P$ is local.

\shunt

{\bf Problem 3:}

Define $rad(R) = \sqrt{0} = \{a \in R: a^n = 0 \text{ for some } n \geq 0\}$.

Then if $r \in rad(R)$, then $r \in \bigcap\limits_{P \in Spec(R)} P$:

\tab We know that $r^n=0$ for some $n \in \N$. For any prime ideal, $P$, $0 \in P$. So this means that either $r$ or $r^{n-1}$ is in $P$. If $r \in P$, we're done.

\tab If $r^{n-1} \in P$, this means that $r$ or $r^{n-2}$ is in $P$. If $r \in P$, we're done.

\tab We can iterate down to $r$; the process above ends, so we can get that $r \in P$ for all prime ideals, $P$. That is, $r \in \bigcap\limits_{P \in Spec(R)} P$.

Next, if $r \in \bigcap\limits_{P \in Spec(R)} P$, then $r \in rad(R)$:

\tab Consider $R_r = S^{-1}R$, where $S = \{r^n: n \geq 0\}$. If $S^{-1}R = R_x \neq (0)$, then there's a nonzero $y \in R_x$. So $R_x$ contains a maximal ideal, $MR_x$. We know that $MR_x$ is prime. Thus, $MR_x$ corresponds to a prime ideal, $M$, in $R$. We also know that $x \in M$. But this is bad, because this would imply that $x/x \in MR_x$, so $1 \in MR_x$, so $MR_x$ cannot be maximal.

\tab That is, $R_x$ can't contain maximal ideals. So $R_x = (0)$. So $0 \in S$. So $x$ is nilpotent.

So $rad(R) = \bigcap\limits_{P \in Spec(R)} P$.

\shunt

{\bf Problem 4:}

Let $u \in R^*$ and $a \in rad(0)$.

Then $a^{2^n} = 0$ for some $n \in \N$: since $a \in rad(0)$, $a^k = 0$ for some $k \in \N$. So for all $j \geq k$, $a^j = 0$. There's an $n \in \N$ such that $2^n \geq k$, so we have what we want.

Now, consider the product

\begin{align*}
(u+a)(u-a)(u^2+a^2)\ldots(u^{2^{n-1}}+a^{2^{n-1}}) &= u^{2^n} - a^{2^n}\\
&=u^{2^n}
\end{align*}

Now, there is a $u^{-1} \in R$ such that $u^{-1}u=1$. It is clear also that $(u^{-1})^{2^n}u^{2^n} = 1$. So we have

\begin{align*}
u^{-2^n}(u+a)(u-a)(u^2+a^2)\ldots(u^{2^{n-1}}+a^{2^{n-1}}) &= u^{-2^n}(u^{2^n} - a^{2^n})\\
&=u^{-2^n}u^{2^n}\\
&=1
\end{align*}

So $(u+a)u^{-2^n}(u-a)(u^2+a^2)\ldots(u^{2^{n-1}}+a^{2^{n-1}})=1$. So $u+a \in R^*$.

\shunt

{\bf Problem 5:}

Let $R$ be a PID, and $P$ be a nonzero prime ideal of $R$.

Then $P = (p)$ for some nonzero $p \in P$. We know that prime elements are irreducible. Ideals generated by a single irreducible element are maximal among proper principal ideals. All ideals are pricipal. So $P$ is a maximal ideal.

\shunt

{\bf Problem 6:}

Let $R$ be a domain, and $a, b \in R$.

Let $(a) \cap (b)$ be a principal ideal. Then $(a) \cap (b) = (c)$ for some $c \in R$. It is clear that $c$ is a multiple of both $a$ and $b$. Further, $(c)$ is the set of all multiples of $c$ (this is clear from the definitions), so this means that every element of $(a) \cap (b)$ is a multiple of $c$. So $a \mid x$ and $b \mid x$ implies that $c \mid x$. That is, $c$ is the $lcm$ of $a$ and $b$; $lcm(a,b)$ exists.

Let $lcm(a,b)$ exist. Then $(lcm(a,b)) = (a) \cap (b)$: first, as $lcm(a,b)$ is a multiple of $a$, it is in $(a)$, so $lcm(a,b) \subset (a)$. Similarly, $lcm(a,b) \subset (b)$. So $(lcm(a,b)) \subset (a) \cap (b)$. Next, if $x \in (a) \cap (b)$, then $x$ is a multiple of both $a$ and $b$. So $x$ is a multiple of $lcm(a,b)$. This means that $x \in (lcm(a,b))$. So $(lcm(a,b)) \supset (a) \cap (b)$, so $(lcm(a,b)) = (a) \cap (b)$. In particular, this means that $(a) \cap (b)$ is a principal ideal.

So $lcm(a,b)$ exists if and only if $(a) \cap (b)$ is a principal ideal, and in this case $(a) \cap (b) = (lcm(a,b))$.

\shunt

\end{document}