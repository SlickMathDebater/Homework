
\documentclass[a4paper,12pt]{article}

\usepackage{fancyhdr}
\usepackage{amssymb}
%\usepackage{mathpazo}
\usepackage{mathtools}
\usepackage{amsmath}
\usepackage{slashed}
\usepackage[mathscr]{euscript}

\newcommand{\tab}{\hspace{4mm}} %Spacing aliases
\newcommand{\shunt}{\vspace{20mm}}

\newcommand{\sd}{\partial} %Squiggle d

\newcommand{\absval}[1]{\lvert #1 \rvert}
\newcommand{\anbrack}[1]{\left\langle #1 \right\rangle}
\newcommand{\norm}[1]{\|#1\|}


\newcommand{\al}{\alpha} %Steal ALL of Dr. Kable's Aliases! MWAHAHAHAHA!
\newcommand{\be}{\beta}
\newcommand{\ga}{\gamma}
\newcommand{\Ga}{\Gamma}
\newcommand{\de}{\delta}
\newcommand{\De}{\Delta}
\newcommand{\ep}{\epsilon}
\newcommand{\vep}{\varepsilon}
\newcommand{\ze}{\zeta}
\newcommand{\et}{\eta}
\newcommand{\tha}{\theta}
\newcommand{\vtha}{\vartheta}
\newcommand{\Tha}{\Theta}
\newcommand{\io}{\iota}
\newcommand{\ka}{\kappa}
\newcommand{\la}{\lambda}
\newcommand{\La}{\Lambda}
\newcommand{\rh}{\rho}
\newcommand{\si}{\sigma}
\newcommand{\Si}{\Sigma}
\newcommand{\ta}{\tau}
\newcommand{\ups}{\upsilon}
\newcommand{\Ups}{\Upsilon}
\newcommand{\ph}{\phi}
\newcommand{\Ph}{\Phi}
\newcommand{\vph}{\varphi}
\newcommand{\vpi}{\varpi}
\newcommand{\ch}{\chi}
\newcommand{\ps}{\psi}
\newcommand{\Ps}{\Psi}
\newcommand{\om}{\omega}
\newcommand{\Om}{\Omega}

\newcommand{\bbA}{\mathbb{A}}
\newcommand{\A}{\mathbb{A}}
\newcommand{\bbB}{\mathbb{B}}
\newcommand{\bbC}{\mathbb{C}}
\newcommand{\C}{\mathbb{C}}
\newcommand{\bbD}{\mathbb{D}}
\newcommand{\bbE}{\mathbb{E}}
\newcommand{\bbF}{\mathbb{F}}
\newcommand{\bbG}{\mathbb{G}}
\newcommand{\G}{\mathbb{G}}
\newcommand{\bbH}{\mathbb{H}}
\newcommand{\HH}{\mathbb{H}}
\newcommand{\bbI}{\mathbb{I}}
\newcommand{\I}{\mathbb{I}}
\newcommand{\bbJ}{\mathbb{J}}
\newcommand{\bbK}{\mathbb{K}}
\newcommand{\bbL}{\mathbb{L}}
\newcommand{\bbM}{\mathbb{M}}
\newcommand{\bbN}{\mathbb{N}}
\newcommand{\N}{\mathbb{N}}
\newcommand{\bbO}{\mathbb{O}}
\newcommand{\bbP}{\mathbb{P}}
\newcommand{\PP}{\mathbb{P}}
\newcommand{\bbQ}{\mathbb{Q}}
\newcommand{\Q}{\mathbb{Q}}
\newcommand{\bbR}{\mathbb{R}}
\newcommand{\R}{\mathbb{R}}
\newcommand{\bbS}{\mathbb{S}}
\newcommand{\bbT}{\mathbb{T}}
\newcommand{\bbU}{\mathbb{U}}
\newcommand{\bbV}{\mathbb{V}}
\newcommand{\bbW}{\mathbb{W}}
\newcommand{\bbX}{\mathbb{X}}
\newcommand{\bbY}{\mathbb{Y}}
\newcommand{\bbZ}{\mathbb{Z}}
\newcommand{\Z}{\mathbb{Z}}

\newcommand{\scrA}{\mathcal{A}}
\newcommand{\scrB}{\mathcal{B}}
\newcommand{\scrC}{\mathcal{C}}
\newcommand{\scrD}{\mathcal{D}}
\newcommand{\scrE}{\mathcal{E}}
\newcommand{\scrF}{\mathcal{F}}
\newcommand{\scrG}{\mathcal{G}}
\newcommand{\scrH}{\mathcal{H}}
\newcommand{\scrI}{\mathcal{I}}
\newcommand{\scrJ}{\mathcal{J}}
\newcommand{\scrK}{\mathcal{K}}
\newcommand{\scrL}{\mathcal{L}}
\newcommand{\scrM}{\mathcal{M}}
\newcommand{\scrN}{\mathcal{N}}
\newcommand{\scrO}{\mathcal{O}}
\newcommand{\scrP}{\mathcal{P}}
\newcommand{\scrQ}{\mathcal{Q}}
\newcommand{\scrR}{\mathcal{R}}
\newcommand{\scrS}{\mathcal{S}}
\newcommand{\scrT}{\mathcal{T}}
\newcommand{\scrU}{\mathcal{U}}
\newcommand{\scrV}{\mathcal{V}}
\newcommand{\scrW}{\mathcal{W}}
\newcommand{\scrX}{\mathcal{X}}
\newcommand{\scrY}{\mathcal{Y}}
\newcommand{\scrZ}{\mathcal{Z}}

\newcommand{\subgp}{\mathrel{\unlhd}}

\DeclarePairedDelimiter\ceil{\lceil}{\rceil}
\DeclarePairedDelimiter\floor{\lfloor}{\rfloor}

\newcommand{\colorcomment}[2]{\textcolor{#1}{#2}} %First of these leaves in comments. Second one kills them.
%\newcommand{\colorcomment}[2]{}


\pagestyle{fancy}
\lhead{Max Jeter}
\rhead{MA503, Assignment 8, Page \thepage}

\begin{document}

{\bf Problem 1:} 

Let $R$ be a UFD and $P$ be a prime ideal.

Let $P$ fail to be principal. Let $a \in P$.

Now, $a$ has a prime factorization, $p_1^{\al_1}\ldots p_n^{\al_n}$.

Then one of the $p_i^{\al_i}$ is in $P$; $a \in P$, so $p_1^{\al_1} \in P$ or $p_2^{\al_2}\ldots p_n^{\al_n} \in P$. If $p_2^{\al_2}\ldots p_n^{\al_n} \in P$, then $p_2^{\al_2} \in P$ or $p_3^{\al_3}\ldots p_n^{\al_n} \in P$. We can iterate this process, so one of the $p_i^{\al_i}$ is in $P$.

So $p_i \in P$, by applying the same method.

So $(p_i) \subset P$. Because $p_i$ is prime, $(p_i)$ is prime (and nonzero). But it's not $P$, as $P$ is not principal.

So $P$ has a proper, nonzero prime ideal. 

\shunt

{\bf Problem 2:} 

Let $k$ be a field and $n \geq 2$.

If $\text{char}(k) = 2$, $x_1^2 + x_2^2 \ldots x_n^2 -1$ is equal to $(x_1+x_2+\ldots+x_n-1)^2$ (when you multiply it out,every term has a factor of $2$ except the $x_i^2$ and $-1$ terms) and so $x_1^2 + x_2^2 \ldots x_n^2 -1$ is reducible.

Now, if $\text{char}(k) \neq 2$, then $x_1^2 +x_2^2 -1$ is irreducible in $k[x_1,x_2]$; $x_1^2+x_2^2 -1$.

We proceed by induction:

\tab Let $x_1^2 + x_2^2 \ldots x_{n-1}^2 -1$ is irreducible in $k[x_1, x_2 \ldots x_{n-1}]$, and set this equal to $p$. It is clear that $x_n^2+p$ is a monic polynomial of degree $2$ in $k[x_1, x_2 \ldots x_{n-1}][x_n] = k[x_1, x_2 \ldots x_{n}]$. So if it factors, it factors into a product of degree $1$ polynomials; so it factors into something of the form $(x_n+s)(x_n+r)$, with $r$ and $s$ both in $k[x_1, x_2 \ldots x_{n-1}]$. But this would mean that $p=rs$ for some $r,s \in k[x_1, x_2 \ldots x_{n-1}]$, so $p$ would be reducible.

\tab So if $x_1^2 + x_2^2 \ldots x_{n-1}^2 -1$ is irreducible in $k[x_1, x_2 \ldots x_{n-1}]$, then $x_1^2 + x_2^2 \ldots x_{n}^2 -1$ is irreducible in $k[x_1, x_2 \ldots x_{n}]$.

So we have our result. 


\shunt

{\bf Problem 3:} %Wrong, fix it.

By the reduction criterion, $x^4 + 3x^3 + 3x^2 -5$ is irreducible in $\Z[x]$ if it is irreducible in $\Z/(11)[x]$.

By Eisenstein's criterion, $x^4 + 3x^3 + 3x^2 -5 = x^4 + 3x^3 + 3x^2 +6$ is irreducible, using the prime $3$.

So $x^4 + 3x^3 + 3x^2 -5$ is irreducible in $\Z[x]$. So $x^4 + 3x^3 + 3x^2 -5$ is irreducible in $\Q[x]$.

\shunt

{\bf Problem 4:} 

Let $R = \Z[\sqrt{-5}]$, and $K = \text{Quot}(R)$. 

Consider $3x^2 + 4x + 3$. By the quadratic formula, if this polynomial has roots, they are $\frac{-2}{3} \pm \frac{\sqrt{-5}}{3}$. A factorization of $3x^2 + 4x + 3$ is given by $3(x+\frac{2}{3} + \frac{\sqrt{-5}}{3})(x+\frac{2}{3} - \frac{\sqrt{-5}}{3})$. So the polynomial is reducible in $K[x]$.

Now, in $R[x]$, $3x^2 + 4x + 3$ cannot have a constant factored out of it. As it is a degree $2$ polynomial, this means that it factors only as a product of two degree $1$ polynomials. So any factorization of that polynomial must be of the form $(rx+r'(2 +\sqrt{-5}))(sx+s'(2 -\sqrt{-5}))$, with $r',s' \in \Z[\sqrt{-5}]$ and $r=3r'$, $s=3s'$. Yet, this means that the leading coefficient of the polynomial is a multiple of $9$, which $3$ isn't. So the polynomial is irreducible in $R[x]$.

\shunt

{\bf Problem 5:} 

Let $R$ be a UFD and $P$ be a prime ideal of $R[x]$ with $P \cap R = 0$.

Let $P$ fail to be principal. Then there are $p,q \in P$ with $p \slashed{\mid} q$ and $q \slashed{\mid} p$. We can pick $p$ to be of minimal degree, and among the $q$ that satisfy these conditions, we can also choose $q$ minimal (note that the degree of $p$ will be less than or equal to the degree of $q$). Moreover, we can choose $p,q$ to have the leading coefficient of $q$ be a multiple of the leading coefficient of $p$. 

Define $r = \gcd(p,q)$. 

Now, $r$ must have a lower degree than $p$. If not, then $r$ must have the same degree as $p$ (it clearly canot have higher degree). So $rs=p$ for some $r \in R$. Also, $r \mid q$ and $p \slashed{\mid} q$. %Prove it.



\shunt

\end{document}