
\documentclass[a4paper,12pt]{article}

\usepackage{fancyhdr}
\usepackage{amssymb}
%\usepackage{mathpazo}
\usepackage{mathtools}
\usepackage{amsmath}
\usepackage{slashed}
\usepackage{cancel}
\usepackage[mathscr]{euscript}
\usepackage{MaxPackage} %Note: You need MaxPackage installed or in the same folder as your .tex file or something.
\usepackage{esint}

\newcommand{\colorcomment}[2]{\textcolor{#1}{#2}} %First of these leaves in comments. Second one kills them.
%\newcommand{\colorcomment}[2]{}


\pagestyle{fancy}
\lhead{Max Jeter}
\chead{MA523}
\rhead{Final, Page \thepage}

%Number of Problems		:7
%Clear					:1,3i,3ii,4a,4b,4c
%Begun					:
%Not started			:
%Can complete via book	:
%Needs Polish			:2

%Pomodoros logged		:13.25

\begin{document}

{\bf Problem 1:}

Consider the Cauchy problem

\begin{align*}
xu_y-yu_x = xyu\\
\Ga = \{(x,y): x=y\}\\
u(x,y) = xy \text{ for } (x,y) \in \Ga
\end{align*}

We apply the method outlined in Zachmanoglou;

Consider the system of equations $\frac{dx}{-y}=\frac{dy}{x} = \frac{dz}{xyz}$. This is solved by $u_1=\frac{x^2}{2} + \frac{y^2}{2}$ and $u_2 = \frac{y^2}{2} - \ln(z)$. If we parameterize the curve $\Ga$ by $x=y=t$, we get that $z=t^2$ on this curve, so that $u_1 = t^2$ and $u_2 = \frac{t^2}{2} - \ln(t^2)$ on this curve. Solving for $t$, we get $u_2 + \ln(u_1) - \frac{u_1}{2} =0$. That is, $\frac{y^2}{2} - \ln(z) + \ln(\frac{x^2}{2} + \frac{y^2}{2}) - \frac{\frac{x^2}{2} + \frac{y^2}{2}}{2} = 0$.

Solving for $z$, we get $\ln(z) = \frac{y^2-x^2}{4} +  \ln(\frac{x^2}{2} + \frac{y^2}{2}) $. That is, our solution is 

\begin{displaymath}
u(x,y) = \frac{1}{2} (x^2+y^2) e^{\frac{y^2-x^2}{4}}
\end{displaymath}

as is readily checked. The solution is unique: $\Ga$ is noncharacteristic for this PDE except at $(0,0)$, so we have uniqueness/existence except at $(0,0)$ (that is, including $(1,1)$. In other words, we can solve this around any $(x,x) \in \Ga$ except perhaps $(0,0)$. However, note that our solution is valid at $(0,0)$, so we can solve this at any point on $\Ga$. 

\shunt

{\bf Problem 2:}

Let $k(x,y) = \frac{1}{2} e^{-\absval{x-y}}$. Fix $f \in C^2_c(\R)$. Set $w(x) = \frac{1}{2} \int\limits_{-\infty}^\infty e^{-\absval{x-y}} f(y) dy = \frac{1}{2} \int\limits_{-\infty}^\infty e^{-\absval{y}} f(x-y) dy$.

Then $w''(x) =  \frac{1}{2} \int\limits_{-\infty}^\infty e^{-\absval{y}} f''(x-y) dy$, because differentiation under the integral works (as $f$ has compact support). So

\begin{align*}
w(x) - w''(x) &= \frac{1}{2} \int\limits_{-\infty}^\infty e^{-\absval{y}} f(x-y) dy-\frac{1}{2} \int\limits_{-\infty}^\infty e^{-\absval{y}} f''(x-y) dy\\
&= \frac{1}{2} \int\limits_{-\infty}^\infty e^{-\absval{y}} f(x-y) dy-\frac{1}{2} \left[ e^{-\absval{y}} f'(x-y) \Bigr|_{-\infty}^\infty - \int\limits_{-\infty}^\infty (e^{-\absval{y}})' f'(x-y) dy \right]\\
&= \frac{1}{2} \int\limits_{-\infty}^\infty e^{-\absval{y}} f(x-y) dy+\frac{1}{2} \left[\int\limits_{-\infty}^\infty (e^{-\absval{y}})' f'(x-y) dy \right]\\
&= \frac{1}{2} \int\limits_{-\infty}^\infty e^{-\absval{y}} f(x-y) dy+\frac{1}{2} \left[\int\limits_{-\infty}^0 (e^{y})' f'(x-y) dy + \int\limits_{0}^\infty (e^{-y})' f'(x-y) dy \right]\\
&= \frac{1}{2} \int\limits_{-\infty}^\infty e^{-\absval{y}} f(x-y) dy+\frac{1}{2} \left[\int\limits_{-\infty}^0 e^{y} f'(x-y) dy - \int\limits_{0}^\infty e^{-y} f'(x-y) dy \right]\\
&= \frac{1}{2} \int\limits_{-\infty}^\infty e^{-\absval{y}} f(x-y) dy+\frac{1}{2} \Big[e^yf(x-y)\Bigr|_{-\infty}^0 - \int\limits_{-\infty}^0 e^yf(x-y) dy \\
& \tab - e^{-y}f(x-y)\Bigr|_{0}^\infty - \int\limits_{0}^\infty e^{-y}f(x-y) dy \Big]\\
&= \frac{1}{2} \int\limits_{-\infty}^\infty e^{-\absval{y}} f(x-y) dy+\frac{1}{2} \left[f(x) - \int\limits_{-\infty}^0 e^yf(x-y) dy +f(x) - \int\limits_{0}^\infty e^{-y}f(x-y) dy \right]\\
&= \frac{1}{2} \int\limits_{-\infty}^\infty e^{-\absval{y}} f(x-y) dy+\frac{1}{2} \left[2f(x) - \int\limits_{-\infty}^\infty e^{-\absval{y}}f(x-y) dy \right]\\
&= f(x)
\end{align*}

as desired. (All steps are are straightforward integration by parts or simply observing that something cancels or is zero.)

\shunt

{\bf Problem 3:}

Let $u(x,t)$ solve

\begin{align*}
u_tt - \De u = 0 \text{ for } x \in \R^3, t>0\\
u(x,0) = 0\\
u_t(x,0) = \chi_{B(0,1)}(x)\\
\end{align*}

That is, $u(x,t) = \fint_{\partial B(x,t)} t\chi_{B(0,1)}(y) dS(y) = \frac{1}{4\pi t} \int_{\partial B(x,t)} \chi_{B(0,1)}(y) dS(y)$.

Part i:

Now, define $\Om(t) = \{x: u(x,t) \neq 0\}$. Note that $u(x,t) \neq 0$ if $\partial B(x,t) \cap B(0,1)$ is nonempty (this isn't exactly ``obvious'', but if the intersection has at least one point, then it's open in the right topology...and so it contains some area, which means the integral above is nonzero).

Thus, $u(x,t)$ is nonzero when $\absval{y-x} =t$ for some $y$ with $\absval{y} < 1$. That is, $u(x,t)$ is nonzero when $\absval{x} \in (t-1,t+1)$. (This is geometrically clear.) %Double check your geometry. 

So, $\Om(t) = \{x: \absval{x} \in (t-1,t+1)\}$. 

\shunt

Part ii:

Observe that $\int_{\partial B(x,t)} \chi_{B(0,1)}(y) dS(y) \leq 4\pi$; this is because the surface area of a portion of a sphere embedded in the unit ball must have surface area no greater than the surface area of the sphere. Thus, $u(x,t) = \frac{1}{4\pi t} \int_{\partial B(x,t)} \chi_{B(0,1)}(y) dS(y) \leq \frac{1}{t} < \frac{2}{t+1}$ if $t \geq 1$.

Thus, $\sup\limits_{x \in \R^3} u(x,t) \leq  \frac{2}{t+1}$. 

Next, notice that because $t\geq 1$, we can pick $x \in \R^3$ so that $\partial B(x,t)$ contains an equator of $\partial B(0,1)$. The surface area of the segment of $\partial B(x,t)$ that intersects $B(0,1)$ is at least $\pi$, because the surface of least area containing any equator is a circle. Thus, we can choose $x$ so that $u(x,t) = \frac{1}{4\pi t} \int_{\partial B(x,t)} \chi_{B(0,1)}(y) dS(y) \geq \frac{1}{4\pi t} \pi = \frac{1}{4t} < \frac{1}{t+1}$ if $t \geq 1$.

Thus, $\sup\limits_{x \in \R^3} u(x,t) \geq \frac{1}{t+1}$. Combining this with the above,

\begin{displaymath}
\frac{1}{t+1} \leq \sup\limits_{x \in \R^3} u(x,t) \leq  \frac{2}{t+1}
\end{displaymath}

as desired. 

%The other direction's easycakes, just pick x correctly and you should win. Split it up into t small and t large, though. 

\shunt

{\bf Problem 4:}

Let $T$ be any rectangle in $\R^2$ with sides parallel to the x and y axes, as pictured in the test. Let $u \in C^2(T) \cap C(\overline{T})$ satisfy $u_{xy} = 0$.

Part a:

Then $u_x = f(x)$ for some differentiable function $f$. Thus, by the fundamental theorem of calculus, $\int\limits_{[B,A]} u_x = u(A) - u(B)$ and $\int\limits_{[D,C]} u_x = u(C) - u(D)$

However, $u_x = f(x)$ does not depend on $y$; this means that $\int\limits_{[D,C]} u_x = \int\limits_{[B,A]} u_x$. That is, $u(A) - u(B) = u(C) - u(D)$.

Rewriting that, we have $u(A) - u(B) - u(C) + u(D) = 0$, as desired.

\shunt

Part b:

Consider the Dirichlet problem,

\begin{align*}
u_{xy} = 0 \text{ in } T\\
u(x,y) = g(x,y) \text { on } \partial T
\end{align*}

Now, a solution to the Dirichlet problem must have that $u \in C^2(T) \cap C(\overline{T})$. So, $u(A) - u(B) - u(C) + u(D) = 0$. However, we can choose $g$ smooth so that this condition is necessarily violated: say, by taking $g(x,y) = xy$ and the points $A=(1,1), B=(0,1), C = (1,0), D= (0,0)$ so that $u(A) - u(B) - u(C) + u(D) = 1+0+0+0=1 \neq 0$. 

Thus, there are $g \in C(\partial T)$ so that the Dirichlet problem has no solution. 

\shunt

Part c:

Consider the Dirchlet problem for this PDE on $B(0,1)$, that is,

Consider the Dirichlet problem,

\begin{align*}
u_{xy} = 0 \text{ in } B(0,1)\\
u(x,y) = g(x,y) \text { on } \partial B(0,1)
\end{align*}

with $g(x,y) = (x+1/\sqrt{2})(y+1/\sqrt{2})$. Let $u$ solve the PDE on the domain $B(0,1)$; then $u$ also solves the PDE on the rectangle with vertices $A=(1/\sqrt{2},1/\sqrt{2}), B=(-1/\sqrt{2},1/\sqrt{2}), C=(1/\sqrt{2},-1/\sqrt{2}), D=(-1/\sqrt{2},-1/\sqrt{2})$. So $u(A) - u(B) - u(C) + u(D) = 0$, by part a. However, we can calculate using the boundary data that $u(A) - u(B) - u(C) + u(D) = (2/\sqrt{2})^2 = 2 \neq 0$, which is a contradiction: there is no solution to the problem. 

That is, the Dirchlet problem is ill-posed on this domain. 

\shunt

\end{document}