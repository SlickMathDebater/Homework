
\documentclass[a4paper,12pt]{article}

\usepackage{fancyhdr}
\usepackage{amssymb}
%\usepackage{mathpazo}
\usepackage{mathtools}
\usepackage{esint}
\usepackage{amsmath}
\usepackage{slashed}
\usepackage{cancel}
\usepackage[mathscr]{euscript}
\usepackage{MaxPackage} %Note: You need MaxPackage installed or in the same folder as your .tex file or something.

\newcommand{\colorcomment}[2]{\textcolor{#1}{#2}} %First of these leaves in comments. Second one kills them.
%\newcommand{\colorcomment}[2]{}


\pagestyle{fancy}
\lhead{Max Jeter}
\chead{MA523}
\rhead{Assignment 2, Page \thepage}

\begin{document}

{\bf Problem 1 (Problem 2 in book):}

Let $u$ be such that $\De u = 0$ and let $v$ be such that $v(x) = u(Ox)$ for some orthogonal $n\times n$  matrix $O$.

Then we have:

\begin{align*}
\De v(x) &= \De u(x)\\
&= \sum\limits_{i=1}^n u_{x_ix_i} (Ox)\\
&= \sum\limits_{i=1}^n u_{x_ix_i} (Ox_1e_1+ Ox_2e_2 \ldots + Ox_ne_n)\\
&= \sum\limits_{i=1}^n \sum\limits_{j=1}^n u_{x_ix_i} (Ox_je_j) \\
&= \sum\limits_{i=1}^n \sum\limits_{j=1}^n u_{x_ix_i} (x_je_j) \\
&= \sum\limits_{i=1}^n u_{x_ix_i} (x)\\
&= 0
\end{align*}

That is, solutions to $\De u = 0$ are rotation-invariant.

\shunt

{\bf Problem 2 (Problem 3 in book):}

Let $u$ be such that 

\begin{displaymath}
   \left\{
     \begin{array}{lr}
       -\De u = f & \text{ in } B(0,r)\\
       u=g & \text{ on }\partial B(0,r)
     \end{array}
   \right.
\end{displaymath} 

with dimension $n \geq 3$.

Then define $\phi(r) = \fint\limits_{\partial B(0,r)} u(y) dS(y) + \frac{1}{n(n-2)\al(n)} \int\limits_{B(0,r)} \left(\frac{1}{\absval{x}^{n-2}} - \frac{1}{r^{n-2}}\right) u(x) dx$.

Then 

\begin{align*}
\phi'(r) &= stuff \\
&= 0 \\ 
\end{align*}



So $\phi'$ is identically zero. So $\phi$ is constant. So $\phi(r) = \lim\limits_{t \to 0} \phi(t) = \lim\limits_{t \to 0} \fint\limits_{\partial B(0,r)} u(y) dS(y) + \frac{1}{n(n-2)\al(n)} \int\limits_{B(0,r)} \left(\frac{1}{\absval{x}^{n-2}} - \frac{1}{r^{n-2}}\right) u(x) dx = u(0)$.

Thus, by replacing $u$ with $f$ and $g$ on the interior/exterior, 

\begin{displaymath}
u(0) = \fint \limits_{\partial B(0,r)} g dS + \frac{1}{n(n-2)\al(n)} \int\limits_{B(0,r)} \left(\frac{1}{\absval{x}^{n-2}} - \frac{1}{r^{n-2}}\right) f dx
\end{displaymath}

\shunt

{\bf Problem 3 (Only homework on page):}

Let $u(x)$ be a $C^2$ solution to 

\begin{align*}
\De u(x) = \absval{x}^2 \text{ on } \R^n
\end{align*}

with $n \geq 3$.

Set $m(r) = \fint\limits_{\partial B(0,r)} u(y) dS(y)$.

Then $u$ solves 

\begin{align*}
\De(u) &= \absval{x}^2 \text{ in } B(0,r)\\
u &= r \text { on } \partial B(0,r)\\
\end{align*}

So by the above problem,

\begin{align*}
u(0) &= \fint \limits_{\partial B(0,r)} u(y) dS(y) + \frac{1}{n(n-2)\al(n)}\int\limits_{B(0,r)} \left( \frac{1}{\absval{x}^{n-2}} - \frac{1}{r^{n-2}}\right) \absval{x}^2 dx \\
&= m(r) + \frac{1}{n(n-2)\al(n)}\int\limits_{B(0,r)} \left( \frac{1}{\absval{x}^{n-2}} - \frac{1}{r^{n-2}}\right) \absval{x}^2 dx \\
&= m(r) + \frac{1}{n(n-2)\al(n)}\int\limits_{B(0,r)} \left( \frac{\absval{x}^2}{\absval{x}^{n-2}} - \frac{\absval{x}^2}{r^{n-2}}\right) dx \\
&= m(r) -\frac{r^4}{4(n+2)} \\
\end{align*}

which is the result we wanted.

If $n =1$, then $u(x) = x^4 + Cx + D$ are the only solutions to this, where $C, D \in \R$; this follows from elementary differential equations.

%explain result

If $n = 2$, then ... %I don't know how to do it...:(

\shunt

\end{document}