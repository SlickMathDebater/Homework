
\documentclass[a4paper,12pt]{article}

\usepackage{fancyhdr}
\usepackage{amssymb}
%\usepackage{mathpazo}
\usepackage{mathtools}
\usepackage{esint}
\usepackage{amsmath}
\usepackage{slashed}
\usepackage{cancel}
\usepackage[mathscr]{euscript}
\usepackage{MaxPackage} %Note: You need MaxPackage installed or in the same folder as your .tex file or something.

\newcommand{\colorcomment}[2]{\textcolor{#1}{#2}} %First of these leaves in comments. Second one kills them.
%\newcommand{\colorcomment}[2]{}


\pagestyle{fancy}
\lhead{Max Jeter}
\chead{MA523}
\rhead{Assignment 2, Page \thepage}

\begin{document}

{\bf Problem 1 (Problem 2 in book):}

Let $u$ be such that $\De u = 0$ and let $v$ be such that $v(x) = u(Ox)$ for some orthogonal $n\times n$  matrix $O$.

Then we have:

\begin{align*}
\De v(x) &= \De u(x)\\
&= \sum\limits_{i=1}^n u_{x_ix_i} (Ox)\\
&= \sum\limits_{i=1}^n u_{x_ix_i} (x) \\
&= 0
\end{align*}

%Note to self; justify step 2->3 way better.

\shunt

{\bf Problem 2 (Problem 3 in book):}

Let $u$ be such that 

\begin{displaymath}
   \left\{
     \begin{array}{lr}
       -\De u = f & \text{ in } B(0,r)\\
       u=g & \text{ on }\partial B(0,r)
     \end{array}
   \right.
\end{displaymath} 

with dimension $n \geq 3$.

Then %hack the mean value formulae proof: define $\phi(r)$ as a fnction of $r$ similar to what is desired, take a derivative, apply Integration by Parts and Green's Formulas. \phi' should be zero, so \phi should be constant. Take a limit as r goes to 0; you should win after this.

Thus, 

\begin{displaymath}
u(0) = \fint \limits_{\partial B(0,r)} g dS + \int\limits_{B(0,r)} \left( \frac{1}{\absval{x}^{n-2}} - \frac{1}{r^{n-2}}\right) f dx
\end{displaymath}

\shunt

{\bf Problem 3 (Only homework on page):}

Let $u(x)$ be a $C^2$ solution to 

\begin{align*}
\De u(x) = \absval{x}^2 \text{ on } \R^n
\end{align*}

Set $m(r) = \fint\limits_{\partial B(0,r)} u(y) dS(y)$.

%This is painfully straightforward from the above problem; u solves the system of PDEs given by \De(u) = \absval{x}^2 in the ball, and u=u on the boundary. Rearrange the result of the above, integrate, win.

\shunt

\end{document}