
\documentclass[a4paper,12pt]{article}

\usepackage{fancyhdr}
\usepackage{amssymb}
%\usepackage{mathpazo}
\usepackage{mathtools}
\usepackage{amsmath}
\usepackage{slashed}
\usepackage{cancel}
\usepackage[mathscr]{euscript}
\usepackage{MaxPackage} %Note: You need MaxPackage installed or in the same folder as your .tex file or something.

\newcommand{\colorcomment}[2]{\textcolor{#1}{#2}} %First of these leaves in comments. Second one kills them.
%\newcommand{\colorcomment}[2]{}

%Number of Problems		:4
%Clear					:
%Begun					:2,3
%Not started			:4
%Can complete via book	:
%Needs Polish			:1

%Pomodoroes logged		:4


\pagestyle{fancy}
\lhead{Max Jeter}
\chead{MA523}
\rhead{Assignment 3, Page \thepage}

\begin{document}

{\bf Problem 1: Problem 6 in textbook:} %Find out where to put that absval of u in...

Let $U$ be a bounded, open subset of $\R^n$.

We freely use the preceding problem's result that if $-\De v \leq 0$, then $\max\limits_{\overline{U}} v = \max\limits_{\partial U} v$.

Define $\la = \max\limits_{\overline{U}} \absval{f}$. Define $M = \max(1,\frac{r^2}{2n})$ where $r$ is an upper bound on the distance of a point in $U$ from $0$.

So we have:

\begin{align*}
\max\limits_{\overline{U}} u &\leq \max\limits_{\overline{U}} (u + \frac{\absval{x}^2}{2n} \la) \\
&= \max\limits_{\partial U} (u + \frac{\absval{x}^2}{2n} \la)\\
&\leq \max\limits_{\partial U} (u) +\max\limits_{\partial U} \frac{\absval{x}^2}{2n} \la)\\
&\leq \max\limits_{\partial U} (u) +M \la\\
&= \max\limits_{\partial U} (g) +M \la\\
&\leq \max\limits_{\partial U} (\absval{g}) + M \max\limits_{\overline{U}} (\absval{f})\\
&\leq M(\max\limits_{\partial U} (\absval{g}) + \max\limits_{\overline{U}} (\absval{f}))\\
\end{align*} 

\shunt

{\bf Problem 2: Problem 9 in textbook:}

Let $u$ be the solution of 

\begin{displaymath}
   \left\{
     \begin{array}{lr}
       \De u = 0 & \text{ in } \R_+^n\\
       u=g & \text{ on }\partial \R_+^n
     \end{array}
   \right.
\end{displaymath}

Assume $g$ is bounded and $g(x) = \absval{x}$ for $x \in \partial \R_+^n$ with $\absval{x}\leq 1$.

Then $u(x) = -\int\limits_{\partial U} \absval{x} (\frac{\partial G}{\partial \nu} (x,y) dS(y))$, where $G$ is the Green's function for the half-space, $G(x,y) = \Phi(y-x) - \Phi(y-\overline{x})$.

Consider $\frac{u(\la e_n) - u(0)}{\la}$ (with $\la >0$). We can see that: %Fix those xs...

\begin{align*}
\frac{u(\la e_n) - u(0)}{\la} &= \frac{\int\limits_{\partial U} \absval{\la e_n} \left[- \frac{\partial G}{\partial \nu}(\la e_n,y)\right]dS(y)}{\la}\\
&=\frac{\int\limits_{\partial U} \absval{\la e_n } \left[- \frac{\partial \Phi}{\partial \nu}(y-\la e_n) + \frac{\partial \Phi}{\partial \nu}(y-\overline{\la e_n}))\right]dS(y)}{\la}\\
&=\frac{\int\limits_{\partial U} \absval{\la e_n} \left[\frac{\partial \Phi}{\partial \nu}(y-\overline{\la e_n}) - \frac{\partial \Phi}{\partial \nu}(y-\la e_n))\right]dS(y)}{\la}\\
&=\int\limits_{\partial U} \left[\frac{\partial \Phi}{\partial \nu}(y-\overline{\la e_n}) - \frac{\partial \Phi}{\partial \nu}(y-\la e_n))\right]dS(y)\\
&=\int\limits_{\partial U} \left[\frac{\partial \Phi}{\partial \nu}(y+\la e_n) - \frac{\partial \Phi}{\partial \nu}(y-\la e_n))\right]dS(y)\\
\end{align*}

%Suggest that the limit as $\la \to 0$ of this is infinite...it asplode.

\shunt

{\bf Problem 3: Problem 10 in textbook:}

Part a:

Let $U^+$ be the open half-ball $\{ x \in \R^n : \absval{x} < 1, x_n > 0\}$. Assume that $u \in C^2(\overline{U^+})$ is harmonic in $U^+$, with $u = 0$ on $\partial U^+ \cap \{x: x_n = 0\}$. Now, set

\begin{displaymath}
v(x) =
   \left\{
     \begin{array}{lr}
       u & \text{ if } x_n \geq 0\\
       -u(x_1,x_2, \ldots x_{n-1}, -x_n) & \text{ if } x_n < 0
     \end{array}
   \right.
\end{displaymath}

for $x$ in the open unit ball, $U$.

First, $v \in C^2( U \setminus \{x: x_n = 0\})$, and this is clear.

Next, %finish up the rest of it...that v is smooth, even on the crease

So, $v \in C^2(U)$, and $\De v = 0$ except perhaps when $x_n = 0$. Thus, $\De v = 0$ even on the line, by continuity of the second partials. So $v$ is harmonic on $U$.

\shunt

Part b:

%You'll actually need to bash it over the head with Green's functions and stuff.

\shunt

{\bf Problem 4: Only problem on sheet:}

\shunt

\end{document}