
\documentclass[a4paper,12pt]{article}

\usepackage{fancyhdr}
\usepackage{amssymb}
%\usepackage{mathpazo}
\usepackage{mathtools}
\usepackage{amsmath}
\usepackage{slashed}
\usepackage{cancel}
\usepackage[mathscr]{euscript}
\usepackage{MaxPackage} %Note: You need MaxPackage installed or in the same folder as your .tex file or something.

\newcommand{\colorcomment}[2]{\textcolor{#1}{#2}} %First of these leaves in comments. Second one kills them.
%\newcommand{\colorcomment}[2]{}

%Number of Problems		:4
%Clear					:
%Begun					:3
%Not started			:
%Can complete via book	:
%Needs Polish			:1,2*,4*

%Pomodoroes logged		:7

%*: Ask someone to look it over...


\pagestyle{fancy}
\lhead{Max Jeter}
\chead{MA523}
\rhead{Assignment 3, Page \thepage}

\begin{document}

{\bf Problem 1: Problem 6 in textbook:}

Let $U$ be a bounded, open subset of $\R^n$.

We freely use the result that if $-\De v \leq 0$, then $\max\limits_{\overline{U}} v = \max\limits_{\partial U} v$, and also the hint given; $-\De(u + \frac{\absval{x}^2}{2n} \la) \leq 0$. %It may be valuable to put these in, but it doesn't seem like the best use of time right now...

Define $\la = \max\limits_{\overline{U}} \absval{f}$. Define $M = \max(1,\frac{r^2}{2n})$ where $r$ is an upper bound on the distance of a point in $U$ from $0$.

So we have:

\begin{align*}
\max\limits_{\overline{U}} u &\leq \max\limits_{\overline{U}} (u + \frac{\absval{x}^2}{2n} \la) \\
&= \max\limits_{\partial U} (u + \frac{\absval{x}^2}{2n} \la)\\
&\leq \max\limits_{\partial U} (u) +\max\limits_{\partial U} \frac{\absval{x}^2}{2n} \la\\
&\leq \max\limits_{\partial U} (u) +M \la\\
&= \max\limits_{\partial U} (g) +M \la\\
&\leq \max\limits_{\partial U} (\absval{g}) + M \max\limits_{\overline{U}} (\absval{f})\\
&\leq M(\max\limits_{\partial U} (\absval{g}) + \max\limits_{\overline{U}} (\absval{f}))\\
\end{align*} 

whenever $u$ is a smooth solution of

\begin{displaymath}
   \left\{
     \begin{array}{lr}
       -\De u = f & \text{ in } U\\
       u=g & \text{ on }\partial U
     \end{array}
   \right.
\end{displaymath}

Noting that we get the same result for $-u$, we have our result.

\shunt

{\bf Problem 2: Problem 9 in textbook:}

Let $u$ be the solution of 

\begin{displaymath}
   \left\{
     \begin{array}{lr}
       \De u = 0 & \text{ in } \R_+^n\\
       u=g & \text{ on }\partial \R_+^n
     \end{array}
   \right.
\end{displaymath}

Assume $g$ is bounded and $g(x) = \absval{x}$ for $x \in \partial \R_+^n$ with $\absval{x}\leq 1$.

Then $u(x) = \frac{2x_n}{n\al(n)} \int\limits_{\partial \R^n_+} \frac{\absval{y}}{\absval{x-y}^n} dy$, by Poisson's formula.

Consider $\frac{u(\la e_n) - u(0)}{\la}$ (with $1> \la >0$). We can see that:

\begin{align*}
\frac{u(\la e_n) - u(0)}{\la} &= \frac{2\la}{\la n\al(n)} \int\limits_{\partial \R^n_+} \frac{\absval{y}}{\absval{\la e_n-y}^n} dy\\
&= \frac{2}{n\al(n)} \int\limits_{\partial \R^n_+} \frac{\absval{y}}{\absval{\la e_n-y}^n} dy\\
&= C \int\limits_{\partial \R^n_+} \frac{\absval{y}}{\absval{\la e_n-y}^n} dy\\
\end{align*}

As $\la \to 0$, that integral approaches $\infty$ (because the integrand approaches $\frac{1}{\absval{y}^{n-1}}$, and the integral of this explodes.) So, because the derivatives are continuous, this means that $Du$ is unbounded around $0$. 

\shunt

{\bf Problem 3: Problem 10 in textbook:}

Part a:

Let $U^+$ be the open half-ball $\{ x \in \R^n : \absval{x} < 1, x_n > 0\}$. Assume that $u \in C^2(\overline{U^+})$ is harmonic in $U^+$, with $u = 0$ on $\partial U^+ \cap \{x: x_n = 0\}$. Now, set

\begin{displaymath}
v(x) =
   \left\{
     \begin{array}{lr}
       u(x) & \text{ if } x_n \geq 0\\
       -u(x_1,x_2, \ldots x_{n-1}, -x_n) & \text{ if } x_n < 0
     \end{array}
   \right.
\end{displaymath}

for $x$ in the open unit ball, $U$.

First, $v \in C^2( U \setminus \{x: x_n = 0\})$, and this is clear.

Next, $v$ is continuous on $\{x: x_n = 0\}$, and this is clear. Also, $v$'s derivatives on $\{x: x_n = 0\}$ are continuous: for the first $n-1$ partials, $-u_{x_i}(x_1,x_2, \ldots 0) = u_{x_i}(x_1,x_2, \ldots 0)$. For the $n$th partial, $blah$. %Fill it out...

Last, $v$'s second derivatives on $\{x: x_n = 0\}$ are continuous: %Fill it out...

So, $v \in C^2(U)$, and $\De v = 0$ except perhaps when $x_n = 0$. Thus, $\De v = 0$ even on the line, by continuity of the second partials. So $v$ is harmonic on $U$.

\shunt

Part b:

Let $u \in C^2(U^+) \cap C(\overline{U^+})$, and define $v$ as above.

First, $v \in C^2( U \setminus \{x: x_n = 0\})$, and this is clear.

Next, $v$ is continuous on $\{x: x_n = 0\}$, and this is clear. Also, $v$'s derivatives on $\{x: x_n = 0\}$ are continuous: %Don't know how to do this.

Last, $v$'s second derivatives on $\{x: x_n = 0\}$ are continuous, because (reasons). %OR this...

So, $v \in C^2(U)$, and $\De v = 0$ except perhaps when $x_n = 0$. Thus, $\De v = 0$ even on the line, by continuity of the second partials. So $v$ is harmonic on $U$.

\shunt

{\bf Problem 4: Only problem on sheet:}

Let $g \in C(\R^{n-1}) \cap L^\infty (\R^{n-1})$.

By appealing to either Theorem 14 or the fact that the question bashes us over the head with it, there's at least one function, $u$, with:

\begin{itemize}
\item $u \in C^\infty( \R^n_+) \cap L^\infty  (\R^n_+) \cap C(\overline{\R^n_+})$
\item $\De u = 0$ in $\R^n_+$
\item $u(x',0) = g(x')$ on $\R^{n-1}$.
\end{itemize}

Let $u$ and $v$ be such functions. Then there's a function $\tilde{u}$ and $\tilde{v}$ with

\begin{displaymath}
\tilde{u}(x) =
   \left\{
     \begin{array}{lr}
       u(x) & \text{ if } x_n \geq 0\\
       -u(x_1,x_2, \ldots x_{n-1}, -x_n) & \text{ if } x_n < 0
     \end{array}
   \right.
\end{displaymath}

and

\begin{displaymath}
\tilde{v}(x) =
   \left\{
     \begin{array}{lr}
       v(x) & \text{ if } x_n \geq 0\\
       -v(x_1,x_2, \ldots x_{n-1}, -x_n) & \text{ if } x_n < 0
     \end{array}
   \right.
\end{displaymath}

Now, consider $w = \tilde{u} - \tilde{v}$. Then $w$ is harmonic on the entire space: it's the sum of two harmonic functions, as explained in the previous problem.

Moreover, $w$ is bounded: both $u$ and $v$ are bounded, so $\tilde{u}$ and $\tilde{v}$ are bounded, so their difference is bounded.

So, by Liouville, $w$ must be constant. However, $\tilde{u}$ and $\tilde{v}$ are the same at a point ($\tilde{u}(0) = u(0) = g(0) = v(0) = \tilde{v}(0)$). So $w = 0$ at a point. So $\tilde{u} = \tilde{v}$. So $u = v$.

\shunt

\end{document}