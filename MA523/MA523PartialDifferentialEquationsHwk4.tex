
\documentclass[a4paper,12pt]{article}

\usepackage{fancyhdr}
\usepackage{amssymb}
%\usepackage{mathpazo}
\usepackage{mathtools}
\usepackage{amsmath}
\usepackage{slashed}
\usepackage{cancel}
\usepackage[mathscr]{euscript}
\usepackage{MaxPackage} %Note: You need MaxPackage installed or in the same folder as your .tex file or something.

\newcommand{\colorcomment}[2]{\textcolor{#1}{#2}} %First of these leaves in comments. Second one kills them.
%\newcommand{\colorcomment}[2]{}


\pagestyle{fancy}
\lhead{Max Jeter}
\chead{MA523}
\rhead{Assignment 4, Page \thepage}

%Number of Problems		:6
%Clear					:
%Begun					:1,2a,2b
%Not started			:3b,3c
%Can complete via book	:
%Needs Polish			:3a

%Pomodoros logged		:3

\begin{document}

{\bf Problem 1:}

Let $g: [0,\infty) \to \R$ with $g(0) = 0$, and let $u(x,t)$ solve

\begin{displaymath}
   \left\{
     \begin{array}{lr}
       u_t - u{xx} = 0 & \text{ in } \R_+ \times (0,\infty) \\
       u=0 & \text{ in } \R_+ \times \{t=0\}\\
       u=g & \text{ on } \{x=0\} \times [0,\infty)
     \end{array}
   \right.
\end{displaymath}

Let $v(x,t) = u(x,t) - g(t)$, and extend $v$ to $\{x<0\}$ by odd reflection (just call the resulting extension $v$). Then $v$ solves

\begin{displaymath}
   \left\{
     \begin{array}{lr}
       v_t - v{xx} = g'(t) & \text{ in } \R_+ \times (0,\infty) \\
       v=-g & \text{ in } \R_+ \times \{t=0\}\\
       v=0 & \text{ on } \{x=0\} \times [0,\infty)
     \end{array}
   \right.
\end{displaymath}

So $v(x,t) = \int\limits_\R -g(y)e^{\frac{-\absval{x-y}^2}{4t}} dy +  \int\limits_0^t \frac{1}{\sqrt{4 \pi (t-s)}} \int\limits_\R g'(y) e^{\frac{-\absval{x-y}^2}{4(t-s)}} dy ds$ by formula 17 in the book. 

%Applying mean value theorems may succeed in getting your desired result...

%This is the extension of the heat equation to non-zero initial values...

\shunt

{\bf Problem 2:}

Let $g \in C(\R^n)$, $g \in L^1(\R^n)$, $\absval{g} < M$ for some $M$. Let $u$ be the bounded solution to 

\begin{displaymath}
   \left\{
     \begin{array}{lr}
       \De u -u_t  = 0 & \text{ for } t>0, x \in \R^n \\
       u(x,0) = g(x) & \text{ for } x \in \R^n
     \end{array}
   \right.
\end{displaymath}

Part a:

Then $u(x,t) = \frac{1}{(4\pi t)^{n/2}}\int\limits_\R e^{\frac{-\absval{x-y}^2}{4t}} g(y) dy$. Let $\ep >0$. Choose $N$ greater than $something$. Then for all $t >N$, $result desired$.

That is, $\lim\limits_{t \to \infty} \sup\limits_{x \in \R^n} \absval{u(x,t)} = 0$, which is the desired result.

\shunt

Part b:

Consider $v(x,t) = u(x,t) - g(x)$. Then $v(x,t)$ solves

\begin{displaymath}
   \left\{
     \begin{array}{lr}
       \De v -v_t  = -\De g(x) & \text{ for } t>0, x \in \R^n \\
       v(x,0) = 0 & \text{ for } x \in \R^n
     \end{array}
   \right.
\end{displaymath}

Thus, $v(x,t) = \int\limits_0^t \int\limits_{\R^n} \Phi(x-y,t-s) \De g(y) dy ds$.

So $\int\limits_{\R^n} v(x,t) dx = \int\limits_{\R^n} \int\limits_0^t \int\limits_{\R^n} \Phi(x-y,t-s) \De g(y) dy ds dx$

Switching the order of integration, we get $ \int\limits_0^t \int\limits_{\R^n} \int\limits_{\R^n} \Phi(x-y,t-s) \De g(y) dx dy ds$.

Yet, this is $ \int\limits_0^t \int\limits_{\R^n} \De g(y) dy ds$, because $\int\limits_{\R^n} \Phi(x-y,t-s) dx = 1$. 

The integral vanishes, for reasons I haven't figured out yet. %Reasons are?

So, $\int\limits_{\R^n} v(x,t) dx = 0$; $\int\limits_{\R^n} u(x,t) - g(x) dx = 0$, which yields the desired result of $\int\limits_{\R^n} u(x,t) dx = \int\limits_{\R^n} g(x) dx$.

\shunt

{\bf Problem 3:}

Part a: %Clean it up a bit.

Fix $\al \in (0,1)$, $\be \geq 0$.

Note first that $z^\be e^{-z} = e^{\be \ln(z) - z}$. So, the desired result is

\begin{displaymath}
e^{\be \ln(z) - z} \leq M e^{-\al z}
\end{displaymath}

for some $M$, which is equivalent to

\begin{displaymath}
\be \ln(z) - z \leq \ln(M) - \al(z)
\end{displaymath}

for some $M$. Now, this is equivalent to 

\begin{displaymath}
- \ln(M) \leq (1-\al)z - \be \ln(z)
\end{displaymath}

for some $M$. By applying basic calculus, the right hand side takes a minimum at $z = \be / (1-\al)$, so taking $M = (1-\al)(\be / (1-\al)) - \be \ln(\be / (1-\al))$ suffices. 

\shunt

Part b:

\shunt

Part c:

\shunt

\end{document}