
\documentclass[a4paper,12pt]{article}

\usepackage{fancyhdr}
\usepackage{amssymb}
%\usepackage{mathpazo}
\usepackage{mathtools}
\usepackage{esint}
\usepackage{amsmath}
\usepackage{slashed}
\usepackage{cancel}
\usepackage[mathscr]{euscript}
\usepackage{MaxPackage} %Note: You need MaxPackage installed or in the same folder as your .tex file or something.

\newcommand{\colorcomment}[2]{\textcolor{#1}{#2}} %First of these leaves in comments. Second one kills them.
%\newcommand{\colorcomment}[2]{}


\pagestyle{fancy}
\lhead{Max Jeter}
\chead{MA523}
\rhead{Assignment 5, Page \thepage}

%Number of Problems		:4
%Clear					:2a,2b,3
%Begun					:
%Not started			:
%Can complete via book	:
%Needs Polish			:1

%Pomodoroes logged		:8

\begin{document}

{\bf Problem 1 (23 in book):}

Let $S$ denote the square in $\R \times (0,\infty)$ with corners $(0,1),(1,2),(0,3),(-1,2)$. Define

\begin{displaymath}
f(x,t) = 
   \left\{
     \begin{array}{lr}
       -1  & \text{ for } (x,t) \in S \cap \{t > x+2\} \\
       -1 & \text{ for } (x,t) \in S \cap \{t < x+2\} \\
       0 & \text{ else}
     \end{array}
   \right.
\end{displaymath}

Let $u$ solve

\begin{displaymath}
   \left\{
     \begin{array}{lr}
       u_{tt} - u_{xx} = f  & \text{ when } t>0 \\
       u=0, u_t = 0 & \text{ when } t=0 \\
     \end{array}
   \right.
\end{displaymath}

Consider $u$ when $t>3$. Then we have

\begin{align*}
u(x,t) &= \int\limits_0^t u(x,t;s)ds
\end{align*}

where $u(x,t;s) = \frac{1}{2} \int\limits_{x-t}^{x+t} f(y,s) dy$ (we get this by Duhamel's principle and the solution of the wave equation in one dimension). In other words,

\begin{align*}
u(x,t) &= \int\limits_0^t \frac{1}{2} \int\limits_{x-t}^{x+t} f(y,s) dyds
\end{align*}

I appear to be pressed on time; I present the below diagram without argument, and hope that the nature of the argument follows from it. I will note that each segment of the curve is a segment of a parabola, and this is somewhat clear given the geometry of the desired shape.

\shunt

\shunt

\shunt

{\bf Problem 2 (24 in book):}

Let $u$ solve the intial value problem for the wave equation in one dimension:

\begin{displaymath}
   \left\{
     \begin{array}{lr}
       u_{tt} - u_{xx} = 0  & \text{ when } t>0 \\
       u=g, u_t = h & \text{ when } t=0 \\
     \end{array}
   \right.
\end{displaymath}

Let $g,h$ have compact support. Consider $k(t) = \frac{1}{2} \int\limits_\R u_t^2(x,t) dx$ and $p(t) = \frac{1}{2} \int\limits_\R u_x^2(x,t) dx$.

Part a:

Consider $k(t) + p(t) = \frac{1}{2}\int\limits_\R u_x^2(x,t) + u_t^2(x,t) dx$.

We know that $u(x,t) = \frac{g(x+t)+g(x-t)}{2} + \frac{1}{2}\int\limits_{x-t}^{x+t} h(y)dy$

So, we have:

\begin{align*}
u_x(x,t) &= \frac{g'(x+t)+g'(x-t)}{2} + \left[\frac{1}{2}\int\limits_{x-t}^{x+t} h(y)dy\right]_x\\
&= \frac{g'(x+t)+g'(x-t)}{2} + \frac{1}{2}\left[h(x+t) - h(x-t)\right]\\
u_t(x,t) &= \frac{g'(x+t)-g'(x-t)}{2} + \left[\frac{1}{2}\int\limits_{x-t}^{x+t} h(y)dy\right]_t\\
&= \frac{g'(x+t)-g'(x-t)}{2} + \frac{1}{2}\left[h(x+t) + h(x-t)\right]\\
\end{align*}

This means that

\begin{align*}
u_x^2 + u_t^2 &= \left(\frac{g'(x+t)+g'(x-t)}{2} + \frac{1}{2}\left[h(x+t) - h(x-t)\right]\right)^2 \\
&\phantom{{}={}}+ \left(\frac{g'(x+t)-g'(x-t)}{2} + \frac{1}{2}\left[h(x+t) + h(x-t)\right]\right)^2\\
&= (\frac{g'(x+t)+g'(x-t)}{2})^2\\ &\phantom{{}={}}+ \frac{g'(x+t)+g'(x-t)}{2}\left[h(x+t) - h(x-t)\right]\\ &\phantom{{}={}}+ (\frac{1}{2}\left[h(x+t) - h(x-t)\right])^2 \\
&\phantom{{}={}}+ (\frac{g'(x+t)-g'(x-t)}{2})^2\\ &\phantom{{}={}}+\frac{g'(x+t)-g'(x-t)}{2}\left[h(x+t) + h(x-t)\right] \\&\phantom{{}={}}+ (\frac{1}{2}\left[h(x+t) + h(x-t)\right])^2\\
&= \frac{1}{4}g'(x+t)^2  + \frac{1}{4} g'(x-t)^2 + \frac{1}{2} g'(x+t)g'(x-t)\\ &\phantom{{}={}}+ \frac{1}{2}\left[h(x+t)g'(x+t) - h(x-t)g'(x+t) +h(x+t)g'(x-t) - h(x-t)g'(x-t) \right]\\ &\phantom{{}={}}+ \frac{1}{4}h(x+t)^2  + \frac{1}{4} h(x-t)^2 - \frac{1}{2} h(x+t)h(x-t)\\
&\phantom{{}={}}+ \frac{1}{4}g'(x+t)^2  + \frac{1}{4} g'(x-t)^2 - \frac{1}{2} g'(x+t)g'(x-t)\\ &\phantom{{}={}}+\frac{1}{2}\left[h(x+t)g'(x+t) + h(x-t)g'(x+t) -h(x+t)g'(x-t) - h(x-t)g'(x-t) \right] \\&\phantom{{}={}}+ \frac{1}{4}h(x+t)^2  + \frac{1}{4} h(x-t)^2 + \frac{1}{2} h(x+t)h(x-t)\\
&= \frac{1}{2}g'(x+t)^2  + \frac{1}{2} g'(x-t)^2 \\ &\phantom{{}={}}+ \left[h(x+t)g'(x+t) -h(x-t)g'(x-t)\right]\\ &\phantom{{}={}}+ \frac{1}{2}h(x+t)^2  + \frac{1}{2} h(x-t)^2\\
\end{align*}

Now, we integrate:

\begin{align*}
\int\limits_\R u_x^2 + u_t^2 dx &= \int\limits_\R \frac{1}{2}g'(x+t)^2  + \frac{1}{2} g'(x-t)^2 \\ &\phantom{{}={}}+ \left[h(x+t)g'(x+t) -h(x-t)g'(x-t)\right]\\ &\phantom{{}={}}+ \frac{1}{2}h(x+t)^2  + \frac{1}{2} h(x-t)^2 dx
\end{align*}

The above is constant (with respect to $t$), and this is clear by applying appropriate substitutions to each term.

\shunt

Part b:

Using the above, consider that 

\begin{align*}
u_x^2 - u_t^2 &= \left(\frac{g'(x+t)+g'(x-t)}{2} + \frac{1}{2}\left[h(x+t) - h(x-t)\right]\right)^2 \\
&\phantom{{}={}}- \left(\frac{g'(x+t)-g'(x-t)}{2} + \frac{1}{2}\left[h(x+t) + h(x-t)\right]\right)^2\\
&= (\frac{g'(x+t)+g'(x-t)}{2})^2\\ &\phantom{{}={}}+ \frac{g'(x+t)+g'(x-t)}{2}\left[h(x+t) - h(x-t)\right]\\ &\phantom{{}={}}+ (\frac{1}{2}\left[h(x+t) - h(x-t)\right])^2 \\
&\phantom{{}={}}- (\frac{g'(x+t)-g'(x-t)}{2})^2\\ &\phantom{{}={}}-\frac{g'(x+t)-g'(x-t)}{2}\left[h(x+t) + h(x-t)\right] \\&\phantom{{}={}}- (\frac{1}{2}\left[h(x+t) + h(x-t)\right])^2\\
&= \frac{1}{4}g'(x+t)^2  + \frac{1}{4} g'(x-t)^2 + \frac{1}{2} g'(x+t)g'(x-t)\\ &\phantom{{}={}}+ \frac{1}{2}\left[h(x+t)g'(x+t) - h(x-t)g'(x+t) +h(x+t)g'(x-t) - h(x-t)g'(x-t) \right]\\ &\phantom{{}={}}+ \frac{1}{4}h(x+t)^2  + \frac{1}{4} h(x-t)^2 - \frac{1}{2} h(x+t)h(x-t)\\
&\phantom{{}={}}- \frac{1}{4}g'(x+t)^2  - \frac{1}{4} g'(x-t)^2 + \frac{1}{2} g'(x+t)g'(x-t)\\ &\phantom{{}={}}-\frac{1}{2}\left[h(x+t)g'(x+t) + h(x-t)g'(x+t) -h(x+t)g'(x-t) - h(x-t)g'(x-t) \right] \\&\phantom{{}={}}- \frac{1}{4}h(x+t)^2  - \frac{1}{4} h(x-t)^2 - \frac{1}{2} h(x+t)h(x-t)\\
&=  g'(x+t)g'(x-t)\\ &\phantom{{}={}}+ \left[ - h(x-t)g'(x+t) +h(x+t)g'(x-t) \right]\\ &\phantom{{}={}}+ h(x+t)h(x-t)\\
\end{align*}

Integrating, we get

\begin{align*}
\int\limits_\R u_x^2 - u_t^2 dx &=  \int\limits_\R g'(x+t)g'(x-t)\\ &\phantom{{}={}}+ \left[ - h(x-t)g'(x+t) +h(x+t)g'(x-t) \right]\\ &\phantom{{}={}}+ h(x+t)h(x-t) dx\\
\end{align*}

Because $g$ and $h$ have compact support, there's a $t$ large enough that all of the above products vanish for all $x$. (Taking $t$ to be twice the diameter of the larger of the sets $g$ and $h$ have support on suffices.)

Thus, the above integral vanishes for some sufficiently large $t$, yielding our result.

\shunt

{\bf Problem 3 (on page):}

Assume $f(x,t) = 1$ if $\absval{x} \leq 1$ and $0\leq t \leq 1$, and $f(x,t) = 0$ otherwise. Let $u$ solve

\begin{displaymath}
   \left\{
     \begin{array}{lr}
       u_{tt} - \De u = f  & \text{ when } t>0 \\
       u=0, u_t = 0 & \text{ when } t=0 \\
     \end{array}
   \right.
\end{displaymath}

Consider $u(0,t)$ when $t>2$. Then $u(0,t) = \int\limits_0^t f(0,t;s) ds$ where $f(x,t;s)$ solves the IVP at time $s$.

If $n = 1$, then this means that $u(0,t) = \frac{1}{2}\int\limits_0^t \int\limits_{-t}^{t} f(y,s) dy ds = \frac{1}{2}\int\limits_0^1 \int\limits_{-1}^{1} 1 dy ds = 1$.

So if $n=1$, then $u(0,t) = 1$ when $t >2$.  %This should be a nonzero constant

If $n = 2$, then:

\begin{align*}
u(0,t) &= \frac{1}{2} \int\limits_0^t \fint\limits_{B(0,t)} \frac{t^2f(y,s)}{\sqrt{t^2-\absval{y}^2}} dy ds\\
&= \frac{t^2}{2} \int\limits_0^1 \frac{1}{\pi t^2} \int\limits_{B(0,1)} \frac{1}{\sqrt{t^2-\absval{y}^2}} dy ds\\
&= \frac{1}{2 \pi} \int\limits_0^1 \int\limits_0^{2\pi}\int\limits_0^1 \frac{r}{\sqrt{t^2-\absval{r}^2}} dr d\theta ds\\
&=\int\limits_0^1\int\limits_0^1 \frac{r}{\sqrt{t^2-\absval{r}^2}} dr ds\\
&=\int\limits_0^1 \frac{r}{\sqrt{t^2-\absval{r}^2}} dr \\
&= \sqrt{t^2} - \sqrt{t^2-1}\\
&= t - \sqrt{t^2-1}
\end{align*} %This is where you relearn how to do radial integration...

So if $n = 2$, then $u(0,t) = t - \sqrt{t^2-1}$ when $t>2$.

If $n = 3$, then this means that $u(0,t) = \int\limits_0^t\fint\limits_{\partial B(0,t)} tf(y,s) dS(y)ds = 0$ (it vanishes because $f(y,s)$ vanishes on the shells we're working on.)

So if $n=3$, then $u(0,t) = 0$ when $t>2$. 

\shunt

\end{document}