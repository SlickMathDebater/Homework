
\documentclass[a4paper,12pt]{article}

\usepackage{fancyhdr}
\usepackage{amssymb}
%\usepackage{mathpazo}
\usepackage{mathtools}
\usepackage{amsmath}
\usepackage{slashed}
\usepackage{cancel}
\usepackage[mathscr]{euscript}
\usepackage{MaxPackage} %Note: You need MaxPackage installed or in the same folder as your .tex file or something.

\newcommand{\colorcomment}[2]{\textcolor{#1}{#2}} %First of these leaves in comments. Second one kills them.
%\newcommand{\colorcomment}[2]{}


\pagestyle{fancy}
\lhead{Max Jeter}
\chead{Class}
\rhead{Assignment, Page \thepage}

%Number of Problems		: 2
%Clear					:
%Begun					: 1
%Not started			: 2
%Can complete via book	:
%Needs Polish			:

%Pomodoros logged 		: 4

\begin{document}

{\bf Problem 1:}

Let $u$ be such that $\De u = 0$ in $\R^2$, $u = 0$, $\frac{\partial u}{\partial x_2} = \frac{1}{n} \sin(nx_1)$ when $x_2=0$.

We try to find $u$ with $u(x_1,x_2) = v(x_1)w(x_2)$ for some $v,w$.

If this holds, then we get $v''(x_1)w(x_2) - v(x_1)w''(x_2) = 0$, which we can rearrange to get $\frac{v''(x_1)}{v(x_1)} = - \frac{w''(x_2)}{w(x_2)} = \mu$.

Also, we get that $v(x_1)w'(0) = \frac{1}{n} \sin(nx_1)$. Thus, we determine that either $v(x_1) = \frac{c}{n} \sin(nx_1)$, where $c = \frac{1}{w'(0)}$. (Note: $w'(0) \neq 0$, $\frac{1}{n} \sin(nx_1)$ is identically zero...)

Because of the situation, we can arbitrarily choose $w'(0) = 1$, by rescaling $v$ and $w$ appropriately. This yields $v(x_1) = \frac{1}{n} \sin(nx_1)$.

So, $v''(x_1) = -n \sin(nx_1)$, so we get $\mu = -n^2$ in the above.

So $\frac{w''(x_2)}{w(x_2)} = n^2$. Using techniques of ODE (note that it's necessary to use that $w(0)= 0$ and $w'(0) = 1$ for this...it's just a second order linear ODE with constant coefficients), we find that $w(x_2) = \frac{1}{n}\sinh(nx_2)$. Thus, 

\begin{align*}
u(x_1,x_2) &= v(x_1)w(x_2) \\
&= \frac{1}{n^2} \sin(nx_1)\sinh(nx_2)
\end{align*}

as desired.

Note that as $n \to \infty$, $u(x_1,x_2)$ vanishes along $x_1=0$ and $x_2=0$. However, in all other cases,

\begin{align*}
\lim\limits_{n \to \infty} \frac{1}{n^2} \sin(nx_1)\sinh(nx_2) &= \lim\limits_{n \to \infty} \frac{n\left[\cos(nx_1)\sinh(nx_2) + \sin(nx_1)\cosh(nx_2)\right]}{2n}\\
&= \lim\limits_{n \to \infty} \frac{\left[\cos(nx_1)\sinh(nx_2) + \sin(nx_1)\cosh(nx_2)\right]}{2}\\
\end{align*}

This limit does not exist; $\frac{\left[\cos(nx_1)\sinh(nx_2) + \sin(nx_1)\cosh(nx_2)\right]}{2}$ oscillates wildly. However, as $n \to \infty$, the Cauchy data goes to $0$. Thus, we can say that the Cauchy Problem for Laplace's Equation isn't well-posed.

\shunt

{\bf Problem 2:}

Consider the line $\{t=0\}$ and the heat equation $u_t-u_{xx} = 0$. Then for this PDE, we have that $-1\nu^{(0,2)}=0$ on the line given (because $\nu = (0,1)$). %Explain why.

So the line is everywhere noncharacteristic for the PDE: it is characteristic. 

Now, assume that there's an analytic solution, $u$, of the heat equation in $\R \times \R$ with $u = \frac{1}{1+x^2}$ on $\{t=0\}$. That is, $u = \int\limits_{\R^2} \frac{1}{y^2+1} \frac{e^{\frac{-(x-y)^2}{4t}}}{4\pi t} dy = \sum\limits{\al}^\infty a_\al (x,t)^\al$.

Consider $u(0,t) = \frac{1}{4\pi t} \int\limits_{\R^2} \frac{e^{\frac{-y^2}{4t}}}{1+y^2} dy = \frac{ce^{-1/4t}}{t}$. 

Note that $u(0,t)$'s second derivative explodes at $0$; thus, $u$ cannot have been analytic. 

\shunt

\end{document}