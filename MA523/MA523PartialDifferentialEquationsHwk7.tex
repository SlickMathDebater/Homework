
\documentclass[a4paper,12pt]{article}

\usepackage{fancyhdr}
\usepackage{amssymb}
%\usepackage{mathpazo}
\usepackage{mathtools}
\usepackage{amsmath}
\usepackage{slashed}
\usepackage{cancel}
\usepackage[mathscr]{euscript}
\usepackage{MaxPackage} %Note: You need MaxPackage installed or in the same folder as your .tex file or something.

\newcommand{\colorcomment}[2]{\textcolor{#1}{#2}} %First of these leaves in comments. Second one kills them.
%\newcommand{\colorcomment}[2]{}


\pagestyle{fancy}
\lhead{Max Jeter}
\chead{MA523}
\rhead{Assignment 7 , Page \thepage}

%Number of Problems		:3
%Clear					:1,2
%Begun					:
%Not started			:3
%Can complete via book	:
%Needs Polish			:

%Pomodoros logged		:10.5

\begin{document}

{\bf Problem 1:}

Consider $u(x,y)$ solving

\begin{align*}
u_y^2u_{xx}+uu_{xy}+u_x^2u_{yy} &= u^2+1\\
u(x,0) = \sin(x), u_y(x,0) &= \cos(x)
\end{align*}

Then the order $2$ (and lower) partials for $u$ at $(0,0)$ are:

\begin{align*}
u(0,0) &= \sin(0) = 0\\
u_x(0,0) &= \cos(0) = 1\\
u_y(0,0) &= \cos(0) = 1\\
u_{xx}(0,0) &= -\sin(0) = 0\\
u_{xy}(0,0) &= -\sin(0) = 0\\
u_{yy}(0,0) &= 1\\
\end{align*}

The first five are obtained by the initial conditions and applying partial derivatives to them, and the last is obtained by plugging this information into the PDE. Thus, the second-order Taylor Approximation of $u$ about the point $(0,0)$ is

\begin{align*}
u(x,y) &\approx x+y+\frac{y^2}{2} \\
\end{align*}

Now, some of the order $2$ (and lower) partials for $u$ at $(\pi/2, 0)$ are:

\begin{align*}
u(\pi/2,0) &= \sin(\pi/2) = 1\\
u_x(\pi/2,0) &= \cos(\pi/2) = 0\\
u_y(\pi/2,0) &= \cos(\pi/2) = 0\\
u_{xx}(\pi/2,0) &= -\sin(\pi/2) = -1\\
u_{xy}(\pi/2,0) &= -\sin(\pi/2) = -1\\
\end{align*} %Look over this once more.

Plugging this information into the PDE yields $-1=2$, which is nonsense; thus, $u$ is inconsistent at $(\pi/2,0)$.

\shunt

{\bf Problem 2:}

Let

\begin{align*}
L[u] = yu_{xx} + (x+y)u_{xy} + xu_{yy} - u_x - u_y
\end{align*}

Part a:

The equation $L$ is hyperbolic when $\De = \left(\frac{x+y}{2}\right)^2 - xy >0$. 

Rewriting this condition, we get:

\begin{align*}
\left(\frac{x+y}{2}\right)^2 - xy &>0\\
\left(\frac{x-y}{2}\right)^2&>0\\
\left(x-y\right)^2&>0\\
x &\neq y\\
\end{align*}

That is, $L$ is hyperbolic except when $x=y$. 

\shunt

Part b:

By the discussion in John, the characteristic curves of this PDE satisfy $\frac{dy}{dx} = \frac{(x+y)/2 \pm (x-y)/2}{y}$. That is, the characteristic curves satisfy either $\frac{dy}{dx} = \frac{x}{y}$ or $\frac{dy}{dx} = \frac{y}{y}$. Solving the ODEs, we get that the characteristic curves are the hyperbolas given by $y^2-x^2 = c$ for some constant $c$, and the lines $y= x + c$. %Can trim those down; I don't think the lines satisfy this. 

\shunt

Part c:

First, consider the solutions to $y\la^2 + (x+y)\la + x = 0$; they are $\la_1=-1$ and $\la_2 = -x/y$, by the quadratic formula.

We want $\xi$ and $\eta$ so that $\xi_x = -\la_2 \xi_y$ and $\eta_x = \la_1 \eta_y$. Choosing $\eta=y-x$ and $\xi=x^2-y^2$ works for this.

Now, say that $u(x,y)$ solves the PDE, and define $v(\xi,\eta) = v(x^2-y^2,y-x) = u(x,y)$

%Also, we can solve for $x$ and $y$ in terms of $\xi$ and $\eta$: $x=\frac{\eta-\xi^2}{2\xi}$ and $y= \frac{\eta-\xi^2}{2\xi}-\xi$.

Using the Chain Rule, we have:

\begin{align*}
u_x &= -2yv_\xi + v_\eta\\
u_y &= 2xv_\xi -v_\eta\\
u_{xx} &= 2v_\xi + 4x^2v_{\xi\xi} -4x v_{\xi\eta} + v_{\eta\eta} \\
u_{yy} &= -2v_\xi + 4y^2v_{\xi\xi} -4yv_{\xi\eta} + v_{\eta \eta}\\
u_{xy} &= -4xyv_{\xi\xi} + 2xv_{\xi\eta} + 2yv_{\xi\eta} +v_{\eta\eta}\\
\end{align*}

Thus, the PDE reduces to the canonical form, $4v_\xi (y-x) + v_{\xi \eta} (2x^2-4xy-2y^2) =4\eta v_\xi + 2\eta^2 v_{\xi \eta}= 0$. We can apply techniques of ODE to determine that $v_\xi = f(\xi)/\eta^2$, and thus $v(\xi,\eta) = F(\xi)/\eta^2 + G(\eta)$ for some $F$, $G$.

Returning to $x$ and $y$ variables, this means that $u(x,y) = v(x^2-y^2,y-x) = \frac{F(x^2-y^2)}{(y-x)^2} + G(y-x)$ is the general solution of the PDE given.

Using the initial data, we get that $x^4-x = \frac{F(x^2)}{x^2} + G(-x)$ and $1=\frac{2F(x^2)}{x^3} + G'(-x)$. Thus, we have

\begin{align*}
x/2&=\frac{F(x^2)}{x^2} + G'(-x)(x/2)\\
x^4-\frac{3}{2}x &= G(-x) - \frac{1}{2}xG'(-x)\\
x^4+\frac{3}{2}x &= G(x) + \frac{1}{2}xG'(x)\\
x^5+3x^2 &= 2xG(x) + x^2G'(x)\\
x^5+3x^2 &= (x^2G(x))'\\
\frac{x^6}{6}+x^3+C &= x^2G(x)\\
\frac{\frac{x^6}{6}+x^3+C}{x^2} &= G(x)\\
\end{align*}

Where $C$ is some unknown constant. Now, plugging this into the other bit of data, we get

\begin{align*}
x^4-x &= \frac{F(x^2)}{x^2}+ G(-x)\\
x^4-x &= \frac{F(x^2)}{x^2}+ \frac{\frac{x^6}{6}-x^3+C}{x^2}\\
x^6-x^3 &= F(x^2)+ \frac{x^6}{6}-x^3+C\\
\frac{5}{6}x^6-2x^3 -C&= F(x^2)\\
\frac{5}{6}x^3-2x^{3/2} -C&= F(x)\\
\end{align*}

So, to summarize, $u(x,y) = \frac{\frac{5}{6}(x^2-y^2)^3-2(x^2-y^2)^{3/2} -C}{(y-x)^2} + \frac{\frac{(y-x)^6}{6}+(y-x)^3+C}{(y-x)^2}$.

After some slight cleaning up, that is

\begin{displaymath}
u(x,y) = \frac{\frac{5}{6}(x^2-y^2)^3-2(x^2-y^2)^{3/2} +\frac{(y-x)^6}{6}+(y-x)^3}{(y-x)^2}
\end{displaymath}

\shunt

{\bf Problem 3:}

Consider $u(x,y)$ solving

\begin{align*}
2yu_x + u_y &= u^2\\
u(x,0)&= \frac{1}{x^2+1} \text{ for } x \in \R
\end{align*}

Part a:

We begin by applying the method of characteristics, as in the textbook.

We get that $\dot{x} = 2y$, $\dot{y} = 1$, $\dot{z} = z^2$. Thus, $x(s) = blah$, $y(s) = s$, and $z(s) = blah$. %mimic this when you're more sharp! You're way out of it right now!

\shunt

Part b:

\shunt


\end{document}