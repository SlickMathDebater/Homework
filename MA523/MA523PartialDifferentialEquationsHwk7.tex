
\documentclass[a4paper,12pt]{article}

\usepackage{fancyhdr}
\usepackage{amssymb}
%\usepackage{mathpazo}
\usepackage{mathtools}
\usepackage{amsmath}
\usepackage{slashed}
\usepackage{cancel}
\usepackage[mathscr]{euscript}
\usepackage{MaxPackage} %Note: You need MaxPackage installed or in the same folder as your .tex file or something.

\newcommand{\colorcomment}[2]{\textcolor{#1}{#2}} %First of these leaves in comments. Second one kills them.
%\newcommand{\colorcomment}[2]{}


\pagestyle{fancy}
\lhead{Max Jeter}
\chead{MA523}
\rhead{Assignment 7 , Page \thepage}

%Number of Problems		:3
%Clear					:1
%Begun					:2
%Not started			:3
%Can complete via book	:
%Needs Polish			:

%Pomodoros logged		:2.25

\begin{document}

{\bf Problem 1:}

Consider $u(x,y)$ solving

\begin{align*}
u_y^2u_{xx}+uu_{xy}+u_x^2u_{yy} &= u^2+1\\
u(x,0) = \sin(x), u_y(x,0) &= \cos(x)
\end{align*}

Then the order $2$ (and lower) partials for $u$ at $(0,0)$ are:

\begin{align*}
u(0,0) &= \sin(0) = 0\\
u_x(0,0) &= \cos(0) = 1\\
u_y(0,0) &= \cos(0) = 1\\
u_{xx}(0,0) &= -\sin(0) = 0\\
u_{xy}(0,0) &= -\sin(0) = 0\\
u_{yy}(0,0) &= 1\\
\end{align*}

The first five are obtained by the initial conditions and applying partial derivatives to them, and the last is obtained by plugging this information into the PDE. Thus, the second-order Taylor Approximation of $u$ about the point $(0,0)$ is

\begin{align*}
u(x,y) &\approx x+y+\frac{y^2}{2} \\
\end{align*}

Now, some of the order $2$ (and lower) partials for $u$ at $(\pi/2, 0)$ are:

\begin{align*}
u(0,0) &= \sin(\pi/2) = 1\\
u_x(0,0) &= \cos(\pi/2) = 0\\
u_y(0,0) &= \cos(\pi/2) = 0\\
u_{xx}(0,0) &= -\sin(\pi/2) = -1\\
u_{xy}(0,0) &= -\sin(\pi/2) = -1\\
\end{align*}

Plugging this information into the PDE yields $-1=2$, which is nonsense; thus, $u$ is inconsistent at $(\pi/2,0)$.

\shunt

{\bf Problem 2:}

Let

\begin{align*}
L[u] = yu_{xx} + (x+y)u_{xy} + xu_{yy} - u_x - u_y
\end{align*}

Part a:

The equation $L$ is hyperbolic when $\De = \left(\frac{x+y}{2}\right)^2 - xy >0$. 

Rewriting this condition, we get:

\begin{align*}
\left(\frac{x+y}{2}\right)^2 - xy &>0\\
\left(\frac{x-y}{2}\right)^2&>0\\
\left(x-y\right)^2&>0\\
x &\neq y\\
\end{align*}

That is, $L$ is hyperbolic except when $x=y$. 

\shunt

Part b:

By the discussion in John, the characteristic curves of this PDE satisfy $\frac{dy}{dx} = \frac{(x+y)/2 \pm (x-y)/2}{y}$. That is, the characteristic curves satisfy either $\frac{dy}{dx} = \frac{x}{y}$ or $\frac{dy}{dx} = \frac{-y}{y}$. Solving the ODEs, we get that the characteristic curves are the hyperbolas given by $y^2-x^2 = c$ for some constant $c$, and the lines $y= -x + c$. %Can trim those down; I don't think the lines satisfy this. 

\shunt

Part c:

First, consider the solutions to $y\la^2 + (x+y)\la + x = 0$; they are $\la_1=-1$ and $\la_2 = -x/y$, by the quadratic formula.

We want $\xi$ and $\eta$ so that $\xi_x = -\la_1 \xi_y$ and $\eta_x = \la_2 \eta_y$. Choosing $\xi=x-y$ and $\eta=y^2-x^2$ works for this. Also, we can solve for $x$ and $y$ in terms of $\xi$ and $\eta$: $x=\frac{\eta-\xi^2}{2\xi}$ and $y= \frac{\eta-\xi^2}{2\xi}-\xi$.

\shunt

{\bf Problem 3:}

Part a:

\shunt

Part b:

\shunt


\end{document}