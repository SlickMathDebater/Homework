
\documentclass[a4paper,12pt]{article}

\usepackage{fancyhdr}
\usepackage{amssymb}
%\usepackage{mathpazo}
\usepackage{mathtools}
\usepackage{amsmath}
\usepackage{slashed}
\usepackage{cancel}
\usepackage[mathscr]{euscript}
\usepackage{MaxPackage} %Note: You need MaxPackage installed or in the same folder as your .tex file or something.

\newcommand{\colorcomment}[2]{\textcolor{#1}{#2}} %First of these leaves in comments. Second one kills them.
%\newcommand{\colorcomment}[2]{}


\pagestyle{fancy}
\lhead{Max Jeter}
\chead{Class}
\rhead{Assignment, Page \thepage}

%Number of Problems		: 2
%Clear					:
%Begun					: 1
%Not started			: 2
%Can complete via book	:
%Needs Polish			: 1

%Pomodoroes logged		: 5

\begin{document}

{\bf Problem 1:}

Consider $\prod\limits_{n=1}^\infty \cos(\frac{z}{2^n})$.

To evaluate this, we apply the telescoping trick; so, note that for each $N \in \N$, $\sin(\frac{z}{2^N}) \prod\limits_{n=1}^N \cos(\frac{z}{2^n}) = \frac{1}{2^N} \sin(z)$.

So $\prod\limits_{n=1}^N \cos(\frac{z}{2^n}) = \frac{\sin(z)}{2^N \sin(\frac{z}{2^N})}$, when $\sin(\frac{z}{2^N}) \neq 0$. By taking limits as $N \to \infty$, we get that $\prod\limits_{n=1}^\infty \cos(\frac{z}{2^n}) = \lim\limits_{N \to \infty} \frac{-\ln(2)2^{-N}\sin(z)}{-\ln(2)2^{-N}\cos(z2^{-N})} = \sin(z)$ when $\sin(\frac{z}{2^N}) \neq 0$ for any $N \in \N$. 

When $\sin(\frac{z}{2^N}) = 0$ for some $N \in \N$, we have that $\sin(z) = 0$. Also, we have that $\cos(\frac{z}{2^{N+k}}) = 0$ for some $k \geq 0$ if $z \neq 0$ (we divide by enough $2$s to reduce it to $\pi/2$ times an odd number); the above formula holds true when $z \neq 0$ , then.

Yet when $z = 0$, the product above is trivially $1$.

So:

\begin{displaymath}
\prod\limits_{n=1}^\infty \cos(\frac{z}{2^n}) = 
   \left\{
     \begin{array}{lr}
       \sin(z)  & \text{ for } z \neq 0 \\
       1 & \text{ for } z=0 
     \end{array}
   \right.
\end{displaymath}

\shunt

{\bf Problem 2:}

Consider that $\pi^2 \csc^2(\pi z) = \sum\limits_{-\infty}^\infty \frac{1}{(z-n)^2}$.

By taking derivatives, we have that $-2\pi^3\csc^2(\pi z)\cot(\pi z) = -2\sum\limits_{-\infty}^\infty \frac{1}{(z-n)^3}$.

Dividing by our original identity, 

\begin{align*}
-2\pi \cot(\pi z) &= \frac{-2\pi^3\csc^2(\pi z)\cot(\pi z)}{\pi^2 \csc^2(\pi z)}\\
&= \frac{(\pi^2 \csc^2(\pi z))'}{\pi^2 \csc^2(\pi z)}\\
&= \sum\limits_{-\infty}^\infty \frac{\frac{-2}{(z-n)^3}}{\frac{1}{(z-n)^2}} \text{ (Because }f'/f = \sum (f_n'/f)).\\
&= \sum\limits_{-\infty}^\infty \frac{-2}{z-n}\\
&= -2\left[\frac{1}{z} + \sum\limits_{n=1}^\infty \frac{1}{z-n} + \frac{1}{z+n} \right]\\
&= -2\left[\frac{1}{z} + \sum\limits_{n=1}^\infty \frac{2z}{z^2-n^2} \right]\\
\end{align*}

Which is equivalent to our desired result.

\shunt

\end{document}