
\documentclass[a4paper,12pt]{article}

\usepackage{fancyhdr}
\usepackage{amssymb}
%\usepackage{mathpazo}
\usepackage{mathtools}
\usepackage{amsmath}
\usepackage{slashed}
\usepackage{cancel}
\usepackage[mathscr]{euscript}
\usepackage{MaxPackage} %Note: You need MaxPackage installed or in the same folder as your .tex file or something.

\newcommand{\colorcomment}[2]{\textcolor{#1}{#2}} %First of these leaves in comments. Second one kills them.
%\newcommand{\colorcomment}[2]{}


\pagestyle{fancy}
\lhead{Max Jeter}
\chead{MA530}
\rhead{Assignment 11, Page \thepage}

%Number of Problems		:7
%Clear					:1,2,4,6,7
%Begun					:
%Not started			:5
%Can complete via book	:
%Needs Polish			:3

%Pomodoros logged		:3

\begin{document}

For reference: in the below, $G(z) = z \prod\limits_{n=1}^\infty (1+\frac{z}{n}) e^{-z/n}$. I don't know if this is standard, so it's worth including.

{\bf Problem 1:}

Consider $\sum\limits_{n=1}^\infty \frac{1}{n^s}$, where $\text{Re}(s) > 1$.

The sum converges if and only if the integral $\int\limits_1^\infty \frac{1}{n^s} dn$ does. We know that $\int\limits_1^\infty \frac{1}{n^s} ds \leq \int\limits_1^\infty \absval{\frac{1}{n^s}} ds = \int\limits_1^\infty \frac{1}{\absval{n^s}} ds \leq \int\limits_1^\infty \frac{1}{n^{\text{Re}(s)}} ds$, and the last integral converges, so the first one must have as well.

Moreover, $\zeta(s) = \sum\limits_{n=1}^\infty \frac{1}{n^s}$ is holomorphic on $\text{Re}(s) >1$: it's a limit of holomorphic functions.

\shunt

{\bf Problem 2:}

Consider $\frac{\sin(\pi z)}{\pi z} = \prod\limits_{n=1}^\infty (1-\frac{z^2}{n^2})$.

The taylor coefficients attached to $z^2$ and $z^4$ of $\frac{\sin(\pi z)}{\pi z}$ are $-\pi/6$ and $\pi^2/120$, respectively, because $\frac{\sin(\pi z)}{\pi z} = \frac{\sum\limits_{n=1}^\infty \frac{(-1)^n(\pi z)^{2n+1}}{(2n+1)!}}{\pi z} = \sum\limits_{n=1}^\infty \frac{(-1)^n(\pi z)^{2n}}{(2n+1)!}$.

The taylor coefficients attached to $z^2$ and $z^4$ of $\prod\limits_{n=1}^\infty (1-\frac{z^2}{n^2})$ are $-\sum\limits_{n=1}^\infty \frac{1}{n^2}$ and $\sum\limits_{i \neq j} \frac{1}{i^2j^2}$, respectively; these follow by multiplying the product out.

Therefore, $\sum\limits_{n=1}^\infty \frac{1}{n^2} = \pi/6$.

We can rewrite the second one as $\sum\limits_{i,j} \frac{1}{i^2j^2} - \sum\limits_{n=1}^\infty \frac{1}{n^4}$.  We evaluate:

\begin{align*}
\sum\limits_{i,j} \frac{1}{i^2j^2} &= \sum\limits_{i=1}^\infty \sum\limits_{j=1}^\infty \frac{1}{i^2}\frac{1}{j^2}\\
&= \sum\limits_{i=1}^\infty \frac{1}{i^2}\sum\limits_{j=1}^\infty \frac{1}{j^2}\\
&= \sum\limits_{i=1}^\infty \frac{1}{i^2}\frac{\pi}{6}\\
&= \pi^2/36\\
\end{align*}

Therefore, $\sum\limits_{n=1}^\infty \frac{1}{n^2} = \pi/6$ and $\sum\limits_{n=1}^\infty \frac{1}{n^4} = \frac{\pi^2}{36} - \frac{\pi^2}{120} = \frac{7\pi^2}{120}$.

\shunt

{\bf Problem 3:}

First: note that $\Ga(n+1) = n\Ga(n)$, and $\Ga(1) = 1$; as discussed in class, 

\begin{align*}
\Ga(1) &= \frac{e^{-\ga}}{G(1)}\\
&= \frac{e^{-\ga}}{\prod\limits_{n=1}^\infty (1+1/n)e^{-1/n}}\\
&= \frac{e^{-\ga}}{e^{-(\sum\limits_{n=1}^\infty (1/n - \ln(1+1/n)))}}\\
&= \frac{e^{-\ga}}{e^{-(\sum\limits_{n=1}^\infty (1/n - \ln(1+1/n)))}}\\
&= \frac{e^{-\ga}}{e^{-\ga}}\\
&= 1
\end{align*} %Explain the last chunk.

So $\Ga(n) = (n-1)!$, by a relatively clear induction argument, recreated below so the problem doesn't look too short:

First, $\Ga(1) = 1!$.

Next, if $\Ga(n) = (n-1)!$, then $\Ga(n+1) = n\Ga(n) = n!$. So by induction, we have our result.

\shunt

{\bf Problem 4:}

Consider $\Ga(z)\Ga(1-z)$.

\begin{align*}
\Ga(z)\Ga(1-z) &= -z\Ga(z)\Ga(-z)\\
&=-z\frac{e^{-\ga z}}{G(z)} \frac{e^{\ga z}}{G(-z)}\\
&=\frac{-z}{G(z)G(-z)}\\
&=\frac{-z}{-z z \prod\limits_{n=1}^\infty (1+z/n) \prod\limits_{n=1}^\infty (1-z/n)}\\
&=\frac{1}{ z \prod\limits_{n=1}^\infty (1-(z/n)^2)}\\
&= \frac{\pi}{\sin(\pi z)}
\end{align*}

Thus, $\Ga(1/2)\Ga(1-1/2) = \frac{\pi}{\sin(\pi/2)} = \pi$.

That is, $\Ga(1/2) = \sqrt{\pi}$. (Note that $\Ga(z) > 0$ if $z >0$, so $\Ga(1/2) \neq - \sqrt{\pi}$).

\shunt

{\bf Problem 5:}

I'm not doing this problem, as I have no idea what to do, and each of my classmates appear to have sunk multiple hours into it.

\shunt

{\bf Problem 6:}

If $a \in \R$, $a>0$, then consider $f: A \to D_1(0)$ given by $ z \mapsto \frac{z-a}{z+a}$, where $A = \{z \in \C: \text{Re}(z) >0\}$.

First, $f$ is holomorphic: it's a product of two holomorphic functions ($z-a$ and $\frac{1}{z+a}$).

Next, $f$ is injective: let $f(x)=f(y)$, with $x,y \in A$. Then $\frac{x-a}{x+a} = \frac{y-a}{y+a}$. So:

\begin{align*}
(x-a)(y+a) &= (x+a)(y-a) \\
xy -ay +ax -a^2 &= xy -ax + ay - a^2 \\
2(ax-ay) &= 0\\
x&=y
\end{align*}

as desired.

Thus, $f$ is a biholomorphism, by the result in class.

\shunt

{\bf Problem 7:}

Define $S_{a,b} = \{z \in \C: a< \text{arg}(z) < b\}$.

The function $f: S_{\al,\be} \to S_{2\al,2\be}$ given by $f(z) = z^2$ is biholomorphic when $\be-\al < \pi$.

First, $f$ is holomorphic, and this is clear.

Next, $f$ is injective: let $a=re^{i\tha}, b=r'e^{i\tha'} \in S_{\al,\be}$ with $f(a)=f(b)$. Then $r^2e^{i2\tha}=r'^2e^{i2\tha'}$. So $e^{i2\tha} = e^{i2\tha'}$, and $r^2 = r'^2$. But because $\be-\al < \pi$, this means that $\absval{\tha-\tha'} < 2\pi$. So This means that $\tha = \tha'$, so we end up with $a=re^{i\tha}=r'e^{i\tha'}=b$, as desired.

So, from the result in class, $f$ is biholomorphic.

Note that if $\be-\al \geq \pi$, there are $z,z' \in S_{\al,\be}$ with $z=re^{i\tha}$ and $z' = re^{i(\tha + \pi)}$, and thus $z^2 = r^2e^{2i\tha} = r^2e^{2i\tha + 2\pi} = r^2e^{2i(\tha + \pi)} = {z'}^2$, so the above condition is the strongest we can get.

\shunt

\end{document}