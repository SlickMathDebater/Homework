
\documentclass[a4paper,12pt]{article}

\usepackage{fancyhdr}
\usepackage{amssymb}
%\usepackage{mathpazo}
\usepackage{mathtools}
\usepackage{amsmath}
\usepackage{slashed}
\usepackage{cancel}
\usepackage[mathscr]{euscript}
\usepackage{MaxPackage} %Note: You need MaxPackage installed or in the same folder as your .tex file or something.

\newcommand{\colorcomment}[2]{\textcolor{#1}{#2}} %First of these leaves in comments. Second one kills them.
%\newcommand{\colorcomment}[2]{}


\pagestyle{fancy}
\lhead{Max Jeter}
\chead{MA530}
\rhead{Assignment 12, Page \thepage}

%Number of Problems		:5
%Clear					:1,2,3,5
%Begun					:4
%Not started			:
%Can complete via book	:
%Needs Polish			:

%Pomodoros logged		:7

\begin{document}

{\bf Problem 1:}

Consider $g: \{z \in \C: 0 < \text{Im}(z) < 2\pi\} \to \C$ given by $g(z) = e^z$. We know that $g$ is holomorphic. Define $A = \{z \in \C: 0 < \text{Im}(z) < 2\pi\}$.

Also, $g$ is injective: let $g(z)=g(w)$, where $z=a+bi \in A$ and $w=c+di \in A$ (and $a,b,c,d \in \R$). Then:

\begin{align*}
e^z&=e^w\\
e^{a+bi} &= e^{c+di}\\
e^ae^{bi} &=e^ce^{di}
\end{align*}

So $e^a=e^c$, so $a=c$. Also, $e^{bi} = e^{di}$, so because $b,d \in (0,2\pi)$, we have that $b=d$. 

So $z=w$, as desired.

So $g$ is an injective holomorphism: it is a biholomorphism between $A$ and $g(A)$.

Moreover, $g(\{z \in \C: 0 < \text{Im}(z) < 2\pi\}) = \C \setminus \R^+$: if $z \in \C \setminus \R^+$, then $\ln(\absval{z}) + i \text{arg}(z) \mapsto z$.

Also, if $z \in \R^+$, then $e^{a+bi} = z$ implies that $b$ is a natural number times $2\pi$, which all lie outside our domain.

\shunt

{\bf Problem 2:}

(Note: I read a stronger version of this proof in Complex Made Simple prior to this problem being assigned; the extra assumption allows for a lot of stripping away of details.)

Let $\Om$ be a convex open set, $\phi \in \scrO(\Om)$, with $\text{Re}(\phi'(z)) > 0$.

We know that $\phi$ is holomorphic.

Consider $\phi(a)-\phi(b)$. Because $\Om$ is convex, we can calculate this by integrating over the line segment $[a,b]$:

\begin{align*}
\absval{\phi(a)-\phi(b)}&= \absval{\int\limits_{[a,b]} \phi'(z) dz}\\
&\geq \absval{\int\limits_{[a,b]} \text{Re}(\phi'(z))dz}\\
\end{align*}

Because $\text{Re}(\phi'(z)) >0$, the absolute value of the integral is greater than zero if $a \neq b$. So $\phi(a)-\phi(b) \neq 0$ if $a \neq b$. That is, $\phi$ is injective.

So $\phi$ is a biholomorphism. 

\shunt

{\bf Problem 3:}

Let $S_{0,\al} = \{z \in \C: 0< \text{arg}(z) < \al\}$ for all $0<\al \leq 2\pi$. 

Consider $S_{0,\al}$ and $S_{0,\be}$. The map $\phi: S_{0,\al} \to S_{0,\be}$ given by $\phi(re^{i\tha}) = re^{\frac{\be}{\al}i\tha}$ is a biholomorphism.

First, $\phi$ is well defined: if $re^{i\tha} \in S_{0,\al}$, then $\phi(z) = re^{\frac{\be}{\al}i\tha}$ has $0<\frac{\be}{\al}\tha <\be$, so that $\phi(z) \in S_{0,\be}$. Moreover, because $0<\al<2\pi$, for each $z$ there is a unique $\tha$ with $r>0$ and $re^{i\tha}=z$.

Next, $\phi$ is a holomorphism, and this is clear using polar coordinates.

Last, $\phi$ is injective: let $\phi(z) = \phi(w)$, with $z = ae^{ib}$ and $w = ce^{id}$ with $a,b,c,d \in \R$. Then:

\begin{align*}
ae^{\frac{\be}{\al}ib} &=ce^{\frac{\be}{\al}id}\\
\end{align*}

So $a = c$, and $e^{\frac{\be}{\al}ib} = e^{\frac{\be}{\al}id}$. Because $\frac{\be}{\al} b,\frac{\be}{\al}d \in (0,2\pi]$, this means that $b=d$. So $z=w$, as desired.

So we have a biholomorphism, $\phi: S_{0,\al} \to S_{0,\be}$. Thus, $S_{0,\al}$ and $ S_{0,\be}$ are conformally equivalent. 

\shunt

{\bf Problem 4:}

Consider $A=\text{Aut}(\{z:\text{Im}(z) > 0\})$ and $B$, the set of maps of the form $z \mapsto \frac{az+b}{cz+d}$ with $a,b,c,d \in \R$ and $ad-bc >0$, as well as $C = \{z:\text{Im}(z) > 0\}$.

Let $\phi \in B$.

First, $\phi$ is injective: let $\phi(z) = \phi(w)$. Then:

\begin{align*}
\frac{az+b}{cz+d} &= \frac{aw+b}{cw+d}\\
({az+b})({cw+d}) &= ({aw+b})({cz+d})\\
acwz + adz+bcw +bd &= acwz+adw+bcz+bd\\
(ad-bc)z&=(ad-bc)w\\
z&=w
\end{align*}

So where $\phi$ is defined, $\phi$ is injective.

Also, $\phi$ is a holomorphism: except in the cases where $c \neq 0$ and at the point $z = -d/c$, this is clear. But because $-d/c$ is real, we don't have to consider this: it lies outside of our domain.

Thus, $\phi$ is a biholomorphism into some set.

Now, $\phi(C) = C$:

Let $z \in C$. Then we have that $\frac{az+b}{cz+d} = \frac{ac\absval{z}^2 +adz+bc\overline{z} + bd}{(cz+d)\overline{(cz+d)}}$. The sign of the imaginary part is determined solely by $adz+bc\overline{z}$; because this has positive imaginary part ($ad-bc >0$ and $\text{Im}(z) >0$), we have that $\phi(z) \in C$. 

Also, if $z \in \phi(C)$, then note that $\phi$ has a holomorphic inverse (it is $\phi^{-1}(z) = \frac{dz-b}{a-cz}$. The same analysis as above verifies that this is holomorphic.) %Do the same fucking analysis.

Thus, $\phi$ is a biholomorphism from $C$ to $C$. That is, $\phi \in A$. 

Now, let $\phi \in A$. Then $\phi$ can be extended to a biholomorphism in $\overline{\C}$. That is, $\tilde{\phi} \in \text{Aut}(\overline{\C})$ for some $\tilde{\phi}$. So $\tilde{\phi}$ is linear-fractional: $\tilde{\phi}(z) = \frac{az+b}{cz+d}$ for some $a,b,c,d \in \C$.

Now, $\tilde{\phi}(C) = C$, because $\phi$ is a biholomorphism from $C$ to $C$. So we can require that $a,b,c,d \in \R$: write $a=re^{i\al}$, $b=se^{i\be}$, $c=te^{i\ga}$, $d=ue^{i\de}$. Then $\phi(z) = \frac{az+b}{cz+d} = \frac{re^{i\al}z+se^{i\be}}{te^{i\ga}z+ue^{i\de}} = e^{i(\al-\ga)}\frac{rz+se^{i(\be-\al)}}{tz+ue^{i(\de-\ga)}}$. But we know that $\frac{rz+se^{i(\be-\al)}}{tz+ue^{i(\de-\ga)}}$ is a biholomorphism from $C$ to $C$ by the above; multiplication by $e^{i\tha}$ rotates the half-plane by $\tha$, so $\al-\ga$ must have been a multiple of $2\pi$. So we have $\phi(z) = \frac{rz+se^{i(\be-\al)}}{tz+ue^{i(\de-\ga)}}$. Similarly, we have that $\phi(z) = \frac{rze^{i(\al-\be)}+s}{tze^{i(\ga-\de)}+u}$. A bit of linear algebra shows that this combination implies that $\al -\be$ and $\de-\ga$ are multiples of $2\pi$. So $\phi(z) = \frac{rz+s}{tz+u}$.

That is, $\phi \in A$.

So $A=B$, as desired.

\shunt

{\bf Problem 5:}

Let $g:D_1(0) \to D_1(0)$ be holomorphic, with $g(0)=g'(0)= \ldots g^{(k)}(0) = 0$.

Then $h(z) = \frac{g(z)}{z^{k+1}}$ is holomorphic; on $D_1(0)$, $h(z) = \frac{\sum\limits_{n=k+1}^\infty a_nz^n}{z^{k+1}} = \sum\limits_{n=k+1} a_nz^{n-(k+1)}$. That is, $h$ is represented as a power series, $h$ is holomorphic.

Now, $\absval{h(z)} \leq \max\limits_{\partial D_r(0)} \absval{h}$ for all $z \in D_r(0)$ with $r<1$.  So $\absval{h(z)} \leq \frac{1}{r}$ for all $r < 1$. By taking limits as $r \to 1$, we get that $\absval{h(z)} \leq 1$ for all $z \in D_1(0)$.

So $g(z) \leq \absval{z}^{k+1}$ for $z \in D_1(0)$.

Now, if we have $g(z) = \absval{z}^{k+1}$ for some $z \in D_1(0)$, we get $\absval{h(z)} =1$. That is, $h$ achieves its maximum. So by the maximum principle, $h$ is constant; say $h = c$. So then we have $g(z) = cz^{k+1}$, for some $c \in \C$ with $\absval{c}=1$. 

\shunt

\end{document}