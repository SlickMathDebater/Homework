
\documentclass[a4paper,12pt]{article}

\usepackage{fancyhdr}
\usepackage{amssymb}
%\usepackage{mathpazo}
\usepackage{mathtools}
\usepackage{amsmath}
\usepackage{slashed}
\usepackage{cancel}
\usepackage[mathscr]{euscript}
\usepackage{MaxPackage} %Note: You need MaxPackage installed or in the same folder as your .tex file or something.

\newcommand{\colorcomment}[2]{\textcolor{#1}{#2}} %First of these leaves in comments. Second one kills them.
%\newcommand{\colorcomment}[2]{}


\pagestyle{fancy}
\lhead{Max Jeter}
\chead{MA530}
\rhead{Assignment 13, Page \thepage}

%Number of Problems		:6
%Clear					:
%Begun					:
%Not started			:1,2,3,4,5,6
%Can complete via book	:
%Needs Polish			:

%Pomodoros logged		:0.5

\begin{document}

{\bf Problem 1:} %May check this to be more careful...

The map described in class is $f\circ g \circ h$, where $f(z) = \frac{z-1}{z+1}$, $g(z) = \sqrt{z}$, and $h(z) = \frac{z-1}{z+1}$.

Its inverse is thus $h^{-1} \circ g^{-1} \circ f^{-1}$, which is $F: D_1(0) \to \overline{\C} \setminus [-1,1]$ where $F(z) = \frac{\left(\frac{z+1}{1-z}\right)^2 + 1}{1-\left(\frac{z+1}{1-z}\right)^2} = \frac{-z^2-1}{2z}$.

Consider the set $\partial D_r(0)$ where $r < 1$. We see that $F(\partial D_r(0)) = \{\frac{-z^2-1}{2z}: \absval{z} = r\}$ %Prove it.

\shunt

{\bf Problem 2:}

%May try bashing this one over the head with definitions. 

\shunt

{\bf Problem 3:}

%This one should fall out of the proof of problem 2...

\shunt

{\bf Problem 4:}

\shunt

{\bf Problem 5:}

\shunt

{\bf Problem 6:}

\shunt

{\bf Problem 7:}

\shunt

\end{document}