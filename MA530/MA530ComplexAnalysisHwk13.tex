
\documentclass[a4paper,12pt]{article}

\usepackage{fancyhdr}
\usepackage{amssymb}
%\usepackage{mathpazo}
\usepackage{mathtools}
\usepackage{amsmath}
\usepackage{slashed}
\usepackage{cancel}
\usepackage[mathscr]{euscript}
\usepackage{MaxPackage} %Note: You need MaxPackage installed or in the same folder as your .tex file or something.

\newcommand{\colorcomment}[2]{\textcolor{#1}{#2}} %First of these leaves in comments. Second one kills them.
%\newcommand{\colorcomment}[2]{}


\pagestyle{fancy}
\lhead{Max Jeter}
\chead{MA530}
\rhead{Assignment 13, Page \thepage}

%Number of Problems		:8
%Clear					:
%Begun					:4,5,6,7
%Not started			:8
%Can complete via book	:
%Needs Polish			:2,3
%Absolutely broken		:1

%Pomodoros logged		:10

\begin{document}

Note: in the below, we adopt the notation $\phi_a: D_1(0) \to D_1(0)$ to be given by $\phi_a(z) = \frac{z-a}{1-\overline{a}z}$. This is the same as the $f_a$ given in class, but that notation lends itself to issues in this homework. 

I should've said this in the other homework as well, but I use $\overline{\C}$ to denote the Riemann Sphere because I can't figure out how to get $\C$ with a hat over it.

{\bf Problem 1:} %Something is very, very wrong.

The map described in class is $f\circ g \circ h$, where $f(z) = \frac{z-1}{z+1}$, $g(z) = \sqrt{z}$, and $h(z) = \frac{z-1}{z+1}$.

Its inverse is thus $h^{-1} \circ g^{-1} \circ f^{-1}$, which is $F: D_1(0) \to \overline{\C} \setminus [-1,1]$ where $F(z) = \frac{\left(\frac{z+1}{1-z}\right)^2 + 1}{1-\left(\frac{z+1}{1-z}\right)^2} = \frac{-z^2-1}{2z}$.

Consider the set $\partial D_r(0)$ where $r < 1$. We see that $F(\partial D_r(0)) = \{\frac{-z^2-1}{2z}: \absval{z} = r\}$....

I have recognized that this is horribly broken ($1/2$ maps to $3/4$, which is on the line segment we excluded), and I have no earthly clue how to fix this.

\shunt

{\bf Problem 2:}

Consider $\phi_{f(0)}(f(z))$. Taking a derivative, we get:

\begin{align*}
(\phi_{f(0)}(f(z)))' &= \phi_{f(0)}'(f(z))f'(z)\\
f'(z) &= \frac{(\phi_{f(0)}(f(z)))'}{\phi_{f(0)}'(f(z))}\\
\absval{f'(z)} &= \absval{\frac{(\phi_{f(0)}(f(z)))'}{\phi_{f(0)}'(f(z))}}\\
\absval{f'(0)} &= \absval{\frac{(\phi_{f(0)}(f(0)))'}{\phi_{f(0)}'(f(0))}}\\
\absval{f'(0)} &= (1-\absval{f(0)}^2)\absval{(\phi_{f(0)}(f(0)))'}\\
\end{align*}

with the last line being because $\absval{\phi_{a}'(a)} = \frac{1}{1-\absval{a}^2}$, which was discussed in class.

Moreover, $\absval{(\phi_{f(0)}(f(0)))'} \leq 1$, by Schwarz's lemma. (Note that $\phi_{f(0)}(f(z))$ is a holomorphic map fixing the origin, so its derivative at the origin is at most $1$.)

Thus, we have $\absval{f'(z)} \leq (1-\absval{f(0)}^2)$.

\shunt

{\bf Problem 3:}

Fix $z \in D_1(0)$. Consider $f(\phi_{-z}(w))$ as a function of $w$. Taking a derivative, we get: 

\begin{align*}
(f(\phi_{-z}(w)))' &= f'(\phi_{-z}(w))\phi_{-z}'(w)\\
(f(\phi_{-z}(0)))' &= f'(\phi_{-z}(0))\phi_{-z}'(0)\\
\frac{(f(\phi_{-z}(0)))'}{\phi_{-z}'(0)} &= f'(\phi_{-z}(0))\\
\frac{(f(\phi_{-z}(0)))'}{\phi_{-z}'(0)} &= f'(z)\\
\absval{\frac{(f(\phi_{-z}(0)))'}{\phi_{-z}'(0)}} &= \absval{f'(z)}\\
\absval{f'(z)}&= \absval{f(\phi_{-z}(0)))'} \frac{1}{1-\absval{-z}^2}\\
\absval{f'(z)}&\leq \frac{1}{1-\absval{z}^2}\\
\end{align*}

With the last line being by Schwarz's lemma, as above.

\shunt

{\bf Problem 4:}

Consider $\{z \in \C: A \absval{z}^2 + 2\text{Re}(Bz^2) + 2\text{Re}(Cz) + D = 0\}$, with $A,D \in \R$, $B,C \in \C$ ($A,B,C,D$ fixed).

This describes a line when $A=B=0$; If $A$ or $B$ is nonzero, then something. %Pick 3 non-colinear points.
However, if $A=B=0$, then the set becomes $\{z \in \C: 2\text{Re}(Cz) = D\}$, which is rather clearly a line.

This describes a circle when %something...not just B=C=0, that is too strong.

\shunt

{\bf Problem 5:}

(Note: I had read this in Complex Made Simple before this was assigned.)

Let $\phi \in \text{Aut}(\overline{\C})$. Say $\scrC$ is the set of all circles and lines in the complex plane.

Note that $\text{Aut}(\overline{\C})$ is the set of linear-fractional transformations. Further note that the set of linear-fractional transformations is generated, as a group, by the maps of the form $z \mapsto az+b$ (with $a,b \in \C$) and the map $z \mapsto 1/z$.

It suffices to show our result for the generating set.

The result is clear for linear maps (note that they're a dilation followed by a translation followed by a rotation.) 

For the map $f(z) = 1/z$, let $\ell$ be a line through the origin: that is, $\ell = \{z \in \C: \exists r \in \overline{\R}: z=\ep r\}$ for some fixed $\ep \in \C$ with $\absval{\ep} = 1$. Then $f(\ell)$ is another line: $f(\ell) = \{z \in \C: \exists r \in \overline{\R}: z=1/(\ep r)\}$; note that $\absval{1/\ep} = 1$ and $1/r$ is an automorphism of $\overline{\R}$. 

If $\ell$ is a line that misses the origin: that is, $\ell = \{z \in \C: \exists r \in \overline{\R}: z=\ep r+ c\}$ for some fixed $\ep \in \C$ and $c \in \C$ with $\absval{\ep} = 1$ . Then $f(\ell)$ is a circle: $f(\ell) = \{z \in \C: \exists r \in \overline{\R}: z = 1/(\ep r + c)\}$, which is a circle. (I am somewhat certain we discussed this in class.)  

Let $\Ga$ be a circle centered at the origin. Then %do it

Let $\Ga$ be a circle not centered at the origin. Then %Again.

So in all cases, $f(z) = 1/z$ maps $\scrC$ to itself.

So we have the desired result. 

\shunt

{\bf Problem 6:}

Let $\Om \subset \C$ be open, $f_n \in \scrO(\Om)$, $\sup(\absval{f_n(z)}) = L < \infty$, $\xi_j \in \Om$, (with each $\xi_j$ distinct), $\xi_j \to \xi \in \Om$, and $f_n(\xi_j) \to \Xi_j$ for some $\Xi_j$.

By Vitali-Montel, there's a subsequence of $f_n$, call it $f_{n_k}$, that converges locally uniformly to some holomorphic function, $f$.

Consider $\scrF$, the set of functions $f$ such that $f_{n_k}$ converges to $f$ for some subsequence $f_{n_k}$. %Show that this set only has f, probably by uniqueness theorem.

So $f_n$ converges to $f$, and $f_n$ has a subsequence that converges locally uniformly to $f$. %Get the last little bit...may do this by contradiction.

\shunt

{\bf Problem 7:}

Consider $\text{Aut}(\C \setminus \{0\})$.

Let $\phi \in \text{Aut}(\C \setminus \{0\})$. Then $\phi$ is an injective holomorphism with singularities at $0$ and $\infty$. By the exam problem, $\phi$ has removable singularities or (first order) poles at $0$ and $\infty$.

If $\phi$ has a removable singularity at $0$, then $\phi$ is extended naturally to an automorphism of $\C$. Thus, $\phi$ is given by $z \mapsto az+b$ for some $a,b \in \C$. Note that $b=0$ in this case, otherwise $\phi(-b/a) = 0$, so that $\phi$ is no longer well defined. Moreover, $a \neq 0$, else $\phi$ isn't injective.

So if $\phi$ has a removable singularity at $0$, then $\phi$ is given by $z \mapsto az$ for some $a \in \C$, $a \neq 0$.

Next, let $\phi$ have a pole at $0$. Then %Apply 1/z and regress to the above situation.

\shunt

{\bf Problem 8:}

\shunt

\end{document}