
\documentclass[a4paper,12pt]{article}

\usepackage{fancyhdr}
\usepackage{amssymb}
%\usepackage{mathpazo}
\usepackage{mathtools}
\usepackage{amsmath}
\usepackage{slashed}
\usepackage{cancel}
\usepackage[mathscr]{euscript}
\usepackage{MaxPackage} %Note: You need MaxPackage installed or in the same folder as your .tex file or something.

\newcommand{\colorcomment}[2]{\textcolor{#1}{#2}} %First of these leaves in comments. Second one kills them.
%\newcommand{\colorcomment}[2]{}


\pagestyle{fancy}
\lhead{Max Jeter}
\chead{MA530}
\rhead{Assignment 13, Page \thepage}

%Number of Problems		:8
%Clear					:
%Begun					:3,5,6,7
%Not started			:2,4,8
%Can complete via book	:
%Needs Polish			:1

%Pomodoros logged		:4.5

\begin{document}

{\bf Problem 1:} %May check this to be more careful...

The map described in class is $f\circ g \circ h$, where $f(z) = \frac{z-1}{z+1}$, $g(z) = \sqrt{z}$, and $h(z) = \frac{z-1}{z+1}$.

Its inverse is thus $h^{-1} \circ g^{-1} \circ f^{-1}$, which is $F: D_1(0) \to \overline{\C} \setminus [-1,1]$ where $F(z) = \frac{\left(\frac{z+1}{1-z}\right)^2 + 1}{1-\left(\frac{z+1}{1-z}\right)^2} = \frac{-z^2-1}{2z}$.

Consider the set $\partial D_r(0)$ where $r < 1$. We see that $F(\partial D_r(0)) = \{\frac{-z^2-1}{2z}: \absval{z} = r\}$. 

\shunt

{\bf Problem 2:}

%Consider composing f with a biholomorphism of the disk and taking the derivative using chain rule.

\shunt

{\bf Problem 3:}

Consider $f(z) = f(\phi_{-z}(0))$. Taking derivatives, we get

\begin{align*}
f'(z) &= (f(\phi_{-z}(0)))'\\
&= f'(\phi_{-z}(0))\phi_{-z}'(0)\\
&= f'(\phi_{-z}(0))\frac{1}{1-\absval{z}^2}\\
\absval{f'(z)}&\leq \frac{1}{1-\absval{z}^2}\\
\end{align*}

with the last line being due to the previous problem after an adjustment. %Perform the adjustment. Also, check that derivative.

\shunt

{\bf Problem 4:}

Consider $\{z \in \C: A \absval{z}^2 + 2\text{Re}(Bz^2) + 2\text{Re}(Cz) + D = 0\}$, with $A,D \in \R$, $B,C \in \C$ ($A,B,C,D$ fixed).

This describes a line when $A=B=0$; If $A$ or $B$ is nonzero, then %Pick 3 non-colinear points.
However, if $A=B=0$, then the set becomes $\{z \in \C: 2\text{Re}(Cz) = D\}$, which is rather clearly a line.

This describes a circle when %something...not just B=C=0, that is too strong.

\shunt

{\bf Problem 5:}

(Note: I had read this in Complex Made Simple before this was assigned.)

Let $\phi \in \text{Aut}(\overline{\C})$. Say $\scrC$ is the set of all circles and lines in the complex plane.

Note that $\text{Aut}(\overline{\C})$ is the set of linear-fractional transformations. Further note that the set of linear-fractional transformations is generated, as a group, by the set of maps $z \mapsto az+b$ (with $a,b \in \C$ and the map $z \mapsto 1/z$.

It suffices to show our result for the generating set.

The result is clear for linear maps (for circles, note that they're isometries. For lines, note that they're a dilation followed by a translation followed by a rotation.) 

For the map $1/z$, consider a line $\ell = \{z \in \C: \}$. %I forget how to do this, kek

\shunt

{\bf Problem 6:}

Let $\Om \subset \C$ be open, $f_n \in \scrO(\Om)$, $\sup(\absval{f_n(z)}) = L < \infty$, $\xi_j \in \Om$, (with each $\xi_j$ distinct), $\xi_j \to \xi \in \Om$, and $f_n(\xi_j) \to \Xi_j$ for some $\Xi_j$.

By Vitali-Montel, there's a subsequence of $f_n$, call it $f_{n_k}$, that converges locally uniformly to some holomorphic function, $f$.

Consider $\scrF$, the set of functions $f$ such that $f_{n_k}$ converges to $f$ for some subsequence $f_{n_k}$. %Show that this set only has f, probably by uniqueness theorem.

So $f_n$ converges to $f$, and $f_n$ has a subsequence that converges locally uniformly to $f$. %Get the last little bit...may do this by contradiction.

\shunt

{\bf Problem 7:}

Consider $\text{Aut}(\C \setminus \{0\})$.

This is the subset of $\text{Aut}(\overline{\C})$ that maps $\{0,\infty\}$ to itself. That is, these are the maps of the form $z \mapsto az$ and $z \mapsto a/z$ for some $a \in \C$. %This may not be clear. 

\shunt

{\bf Problem 8:}

\shunt

\end{document}