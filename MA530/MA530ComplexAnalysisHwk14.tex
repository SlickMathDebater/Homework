
\documentclass[a4paper,12pt]{article}

\usepackage{fancyhdr}
\usepackage{amssymb}
%\usepackage{mathpazo}
\usepackage{mathtools}
\usepackage{amsmath}
\usepackage{slashed}
\usepackage{cancel}
\usepackage[mathscr]{euscript}
\usepackage{MaxPackage} %Note: You need MaxPackage installed or in the same folder as your .tex file or something.

\newcommand{\colorcomment}[2]{\textcolor{#1}{#2}} %First of these leaves in comments. Second one kills them.
%\newcommand{\colorcomment}[2]{}


\pagestyle{fancy}
\lhead{Max Jeter}
\chead{MA530}
\rhead{Assignment 14, Page \thepage}

%Number of Problems		:6
%Clear					:2,3,4,5,6
%Begun					:1
%Not started			:
%Can complete via book	:
%Needs Polish			:

%Pomodoros logged		:11

\begin{document}

{\bf Problem 1:}

Consider $a, b \in \R$, and consider the set of functions $u \in C^1([0,1])$ such that $u(0)=a, u(1)=b$. Without loss of generality, we can take $a=0$ and $b$ positive. (We are focused on the derivative's absolute value, so we can add/subtract constants and multiply by $-1$ as desired.)

If $b=0$, then the integral $\int\limits_0^1 \absval{u'(x)}^2 dx$ is minimized only by $u(x) =0$, and this is clear.

Consider the set of functions $\scrU_b$ that minimize the integral $\int\limits_0^1 \absval{u'(x)}^2 dx$ with respect to the set of functions $u \in C^1([0,1])$ such that $u(0)=0$ and $u(1)=b$. This set injects into the set of functions $\scrU_b - bx = \{u-bx: u \in \scrU\}$.

Now, functions in $\scrU_b -bx$ minimize the integral with respect to $u(0)=0$ and $u(1)=0$, and I can't figure out why. I think that a good proof of this would probably go through Normal Families somehow.

Either way, the point is that because $\scrU_b -bx = \{0\}$, the only function that minimizes the integral $\int\limits_0^1 \absval{u'(x)}^2 dx$ subject to $u(0)=0$ and $u(1) = b$ is $bx$. This yields the result desired.

\shunt

{\bf Problem 2:}

Consider the set $A=\{z \in \C: \text{Re}(z) >0\} \setminus [0,a]$ with $a \in \R^+$.

Define the sets $B = \C \setminus \{z \in \R: z \leq a^2\}$, and $C = \C \setminus \{z \in \R: z \leq 0\}$.

First, the map $\phi: A \to B$ given by $z \mapsto z^2$ is a biholomorphism from $A$ to $B$, and this is clear; the argument that $z \mapsto z^2$ gives a biholomorphism from the half-plane to the slit plane was made in class, and note that the image of $[0,a]$ under this map is $[0,a^2]$: so because the map is a biholomorphism, the half plane exluding $[0,a]$ has the image of the slit plane excluding $[0,a^2]$.

Second, the map $\psi: B \to C$ given by $z \mapsto z-a^2$ is a biholomorphism from $B$ to $C$, and this is clear (this is a straight translation).

Third, the map $\xi: C \to \{\text{Re}(z) >0\}$ given by $z \mapsto \sqrt{z}$ (using the branch of $\sqrt{z}$ that is the natural inverse of $z^2$, of course) is a biholomorphism from $C$ to $\{\text{Re}(z) >0\}$, and this was discussed in class.

So their composition is a biholomorphism from $A$ to $\{\text{Re}(z) >0\}$; that is, the map $f(z) = \sqrt{z^2-a^2}$ is a biholomorphism from the above set to $\{\text{Re}(z) >0\}$.

\shunt

{\bf Problem 3:}

Let $\Om$ be open and symmetric about the $\R$-axis.

Let $f \in C(\Om)$, and $f$ be holomorphic except perhaps on the $\R$-axis. Note that $f=0$ on the $\R$-axis.

Our goal is to show that $f \in \scrO(\Om)$; we only need to check that $f$ is holomorphic on the $\R$-axis. So, let $z \in \R \cap \Om$. Then there is an open ball centered at $z$, call it $D_r(z)$, contained in $\Om$. This open ball is simply connected. Now, the real part of $f$, say $u = \text{Re}(f)$, is harmonic on $D_r(z) \setminus \R$. By the reflection principle discussed in class, $u$ is harmonic on all of $D_r(z)$.

Now, $u$ is the real part of some holomorphic function, $g$, and this holomorphic function is unique up to addition of a constant. So, we can take $g(z) = 0$. %Now: show that f=g...

Now, $h=f-g$ is holomorphic except perhaps on the real axis, where it is $0$. Moreover, the real part of $h$ is $0$; by the Cauchy-Riemann equations, the imaginary part of $h$ must be constant (except perhaps on the real axis). Thus, because the imaginary part of $h$ is $0$ on the real axis (and $h$ is continuous), the imaginary part of $h$ is $0$. So, $h=0$; that is, $f=g$.

So, $f$ is holomorphic on $D_r(z)$; in particular, $f$ is holomorphic at $z$.

Because holomorphy is a local property, this yields the desired result; $f$ is holomorphic on $\Om$. 

\shunt

{\bf Problem 4:}

Let $\phi \in \text{Aut}(\overline{\C})$ be such that $\phi(\{z\in \C:\text{Im}(z)>0\}) = D_1(0)$. %Move everything to the half-plane (Im(z) >0).

One biholomorphism that takes the disk $D_1(0)$ to $\{z\in \C:\text{Im}(z)>0\}$ is $\phi_C: \C \to \C$ given by $z \mapsto i\frac{1+z}{1-z}$; this is the Cayley transform. Its inverse is $\psi_C: \C \to \C$ given by $z \mapsto \frac{z-i}{z+i}$. (I pulled these maps from Complex Made Simple; any other biholomorphism would've probably worked).

So, $\psi = \phi_C \circ \phi$ is a biholomorphism of the plane that fixes $\{z\in \C:\text{Im}(z)>0\}$; by an earlier homework problem, this means that $\phi_C \circ \phi$ is of the form $z \mapsto \frac{az+b}{cz+d}$ with $a,b,c,d \in \R$.

Fix $z \in \C$. Let $w = \overline{z}$. Then

\begin{align*}
\psi(w) &= \frac{aw+b}{cw+d}\\
&= \frac{a\overline{z}+b}{c\overline{z}+d}\\
&= \frac{\overline{az+b}}{\overline{cz+d}}\\
&= \overline{\psi(z)}\\
\end{align*}

Now, $\psi_C \circ \psi = \phi$. So,

\begin{align*}
\phi(w) &= \psi_C(\psi(w))\\
&= \psi_C(\overline{\psi(z)})\\
&= \frac{\overline{\psi(z)}-i}{\overline{\psi(z)}+i}\\
&= \frac{\overline{i\frac{1+\phi(z)}{1-\phi(z)}}-i}{\overline{i\frac{1+\phi(z)}{1-\phi(z)}}+i}\\
&= \frac{-i\overline{\frac{1+\phi(z)}{1-\phi(z)}}-i}{-i\overline{\frac{1+\phi(z)}{1-\phi(z)}}+i}\\
&= \frac{\overline{\frac{1+\phi(z)}{1-\phi(z)}}+1}{\overline{\frac{1+\phi(z)}{1-\phi(z)}}-1}\\
&= \frac{\frac{1+\overline{\phi(z)}}{1-\overline{\phi(z)}}+1}{\frac{1+\overline{\phi(z)}}{1-\overline{\phi(z)}}-1}\\
&= \frac{\frac{2}{1-\overline{\phi(z)}}}{\frac{2\overline{\phi(z)}}{1-\overline{\phi(z)}}}\\
&=\frac{1}{\overline{\phi(z)}}\\
&=\frac{\phi(z)}{\absval{\phi(z)}^2}\\
\end{align*}

which is the desired result.

\shunt

{\bf Problem 5:} %You can fix this by going to harmonic functions in your first step. The overarching idea is right, but that key point is borked.

Let $f \in \scrO(\Om)$, where $\Om$ is a symmetric domain (with respect to $\R$), and $\R \cap \Om \neq \emptyset$. Moreover, let $f(\R \cap \Om) \subset \R$.

Because $f$ is holomorphic on a domain, it can be written as $f=u+iv$, with $u$ and $v$ real-valued harmonic functions. 

So, $v$ is a harmonic function with $v=0$ on $\R \cap \Om$ (because $f(\R \cap \Om) \subset \R$). So $v(\overline{z}) = -v(z)$; if not, then $v$ restricted to $\Om^+$ extends to the harmonic function on $\Om$ that we get via reflection principle (call it $v'$), and then $v-v'$ vanishes on $\Om^+$...so $v-v'$ vanishes everywhere by minimum principle, so $v=v'$. 

So, $v(\overline{z}) = -v(z)$, for all $z \in \Om$.

Now, $u(\overline{z}) = u(z)$ by Cauchy Riemann: $u_y(\overline{z}) = -v_x(\overline{z}) = v_x(z) = -u_y(z)$. That is, $u_y$ is an odd function of $y$, so $u$ is an even function of $y$ (which is another way of saying the first sentence.)

Thus, $f(\overline{z}) = u(\overline{z}) + iv(\overline{z}) = u(z) -iv(z) = \overline{f(z)}$, as desired.

\shunt

{\bf Problem 6:}

Consider $\psi(z) = z + \frac{1}{z}$. Fix $a \in [0,1]$. Consider $U = D_1(0) \setminus ([-1,-a] \cup [a,1])$.

Then 

\begin{align*}
\psi(U) &= \psi(D_1(0)) \setminus \psi([-1,-a] \cup [a,1])\\
&= \C \setminus ([-2,2] \cup \psi([-1,-a] \cup [a,1]))\\
&= \C \setminus ([-2,2] \cup [-a-\frac{1}{a},-2]\cup [2,a+\frac{1}{a}])\\
&= \C \setminus [-a-\frac{1}{a},a+\frac{1}{a}]
\end{align*}

So we can dilate $\psi(U)$ to yield $\C \setminus [-1,1]$, by the map $\al$ given by $z \mapsto \frac{z}{a+\frac{1}{a}}$. That is, $\al \circ \psi (U) = \C \setminus [-1,1]$.

So by using the biholomorphism discussed in class, $\phi(z): \C \setminus [-1,1] \to D_1(0)$ given by $z \mapsto \sqrt{z^2-1} -z$, we have that $\phi \circ \al \circ \psi$ is a biholomorphism from $U$ to $D_1(0)$.

That is, the map $\be: U \to D_1(0)$ given by $z \mapsto \sqrt{(\frac{z + \frac{1}{z}}{a+\frac{1}{a}})^2 - 1} - \frac{z + \frac{1}{z}}{a+\frac{1}{a}}$ is a biholomorphism from $U$ to $D_1(0)$, as desired.

\shunt


\end{document}