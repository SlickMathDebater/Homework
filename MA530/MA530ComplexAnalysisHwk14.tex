
\documentclass[a4paper,12pt]{article}

\usepackage{fancyhdr}
\usepackage{amssymb}
%\usepackage{mathpazo}
\usepackage{mathtools}
\usepackage{amsmath}
\usepackage{slashed}
\usepackage{cancel}
\usepackage[mathscr]{euscript}
\usepackage{MaxPackage} %Note: You need MaxPackage installed or in the same folder as your .tex file or something.

\newcommand{\colorcomment}[2]{\textcolor{#1}{#2}} %First of these leaves in comments. Second one kills them.
%\newcommand{\colorcomment}[2]{}


\pagestyle{fancy}
\lhead{Max Jeter}
\chead{MA530}
\rhead{Assignment 14, Page \thepage}

%Number of Problems		:6
%Clear					:2,3
%Begun					:1
%Not started			:
%Can complete via book	:
%Needs Polish			:4,6

%Needs fixing			:5

%Pomodoros logged		:8

\begin{document}

{\bf Problem 1:} %Minimization of that integral is somehow invariant with respect to biholomorphism, and rotation of the plane is somehow a biholomorphism. Biholomorph everything to u(0)=u(1)=0 and you should get it. 

Consider $a, b \in \R$, and consider the set of functions $u \in C^1([0,1])$ such that $u(0)=a, u(1)=b$. Without loss of generality, we can take both $a$ and $b$ nonnegative. Call the set of such functions $\scrU$. %Why do we get that WLOG?

Let $u$ and $v$ both minimize the integral $\int\limits_0^1 \absval{f'(x)}^2 dx$ among functions in $\scrU$. Then $\min(u,v)$ minimizes the same integral. %Correct for signs...you're assuming that these things are all positive.
Moreover, $\min(u,v) < u$ or $\min(u,v) < v$ at some point if $u \neq v$. But if that were true at any point, then $u$ or $v$ would fail to minimize that integral; thus, $u = v$ at every point.

Next: the linear function minimizes the integral: let $l$ be the linear function with $l(0) = a$, $l(1)=b$, and let $u \in \scrU$ with $u \neq l$ minimize the integral. Also, define $M_1 = \int\limits_0^1 \absval{u'(x)}^2 dx$ and $M_2 = \int\limits_0^1 \absval{l'(x)}^2 dx$. Then $\int\limits_0^1 \absval{u'(x)-l'(x)}^2 dx \neq 0$; that is, $u-l$ fails to minimize the integral $\int\limits_0^1 \absval{v'(x)}^2 dx$ subject to $v(0)=v(1)=0$. %Explain why that makes this go bad.

Thus, the linear function is the unique function in $C^1([0,1])$ that minimizes the integral $\int\limits_0^1 \absval{f'(x)}^2 dx$.

\shunt

{\bf Problem 2:}

Consider the set $A=\{z \in \C: \text{Re}(z) >0\} \setminus [0,a]$ with $a \in \R^+$.

Define the sets $B = \C \setminus \{z \in \R: z \leq a^2\}$, and $C = \C \setminus \{z \in \R: z \leq 0\}$.

First, the map $\phi: A \to B$ given by $z \mapsto z^2$ is a biholomorphism from $A$ to $B$, and this is clear; the argument that $z \mapsto z^2$ gives a biholomorphism from the half-plane to the slit plane was made in class, and note that the image of $[0,a]$ under this map is $[0,a^2]$: so because the map is a biholomorphism, the half plane exluding $[0,a]$ has the image of the slit plane excluding $[0,a^2]$.

Second, the map $\psi: B \to C$ given by $z \mapsto z-a^2$ is a biholomorphism from $B$ to $C$, and this is clear (this is a straight translation).

Third, the map $\xi: C \to \{\text{Re}(z) >0\}$ given by $z \mapsto \sqrt{z}$ (using the branch of $\sqrt{z}$ that is the natural inverse of $z^2$, of course) is a biholomorphism from $C$ to $\{\text{Re}(z) >0\}$, and this was discussed in class.

So their composition is a biholomorphism from $A$ to $\{\text{Re}(z) >0\}$; that is, the map $f(z) = \sqrt{z^2-a^2}$ is a biholomorphism from the above set to $\{\text{Re}(z) >0\}$.

\shunt

{\bf Problem 3:}

Let $\Om$ be open and symmetric about the $\R$-axis.

Let $f \in C(\Om)$, and $f$ be holomorphic except perhaps on the $\R$-axis. Note that $f=0$ on the $\R$-axis.

Our goal is to show that $f \in \scrO(\Om)$; we only need to check that $f$ is holomorphic on the $\R$-axis. So, let $z \in \R \cap \Om$. Then there is an open ball centered at $z$, call it $D_r(z)$, contained in $\Om$. This open ball is simply connected. Now, the real part of $f$, say $u = \text{Re}(f)$, is harmonic on $D_r(z) \setminus \R$. By the reflection principle discussed in class, $u$ is harmonic on all of $D_r(z)$.

Now, $u$ is the real part of some holomorphic function, $g$, and this holomorphic function is unique up to addition of a constant. So, we can take $g(z) = 0$. %Now: show that f=g...

Now, $h=f-g$ is holomorphic except perhaps on the real axis, where it is $0$. Moreover, the real part of $h$ is $0$; by the Cauchy-Riemann equations, the imaginary part of $h$ must be constant (except perhaps on the real axis). Thus, because the imaginary part of $h$ is $0$ on the real axis (and $h$ is continuous), the imaginary part of $h$ is $0$. So, $h=0$; that is, $f=g$.

So, $f$ is holomorphic on $D_r(z)$; in particular, $f$ is holomorphic at $z$.

Because holomorphy is a local property, this yields the desired result; $f$ is holomorphic on $\Om$. 

\shunt

{\bf Problem 4:}

Let $\phi \in \text{Aut}(\overline{\C})$ be such that $\phi(\{z\in \C:\text{Im}(z)>0\}) = D_1(0)$. %Move everything to the half-plane (Im(z) >0).

A biholomorphism that takes the disk $D_1(0)$ to $\{z\in \C:\text{Im}(z)>0\}$ is $\phi_C: \C \to \C$ given by $z \mapsto i\frac{1+z}{1-z}$; this is the Cayley transform. Its inverse is $\psi_C: \C \to \C$ given by $z \mapsto \frac{z-i}{z+i}$. (I pulled these maps from Complex Made Simple; any other such map would've probably worked).

So, $\psi = \phi_C \circ \phi$ is a biholomorphism of the plane that fixes $\{z\in \C:\text{Im}(z)>0\}$; by an earlier homework problem, this means that $\phi_C \circ \phi$ is of the form $z \mapsto \frac{az+b}{cz+d}$ with $a,b,c,d \in \R$, as proven in an earlier homework problem.

Fix $z \in \C$. Let $w = \overline{z}$. Then

\begin{align*}
\psi(w) &= \frac{aw+b}{cw+d}\\
&= \frac{a\overline{z}+b}{c\overline{z}+d}\\
&= \frac{\overline{az+b}}{\overline{cz+d}}\\
&= \overline{\psi(z)}\\
\end{align*}

Now, $\psi_C \circ \psi = \phi$. So,

\begin{align*}
\phi(w) &= \psi_C(\psi(w))\\
&= \psi_C(\overline{\psi(z)})\\
&= \frac{\overline{\psi(z)}-i}{\overline{\psi(z)}+i}\\
&= \frac{\overline{i\frac{1+\phi(z)}{1-\phi(z)}}-i}{\overline{i\frac{1+\phi(z)}{1-\phi(z)}}+i}\\
&= \frac{-i\overline{\frac{1+\phi(z)}{1-\phi(z)}}-i}{-i\overline{\frac{1+\phi(z)}{1-\phi(z)}}+i}\\
&= \frac{\overline{\frac{1+\phi(z)}{1-\phi(z)}}+1}{\overline{\frac{1+\phi(z)}{1-\phi(z)}}-1}\\
&= \frac{\frac{1+\overline{\phi(z)}}{1-\overline{\phi(z)}}+1}{\frac{1+\overline{\phi(z)}}{1-\overline{\phi(z)}}-1}\\
&= \frac{\frac{2}{1-\overline{\phi(z)}}}{\frac{2\overline{\phi(z)}}{1-\overline{\phi(z)}}}\\
&=\frac{1}{\overline{\phi(z)}}\\
&=\frac{\phi(z)}{\absval{\phi(z)}^2}\\
\end{align*}

which is the desired result.

\shunt

{\bf Problem 5:}

Let $f \in \scrO(\Om)$, where $\Om$ is a symmetric domain (with respect to $\R$), and $\R \cap \Om \neq \emptyset$. Moreover, let $f(\R \cap \Om) \subset \R$. Then the function $g(z)=f(Re(z))$ is a holomorphic function. %Is that true?...No, it's absolute bullshit. That'd mean (in general) that we have two different holomorphic functions taking the same values on a line segment. However, it is clear that this IDEA more or less works....it just needs a major overhaul in the first half.

Now, consider $h=f-g$; this is holomorphic. Note that $h$ restricted to $A=\Om^+ \cup (\Om \cap \R)$ satisfies the requirements for the reflection principle; $h$ restricted to $A$ extends to $\Om$; call this extension $j$. (Note that $j(\overline{z}) = \overline{j(z)}$, because of the construction of the extension.) Now, $j-h$ is identically $0$ on $A$. Because $j-h$ is $0$ on an open subset of $\Om$, it is $0$ on all of $\Om$ (This follows by uniqueness principle, as $\Om$ is a domain...it is connected.)

So $j=h$. So $h(\overline{z}) = \overline{h(z)}$. So 

\begin{align*}
f(\overline{z}) &= h(\overline{z}) - g(\overline{z})\\
&= \overline{h(z)} - g(z)\\
&= \overline{h(z)} - \overline{g(z)}\\
&= \overline{f(z)}
\end{align*}

as desired.

\shunt

{\bf Problem 6:}

Consider $\psi(z) = z + \frac{1}{z}$. Fix $a \in [0,1]$. Consider $U = D_1(0) \setminus ([-1,-a] \cup [a,1])$.

Then 

\begin{align*}
\psi(U) &= \psi(D_1(0)) \setminus \psi([-1,-a] \cup [a,1])\\
&= \C \setminus ([-2,2] \cup \psi([-1,-a] \cup [a,1]))\\
&= \C \setminus ([-2,2] \cup [-a-\frac{1}{a},-2]\cup [2,a+\frac{1}{a}])\\
&= \C \setminus [-a-\frac{1}{a},a+\frac{1}{a}]
\end{align*}

So we can dilate $\psi(U)$ to yield $\C \setminus [-1,1]$, by the map $\al$ given by $z \mapsto \frac{z}{a+\frac{1}{a}}$. That is, $\al \circ \psi (U) = \C \setminus [-1,1]$.

So by using the map discussed in class, $\phi(z): \C \setminus [-1,1]$ given by $z \mapsto \sqrt{z^2-1} -z$, we have that $\phi \circ \al \circ \psi$ is a biholomorphism from $U$ to $D_1(0)$.

That is, the map $\be: U \to D_1(0)$ given by $z \mapsto \sqrt{(\frac{z + \frac{1}{z}}{a+\frac{1}{a}})^2 - 1} - \frac{z + \frac{1}{z}}{a+\frac{1}{a}}$ is a biholomorphism from $U$ to $D_1(0)$, as desired.

\shunt


\end{document}