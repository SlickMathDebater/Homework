
\documentclass[a4paper,12pt]{article}

\usepackage{fancyhdr}
\usepackage{amssymb}
\usepackage{esvect}
%\usepackage{mathpazo}
\usepackage{mathtools}
\usepackage{amsmath}
\usepackage{slashed}
\usepackage{cancel}
\usepackage[mathscr]{euscript}
\usepackage{MaxPackage} %Note: You need MaxPackage installed or in the same folder as your .tex file or something.

\newcommand{\colorcomment}[2]{\textcolor{#1}{#2}} %First of these leaves in comments. Second one kills them.
%\newcommand{\colorcomment}[2]{}


\pagestyle{fancy}
\lhead{Max Jeter}
\chead{MA530}
\rhead{Assignment 2, Page \thepage}

\begin{document}

{\bf Problem 1:}

(Prove: The real 2x2 matrix blah represents a complex-linear map iff a=d, c=-b)

Let the real matrix $A = \left[\begin{smallmatrix}a&b\\ c&d \end{smallmatrix}\right]$ represent a complex-linear map. Then we have:

\begin{align*}
i\left[\begin{smallmatrix}a&b\\ c&d \end{smallmatrix}\right]\left[\begin{smallmatrix}0\\ 1 \end{smallmatrix}\right] &= i\left[\begin{smallmatrix}b\\ d \end{smallmatrix}\right]\\
&= \left[\begin{smallmatrix}-d\\ b \end{smallmatrix}\right]
\end{align*}

And also

\begin{align*}
i\left[\begin{smallmatrix}a&b\\ c&d \end{smallmatrix}\right]\left[\begin{smallmatrix}0\\ 1 \end{smallmatrix}\right] &= \left[\begin{smallmatrix}a&b\\c&d\end{smallmatrix}\right]\left[\begin{smallmatrix}-1\\ 0 \end{smallmatrix}\right]\\
&= \left[\begin{smallmatrix}-a\\ -c \end{smallmatrix}\right]
\end{align*}

So $\left[\begin{smallmatrix}-d\\ b \end{smallmatrix}\right] = \left[\begin{smallmatrix}-a\\ -c \end{smallmatrix}\right]$; that is, $a=d$ and $b=-c$.

Now, let the real matrix $A$ have that $a=d$ and $c=-b$. Write $A=\left[\begin{smallmatrix}a&b\\ -b&a \end{smallmatrix}\right]$, and let $T$ be the linear map associated with $A$. Also, let $\vv{z}$ be the vector associated with $z$, for any $z \in \C$.

Let $z=x+iy, w=x'+iy' \in \C$. Then:

\begin{align*}
T(wz) &= A\vv{wz} \\
&=\left[\begin{smallmatrix}a&b\\ -b&a \end{smallmatrix}\right]\left[\begin{smallmatrix}xx'-yy'\\ xy'+yx' \end{smallmatrix}\right]\\
&=\left[\begin{smallmatrix}a(xx'-yy')+b(xy'+yx')\\ a(xy'+yx')-b(xx'-yy') \end{smallmatrix}\right]\\
&=\left[\begin{smallmatrix}x'\\y' \end{smallmatrix}\right] \times \left[\begin{smallmatrix}ax+by\\ ax-by \end{smallmatrix}\right]\\
&= \vv{w} \times T(\vv{z})
\end{align*}

(with $\times$ being complex multiplication).

That is, $A$'s associated linear map is $\C$-linear.

So, we have that the real matrix $A= \left[\begin{smallmatrix}a&b\\ c&d \end{smallmatrix}\right]$ represents a complex-linear map if and only if $a=d$ and $b=-c$.

\shunt

{\bf Problem 2:}

Consider $\int\limits_{\absval{z} = R} \overline{z}^n dz$. Define $\al: [0,2\pi] \to \C$ by $\al(t) = R(\cos(t)+i\sin(t))$.

If $n \neq 1$, we have

\begin{align*}
\int\limits_{\absval{z} = R} \overline{z}^n dz &= \int\limits_\al \overline{z}^n dz\\
&= \int\limits_0^{2\pi} [R(\cos(t) - i \sin(t))]^n [R(-\sin(t)+i\cos(t))] dt \\
&=R^{n+1}\int\limits_0^{2\pi} \frac{-\sin(t)+i\cos(t)}{(\cos(t)+i\sin(t))^n}dt\\
&= R^{n+1} \frac{1}{n-1} [((\cos(2\pi)+i\sin(2\pi)))^{1-n} - (\cos(0)+i\sin(0))^{1-n}\\
&= R^{n+1} \frac{1}{n-1} [1-1]\\
&= 0
\end{align*}

(Here, we're using freely the fact that $\overline{z} = 1/z$ if $\absval{z}=1$, and we gloss over the $u$-substitution with $u = \cos(t)+i\sin(t)$.)

And if $n=1$, we have

\begin{align*}
\int\limits_{\absval{z} = R} \overline{z}^n dz &= \int\limits_\al \overline{z} dz\\
&= \int\limits_0^{2\pi} [R(\cos(t) - i \sin(t))] [R(-\sin(t)+i\cos(t))] dt \\
&=R^{2}\int\limits_0^{2\pi} \frac{-\sin(t)+i\cos(t)}{\cos(t)+i\sin(t)}dt\\
&= R^2\int\limits_0^{2\pi} i\frac{\cos(t)+i\sin(t)}{\cos(t)+i\sin(t)}dt\\
&= R^2\int\limits_0^{2\pi} idt\\
&= R^22\pi i 
\end{align*}

\shunt

{\bf Problem 3:}

Let $D \subset \C$ be open, and let $D^*$ be $D$'s reflection about the $x$-axis. Let $f \in \scrO (D)$, and define $g(z) = \overline{f(\overline{z})}$. As an intermediate step, define $h(z) = f(\overline{z})$.

Let $f=u+iv$. Then $h((x,y))=f((x,-y))$; say that $h=u`-iv`$. By definition, $g=u`+iv`$. %the h approach is bad; explicitly write u` in terms of u and v...

Also, $u_x(z)=v_y(z)$ for all $z \in \C$. Thus, $u`_x(z) = $.%Bash it over head with Cauchy-Riemann

Moreover, $u_y(z)=-v_x(z)$ for all $z \in \C$. So %Same tactic

\shunt

(I remember someone saying something about more problems but I must be missing one...:/)

\end{document}