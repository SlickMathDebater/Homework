
\documentclass[a4paper,12pt]{article}

\usepackage{fancyhdr}
\usepackage{amssymb}
\usepackage{esvect}
%\usepackage{mathpazo}
\usepackage{mathtools}
\usepackage{amsmath}
\usepackage{slashed}
\usepackage{cancel}
\usepackage[mathscr]{euscript}
\usepackage{MaxPackage} %Note: You need MaxPackage installed or in the same folder as your .tex file or something.

\newcommand{\colorcomment}[2]{\textcolor{#1}{#2}} %First of these leaves in comments. Second one kills them.
%\newcommand{\colorcomment}[2]{}


\pagestyle{fancy}
\lhead{Max Jeter}
\chead{MA530}
\rhead{Assignment 3, Page \thepage}

\begin{document}

{\bf Problem 1:}

Let $f \in \scrO(\C)$. Then consider $\int\limits_{\absval{z} = 2} \frac{f(z)}{z-1} dz$.

Note that $\{z: \absval{z}=2\}$ is the boundary of the open disc of radius $2$, and that $1$ is a point in this disc. Thus, Cauchy's formula applies; $f(1) = 1/(2\pi i)  \int\limits_{\absval{z} = 2} \frac{f(z)}{z-1} dz$, so $\int\limits_{\absval{z} = 2} \frac{f(z)}{z-1} dz = 2 \pi i f(1)$.

\shunt

{\bf Problem 2:}

Let $f \in \scrO(\C)$. Then consider $\int\limits_{\absval{z} = 2} \frac{f(z)}{z^2-1} dz$.

Note that $\int\limits_{\absval{z} = 2} \frac{f(z)}{z^2-1} dz = \int\limits_{\absval{z} = 2} \frac{f(z)}{(z+1)(z-1)} dz$. Now, this function is holomorphic except at $1$ and $-1$. Thus, Cauchy's Theorem applies: the integral of $\frac{f(z)}{z^2-1}$ over a closed loop not containing $1$ or $-1$ is zero. Thus:

\begin{align*}
\int\limits_{\absval{z} = 2} \frac{f(z)}{(z+1)(z-1)} dz &= \int\limits_{\absval{z-1} = 1} \frac{f(z)}{(z+1)(z-1)} dz + \int\limits_{\absval{z+1} = 1} \frac{f(z)}{(z+1)(z-1)} dz
\end{align*}

(This becomes clear given the following picture. Moreover, I freely use this sort of trick in future problems:)

\shunt

Now, $f/(z+1)$ is holomorphic except at $-1$, and $f/(z-1)$ is holomorphic except at $1$. So, we can apply Cauchy's Formula to the two integrals;

\begin{align*}
\int\limits_{\absval{z} = 2} \frac{f(z)}{(z+1)(z-1)} dz &= \int\limits_{\absval{z-1} = 1} \frac{f(z)}{(z+1)(z-1)} dz + \int\limits_{\absval{z+1} = 1} \frac{f(z)}{(z+1)(z-1)} dz\\
&= 2\pi i f(1)/(1+1) + 2\pi i f(-1)/(-1-1)\\
&= \pi i (f(1)-f(-1))
\end{align*}

\shunt

{\bf Problem 3:}

If $\Om$ is an open set, $f$ is holomorphic on some open set containing $\Om$'s closure, and $w \notin \Om$, then $\frac{f(z)}{z-w}$ is a product of two holomorphic functions and is thus holomorphic, so $ \int\limits_{\partial \Om} \frac{f(z)}{z-w} dz$ vanishes by Cauchy's Theorem.

\shunt

{\bf Problem 4:}

Let $f$ be a holomorphic function on some open set, $\Om$.

Let $c \in \Om$. Then 

\begin{align*}
\frac{df}{dz}(c) &= \frac{1}{2}(\frac{df}{dx}(c)-i\frac{df}{dy}(c))\\
\frac{d}{d\overline{z}}\frac{df}{dz}(c) &= \frac{1}{2} \frac{d}{dx}[\frac{1}{2}(\frac{df}{dx}(c)-i\frac{df}{dy}(c))] +i\frac{d}{dy}[\frac{1}{2}(\frac{df}{dx}(c)-i\frac{df}{dy}(c))]\\
&=\frac{1}{4}[\frac{d^2}{dx^2}(c) -i\frac{d^2}{dxdy}(c) +i\frac{d^2}{dxdy}(c) + \frac{d^2}{dy^2}(c)]\\
&=\frac{1}{4}\De f
\end{align*}

To summarize, $\frac{d^2f}{dzd\overline{z}} = \frac{1}{4} \De f$.

(Note that we have freely used the symmetry of the second partial derivatives here.)

\shunt

{\bf Problem 5:} %Redo this with Cauchy's Formula...It comes out way cleaner.

Consider $\int\limits_{\absval{z} = 2} z^n(z-1)^m dz$ with $n, m \in \Z$.

First, if $n \geq 0$ and $m \geq 0$, $z^n(z-1)^m$ is holomorphic; the integral is $0$.

\vspace{5mm}

Next, if $n \geq 0$ and $m < 0$, then $f(z)=z^n$ is holomorphic. So, by Cauchy's Formula, 

\begin{align*}
f^{-(m+1)}(1) &= \frac{(-(m+1))!}{2\pi i} \int\limits_{\absval{z}=2} \frac{f(z)}{(z-1)^{-(m+1)+1}} dz\\
&= \frac{(-(m+1))!}{2\pi i} \int\limits_{\absval{z}=2} z^n(z-1)^m dz\\
\frac{f^{-(m+1)}(1)}{(-(m+1))!} 2\pi i &= \int\limits_{\absval{z}=2} z^n(z-1)^m dz\\
\frac{n!}{(-(m+1))!(n+m+1)!} 2\pi i &= \int\limits_{\absval{z}=2} z^n(z-1)^m dz\\
\end{align*}

\vspace{5mm}

Next, if $n < 0$ and $m \geq 0$, then $f(z)=(z-1)^m$ is holomorphic. So, by Cauchy's Formula, 

\begin{align*}
f^{-(n+1)}(0) &= \frac{(-(n+1))!}{2\pi i} \int\limits_{\absval{z}=2} \frac{f(z)}{z^{-(n+1)+1}} dz\\
&= \frac{(-(n+1))!}{2\pi i} \int\limits_{\absval{z}=2} z^n(z-1)^m dz\\
\frac{f^{-(n+1)}(0)}{(-(n+1))!} 2\pi i &= \int\limits_{\absval{z}=2} z^n(z-1)^m dz\\
\frac{(-1)^{m-(n+1)}m!}{(-(n+1))!(n+m+1)!} 2\pi i &= \int\limits_{\absval{z}=2} z^n(z-1)^m dz\\
\end{align*}
\vspace{5mm}

Last, if $n < 0$ and $m < 0$, then $z^n(z-1)^m$ is holomorphic except at $0$ and $1$. So, $\int\limits_{\absval{z}=2} z^n(z-1)^m dz = \int\limits_{\absval{z}=1/2} z^n(z-1)^m dz + \int\limits_{\absval{z-1}=1/2} z^n(z-1)^m dz = \int\limits_{\absval{z}=1/2} z^n(z-1)^m dz + \int\limits_{\absval{z}=1/2} (z+1)^nz^m dz$.

By this and Cauchy's Formula, we have, by letting $f(z)=(z+1)^n$ and $g(z) = (z-1)^m$, %something's wrong here...

\begin{align*}
\int\limits_{\absval{z}=2} z^n(z-1)^m dz  &= \int\limits_{\absval{z}=1/2} z^n(z-1)^m dz + \int\limits_{\absval{z}=1/2} (z+1)^nz^m dz \\
&= \int\limits_{\absval{z}=1/2} \frac{g(z)}{z^{-n}} dz + \int\limits_{\absval{z}=1/2} \frac{f(z)}{z^{-m}} dz \\
&= 2 \pi i \left[\frac{g^{(-(n+1))}(0)}{(-(n+1))!} + \frac{f^{(-(m+1))}(0)}{(-(m+1))!}\right]\\
&= 2 \pi i \left[\frac{(-1)^m \frac{(-(m+n+1))!}{(-(m+1))!}}{(-(n+1))!} + \frac{(-1)^{-(n+1)} \frac{(-(m+n+1))!}{(-(n+1))!}}{(-(m+1))!}\right]
\end{align*}

So in each case, we have a formula for the integral; the problem is satisfied.

\shunt

{\bf Problem 6:}

Let $g(z) = \overline{z}$ for all $z: \absval{z} = 1$.

Consider 

\begin{align*}
\int\limits_{\absval{z}=1} g(z)dz &= \int\limits_{\absval{z}=1} \overline{z}dz\\
&= \int\limits_0^{2\pi} (\cos(t)-i\sin(t))(-\sin(t)+i\cos(t))dt\\
&= \int\limits_0^{2\pi} \sin(t)\cos(t)-\sin(t)\cos(t) + i[\sin^2(t)+\cos^2(t)]dt\\
&= \int\limits_0^{2\pi} idt\\
&=2\pi i
\end{align*}

That is, Cauchy's Theorem would fail if $g$ was holomorphic; $g$ cannot be holomorphic on any open disk containing the set $\{z: \absval{z} = 1\}$, let alone $\C$!

\shunt

{\bf Problem 7:}

Let $f \in \scrO( \C)$, with $\absval{f(z)} \leq A + B\absval{z}^n$ for some fixed $A, B \in \R$, $n \in \N$, and for all $z \in \C$.

Then for all $z \in \C$, $r>0 \in \R$, $k \in \N$, we have $\absval{f^{(k)}(z)} \leq k! \frac{\sup(f(B(z,r)))}{r^k}$.

So $\absval{f^{(k)}(z)} \leq k! \frac{A+B(\absval{z}+r)^n}{r^k}$.

So $\absval{f^{(n+1)}(z)} \leq (n+1)! \frac{A+B(\absval{z}+r)^n}{r^{n+1}}$.

By taking a limit as $r \to \infty$, we see that $\absval{f^{(n+1)}(z)}=0$ for all $z \in \C$. That is, the $n+1$th derivative of $f$ is identically $0$; $f$ is a polynomial of degree at most $n+1$.

\shunt

{\bf Problem 8:}

Let $S \subset \C$ be an arbitrary set, $U \subset \C$ be open, and $K \in C(S \times U)$ be such that for all $s \in S$, $f_s(w) = K(s,w)$ is holomorphic on $U$.

Then $\frac{\partial K(s,w)}{\partial w} = f_s'(w)$. Because $f_s$ is holomorphic on $U$, $f_s'(w)$ is continuous (if this isn't clear, consider that Cauchy's Formula implies that $f$ is infinitely differentiable...and thus, each derivative is differentiable, and thus continuous). So, $\frac{\partial K(s,w)}{\partial w}$ is continuous for fixed $s$.

Next, let $s_n \to s$, with each $s_n, s \in S$. Then $f_{s_n}$ converges to $f_s$ uniformly; %reasons.
By Weierstrauss' theorem, this means that $f'_{s_n}$ converges to $f'(s)$ uniformly. That is, $\frac{\partial K(s,w)}{\partial w}$ is continuous for fixed $w$.

So $\frac{\partial K(s,w)}{\partial w}$ is continuous with either variable fixed; it is continuous.

\shunt

\end{document}