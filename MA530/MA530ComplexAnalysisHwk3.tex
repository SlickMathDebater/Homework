
\documentclass[a4paper,12pt]{article}

\usepackage{fancyhdr}
\usepackage{amssymb}
\usepackage{esvect}
%\usepackage{mathpazo}
\usepackage{mathtools}
\usepackage{amsmath}
\usepackage{slashed}
\usepackage{cancel}
\usepackage[mathscr]{euscript}
\usepackage{MaxPackage} %Note: You need MaxPackage installed or in the same folder as your .tex file or something.

\newcommand{\colorcomment}[2]{\textcolor{#1}{#2}} %First of these leaves in comments. Second one kills them.
%\newcommand{\colorcomment}[2]{}


\pagestyle{fancy}
\lhead{Max Jeter}
\chead{MA530}
\rhead{Assignment 3, Page \thepage}

\begin{document}

{\bf Problem 1:}

Let $f \in \scrO(\C)$. Then consider $\int\limits_{\absval{z} = 2} \frac{f(z)}{z-1} dz$.

Note that $\{z: \absval{z}=2\}$ is the boundary of the open disc of radius $2$, and that $1$ is a point in this disc. Thus, Cauchy's formula applies; $f(1) = 1/(2\pi i)  \int\limits_{\absval{z} = 2} \frac{f(z)}{z-1} dz$, so $\int\limits_{\absval{z} = 2} \frac{f(z)}{z-1} dz = 2 \pi i f(1)$.

\shunt

{\bf Problem 2:}

Let $f \in \scrO(\C)$. Then consider $\int\limits_{\absval{z} = 2} \frac{f(z)}{z^2-1} dz$.

Note that $\int\limits_{\absval{z} = 2} \frac{f(z)}{z^2-1} dz = \int\limits_{\absval{z} = 2} \frac{f(z)}{(z+1)(z-1)} dz$. Now, the function $f/(z+1)$ is holomorphic except at $-1$, and the function $f/(z-1)$ is holomorphic except at $1$. So, we %sever the ball in half, and work from there.

\shunt

{\bf Problem 3:}

If $\Om$ is an open set, $f$ is holomorphic on some open set containing $\Om$'s closure, and $w \notin \Om$, then $ \int\limits_{\partial \Om} \frac{f(z)}{z-w} dz$ vanishes; $\frac{f(z)}{z-w}$ is a product of two holomorphic functions and is thus holomorphic, so the integral vanishes by the theorem we use to prove Cauchy's Formula.

\shunt

{\bf Problem 4:}

%This looks like a bitch and I don't want to handle it.

\shunt

\end{document}