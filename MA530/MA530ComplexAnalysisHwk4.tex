
\documentclass[a4paper,12pt]{article}

\usepackage{fancyhdr}
\usepackage{amssymb}
%\usepackage{mathpazo}
\usepackage{mathtools}
\usepackage{amsmath}
\usepackage{slashed}
\usepackage{cancel}
\usepackage[mathscr]{euscript}
\usepackage{MaxPackage} %Note: You need MaxPackage installed or in the same folder as your .tex file or something.

\newcommand{\colorcomment}[2]{\textcolor{#1}{#2}} %First of these leaves in comments. Second one kills them.
%\newcommand{\colorcomment}[2]{}


\pagestyle{fancy}
\lhead{Max Jeter}
\chead{MA530}
\rhead{Assignment 4, Page \thepage}

\begin{document}

{\bf Problem 1:}

Consider $\sum\limits_{n=0}^\infty (-1)^n \frac{z^{2n+1}}{(2n+1)!}$ and $\sum\limits_{n=0}^\infty (-1)^n \frac{z^{2n}}{(2n)!}$. Define $s_n = (-1)^n \frac{z^{2n+1}}{(2n+1)!}$. 

Fix $z \in \C$. Then:

\begin{align*}
\absval{\frac{s_{n+1}}{s_n}} &= \absval{\frac{(-1)^{n+1} \frac{z^{2n+3}}{(2n+3)!}}{(-1)^n \frac{z^{2n+1}}{(2n+1)!}}}\\
&= \absval{\frac{(-1)z^{2}}{(2n+3)(2n+2)}}\\
&= \absval{\frac{z^2}{4n^2+10n+6}}\\
\end{align*}

And it is clear that as $n \to \infty$, $\absval{\frac{z^2}{4n^2+10n+6}} \to 0$ for all $z \in \C$. So by the ratio test, $\sum\limits_{n=0}^\infty (-1)^n \frac{z^{2n+1}}{(2n+1)!}$ absolutely converges (and thus, converges) for all $z \in \C$. That is, $\sum\limits_{n=0}^\infty (-1)^n \frac{z^{2n+1}}{(2n+1)!}$ has infinite radius of convergence.

Similarly (by defining $t_n = (-1)^n \frac{z^{2n}}{(2n)!}$ and applying the ratio test), $\sum\limits_{n=0}^\infty (-1)^n \frac{z^{2n}}{(2n)!}$ has infinite radius of convergence.

Typically, we define $\sin(z) = \sum\limits_{n=0}^\infty (-1)^n \frac{z^{2n+1}}{(2n+1)!}$ and $\cos(z) = \sum\limits_{n=0}^\infty (-1)^n \frac{z^{2n}}{(2n)!}$

\shunt

{\bf Problem 2:}

Consider $\sum\limits_{k=1}^\infty \frac{z^k}{1-z^k}$.

Define $s_k = \frac{z^k}{1-z^k}$. Fix $z \in D_1(0)$. Then:

\begin{align*}
\absval{\frac{s_{k+1}}{s_k}} &= \absval{\frac{\frac{z^{k+1}}{1-z^{k+1}}}{\frac{z^k}{1-z^k}}}\\
&= \absval{z\frac{1-z^k}{1-z^{k+1}}}\\
&\leq \absval{z}\absval{\frac{1-z^k}{1-z^{k+1}}} \text{     (By Cauchy-Schwarz)}\\
&\leq \absval{z} \text{     (Because } \absval{z} < 1\text{ and } \absval{z^k} \text{ is a decreasing sequence when } \absval{z} < 1)\\
&< 1
\end{align*}

So, by the ratio test, $\sum\limits_{k=1}^\infty \frac{z^k}{1-z^k}$ converges on the unit disk. 

Moreover, that sum converges uniformly on compact subsets of $D_1(0)$; %So the sum is holomorphic at each point by Weierstrauss, so it's holomorphic on the whole disk.

\shunt

{\bf Problem 3:}

Consider $e^{\overline{z}}$, with $z \in \C$.

\begin{align*}
e^{\overline{z}} &= \sum \frac{\overline{z}^n}{n!} \\
&= \sum \frac{\overline{z^n}}{n!}\\
&= \sum \overline{\frac{z^n}{n!}}\\
&= \overline{\sum\frac{z^n}{n!}}\\
&= \overline{e^z}
\end{align*}

So $e^{\overline{z}} = \overline{e^z}$, for all $z \in \C$.

\shunt

{\bf Problem 4:}

(In the following, $\dotplus$ is used for curve concatenation.)

Let $\Om \subset \C$ be open and connected. Fix $w \in \Om$, define $\Om_1$ to be the set of all points that can be joined to $w$ by a curve. Define $\Om_2$ to be the set of all points that cannot.

Then it is clear that $\Om=\Om_1 \cup \Om_2$, that $\Om_1 \cap \Om_2 = \emptyset$, and that $w \in \Om_1$.

Now, $\Om_1$ is open:

\tab Let $x \in \Om_1$. There is an $\ep>0$ such that $D_\ep(x) \subset \Om$. There is a curve, $\Ga$, connecting $w$ to $x$. For all $y \in D_\ep(x)$, the line segment $[x,y] \subset D_\ep(x) \subset \Om$ (because disks are convex); we can parameterize a curve $\phi$ whose image is $[x,y]$. It is clear that $\Ga \dotplus \phi$ is a curve connecting $w$ and $y$. So for all $y \in D_\ep(x)$, $y \in \Om_1$; $D_\ep(x) \subset \Om_1$.

\tab So for all $x \in \Om_1$, there's an open disk that contains $x$ that is a subset of $\Om_1$. So $\Om_1$ is open.

Also, $\Om_2$ is open: 

\tab Let $x \in \Om_2$. There is an $\ep>0$ such that $D_\ep(x) \subset \Om$. There is no curve connecting $w$ to $x$. For all $y \in D_\ep(x)$, the line segment $[y,x] \subset D_\ep(x) \subset \Om$ (because disks are convex); we can parameterize a curve $\phi$ whose image is $[y,x]$. 

\tab If for any $y \in D_\ep(x)$ there is a curve, $\Ga$, connecting $y$ and $w$, then $\Ga \dotplus \phi$ is a curve connecting $w$ and $x$. This would contradict the fact that $x \in \Om_2$. So for all $y \in D_\ep(x)$, $y \in \Om_2$; $D_\ep(x) \subset \Om_2$.

\tab So for all $x \in \Om_2$, there's an open disk that contains $x$ that is a subset of $\Om_2$. So $\Om_2$ is open.

So if $\Om_2$ was nonempty, we would have two nonempty open sets whose union was $\Om$. That is, we would be able to disconnect $\Om$, which is contrary to our assumption.

Thus, $\Om_2$ is empty, so $\Om_1=\Om$. That is, an open, connected subset of $\C$ is path-connected.

\shunt

\end{document}