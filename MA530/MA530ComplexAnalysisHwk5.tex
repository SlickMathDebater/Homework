
\documentclass[a4paper,12pt]{article}

\usepackage{fancyhdr}
\usepackage{amssymb}
%\usepackage{mathpazo}
\usepackage{mathtools}
\usepackage{amsmath}
\usepackage{slashed}
\usepackage{cancel}
\usepackage[mathscr]{euscript}
\usepackage{MaxPackage} %Note: You need MaxPackage installed or in the same folder as your .tex file or something.

\newcommand{\colorcomment}[2]{\textcolor{#1}{#2}} %First of these leaves in comments. Second one kills them.
%\newcommand{\colorcomment}[2]{}


\pagestyle{fancy}
\lhead{Max Jeter}
\chead{MA530}
\rhead{Assignment 5, Page \thepage}

\begin{document}

{\bf Problem 1:}

Part a:

Consider the set $A= \{z \in \C: e^z = 0\}$.

If $z=a+bi \in A$ (with $a, b \in \R$), then $e^z =0$. So $e^ae^{bi} = 0$. 

For $a \in \R$, $e^a \neq 0$. So this means that $e^{bi} = 0$. But this never happens either, because $\absval{e^{bi}} = 1$ for all $b \in \R$ (because $\absval{e^{bi}} = \absval{\cos(b)+i\sin(b)} = \sqrt{\cos^2(b) + \sin^2(b)} = 1$).

So we have a contradiction. So $A = \emptyset$.

\shunt

Part b:

Consider the set $B= \{z \in \C: e^z = 1\}$.

If $z=a+bi \in B$ (with $a,b \in \R$), then $e^z = 1$. So $e^z = e^ae^{bi} = 1$.

This means that $\absval{e^ae^{bi}} = \absval{e^a}\absval{e^{bi}}= 1$. But $\absval{e^{bi}} = 1$ for all $b \in \R$. So, $\absval{e^a} = 1$, so $e^a = 1$, so $a = 0$.

So $z=bi$ for some $b \in \R$.

By applying the equivalence of polar and trigonometric forms, this means that $e^{ib} = \cos(b) + isin(b) = 1$. So, $\cos(b) =1$ and $\sin(b) = 0$. This means that $b = 2k \pi$ for some $k \in \Z$.

So $B \subset \{2k\pi i \in \C: k \in \Z\}$.

Now, if $z=2k\pi i$ for some $k \in \Z$, then $e^z = \cos(2k \pi ) + i \sin(2k \pi) =1$. So $z \in B$.

So $B = \{2k\pi i \in \C: k \in \Z\}$.

\shunt

Part c:

Consider the set $C= \{z \in \C: \sin(z) = 0\}$.

Let $z =a+bi \in C$. Then $sin(z) = 0$. So $\frac{e^{iz}-e^{-iz}}{2i} = 0$, so that $e^{iz} = e^{-iz}$.

In other words, $e^{-b}e^{ai} = e^{b}e^{-ai}$. So, $e^{2b} = e^{2ai}$. Because  $\absval{e^{2ai}} = 1$, this means that $e^{2b} = 1$. So, $b=0$, and $e^{2ai}=1$. So $2ai = 2k\pi i $ for some $k \in \Z$.

So, $z=k \pi$ for some $k \in \Z$. So $C \subset \{k\pi  \in \C: k \in \Z\}$.

Now, if $z = k\pi$ for some $k \in \Z$, then $\sin(z) = 0$, and this is very well known. So $z \in C$.

So $C = \{k\pi  \in \C: k \in \Z\}$.

\shunt

{\bf Problem 2:}

Let $\Om \subset \C$ be an open connected set, and $f \in C(\Om)$ be such that for all closed, piecewise continuous curves, $\Ga$, with $\Ga \subset \Om$, $\int\limits_\Ga f(z) dz = 0$.

Pick $z \in \Om$. Let $p \in \Om$, and $\ga$ be a curve from $p$ to $z$. We showed in class that $\int\limits_\ga f(\xi) d\xi$ is independent of $\ga$; that is, $\int\limits_\ga f(\xi) d\xi$ only depends on $p$ and $z$.

So, we can define $g(z) = \int\limits_\ga f$, where $\ga$ is a curve from a chosen fixed point,$p$, to $z$.

Now, fix $z_0 \in \Om$ and let $\ep >0$ with $\ep$ small enough that $\overline{D_\ep(z_0)} \subset \Om$. Then $\absval{f} \leq M$ for some $M \in \R$ on $D_{\ep(z_0)}$. Choose $\de = \min(\ep, \ep/M)$.

\tab If $\absval{z-z_0} < \de$, we have:

\begin{align*}
\absval{\frac{g(z)-g(z_0)}{z-z_0} -f(z_0)} &= \absval{\frac{\int\limits_{z_0}^z f(x) dx}{z-z_0} -f(z_0)}\\
&\leq \\
\end{align*}

%...Doing this well is going to require that the limit of M as \de \to 0 is f(z_0)...

%Take the derivative, it's f, so f has a primitive. Done.

\shunt

{\bf Problem 3:}

(I must admit to having read this in Complex Made Simple by David C. Ullrich prior to the assignment of this problem.)

Let $f \in \scrO(D_1(0))$, with $f = \sum\limits_{n=0}^\infty a_nz^n$.

Then consider $\int\limits_0^{2\pi} \absval{f(re^{it})}^2 dt$. %Prove Parseval's formula

\shunt

{\bf Problem 4:}

Let $\phi, \psi : [a,b] \to \R$ be log-convex.

Then $\ln(\phi)$ and $\ln(\psi)$ are convex.

So for all $x,y \in [a,b]$ with $x \leq y$ and for all $t \in [0,1]$, $\ln(\psi(tx+(1-t)y)) \leq t\ln\psi(x) + (1-t)\ln\psi(y)$ and $\ln(\phi(tx+(1-t)y)) \leq t\ln\phi(x) + (1-t)\ln\phi(y)$.

%Then \phi + \psi is.

\shunt

{\bf Problem 5:}

Let $\Om \subset \C$ be a simply connected domain, $f \in \scrO(\Om)$, $f(z) \neq 0$ for any $z \in \Om$.

We showed in class that $g(z) = \int\limits_p^z \frac{f'(w)}{f(w)} dw + \la$ with $p$ chosen arbitrarily in $\Om$ and $e^\la = f(p)$ satisfies $f=e^g$, and that $g \in \scrO(\Om)$. 

Now, let $h \in \scrO(\Om)$ be such that $f=e^h$.

Then $\frac{f}{f} = \frac{e^g}{e^h}$, so that $1 = e^{g-h}$. Thus, by problem 1, we have that $g-h = 2k \pi  i$ for some $k \in \Z$. %why not different k for different values of z? This is something Pete will bring up, and he will pwn you for it.

That is, any two functions, $g$ and $h$, satisfying $e^g=e^h=f$ differ only by $2k\pi i $ for some $k \in \Z$.

\shunt

{\bf Problem 6:}

(I must admit to having read this in Complex Made Simple by David C. Ullrich prior to the assignment of this problem.)

Let $\phi \in \scrO(D_1(0))$. Suppose that $\phi$ takes its maximum at $0$.

%Prove that \phi is constant. You should be using Parseval to do this.

\shunt

{\bf Problem 7:}

Suppose that $\phi \in \scrO(\Om)$ with $\Om$ a domain, and that there is a $c \in \Om$ such that $\absval(\phi(c)) = max(\absval(\phi))$.

Then $\phi$ is constant on any ball centered at $c$, by problem 6. %Might clean that up.

Now, $\Om$ is path connected (it is a domain).

Let $z \in \Om$, and let $\ga$ be a path from $z$ to $c$ with $\ga \subset \Om$. We can cover the path with open balls, as $\Om$ is open. $\phi$ is constant on each of these open balls, as $\phi$ takes its maximum (or minimum) on these open balls. So $\phi$ is constant along the path: $\phi(z) = \phi(c)$. %Clean it up.

So for all $z \in \Om$, $\phi(z) = \phi(c)$. So $\phi$ is constant.

\shunt

\end{document}