
\documentclass[a4paper,12pt]{article}

\usepackage{fancyhdr}
\usepackage{amssymb}
%\usepackage{mathpazo}
\usepackage{mathtools}
\usepackage{amsmath}
\usepackage{slashed}
\usepackage{cancel}
\usepackage[mathscr]{euscript}
\usepackage{MaxPackage} %Note: You need MaxPackage installed or in the same folder as your .tex file or something.

\newcommand{\colorcomment}[2]{\textcolor{#1}{#2}} %First of these leaves in comments. Second one kills them.
%\newcommand{\colorcomment}[2]{}


\pagestyle{fancy}
\lhead{Max Jeter}
\chead{MA530}
\rhead{Assignment 5, Page \thepage}

%Status: 
%Clear					:1,2,7
%Cleared only via book	:3,6
%Begun					:
%Not started			:
%Can complete via book	:
%Needs Polish			:4,5

\begin{document}

(I worked with Sarah Percival and Frankie Chan a little).

{\bf Problem 1:}

Part a:

Consider the set $A= \{z \in \C: e^z = 0\}$.

If $z=a+bi \in A$ (with $a, b \in \R$), then $e^z =0$. So $e^ae^{bi} = 0$. 

For $a \in \R$, $e^a \neq 0$. So this means that $e^{bi} = 0$. But this never happens either, because $\absval{e^{bi}} = 1$ for all $b \in \R$ (because $\absval{e^{bi}} = \absval{\cos(b)+i\sin(b)} = \sqrt{\cos^2(b) + \sin^2(b)} = 1$).

So we have a contradiction. So $A = \emptyset$.

\shunt

Part b:

Consider the set $B= \{z \in \C: e^z = 1\}$.

If $z=a+bi \in B$ (with $a,b \in \R$), then $e^z = 1$. So $e^z = e^ae^{bi} = 1$.

This means that $\absval{e^ae^{bi}} = \absval{e^a}\absval{e^{bi}}= 1$. But $\absval{e^{bi}} = 1$ for all $b \in \R$. So, $\absval{e^a} = 1$, so $e^a = 1$, so $a = 0$.

So $z=bi$ for some $b \in \R$.

By applying the equivalence of polar and trigonometric forms, this means that $e^{ib} = \cos(b) + isin(b) = 1$. So, $\cos(b) =1$ and $\sin(b) = 0$. This means that $b = 2k \pi$ for some $k \in \Z$.

So $B \subset \{2k\pi i \in \C: k \in \Z\}$.

Now, if $z=2k\pi i$ for some $k \in \Z$, then $e^z = \cos(2k \pi ) + i \sin(2k \pi) =1$. So $z \in B$.

So $B = \{2k\pi i \in \C: k \in \Z\}$.

\shunt

Part c:

Consider the set $C= \{z \in \C: \sin(z) = 0\}$.

Let $z =a+bi \in C$. Then $sin(z) = 0$. So $\frac{e^{iz}-e^{-iz}}{2i} = 0$, so that $e^{iz} = e^{-iz}$.

In other words, $e^{-b}e^{ai} = e^{b}e^{-ai}$. So, $e^{2b} = e^{2ai}$. Because  $\absval{e^{2ai}} = 1$, this means that $e^{2b} = 1$. So, $b=0$, and $e^{2ai}=1$. So $2ai = 2k\pi i $ for some $k \in \Z$.

So, $z=k \pi$ for some $k \in \Z$. So $C \subset \{k\pi  \in \C: k \in \Z\}$.

Now, if $z = k\pi$ for some $k \in \Z$, then $\sin(z) = 0$, and this is very well known. So $z \in C$.

So $C = \{k\pi  \in \C: k \in \Z\}$.

\shunt

{\bf Problem 2:}

Let $\Om \subset \C$ be an open connected set, and $f \in C(\Om)$ be such that for all closed, piecewise continuous curves, $\Ga$, with $\Ga \subset \Om$, $\int\limits_\Ga f(z) dz = 0$.

Pick $z \in \Om$. Let $p \in \Om$, and $\ga$ be a curve from $p$ to $z$. We showed in class that $\int\limits_\ga f(\xi) d\xi$ is independent of $\ga$; that is, $\int\limits_\ga f(\xi) d\xi$ only depends on $p$ and $z$.

So, we can define $g(z) = \int\limits_\ga f$, where $\ga$ is a curve from a chosen fixed point,$p$, to $z$.

Now, fix $z_0 \in \Om$. It is clear that $\lim\limits_{z\to z_0} \frac{\int\limits_{z_0}^z f(w) dw}{z-z_0} = f(z_0)$, because $ \frac{\int\limits_{z_0}^z f(w) dw}{z-z_0}$ is the average value of $f(w)$ on the line segment. Now, because $\frac{g(z)-g(z_0)}{z-z_0} = \frac{\int\limits_{z_0}^z f(w) dw}{z-z_0}$, this means that $g'(z_0) = \lim\limits_{z\to z_0}\frac{g(z)-g(z_0)}{z-z_0} = \lim\limits_{z\to z_0}\frac{\int\limits_{z_0}^z f(w) dw}{z-z_0} = f(z_0)$

That is, $g'(z_0) = f(z_0)$ for all $z_0 \in \Om$; $g$ is a primitive of $f$.

\shunt

{\bf Problem 3:}

(I must admit to having read this in Complex Made Simple by David C. Ullrich prior to the assignment of this problem, and that I used it as a reference.)

Let $f \in \scrO(D_1(0))$, with $f = \sum\limits_{n=0}^\infty a_nz^n$.

Then consider $f_N = \sum\limits_{n=0}^N a_nz^n$.

\begin{align*}
\int\limits_0^{2\pi} \absval{f_N(re^{it})}^2 dt &= \int\limits_0^{2\pi} f_N(re^{it})\overline{f_N(re^{it})} dt\\
&= \int\limits_0^{2\pi} \sum\limits_0^N a_nr^ne^{int}\overline{\sum\limits_0^Na_n(r^ne^{int})} dt\\
&= \int\limits_0^{2\pi} \sum\limits_{n,m=0,0}^{N,N} a_n\overline{a_m}r^{2n}e^{i(n-m)t} dt\\
\end{align*}

It is readily checked that all of the terms in the above, except for those where $n=m$, vanish; this is because $\int\limits_0^{2\pi} e^{int}dt = 0$ when $n \neq 0$. Thus, we have

\begin{align*}
\int\limits_0^{2\pi} \absval{f_N(re^{it})}^2 dt &= \int\limits_0^{2\pi} \sum\limits_{n=0}^{N} a_n\overline{a_n}r^{2n} dt\\
&= \sum\limits_0^N 2\pi \absval{a_n}^2 r^{2n}\\
\end{align*}

So, for all $N \in \N$, $\int\limits_0^{2\pi} f_N(re^{it})dt = \sum\limits_0^N 2\pi \absval{a_n}^2 r^{2n}$.

Taking limits as $N \to \infty$, we have $\int\limits_0^{2\pi} f(re^{it})dt = \sum\limits_0^\infty 2\pi \absval{a_n}^2 r^{2n}$, which is what we wanted.

\shunt

{\bf Problem 4:}

Let $\phi, \psi : [a,b] \to \R$ be log-convex.

Then $\ln(\phi)$ and $\ln(\psi)$ are convex.

So for all $x,y \in [a,b]$ with $x \leq y$ and for all $t \in [0,1]$, $\ln(\psi(tx+(1-t)y)) \leq t\ln\psi(x) + (1-t)\ln\psi(y)$ and $\ln(\phi(tx+(1-t)y)) \leq t\ln\phi(x) + (1-t)\ln\phi(y)$. Note that because $e^x$ is an increasing function, $a<b$ if and only if $e^a < e^b$, so that these are equivalent to $\phi(tx+(1-t)y) \leq \phi(x)^t\phi(y)^{(1-t)}$ and $\psi(tx+(1-t)y) \leq \psi(x)^t\psi(y)^{(1-t)}$. 

Consider $\ln(\phi + \psi)$. Note that because $e^x$ is an increasing function, $a<b$ if and only if $e^a < e^b$.

Now, fix $x,y \in [a,b]$ with $x<y$ and fix $t \in [0,1]$.

\begin{align*}
e^{\ln(\phi + \psi)(tx+(1-t)y)} &= (\phi + \psi)(tx+(1-t)y)\\
&= \phi(tx+(1-t)y) + \psi(tx+(1-t)y)\\
&\leq \phi(x)^t\phi(y)^{(1-t)} + \psi(x)^t\psi(y)^{(1-t)}\\
&\leq e^{t\ln((\phi+\psi)(x)) + (1-t)\ln((\phi+\psi)(y))} \\
&\text{(I must admit to not being sure how to do that last step,}\\
&\text{but it's clear this is what is needed.)}
\end{align*} %That last step is not so clear...

So $\ln(\phi + \psi)(tx+(1-t)y) \leq \ln(t\ln\phi(x) + (1-t)\ln\phi(y) +t\ln\psi(x) + (1-t)\ln\psi(y))$.

That is, $\phi + \psi$ is log-convex if $\phi$ and $\psi$ are.

\shunt

{\bf Problem 5:}

Let $\Om \subset \C$ be a simply connected domain, $f \in \scrO(\Om)$, $f(z) \neq 0$ for any $z \in \Om$.

We showed in class that $g(z) = \int\limits_p^z \frac{f'(w)}{f(w)} dw + \la$ with $p$ chosen arbitrarily in $\Om$ and $e^\la = f(p)$ satisfies $f=e^g$, and that $g \in \scrO(\Om)$. 

Now, let $h \in \scrO(\Om)$ be such that $f=e^h$.

Then $\frac{f}{f} = \frac{e^g}{e^h}$, so that $1 = e^{g-h}$. Thus, by problem 1, we have that for all $z \in \C$, $g(z)-h(z) = 2k \pi  i$ for some $k \in \Z$. All that remains is to show that $k$ does not depend on $z$: consider $(g-h)'$. Now, $g'=h' = f'/f$, as was discussed in class. So $(g-h)'$ is zero; $g-h$ is constant. So $g-h$ doesn't depend on $z$; $g(z)-h(z) =2k\pi i$ for some fixed $k$.

That is, any two functions, $g$ and $h$, satisfying $e^g=e^h=f$ differ only by $2k\pi i $ for some $k \in \Z$.

\shunt

{\bf Problem 6:}

(Once again, I used Complex Made Simple as a reference for this.)

Let $\phi \in \scrO(D_1(0))$. Suppose that $\phi$ takes its maximum at $0$.

Because $\phi$ is holomorphic on $D_1(0)$, we know that $\phi$ has a power series representation, $\phi(z)=\sum\limits_0^\infty a_nz^n $, on any disk $\overline{D_r(0)}$ with $r \in (0,1)$.

So, problem 3 applies: $\int\limits_0^{2\pi} \absval{\phi(re^{it})}^2dt = \sum\limits_0^\infty 2\pi \absval{a_n}^2 r^{2n}$. Now, $\phi(0) = a_0$. So, $\int\limits_0^{2\pi} \absval{\phi(re^{it})}^2dt = \sum\limits_0^\infty 2\pi \absval{a_n}^2 r^{2n} = 2\pi \absval{a_0}^2 + \sum\limits_1^\infty 2\pi \absval{a_n}^2 r^{2n}$. So $\int\limits_0^{2\pi} \absval{\phi(re^{it})}^2dt \geq 2\pi\absval{\phi(0)}^2$.

Thus, for all $r$, $\int\limits_0^{2\pi} \absval{\phi(re^{it})}dt \geq 2\pi \absval{\phi(0)}^2$. But because $\phi(re^{it}) \leq \phi(0)$ for all $r,t \in \C$, this means that $\phi(re^{it}) = \phi(0)$ for all $r,t \in \R$. That is, $\phi(z) = \phi(0)$ for all $z \in D_1(0)$.

\shunt

{\bf Problem 7:}

Suppose that $\phi \in \scrO(\Om)$ with $\Om$ a domain, and that there is a $c \in \Om$ such that $\absval(\phi(c)) = max(\absval(\phi))$.

Then $\phi$ is constant on any disk centered at $c$, by problem 6 (by expanding and translating appropriately).

Now, $\Om$ is path connected (it is a domain).

Let $z \in \Om$, and let $\ga: [0,1] \to \Om$ be a path from $z$ to $c$ with $\ga \subset \Om$. We can cover the image of the path with a finite number of open disks contained in $\Om$, because paths are compact. Also $\phi$ is constant on each of these open balls: if not, then $\sup\{t \in [0,1]: \phi\ga(t) \neq c\} = s$ for some $s \in [0,1]$. But then there's an $\ep$-ball around $s$ where $\phi\circ\ga$ takes the value $c$ somewhere...which means that $\phi(\ga(s)) = c$, which is a contradiction. 

So $\phi$ is constant along the path: $\phi(z) = \phi(c)$.

So for all $z \in \Om$, $\phi(z) = \phi(c)$. So $\phi$ is constant.

\shunt

\end{document}