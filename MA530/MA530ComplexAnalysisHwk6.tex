
\documentclass[a4paper,12pt]{article}

\usepackage{fancyhdr}
\usepackage{amssymb}
%\usepackage{mathpazo}
\usepackage{mathtools}
\usepackage{amsmath}
\usepackage{slashed}
\usepackage{cancel}
\usepackage[mathscr]{euscript}
\usepackage{MaxPackage} %Note: You need MaxPackage installed or in the same folder as your .tex file or something.

\newcommand{\colorcomment}[2]{\textcolor{#1}{#2}} %First of these leaves in comments. Second one kills them.
%\newcommand{\colorcomment}[2]{}


\pagestyle{fancy}
\lhead{Max Jeter}
\chead{MA530}
\rhead{Assignment 6, Page \thepage}

%Number of Problems		:6
%Clear					:3,5
%Begun					:1
%Not started			:2,4,6
%Can complete via book	:
%Needs Polish			:

%Pomodoroes logged		:3

\begin{document}

{\bf Problem 1:}

Consider $\int\limits_0^\infty \frac{1-\cos(x)}{x^2} dx$. 

Now, $\int\limits_0^T\frac{1-\cos(z)}{z^2} dz = \int\limits_0^T\frac{1-\frac{e^{iz}+e^{-iz}}{2}}{z^2} dz = -\left[\int\limits_0^T \frac{e^{iz}-1}{2z^2} dz+ \int\limits_0^T \frac{e^{-iz}-1}{2z^2} dz\right]$. Both of the functions under the integrands are holomorphic, except at the origin. Using a $u$-substitution, we get $\int\limits_0^T \frac{e^{-iz}-1}{2z^2} dz = -\int\limits_0^{-T} \frac{e^{iz}-1}{2z^2} dz$.

So, 

\begin{align*}
\int\limits_0^T\frac{1-\cos(z)}{z^2} dz &= -\left[\int\limits_0^T \frac{e^{iz}-1}{2z^2} dz- \int\limits_0^{-T} \frac{e^{iz}-1}{2z^2} dz dz\right]\\
&= \frac{1}{2}\left[\int\limits_0^T \frac{1-e^{iz}}{z^2} dz-\int\limits_0^{-T} \frac{1-e^{iz}}{z^2} dz \right]\\
&= \frac{1}{2}\left[\int\limits_{-T}^T \frac{1-e^{iz}}{z^2} dz \right]\\
\end{align*} %Break it up into three appropriate integrals, estimate a lot of it away, win.

\shunt

{\bf Problem 2:}

Let $\Om \subset \C$ be open and simply connected, $f \in \scrO (\Om)$, $f$ is always nonzero, $k \in \Z$.

There is an $h \in \scrO(\Om)$ such that $e^h =f$. Define $\overline{h} = h/k$. Then $e^{\overline{h}k} = f$. %Scheme fails: if f is ever 1, then h will be zero. We can't find the logarithm in that case. Find a new scheme. Also, overbar is shit notation, case conjugates.

\shunt

{\bf Problem 3:}

Consider $\sqrt[\sqrt{-1}]{-1} = (-1)^{sqrt{-1}} = (e)^{\ln(-1)\sqrt{-1}} = e^{\ln{-1}e^{\frac{1}{2}\ln(-1)}}$. As discussed in class, the logarithms of $-1$ are $(2k+1)\pi i$ for each $k \in \Z$. That is, the possible values of $\sqrt[\sqrt{-1}]{-1}$ are given by 

\begin{align*}
e^{((2k+1)\pi i)e^{\frac{1}{2}((2j+1)\pi i)}} 
\end{align*} %Make this bigger

for any given $k, j \in \Z$.

Yet, this is an intractible mess. Consider that $e^{\frac{1}{2}((2j+1)\pi i)} = e^{j \pi i + \frac{1}{2} \pi i} = e^{j\pi i} e^{\frac{1}{2} \pi i} = (-1)^j e^{\frac{1}{2} \pi i}$. Thus, our original expression becomes

\begin{align*}
e^{((2k+1)\pi i)(-1)^j e^{\frac{1}{2} \pi i}} 
\end{align*}

To clean this up even more, $e^{\frac{1}{2} \pi i}=i$. So, we have our original expression as 

\begin{align*}
e^{((2k+1)\pi i)(-1)^j i}  = e^{-((2k+1)\pi)(-1)^j}
\end{align*}

Now, it is somewhat clear that $\{e^{-((2k+1)\pi)(-1)^j}: j, k \in \Z \} = \{ e^{ \pm ((2k+1)\pi)}: k \in \Z \} = \{e^{-((2k+1)\pi)}: k \in \Z \}$.

So, the set of values $\sqrt[\sqrt{-1}]{-1}$ are $\{e^{-((2k+1)\pi)}: k \in \Z \}$. 

And yes, taking $k=-1$ yields a value of $e^\pi$, which is ``about $23$''.

\shunt

{\bf Problem 4:}

Let $\ln(z)$ be the principal branch of the logarithm of $z$, and let $z_1, z_2$ have positive real component.

\shunt

{\bf Problem 5:}

Consider $\sin(\frac{1}{z})$. We know that $\sin(z) = \sum\limits_{n=0}^\infty (-1)^n \frac{z^{2n+1}}{(2n+1)!}$. So, where defined, $\sin(\frac{1}{z}) = \sum\limits_{n=0}^\infty (-1)^n \frac{\frac{1}{z}^{2n+1}}{(2n+1)!} = \sum\limits_{n=0}^\infty (-1)^n \frac{1}{z^{2n+1}(2n+1)!}$.

That is, we have found a Laurent series for $\sin(\frac{1}{z})$ about $0$. We are done.

\shunt

{\bf Problem 6:}

\shunt

{\bf Problem 7:}

\shunt

\end{document}