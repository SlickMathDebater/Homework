
\documentclass[a4paper,12pt]{article}

\usepackage{fancyhdr}
\usepackage{amssymb}
%\usepackage{mathpazo}
\usepackage{mathtools}
\usepackage{amsmath}
\usepackage{slashed}
\usepackage{cancel}
\usepackage[mathscr]{euscript}
\usepackage{MaxPackage} %Note: You need MaxPackage installed or in the same folder as your .tex file or something.

\newcommand{\colorcomment}[2]{\textcolor{#1}{#2}} %First of these leaves in comments. Second one kills them.
%\newcommand{\colorcomment}[2]{}


\pagestyle{fancy}
\lhead{Max Jeter}
\chead{MA530}
\rhead{Assignment 6, Page \thepage}

%Number of Problems		:9
%Clear					:2,3,5,6
%Begun					:
%Not started			:
%Can complete via book	:
%Needs Polish			:1*,4*,7*,8*,9*

%Pomodoroes logged		:10

%*:Get someone else to read it.

\begin{document}

(I worked with Dan McNall a little).

{\bf Problem 1:}

Consider $\int\limits_0^\infty \frac{1-\cos(x)}{x^2} dx$. Note that this must be positive. (If this is not clear, we can estimate the integral from below by an alternating series that converges to something positive.)

Now, $\int\limits_0^T\frac{1-\cos(z)}{z^2} dz = \int\limits_0^T\frac{1-\frac{e^{iz}+e^{-iz}}{2}}{z^2} dz = -\left[\int\limits_0^T \frac{e^{iz}-1}{2z^2} dz+ \int\limits_0^T \frac{e^{-iz}-1}{2z^2} dz\right]$. Both of the functions under the integrands are holomorphic, except at the origin. Using a $u$-substitution, we get $\int\limits_0^T \frac{e^{-iz}-1}{2z^2} dz = -\int\limits_0^{-T} \frac{e^{iz}-1}{2z^2} dz$.

So, 

\begin{align*}
\int\limits_0^T\frac{1-\cos(z)}{z^2} dz &= -\left[\int\limits_0^T \frac{e^{iz}-1}{2z^2} dz- \int\limits_0^{-T} \frac{e^{iz}-1}{2z^2} dz dz\right]\\
&= \frac{1}{2}\left[\int\limits_0^T \frac{1-e^{iz}}{z^2} dz-\int\limits_0^{-T} \frac{1-e^{iz}}{z^2} dz \right]\\
&= \frac{1}{2}\left[\int\limits_{-T}^T \frac{1-e^{iz}}{z^2} dz \right]\\
\end{align*} %Break it up into three appropriate integrals, estimate a lot of it away, win...but also try to throw out that 1/z^2 term...

Now, we know that $\int\limits_{-T}^T \frac{1}{z^2} dz = \infty$. The remaining term is more difficult to handle.

First, by Cauchy's Theorem, we can integrate the remaining term along the path $\ga$, pictured below:

\shunt %PICTURE GOES HERE

So, 

\begin{align*}
\frac{1}{2}\left[\int\limits_{-T}^T \frac{1-e^{iz}}{z^2} dz \right] &= \frac{1}{2}\left[\int\limits_{-T}^{-T+i\sqrt{T}} \frac{1-e^{iz}}{z^2} dz \right]\\
&+ \frac{1}{2}\left[\int\limits_{-T+i\sqrt{T}}^{T+i\sqrt{T}} \frac{1-e^{iz}}{z^2} dz \right]\\
&+ \frac{1}{2}\left[\int\limits_{T+i\sqrt{T}}^T \frac{1-e^{iz}}{z^2} dz \right]
\end{align*} %sqrt might not be what you need here...

Using the $ML$-inequality/trivial estimate, we can estimate the first term: $\int\limits_{-T}^{-T+i\sqrt{T}} \frac{1-e^{iz}}{z^2} dz \leq \sqrt{T} \sup(\absval{\frac{1-\cos(z)}{z^2}}) = \frac{1}{\sqrt{T}}$

Similarly, we can estimate the last term: $\int\limits_{T+i\sqrt{T}}^T \frac{1-e^{iz}}{z^2} dz \leq \sqrt{T} \sup(\absval{\frac{1-\cos(z)}{z^2}}) = \frac{1}{\sqrt{T}}$

And we can estimate the middle term: $\int\limits_{-T+i\sqrt{T}}^{T+i\sqrt{T}} \frac{1-e^{iz}}{z^2} dz \leq 2T \sup(\absval{\frac{1-\cos(z)}{z^2}}) = 2/T$.

So as $T \to \infty$, all of these terms vanish. So $\frac{1}{2}\left[\int\limits_{-T}^T \frac{1-e^{iz}}{z^2} dz \right] = \frac{1}{2}\left[\int\limits_{-T}^{-T+i\sqrt{T}} \frac{1-e^{iz}}{z^2} dz \right]$ vanishes as $T \to \infty$.

So $\int\limits_0^\infty \frac{1-\cos(x)}{x^2} dx = 0$.

\shunt

{\bf Problem 2:}

Let $\Om \subset \C$ be open and simply connected, $f \in \scrO (\Om)$, $f$ is always nonzero, $k \in \Z^+$.

There is an $h \in \scrO(\Om)$ such that $e^h =f$. Define $\tilde{h} = h/k$. Then:

\begin{align*}
e^{\tilde{h}k} &= f\\
e^{\tilde{h}+\tilde{h}+\tilde{h} \ldots +\tilde{h}} &= f\\
e^{\tilde{h}}e^{\tilde{h}}e^{\tilde{h}} \ldots e^{\tilde{h}} &= f\\
(e^{\tilde{h}})^k &= f\\
\end{align*}

So, if $\Om \subset \C$ be open and simply connected, $f \in \scrO (\Om)$, $f$ is always nonzero, $k \in \Z^+$, then there's a $g \in \scrO(\Om)$ with $g^k = f$.

Now, if $k \in \Z^-$, then find $h$ with $h^{-k} = f$. Next, define $g = 1/h$. Then we have that $g^k = \frac{1}{h}^k =e^{\ln(1/h)k} = e^{-ln(h)k}= h^{-k} = f$, which yields our result.

\shunt

{\bf Problem 3:}

Consider $\sqrt[\sqrt{-1}]{-1} = (-1)^{sqrt{-1}} = (e)^{\ln(-1)\sqrt{-1}} = e^{\ln{-1}e^{\frac{1}{2}\ln(-1)}}$. As discussed in class, the logarithms of $-1$ are $(2k+1)\pi i$ for each $k \in \Z$. That is, the possible values of $\sqrt[\sqrt{-1}]{-1}$ are given by 

\begin{align*}
e^{((2k+1)\pi i)e^{\frac{1}{2}((2j+1)\pi i)}} 
\end{align*} %Make this bigger

for any given $k, j \in \Z$.

Yet, this is an intractible mess. Consider that $e^{\frac{1}{2}((2j+1)\pi i)} = e^{j \pi i + \frac{1}{2} \pi i} = e^{j\pi i} e^{\frac{1}{2} \pi i} = (-1)^j e^{\frac{1}{2} \pi i}$. Thus, our original expression becomes

\begin{align*}
e^{((2k+1)\pi i)(-1)^j e^{\frac{1}{2} \pi i}} 
\end{align*}

To clean this up even more, $e^{\frac{1}{2} \pi i}=i$. So, we have our original expression as 

\begin{align*}
e^{((2k+1)\pi i)(-1)^j i}  = e^{-((2k+1)\pi)(-1)^j}
\end{align*}

Now, it is somewhat clear that $\{e^{-((2k+1)\pi)(-1)^j}: j, k \in \Z \} = \{ e^{ \pm ((2k+1)\pi)}: k \in \Z \} = \{e^{-((2k+1)\pi)}: k \in \Z \}$.

So, the set of values $\sqrt[\sqrt{-1}]{-1}$ are $\{e^{-((2k+1)\pi)}: k \in \Z \}$. 

And yes, taking $k=-1$ yields a value of $e^\pi$, which is ``about $23$''.

\shunt

{\bf Problem 4:}

Let $\ln(z)$ be the principal branch of the logarithm of $z$, and let $z_1, z_2$ have positive real component.

Then $e^{\ln(z_1)+\ln(z_2)} = e^{\ln(z_1)}e^{\ln(z_2)} = z_1z_2 = e^{\ln(z_1z_2)}$.

Now, $e^{a+bi}$ is one-to-one given that $a \in \R$ and $b \in (-\pi,\pi)$. Because we're working in the principal branch and the real components of $z_1$ and $z_2$ are (strictly) positive, $z_1z_2 = e^{a+bi}$ has $a \in \R$ and $b \in (-\pi,\pi)$. For the same reason, $\ln(z_1)+\ln(z_2) = e^{a'+b'i}$ has $a' \in \R$ and $b' \in (-\pi, \pi)$. So $e^{z}$ is one-to-one for a domain containing both $\ln(z_1)+\ln(z_2)$ and $\ln(z_1z_2)$. Thus, $\ln(z_1)+\ln(z_2) = \ln(z_1z_2)$.

\shunt

{\bf Problem 5:}

Consider $\sin(\frac{1}{z})$. We know that $\sin(z) = \sum\limits_{n=0}^\infty (-1)^n \frac{z^{2n+1}}{(2n+1)!}$. So, where defined, $\sin(\frac{1}{z}) = \sum\limits_{n=0}^\infty (-1)^n \frac{\frac{1}{z}^{2n+1}}{(2n+1)!} = \sum\limits_{n=0}^\infty (-1)^n \frac{1}{z^{2n+1}(2n+1)!}$.

That is, we have found a Laurent series for $\sin(\frac{1}{z})$ about $0$. We are done.

\shunt

{\bf Problem 6:}

Consider $\frac{\sin(z)}{1-z}$. Because $\frac{1}{1-z} = \sum\limits_{n=0}^\infty z^n$ (when $z \in D_1(0)$, which we are working on because of the singularity at $1$) and $\sin(z) = \sum\limits_{n=0}^\infty \frac{(-1)^n z^{2n+1}}{(2n+1)!}$, we have $\frac{\sin(z)}{1-z} = \sum\limits_{n=0}^\infty z^n \sum\limits_{n=0}^\infty \frac{(-1)^n z^{2n+1}}{(2n+1)!} = \sum\limits_{n=0}^\infty a_nz^n$.

The first seven coefficients of this expansion (that is, those with $n \leq 6$), are as follows (this follows trivially by computation, which I will invariably screw up.)

\begin{align*}
a_0 &= 0\\
a_1 &= 1\\
a_2 &= 1\\
a_3 &= 5/6\\
a_4 &= 5/6\\
a_5 &= 5/6 + 1/60\\
a_6 &= 5/6 + 1/60\\
\end{align*}

\shunt

{\bf Problem 7:}

(Preliminary note: these integrals land in the real numbers...there's no ambiguity about what $\ln$ is...).

Let $f \in \scrO(D_R(0))$. Consider $\ln(\int\limits_0^{2\pi} \absval{f(e^{s+it})}^2 dt)$ as a function of $s$.

We can apply Parseval's Formula (one of the earlier homeworks): let $f(z) = \sum\limits_{n=0}^\infty a_n z^n$.

Then $\ln( \int\limits_0^{2\pi} \absval{f(e^{s+it})}^2 dt) = \ln(2\pi \sum\limits_{n=0}^\infty \absval{a_n}^2 e^{2sn})$. Moreover, we have $\ln(2\pi \sum\limits_{n=0}^\infty \absval{a_n}^2 e^{2sn}) =  \ln(2\pi \lim\limits_{N \to \infty}\sum\limits_{n=0}^N \absval{a_n}^2 e^{2sn})$. 

By the earlier homework, the sums of log-convex functions are log-convex. By a trivial induction argument, all finite sums of log-convex functions are log-convex. Moreover, $\ln(2\pi \absval{a_n}^2 e^{2sn})$ is convex; $\ln(2\pi \absval{a_n}^2 e^{2sn}) = 2sn \ln([2\pi\absval{a_n}^2]^{-2sn}) = 2snc$ for some $c \in \R$, which is clearly convex as a function of $s$. %Prove it.

So we have that $2\pi \sum\limits_{n=0}^N \absval{a_n}^2 e^{2sn}$ is log-convex, for all $N \in \N$; in other words, $\ln(2\pi \sum\limits_{n=0}^N \absval{a_n}^2 e^{2sn})$ is convex for all $N$.

Now, the limit of a sequence of log-convex functions is log-convex: let $x,y \in \R$, and $t \in [0,1]$, and let $\phi_N \to \phi$ be a sequence of log-convex functions. Then:

\begin{align*}
t\ln(\phi_N(x)) + (1-t)\ln(\phi_N(y)) &\leq \ln(\phi_N(tx+(1-t)y))\\
t\ln(\phi(x)) + (1-t)\ln(\phi(y)) &\leq \ln(\phi(tx+(1-t)y))\\
\end{align*}

because inequality is preserved over limits, because $\ln$ is continuous.

Thus, $2\pi \lim\limits_{N \to \infty}\sum\limits_{n=0}^N \absval{a_n}^2 e^{2sn}$ is log-convex: So, $\ln(2\pi \lim\limits_{N \to \infty}\sum\limits_{n=0}^N \absval{a_n}^2 e^{2sn}) = \ln(\int\limits_0^{2\pi} \absval{f(e^{s+it})}^2 dt)$ is convex; this is the result we wanted.

\shunt

{\bf Problem 8:}

Let $\psi, \phi \in \scrO(\C)$, and $\absval{\psi} \leq \absval{\phi}$ on $\C$.

First, $\phi = 0$, $\psi = 0$ trivially by the assumption.

Next, $\absval{\psi}/\absval{\phi} \leq 1$ on $\C$, except where $\phi = 0$. Thus, $\absval{\frac{\psi}{\phi}} \leq 1$ on $\C$, except where $\phi = 0$. Because $\frac{\psi}{\phi}$ is bounded, all of its singularities are removable; we can define $\xi$ holomorphic and equal to $\frac{\psi}{\phi}$ except where $\phi = 0$. 

Now, $\xi$ is a bounded, entire function; it is constant, by Liouville.

So $\frac{\psi}{\phi} = c$ on $\C$, except where $\phi = 0$, for some $c \in \C$. Also, $\phi = \psi = c\phi$ where $\phi = 0$. Thus, $\psi = c \phi$.

\shunt

{\bf Problem 9:}

(Without loss of generality, let $c=0$.)

Let $f$ have an essential singularity at $0$. Then for all $r>0$, $f(D_r(0))$ is dense in $\C$. So the set $\{1/z: z \in f(D_r(0))\}$ is dense in $\C$, for all $r >0$. So $1/f$ has an essential singularity at $0$.

\shunt

\end{document}