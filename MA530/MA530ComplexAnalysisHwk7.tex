
\documentclass[a4paper,12pt]{article}

\usepackage{fancyhdr}
\usepackage{amssymb}
%\usepackage{mathpazo}
\usepackage{mathtools}
\usepackage{amsmath}
\usepackage{slashed}
\usepackage{cancel}
\usepackage[mathscr]{euscript}
\usepackage{MaxPackage} %Note: You need MaxPackage installed or in the same folder as your .tex file or something.

\newcommand{\colorcomment}[2]{\textcolor{#1}{#2}} %First of these leaves in comments. Second one kills them.
%\newcommand{\colorcomment}[2]{}


\pagestyle{fancy}
\lhead{Max Jeter}
\chead{MA530}
\rhead{Assignment 7, Page \thepage}

%Number of Problems		:8
%Clear					:3,6,7
%Begun					:4
%Not started			:5,8
%Can complete via book	:
%Needs Polish			:1,2
%Pomodoros logged		:6
\begin{document}

{\bf Problem 1:}

Let $f, g \in \scrO(D_r(c))$, $g(c) = 0$, and $g'(c) \neq 0$. 

Without loss of generality, $c = 0$. Now, let $f(z) = \sum\limits_{n=0}^\infty a_nz^n$ and $g(z) = \sum\limits_{n=0}^\infty b_nz^n$. Because $g(0) = 0$, we have that $b_0 = 0$. So,

\begin{align*}
\text{Res}_0 \frac{f}{g} (c) &= \frac{1}{2\pi i} \int\limits_{D_r(0)} \frac{f}{g} dz\\
&= \frac{1}{2\pi i} \int\limits_{D_r(0)} \frac{\sum\limits_{n=0}^\infty a_nz^n}{\sum\limits_{n=0}^\infty b_nz^n} dz\\
&= \frac{1}{2\pi i} \int\limits_{D_r(0)} \frac{\sum\limits_{n=0}^\infty a_nz^n}{\sum\limits_{n=1}^\infty b_nz^n} dz\\
&= \frac{1}{2\pi i} \int\limits_{D_r(0)} \frac{\sum\limits_{n=0}^\infty a_nz^n}{z\sum\limits_{n=0}^\infty b_{n+1}z^n} dz\\
&= \frac{1}{2\pi i} \sum\limits_{n=0}^\infty \int\limits_{D_r(0)} \frac{a_nz^n}{z\sum\limits_{m=0}^\infty b_{m+1}z^m} dz\\
\end{align*} 

All but the first of those terms vanish; $\frac{z^na_n}{zb_1 + z^2b_2 \ldots } = \frac{z^na_n}{zh(z)} = \frac{z^{n-1}a_n}{h(z)}$ is holomorphic on a sufficiently small disk around $0$ ($h(z)$ is nonzero on a small enough disk, else $g$ is identically zero...and so $g' = 0$).

So, 

\begin{align*}
\text{Res}_0 \frac{f}{g} (c) &= \frac{1}{2\pi i} \sum\limits_{n=0}^\infty \int\limits_{D_r(0)} \frac{a_nz^n}{z\sum\limits_{m=0}^\infty b_{m+1}z^m} dz\\
&= \frac{1}{2\pi i}  \int\limits_{D_r(0)} \frac{a_1}{\sum\limits_{m=0}^\infty b_{m+1}z^m} dz\\
&= a_1/b_0\\
&= f'(c)/g(c)
\end{align*}  %This needs justification

Yielding our result. 

\shunt

{\bf Problem 2:}

Let $f \in \scrO(\dot{D}_r(c))$ with $c$ not an essential singularity. Without loss of generality, $c=0$.

Consider $\text{Res}_0 \frac{f'}{f}$.  Now, let $f(z) = \sum\limits_{n=k}^\infty a_nz^n$ with $a_k$ nonzero, so that $f'(z) = \sum\limits_{n=k}^\infty na_nz^{n-1}$; $k$ will be the order of zero if positive, and the order of pole if negative, and this is clear. So, %Double-check that...

\begin{align*} %Adapt it...it's fundamentally the same as above.
\text{Res}_0 \frac{f'}{f} (c) &= \frac{1}{2\pi i} \int\limits_{D_r(0)} \frac{f'}{f} dz\\
&= \frac{1}{2\pi i} \int\limits_{D_r(0)} \frac{\sum\limits_{n=k}^\infty na_nz^{n-1}}{\sum\limits_{n=k}^\infty a_nz^n} dz\\
&= \frac{1}{2\pi i} \int\limits_{D_r(0)} \frac{\sum\limits_{n=0}^\infty na_nz^{n-1}}{z\sum\limits_{n=0}^\infty a_{n}z^{n-1}} dz\\
&= \frac{1}{2\pi i} \sum\limits_{n=0}^\infty \int\limits_{D_r(0)} \frac{na_nz^{n-1}}{z\sum\limits_{m=0}^\infty a_{m}z^{m-1}} dz\\
\end{align*} 

All but the first of those terms vanish; $\frac{nz^na_n}{zb_1 + z^2b_2 \ldots } = \frac{z^na_n}{zh(z)} = \frac{z^{n-1}a_n}{h(z)}$ is holomorphic on a sufficiently small disk around $0$ ($h(z)$ is nonzero on a small enough disk, else $a_k$ was zero...).

So, 

\begin{align*}
\text{Res}_0 \frac{f'}{f} (c) &= \frac{1}{2\pi i} \sum\limits_{n=0}^\infty \int\limits_{D_r(0)} \frac{na_nz^{n-1}}{z\sum\limits_{m=0}^\infty a_{m}z^{m-1}} dz\\
&= \frac{1}{2\pi i} \int\limits_{D_r(0)} \frac{ka_kz^{n-1}}{z\sum\limits_{m=0}^\infty a_{m}z^{m-1}} dz\\
&= k \text{ (as above).}
\end{align*}

Yielding our result. 

\shunt

{\bf Problem 3:}

A real-variable analogue of Rouche's Theorem would be:

``Let $I$ be an open interval $(a,b)$, $f,g$ be differentiable on $I$, and let $J$ be an open interval containing the closure of $I$.

If $\absval{f(a)} < \absval{g(a)}$ and $\absval{f(b)} < \absval{g(b)}$, then $g$, $g-f$ have the same number of zeroes in $I$.''

The obvious counterexample is $f(x) = 0$ if $x = 0$, $f(x) = \sin(1/x)$ otherwise, and $g(x) =1$ on the interval $(0,1/2\pi)$. Now, $f(x) = 0$ at $0,1/2\pi$, and $g(x) =1$, so $\absval{f} < \absval{g}$ on the boundary of the interval. But $g$ has no zeroes, and $g-f$ has infinitely many zeroes. So this breaks.

\shunt

{\bf Problem 4:}

Consider $\int\limits_{-\infty}^\infty \frac{\cos(x)}{x^2+a^2} dx$, with $a \in \R$ and $a > 0$.

\begin{align*}
\int\limits_{-\infty}^\infty \frac{\cos(x)}{x^2+a^2} dx &= \frac{1}{2} \int\limits_{-\infty}^\infty \frac{e^{iz}+e^{-iz}}{z^2+a^2}dz\\
&=\int\limits_{0}^\infty \frac{e^{iz}+e^{-iz}}{z^2+a^2}dz \text{ (because the function is even...)}\\
&= \int\limits_{0}^\infty \frac{e^{iz}}{z^2+a^2}dz + \int\limits_{0}^\infty \frac{e^{-iz}}{z^2+a^2}dz\\
&= \int\limits_{0}^\infty \frac{e^{iz}}{z^2+a^2}dz - \int\limits_{0}^{-\infty} \frac{e^{iz}}{z^2+a^2}dz \text{ (u-substitute -z)} \\
&= \int\limits_{-\infty}^\infty \frac{e^{iz}}{z^2+a^2}dz \\
&= \int\limits_{-\infty}^\infty \frac{\sum\limits_{n=0}^\infty \frac{(iz)^n}{n!}}{z^2+a^2}dz \\
&= \sum\limits_{n=0}^\infty \int\limits_{-\infty}^\infty \frac{\frac{(iz)^n}{n!}}{z^2+a^2}dz \\
&= \sum\limits_{n=0}^\infty \frac{i^n}{n!} \int\limits_{-\infty}^\infty \frac{z^n}{z^2+a^2}dz \\
&= \sum\limits_{n=0}^\infty \frac{i^n}{n!} 2\pi i \sum\limits_{c \in \C} \text{Res}_c \frac{z^n}{z^2+a^2} \text{ (As discussed in class) } \\
\end{align*} %Have to finish it out, but this is clearly the right path.

\shunt

{\bf Problem 5:}

Consider $\int\limits_{\Ga_T} z^\al R(z) dz$ (with $R$ a rational function, and $\Ga_T$ as pictured below.)

\shunt %DRAW \Ga_T



\shunt

{\bf Problem 6:}

Consider $e^z = 6z^2 + 1$. This is equivalent to $0=6z^2 + 1 -e^z$.

Define $g(z) = 6z^2 +1$ and $f(z) = e^z$. When $\absval{z} = 2$, $\absval{g} \geq \absval{6z^2} -1 = 23$ and $\absval{f} \leq e^2 \leq 9$. So $g>f$ when $\absval{z} = 2$.

So Rouche's Theorem applies: $e^z = 6z^2 +1$ has the same number of solutions as $0=6z^2 +1$ on the disk bounded by $\absval{z} = 2$.

Now, $6z^2 +1$ has two solutions, by the fundamental theorem of algebra. Moreover, $\pm \frac{i}{\sqrt{6}}$ are solutions, as is readily checked. These solutions are both in that disk. So $6z^2-1$ has two zeroes on the disk bounded by $\absval{z} = 2$.

So  $e^z = 6z^2 +1$ has $2$ solutions on the disk bounded by $\absval{z} = 2$.

\shunt

{\bf Problem 7:}

Consider a polynomial, $f(z) = \sum\limits_{n=0}^N a_nz^n$.

Define $M = 9000N \sum \limits \absval{a_n}$ (Note: $M$ is chosen so that $a_NM^N > \sum\limits_{i=0}^n \absval{a_iM^i}$ for any $n <N$). Define $g_0(z) = a_0$. Now, $\absval{f} > \absval{g_0}$ on the boundary of the disk of radius $M$ centered at $0$. So $f-g_0$ and $f$ have the same number of zeroes in this disk.

Define $g_1(z) = a_1z$. Now, $\absval{f-g_0} > \absval{g_1}$ on the boundary of the disk of radius $M$ centered at $0$. So $f-g_0-g_1$ and $f-g_0$ and $f$ have the same number of zeroes in this disk.

The above process can be iterated: define $g_n(z) = a_nz^n$. Then $\absval{f-\sum\limits_{m=0}^{n-1} g_m} > \absval{g_n}$. So $f-\sum\limits_{m=0}^{n} g_m$ and $f$ have the same number of zeroes in that disk.

So $f$ and $a_nz^n$ have the same number of zeroes on the disk of radius $M$ centered at $0$. So $f$ has $n$ zeroes.

Note that we can pick $M$ arbitrarily large (that was the point of $M$) and have this work. Thus, $f$ has $n$ zeroes on $\C$; this is the fundamental theorem of algebra.

\shunt

{\bf Problem 8:}

Let $\Om$ be ``standard'' (open, bounded, boundary is finitely many piecewise $C^1$ Jordan curves). Let $f \in \scrO(G)$, where $G \supset \overline{\Om}$, and $f \neq 0$ anywhere on $\partial \Om$.

Consider $\frac{1}{2\pi i} \int\limits_{\partial \Om} \frac{z^k f'(z)}{f(z)} dz$, where $k \in \N$.

\shunt

\end{document}