
\documentclass[a4paper,12pt]{article}

\usepackage{fancyhdr}
\usepackage{amssymb}
%\usepackage{mathpazo}
\usepackage{mathtools}
\usepackage{amsmath}
\usepackage{slashed}
\usepackage{cancel}
\usepackage[mathscr]{euscript}
\usepackage{MaxPackage} %Note: You need MaxPackage installed or in the same folder as your .tex file or something.

\newcommand{\colorcomment}[2]{\textcolor{#1}{#2}} %First of these leaves in comments. Second one kills them.
%\newcommand{\colorcomment}[2]{}


\pagestyle{fancy}
\lhead{Max Jeter}
\chead{MA530}
\rhead{Assignment 7, Page \thepage}

%Number of Problems		:8
%Clear					:1,2,3,4,6,7,8
%Begun					:5
%Not started			:
%Can complete via book	:
%Needs Polish			:
%Pomodoros logged		:14
\begin{document}

{\bf Problem 1:}

Let $f, g \in \scrO(D_r(c))$, $g(c) = 0$, and $g'(c) \neq 0$. 

Without loss of generality, $c = 0$. Now, let $f(z) = \sum\limits_{n=0}^\infty a_nz^n$ and $g(z) = \sum\limits_{n=0}^\infty b_nz^n$. Because $g(0) = 0$, we have that $b_0 = 0$. So,

\begin{align*}
\text{Res}_0 \frac{f}{g} &= \frac{1}{2\pi i} \int\limits_{ \partial D_r(0)} \frac{f}{g} dz\\
&= \frac{1}{2\pi i} \int\limits_{ \partial D_r(0)} \frac{\sum\limits_{n=0}^\infty a_nz^n}{\sum\limits_{n=0}^\infty b_nz^n} dz\\
&= \frac{1}{2\pi i} \int\limits_{ \partial D_r(0)} \frac{\sum\limits_{n=0}^\infty a_nz^n}{\sum\limits_{n=1}^\infty b_nz^n} dz\\
&= \frac{1}{2\pi i} \int\limits_{ \partial D_r(0)} \frac{\sum\limits_{n=0}^\infty a_nz^n}{z\sum\limits_{n=0}^\infty b_{n+1}z^n} dz\\
&= \frac{1}{2\pi i} \sum\limits_{n=0}^\infty \int\limits_{ \partial D_r(0)} \frac{a_nz^n}{z\sum\limits_{m=0}^\infty b_{m+1}z^m} dz\\
\end{align*} 

All but the first of those terms vanish; $\frac{z^na_n}{zb_1 + z^2b_2 \ldots } = \frac{z^na_n}{zh(z)} = \frac{z^{n-1}a_n}{h(z)}$ is holomorphic on a sufficiently small disk around $0$ if $n \geq 1$ ($h(z)$ is nonzero on a small enough disk, else $g$ is identically zero...and so $g' = 0$. It's also nonzero at $0$, because $b_1 \neq 0$ (else $g'(0) = 0$)).

So, using $h$ as above,

\begin{align*}
\text{Res}_0 \frac{f}{g} (c) &= \frac{1}{2\pi i} \sum\limits_{n=k}^\infty \int\limits_{ \partial D_r(0)} \frac{a_nz^n}{z\sum\limits_{m=0}^\infty b_{m+1}z^m} dz\\
&= \frac{1}{2\pi i}  \int\limits_{ \partial D_r(0)} \frac{a_0}{z\sum\limits_{m=0}^\infty b_{m+1}z^m} dz\\
&= \frac{a_0}{2\pi i}  \int\limits_{ \partial  \partial D_r(0)} \frac{1}{zh(z)} dz\\
&= \frac{a_0}{2\pi i}  2\pi i \frac{1}{h(0)}\\
&= a_0/b_1\\
&= f(0)/g'(0)
\end{align*}

Yielding our result. 

\shunt

{\bf Problem 2:} %There's a fiddly bit left...indices n'shit.

Let $f \in \scrO(\dot{D}_r(c))$ with $c$ not an essential singularity. Without loss of generality, $c=0$.

Consider $\text{Res}_0 \frac{f'}{f}$.  Now, let $f(z) = \sum\limits_{n=k}^\infty a_nz^n$ with $a_k$ nonzero, so that $f'(z) = \sum\limits_{n=k}^\infty na_nz^{n-1}$; $k$ will be the order of zero if positive, and $-1$ times the order of pole if negative, and this is clear. So,

\begin{align*} %Adapt it...it's fundamentally the same as above.
\text{Res}_0 \frac{f'}{f}  &= \frac{1}{2\pi i} \int\limits_{ \partial D_r(0)} \frac{f'}{f} dz\\
&= \frac{1}{2\pi i} \int\limits_{ \partial D_r(0)} \frac{\sum\limits_{n=k}^\infty na_nz^{n-1}}{\sum\limits_{n=k}^\infty a_nz^n} dz\\
&= \frac{1}{2\pi i} \int\limits_{ \partial D_r(0)} \frac{\sum\limits_{n=k}^\infty na_nz^{n-1}}{z\sum\limits_{n=k}^\infty a_{n}z^{n-1}} dz\\
&= \frac{1}{2\pi i} \sum\limits_{n=k}^\infty \int\limits_{ \partial D_r(0)} \frac{na_nz^{n-1}}{z\sum\limits_{m=k}^\infty a_{m}z^{m-1}} dz\\
\end{align*} 

All but the first of those terms vanish;$\frac{z^na_n}{z(a_{k}z^{k}+ a_{k+1}z^{k+1}+ \ldots )} = \frac{z^na_n}{zz^{k}h(z)} = \frac{z^{n-k}a_n}{zh(z)}$ is holomorphic on a sufficiently small disk around $0$ if $n > k$ ($h(z)$ is nonzero on a small enough disk, else $a_k$ was zero...).

So, 

\begin{align*}
\text{Res}_0 \frac{f'}{f}  &= \frac{1}{2\pi i} \sum\limits_{n=-k}^\infty \int\limits_{ \partial D_r(0)} \frac{na_nz^{n-1}}{z\sum\limits_{m=-k}^\infty a_{m}z^{m-1}} dz\\
&= \frac{1}{2\pi i} \int\limits_{ \partial D_r(0)} \frac{ka_kz^{k-1}}{z\sum\limits_{m=-k}^\infty a_{m}z^{m-1}} dz\\
&= \frac{ka_k}{2\pi i}  \int\limits_{ \partial D_r(0)} \frac{1}{zh(z)} dz\\
&= \frac{ka_k}{2\pi i}  2 \pi i \frac{1}{h(0)}\\
&= k
\end{align*}

Yielding our result. 

\shunt

{\bf Problem 3:}

A real-variable analogue of Rouche's Theorem would be:

``Let $I$ be an open interval $(a,b)$, $f,g$ be differentiable on $I$, and let $J$ be an open interval containing the closure of $I$.

If $\absval{f(a)} < \absval{g(a)}$ and $\absval{f(b)} < \absval{g(b)}$, then $g$, $g-f$ have the same number of zeroes in $I$.''

The obvious counterexample is $f(x) = 0$ if $x = 0$, $f(x) = \sin(1/x)$ otherwise, and $g(x) =1$ on the interval $(0,1/2\pi)$. Now, $f(x) = 0$ at $0,1/2\pi$, and $g(x) =1$, so $\absval{f} < \absval{g}$ on the boundary of the interval. But $g$ has no zeroes, and $g-f$ has infinitely many zeroes. So this breaks.

\shunt

{\bf Problem 4:}

Consider $\int\limits_{-\infty}^\infty \frac{\cos(x)}{x^2+a^2} dx$, with $a \in \R$ and $a > 0$.

\begin{align*}
\int\limits_{-\infty}^\infty \frac{\cos(x)}{x^2+a^2} dx &= \frac{1}{2} \int\limits_{-\infty}^\infty \frac{e^{iz}+e^{-iz}}{z^2+a^2}dz\\
&=\int\limits_{0}^\infty \frac{e^{iz}+e^{-iz}}{z^2+a^2}dz \text{ (because the function is even...)}\\
&= \int\limits_{0}^\infty \frac{e^{iz}}{z^2+a^2}dz + \int\limits_{0}^\infty \frac{e^{-iz}}{z^2+a^2}dz\\
&= \int\limits_{0}^\infty \frac{e^{iz}}{z^2+a^2}dz - \int\limits_{0}^{-\infty} \frac{e^{iz}}{z^2+a^2}dz \text{ (u-substitute -z)} \\
&= \int\limits_{-\infty}^\infty \frac{e^{iz}}{z^2+a^2}dz \\
&= \int\limits_{-\infty}^\infty \frac{\sum\limits_{n=0}^\infty \frac{(iz)^n}{n!}}{z^2+a^2}dz \\
&= \sum\limits_{n=0}^\infty \int\limits_{-\infty}^\infty \frac{\frac{(iz)^n}{n!}}{z^2+a^2}dz \\
&= \sum\limits_{n=0}^\infty \frac{i^n}{n!} \int\limits_{-\infty}^\infty \frac{z^n}{z^2+a^2}dz \\
&= \sum\limits_{n=0}^\infty \frac{i^n}{n!} 2\pi i \sum\limits_{c \in \C^+} \text{Res}_c \frac{z^n}{z^2+a^2} \text{ (As discussed in class) } \\
\end{align*} %Have to finish it out, but this is clearly the right path. goal is \pi e^{-a}/a

With the last line being discussed in class, and $C^+$ being the upper half of the complext plane. Now, $\frac{z^n}{z^2 + a^2}$ can only have poles where $z^2 + a^2 = 0$; that is, where $z= \pm i a$.

So, we have 

\begin{align*}
\int\limits_{-\infty}^\infty \frac{\cos(x)}{x^2+a^2} dx &= \sum\limits_{n=0}^\infty \frac{i^n}{n!} 2\pi i \sum\limits_{c \in \C} \text{Res}_c \frac{z^n}{z^2+a^2}\\
&= \sum\limits_{n=0}^\infty \frac{i^n}{n!} 2\pi i \left[ \text{Res}_{ia} \frac{z^n}{z^2+a^2}\right]\\
&= \sum\limits_{n=0}^\infty \frac{i^n}{n!} 2\pi i \left[ \text{Res}_0 \frac{(z+ia)^n}{(z+ia)^2+a^2} \right]\\
&= \sum\limits_{n=0}^\infty \frac{i^n}{n!} 2\pi i \left[ \text{Res}_0 \frac{\sum\limits_{m=0}^n \binom{n}{m} z^m (ia)^{n-m}}{z^2 + 2ia} \right]\\
&= \sum\limits_{n=0}^\infty \frac{i^n}{n!} 2\pi i \left[ \text{Res}_0 \frac{\sum\limits_{m=0}^n \binom{n}{m} z^m (ia)^{n-m}}{z^2 + 2iaz} \right]\\
\end{align*}

Applying problem $1$ to the above, we get

\begin{align*}
\int\limits_{-\infty}^\infty \frac{\cos(x)}{x^2+a^2} dx &=\sum\limits_{n=0}^\infty \frac{i^n}{n!} 2\pi i \left[ \text{Res}_0 \frac{\sum\limits_{m=0}^n \binom{n}{m} z^m (ia)^{n-m}}{z^2 + 2iaz} \right]\\
&= \sum\limits_{n=0}^\infty \frac{i^n}{n!} 2\pi i \left[ \frac{ (ia)^{n}}{2ia} \right]\\
&= \frac{\pi}{a} \sum\limits_{n=0}^\infty \frac{i^n}{n!}  \left[ (ia)^{n}\right]\\
&= \frac{\pi}{a} \sum\limits_{n=0}^\infty \frac{i^n}{n!}  \left[(ia)^{n} \right]\\
&= \frac{\pi}{a} \sum\limits_{n=0}^\infty \frac{(aii)^n}{n!} \\
&= \frac{\pi}{a} \sum\limits_{n=0}^\infty \frac{(-a)^n}{n!}\\
&= \frac{\pi}{a} e^{-a}
\end{align*}

Which is the desired result.

\shunt

{\bf Problem 5:}

Consider $\int\limits_{\Ga_T} z^\al R(z) dz$ with $R(z) =P(z)/Q(z)$ (with $R$ a rational function, $P$ and $Q$ polynomials, and $\Ga_T$ as pictured below.)

\shunt %DRAW \Ga_T

For this problem, we can take $T$ large enough that the above closed curve fails to enclose any complex zeroes of $Q$, but encloses all real zeroes of $Q$.

\begin{align*}
\int\limits_{\Ga_T} z^\al R(z) dz &= \int\limits_{\ga_1} z^\al R(z) dz - \int\limits_{\ga_2} z^\al R(z) dz\\
\end{align*} %Apply residue theorem, perhaps, and integrate that thing.

Now, $\int\limits_{\ga_1} z^\al R(z) dz =\sum\limits_{z \in \R} \text{Res}_z z^\al R(z)$.

Consider $\int\limits_{\ga_2} z^\al R(z) dz$.

\begin{align*}
\int\limits_{\ga_2} z^\al R(z) dz &= \int\limits_{-1}^1 (T+it/T)^\al R(T+it/T) (i/T) dt\\
\end{align*}

The above being readily computed if $R$ is known. So,

\begin{align*}
\int\limits_{\Ga_T} z^\al R(z) dz &= \int\limits_{\ga_1} z^\al R(z) dz - \int\limits_{\ga_2} z^\al R(z) dz\\
&= \sum\limits_{z \in \R} \text{Res}_z z^\al R(z)-\int\limits_{-1}^1 (T+it/T)^\al R(T+it/T) (i/T) dt
\end{align*}

Although the above expression appears disgusting, it suffices for the desired purpose.

Now, let $T \to \infty$. The integral $\int\limits_{-1}^1 (T+it/T)^\al R(T+it/T) (i/T) dt$ vanishes, which is clear by applying the $ML$-inequality/trivial estimate.

The limit above represents the integral $\int\limits_{\infty}^0 x^\al R(x) dx + \int\limits_{0}^\infty x^\al R(x) dx$. Intuition demands that this integral vanish, but weird things happen at infinity.

\shunt

{\bf Problem 6:}

Consider $e^z = 6z^2 + 1$. This is equivalent to $0=6z^2 + 1 -e^z$.

Define $g(z) = 6z^2 +1$ and $f(z) = e^z$. When $\absval{z} = 2$, $\absval{g} \geq \absval{6z^2} -1 = 23$ and $\absval{f} \leq e^2 \leq 9$. So $g>f$ when $\absval{z} = 2$.

So Rouche's Theorem applies: $e^z = 6z^2 +1$ has the same number of solutions as $0=6z^2 +1$ on the disk bounded by $\absval{z} = 2$.

Now, $6z^2 +1$ has two solutions, by the fundamental theorem of algebra. Moreover, $\pm \frac{i}{\sqrt{6}}$ are solutions, as is readily checked. These solutions are both in that disk. So $6z^2-1$ has two zeroes on the disk bounded by $\absval{z} = 2$.

So  $e^z = 6z^2 +1$ has $2$ solutions on the disk bounded by $\absval{z} = 2$.

\shunt

{\bf Problem 7:}

Consider a polynomial, $f(z) = \sum\limits_{n=0}^N a_nz^n$.

Define $M = 9000N \sum \limits \absval{a_n}$ (Note: $M$ is chosen so that $a_NM^N > \sum\limits_{i=0}^n \absval{a_iM^i}$ for any $n <N$). Define $g_0(z) = a_0$. Now, $\absval{f} > \absval{g_0}$ on the boundary of the disk of radius $M$ centered at $0$. So $f-g_0$ and $f$ have the same number of zeroes in this disk.

Define $g_1(z) = a_1z$. Now, $\absval{f-g_0} > \absval{g_1}$ on the boundary of the disk of radius $M$ centered at $0$. So $f-g_0-g_1$ and $f-g_0$ and $f$ have the same number of zeroes in this disk.

The above process can be iterated: define $g_n(z) = a_nz^n$. Then $\absval{f-\sum\limits_{m=0}^{n-1} g_m} > \absval{g_n}$. So $f-\sum\limits_{m=0}^{n} g_m$ and $f$ have the same number of zeroes in that disk.

So $f$ and $a_nz^n$ have the same number of zeroes on the disk of radius $M$ centered at $0$. So $f$ has $n$ zeroes.

Note that we can pick $M$ arbitrarily large (that was the point of $M$) and have this work. Thus, $f$ has $n$ zeroes on $\C$; this is the fundamental theorem of algebra.

\shunt

{\bf Problem 8:}

Let $\Om$ be ``standard'' (open, bounded, boundary is finitely many piecewise $C^1$ Jordan curves). Let $f \in \scrO(G)$, where $G \supset \overline{\Om}$, and $f \neq 0$ anywhere on $\partial \Om$.

Consider $\frac{1}{2\pi i} \int\limits_{\partial \Om} \frac{z^k f'(z)}{f(z)} dz$, where $k \in \N$.

This is equal to $\sum\limits_{c \in \Om} \text{Res}_c z^k f'/f$.

Consider any individual singularity, $c \in \Om$. Without loss of generality, $c=0$. %Adapt it...

Now, let $f(z) = \sum\limits_{n=l}^\infty a_nz^n$ with $a_l$ nonzero, so that $f'(z) = \sum\limits_{n=l}^\infty na_nz^{n-1}$; $l$ will be the order of zero. It's positive, because $f \in \scrO(\Om)$.

\begin{align*} %Adapt it...it's fundamentally the same as above.
\text{Res}_0 z^k\frac{f'}{f}  &= \frac{1}{2\pi i} \int\limits_{ \partial D_r(0)} \frac{z^kf'}{f} dz\\
&= \frac{1}{2\pi i} \int\limits_{ \partial D_r(0)} \frac{z^k\sum\limits_{n=l}^\infty na_nz^{n-1}}{\sum\limits_{n=l}^\infty a_nz^n} dz\\
&= \frac{1}{2\pi i} \int\limits_{ \partial D_r(0)} \frac{z^k\sum\limits_{n=l}^\infty na_nz^{n-1}}{z\sum\limits_{n=l}^\infty a_{n}z^{n-1}} dz\\
&= \frac{1}{2\pi i} \sum\limits_{n=l}^\infty \int\limits_{ \partial D_r(0)} \frac{z^k na_nz^{n-1}}{z\sum\limits_{m=l}^\infty a_{m}z^{m-1}} dz\\
\end{align*} 

All but the $l-k$th of those terms vanish; $\frac{z^kz^na_n}{z(a_{l}z^{l}+ a_{l+1}z^{l+1}+ \ldots )} = \frac{z^kz^na_n}{zz^lh(z)} = \frac{z^{n+k-l}a_n}{zh(z)}$ is holomorphic on a sufficiently small disk around $0$ if $n +k-l > 0$ ($h(z)$ is nonzero on a small enough disk, else $a_l$ was zero...).

So, 

\begin{align*}
\text{Res}_0 z^k\frac{f'}{f} &= \frac{1}{2\pi i} \sum\limits_{n=l}^\infty \int\limits_{ \partial D_r(0)} \frac{z^kna_nz^{n-1}}{z\sum\limits_{m=l}^\infty a_{m}z^{m-1}} dz\\
&= \frac{1}{2\pi i} \int\limits_{ \partial D_r(0)} \frac{(l-k)a_{l-k}z^{l-k-1}}{z\sum\limits_{m=l}^\infty a_{m}z^{m-1}} dz\\
\end{align*}

This vanishes if $k >l$, because $a_{l-k} = 0$ then. Else,

\begin{align*}
\text{Res}_0 z^k\frac{f'}{f} &= \frac{(l-k)a_{l-k}}{2\pi i}  \int\limits_{ \partial D_r(0)} \frac{1}{zh(z)} dz\\
&= \frac{a_{l-k}}{2\pi i}  2 \pi i \frac{1}{h(0)}\\
&= l-k %not sure about that line... 
\end{align*}

So, back to our original problem; $\frac{1}{2\pi i} \int\limits_{\partial \Om} \frac{z^k f'(z)}{f(z)} dz =\sum\limits_{c \in \Om} \text{Res}_c z^k f'/f = \sum\limits_{c \in \Om} \max(l_c-k,0)$, where $l$ is the order of zero at $c$.

In words, $\frac{1}{2\pi i} \int\limits_{\partial \Om} \frac{z^k f'(z)}{f(z)} dz$ is equal to the sum of the orders of zero at points with order of zero at least $k$, minus the number of such zeroes times $k$.

\shunt

\end{document}