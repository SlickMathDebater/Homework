
\documentclass[a4paper,12pt]{article}

\usepackage{fancyhdr}
\usepackage{amssymb}
%\usepackage{mathpazo}
\usepackage{mathtools}
\usepackage{amsmath}
\usepackage{slashed}
\usepackage{cancel}
\usepackage[mathscr]{euscript}
\usepackage{MaxPackage} %Note: You need MaxPackage installed or in the same folder as your .tex file or something.

\newcommand{\colorcomment}[2]{\textcolor{#1}{#2}} %First of these leaves in comments. Second one kills them.
%\newcommand{\colorcomment}[2]{}


\pagestyle{fancy}
\lhead{Max Jeter}
\chead{Class}
\rhead{Assignment, Page \thepage}

%Number of Problems		:11
%Clear					:1,2,3,4,5,6,7,8,10
%Begun					:
%Not started			:9
%Can complete via book	:
%Needs Polish			:11

%Pomodoroes logged		:11.5

\begin{document}

{\bf Problem 1:} %Something's subtly wrong. Fix it. 

Consider $h(z) = p(z)$ if $z \in \C$, and $h(z) = \infty$ if $z = \infty$, with $p(z) = a_0 + a_1z + \ldots + a_nz^n$ a nonconstant polynomial of degree $n$.

Then define $k(z) = 1/p(z)$ if $p(z) \in \C$, and $k(z) = 0$ if $z = \infty$, and define $g(z) = k(1/z)$ when $z \neq 0$ and $g(z) = 0$ when $z = 0$.

Now, when $\absval{z} < \infty$, $h$ is complex-differentiable at $z$ because it is a polynomial.

If $z = \infty$, then $h(z) = \infty$. First, note that $h$ has exclusively non-infinite values except at $\infty$, so on any neighborhood of $\infty$, $h$ takes noninfinite values on that neighborhood. Next, we want to show that $k$ is complex differentiable at $\infty$. So, we want $g$ to be complex differentiable at $0$; consider $g$ at $0$; $g(0) = 0$. Elsewhere, $g(\zeta) = k(1/\zeta) = 1/p(1/\zeta) = 1/(a_0+a_1z^{-1} + a_2z^{-2} \ldots a_nz^{-n})$.

Except at $0$, this means that $g(\zeta) = \frac{z^n}{a_0z^n + a_1z^{n-1} + \ldots + a_n}$. This is a holomorphic function except possibly at $0$ (where it wasn't explicitly defined). However, the singularity here is removable, and this is clear (the limit as $z \to 0$ of that is just $0$...so it's bounded around $0$, so the singularity's removable). So, $g$ is complex-differentiable at $0$, so $k$ is complex-differentiable at $\infty$, so $h$ is complex-differentiable at $\infty$. 

...So $h$ is complex-differentiable everywhere.

\shunt

{\bf Problem 2:}

Let $\Om \subset \C$ be bounded, $f_n \in C(\overline{\Om})$, $f_n$ all holomorphic on $\Om$, and $f_n \to f$ uniformly on $\partial \Om$ where $f$ is holomorphic on $\Om$.

Let $\ep >0$. Pick $N$ such that for all $n \geq N$, $\absval{f_n - f} < \ep$ on $\partial \Om$.

Now, $f_n - f$ is holomorphic on a domain; the maximum principles apply. So, $\absval{f_n - f} < \ep$ for all $z \in \Om$ (And also, for all $z \in \overline{\Om}$) if $n \geq N$.

So for all $\ep>0$ there is an $N$ such that for all $n \geq N$, $\absval{f_n-f} < \ep$ on $\overline{\Om}$. That is, $f_n \to f$ uniformly on $\overline{\Om}$, as desired.  

\shunt

{\bf Problem 3:}

Consider $e^{z^2}$. 

First, the 99th derivative of this at $0$ is $\frac{99!}{2\pi i} \int\limits_{\absval{z} = 1} \frac{e^{z^2}}{z^{100}} dz$. Now, $\frac{e^{z^2}}{z^{100}}$ is holomorphic except at $0$; we can apply the residue theorem;

\begin{align*}
\frac{99!}{2\pi i} \int\limits_{\absval{z} = 1} \frac{e^{z^2}}{z^{100}} dz &= 99! \text{Res}_0 \frac{e^{z^2}}{z^{100}}\\
\end{align*}

An expansion of $\frac{e^{z^2}}{z^{100}}$ is $\sum\limits_{n=0}^\infty \frac{z^{2n-100}}{n!}$. This has no $1/z$ term; thus, the residue of it is $0$. That is,

\begin{align*}
\frac{99!}{2\pi i} \int\limits_{\absval{z} = 1} \frac{e^{z^2}}{z^{100}} dz &= 99! \text{Res}_0 \frac{e^{z^2}}{z^{100}}\\
&= 0
\end{align*}

So the 99th derivative of $e^{z^2}$ at $0$ is $0$.

Next, the 100th derivative of $e^{z^2}$ at $0$ is $\frac{100!}{2\pi i} \int\limits_{\absval{z} = 1} \frac{e^{z^2}}{z^{101}} dz$. Now, $\frac{e^{z^2}}{z^{101}}$ is holomorphic except at $0$; we can apply the residue theorem;

\begin{align*}
\frac{100!}{2\pi i} \int\limits_{\absval{z} = 1} \frac{e^{z^2}}{z^{101}} dz &= 100! \text{Res}_0 \frac{e^{z^2}}{z^{101}}\\
\end{align*}

An expansion of $\frac{e^{z^2}}{z^{101}}$ is $\sum\limits_{n=0}^\infty \frac{z^{2n-101}}{n!}$. The coefficient attached to the $1/z$ term is $1/(51!)$; thus, the residue of it is $1/(51!)$. That is,

\begin{align*}
\frac{100!}{2\pi i} \int\limits_{\absval{z} = 1} \frac{e^{z^2}}{z^{101}} dz &= 100! \text{Res}_0 \frac{e^{z^2}}{z^{101}}\\
&= 100!/51!
\end{align*}

So the 100th derivative of $e^{z^2}$ at $0$ is $100!/51!$.

\shunt

{\bf Problem 4:}

(Note: I wikipedia'd this to make sure I got the right formula. I got it right the first time; hopefully walking through the logic is convincing enough that I'm not just copying from wikipedia.)

Let $P = (p_1,p_2,p_3) \in S^2$ be a point on the sphere in $\R^3$.

We define the stereographic projection of $P$ onto $\C$ by $SP: S^2 \to \C$ using the following logic:

First, we choose an argument in $[0, 2\pi)$ to be the same as the argument of the complex number $(p_1,p_2)$ in $[0,2\pi)$; this is because the line we use to define the stereographic projection goes in the same direction as $(p_1, p_2)$.

\shunt %DRAW A CIRCLES HERE

Next, we define a magnitude using similar triangles: we say that $r = \frac{\sqrt{p_1^2 + p_2^2}}{1-p_3}$. This works in both the case where $p_3\geq 0$ and $p_3 < 0$.

So, $SP(P) = \frac{\sqrt{p_1^2 + p_2^2}}{1-p_3} e^{i \text{arg}(p_1,p_2)}$.

\shunt %DRAW A TRIANGLES HERE

That is, $SP(P) = \frac{\sqrt{p_1^2 + p_2^2}}{1-p_3} \frac{p_1}{\sqrt{p_1^2+p_2^2}} + i\frac{\sqrt{p_1^2 + p_2^2}}{1-p_3} \frac{p_2}{\sqrt{p_1^2+p_2^2}}$

Simplifying, $SP(P) = \frac{p_1}{1-p_3} + i \frac{p_2}{1-p_3}$.

\shunt

{\bf Problem 5:}

Let $h: \C \to \C$ be a real-differentiable function.

\begin{align*}
\overline{h_z} &= \frac{1}{2} \overline{h_x-ih_y}\\
&= \frac{1}{2} \overline{h_x}-\overline{ih_y}\\
&= \frac{1}{2} \overline{h_x}+i\overline{h_y}\\
&= \overline{h}_{\overline{z}}
\end{align*}

As desired.

\shunt

{\bf Problem 6:}

Let $z_1,z_2$ be fixed on the unit circle in $\R^2$. Define $\al(z): D_1(0) \to \R$ to be the angle between the line segments $\overline{zz_1}$ and $\overline{zz_2}$. That is, $\al(z) = \text{arg}(z_1-z) - \text{arg}(z_2-z)$ Because $\al$ is continuous and is never zero, this means that $\al$ is either strictly positive or strictly negative. Without loss of generality, we can take $\al$ strictly positive, by relabeling points.

We know that $\text{arg}(z)$ is harmonic (where it matters) from the discussion in class*. So because sums/differences of harmonic functions are harmonic, the result is clear.

*(We know that $\log(z) = \log(\absval{z}) + i\text{arg}(z)$ is holomorphic except at $0$...because $D_1(0)$ is simply connected, there's a branch of log that $\text{arg}(z)$ is harmonic on...which is good enough.)

\shunt

{\bf Problem 7:}

Let $\Om$ be an annulus, $u(z) = \log(\absval{z})$ for all $z \in \Om$.

Let $f$ be a holomorphic function on $\Om$ with $u = \text{Re}(f)$.

Then $u_x = \frac{x}{x^2+y^2}$ and $u_y = \frac{y}{x^2+y^2}$. But this means that $v_y = \frac{x}{x^2+y^2}$ and $v_x = -\frac{y}{x^2+y^2}$. 

However, that system of equations is inconsistent; integrating $v_y$ with respect to $y$ yields $\arctan(y/x) + g(x)$ for some $g$, and differentiating this with respect to $x$ yields $g'(x) + y/x \frac{1}{y^2+x^2}$, which is necessarily inconsistent with our given $v_x$.

So $u$ cannot have been the real part of $f$. That is, $u$ isn't the real part of any holomorphic function.

\shunt

{\bf Problem 8:}

Let $u$ and $u^2$ be harmonic. Define $v = u^2 - u$.

Then:

\begin{align*}
v_{xx} &= 2(u_x)^2 + u_{xx}(2u-1)\\
v_{yy} &= 2(u_y)^2 + u_{yy}(2u-1)\\
v_{xx} + v_{yy} &=0\\
&= 2(u_x)^2 + u_{xx}(2u-1) + 2(u_y)^2 + u_{yy}(2u-1)\\
&= 2(u_x)^2+ 2(u_y)^2 \text{ (Cancellation because u is harmonic.)}\\
\end{align*}

Because $(u_x)^2$ and $(u_y)^2$ must be nonnegative, this means that $u_x$ and $u_y$ are both identically zero.

So, $u$ is constant.

\shunt

{\bf Problem 9:}

Let $u \geq 0$ be harmonic on $D_R(0)$ with $R > 1$ and $u(0) = 1$. Let $r <1$.

This is the only one I couldn't quite figure out. 

\shunt

{\bf Problem 10:}

Consider $\cos(\sin(z))$.

By laboriously taking derivatives (that is, by applying wolframalpha rigorously), we get

\begin{align*}
a_0 &= \cos(\sin(0)) = 1\\
a_1 &= \sin(\sin(0))(-\cos(0)) = 0\\
a_2 &= \frac{1}{2}(\sin(0)\sin(\sin(0)) - \cos^2(0)\cos(\sin(0))) = \frac{-1}{2}\\
a_3 &= \frac{1}{6}(\sin(\sin(0))\cos^3(0)+ 3\sin(0)\cos(0)\cos(\sin(0)) + \sin(\sin(0))\cos(0)) = 0\\
a_4 &= \frac{5}{24}\\
\end{align*}

\shunt

{\bf Problem 11:}

Let $\Om$ be a ``standard'' open set, let $a_0, a_1, \ldots a_n \in \Om$. Let $h \in \scrO(G)$, with $G$ containing $\overline{\Om}$. Define $w(z) = \prod\limits_{j=0}^n (z-a_j)$.

Consider $P(\xi) = \frac{1}{2\pi i}\int\limits_{\partial \Om} \frac{h(z)}{w(z)} \frac{w(z)-w(\xi)}{z-\xi} dz$

\begin{align*}
\frac{1}{2\pi i} \int\limits_{\partial \Om} \frac{h(z)}{w(z)} \frac{w(z)-w(\xi)}{z-\xi} dz &= \frac{1}{2\pi i} \int\limits_{\partial \Om} \frac{h(z)}{z-\xi} - \frac{h(z)w(\xi)}{w(z)(z-\xi)} dz\\
&= h(\xi) - \frac{1}{2\pi i}\int\limits_{\partial \Om} \frac{h(z)w(\xi)}{w(z)(z-\xi)} dz\\
&= h(\xi) - w(\xi)\frac{1}{2\pi i}\int\limits_{\partial \Om} \frac{h(z)}{w(z)(z-\xi)} dz\\
&= h(\xi) - w(\xi)\frac{1}{2\pi i}\int\limits_{\partial \Om} \frac{h(z)}{w(z)(z-\xi)} dz\\
\end{align*}

Note that at each $a_j$, $w$ vanishes; so, $P(a_j) = h(a_j)$ for all $j$.

Next, the $n+1$st derivative of $P$ vanishes, although I don't have the time to work it out this week.

Thus, $P$ is a polynomial of degree of at most $n$.

\shunt

\end{document}