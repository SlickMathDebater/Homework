
\documentclass[a4paper,12pt]{article}

\usepackage{fancyhdr}
\usepackage{amssymb}
%\usepackage{mathpazo}
\usepackage{mathtools}
\usepackage{amsmath}
\usepackage{slashed}
\usepackage{cancel}
\usepackage[mathscr]{euscript}
\usepackage{MaxPackage} %Note: You need MaxPackage installed or in the same folder as your .tex file or something.

\newcommand{\colorcomment}[2]{\textcolor{#1}{#2}} %First of these leaves in comments. Second one kills them.
%\newcommand{\colorcomment}[2]{}


\pagestyle{fancy}
\lhead{Max Jeter}
\chead{MA530}
\rhead{Assignment 9, Page \thepage}

%Number of Problems		:4
%Clear					:1,2
%Begun					:3
%Not started			:
%Can complete via book	:
%Needs Polish			:4

%Pomodoroes				:8

\begin{document}

{\bf Problem 1:}

Let $f \in \scrO(\C)$ be such that $f(1/\nu) = (-1)^\nu/\nu$ for all $\nu \in \N$. 

Consider the sequences $\anbrack{a_n} = \frac{1}{2n}$ and $\anbrack{b_n} = \frac{1}{2n+1}$. We know $f(a_n)$ and $f(b_n)$, and both sequences converge to $0$. So by the uniqueness theorem, $f$ is uniquely determined by either one of these sequences. Yet, the holomorphic function $g(z) = z$ matches $f(a_n)$ at all points, and $h(z)= -z$ matches $f(b_n)$ at all points; this contradicts the uniqueness theorem.

\shunt

{\bf Problem 2:}

Consider $z^7 -2z^5+6z^3-z+1$.

First, on $\partial D_1(0)$, $\absval{z^7-2z^5+6z^3} \geq 3$ and $\absval{-z+1} \leq 2$. So by Rouche's Theorem, $z^7-2z^5+6z^3$ and $z^7 -2z^5+6z^3-z+1$ have the same number of zeroes on $D_1(0)$.

Now, $z^7-2z^5+6z^3$ has three zeroes (up to multiplicity) on $D_1(0)$; $z^7-2z^5+6z^3 = z^3(z^4-2z^2+6) = z^3(z^2-1+i\sqrt{20}) (z^2-1-i\sqrt{20})$. The zeroes are $0$ (Having multiplicity $3$) and four other complex numbers whose values have absolute value greater than $1$ (this is clear by inspection.)

So $z^7 -2z^5+6z^3-z+1$ has three zeroes on $D_1(0)$.

\shunt

{\bf Problem 3:}

Let $u$ be harmonic on $A_{r,R}$, the open annulus with inner radius $r$ and outer radius $R$.



Consider $u_z$; this is holomorphic on the annulus. So $u_z$ has a local primitive on the annulus.

So there is a $g$ such that $g' - \log{\absval{z}} = u_z$. So $(g + \overline{g} - u - c\log{\absval{z}})_z  = 0$.

Define $v = g + \overline{g} - u - c\log{\absval{z}}$. Then $v_x - iv_y = 0$. But $v_x$ and $v_y$ are both real. So $v_x = v_y = 0$. So $v$ is constant, $D$, on the annulus.

So $0 = g + \overline{g} - u - c\log{\absval{z}} - D$. That is, $u = g + \overline{g} - c\log{\absval{z}} = 2\text{Re}(g) - c\log{\absval{z}}$, which is equivalent to the desired result. 
\shunt

{\bf Problem 4:}

We proceed by induction. (I have a feeling that there's a proof directly from some deep theorem of Algebra, but I don't know any Algebra. :( )

Let $Q$ be a polynomial of degree $2$. Note that in this case, we have the desired result if $\sum\limits_{z_j} \frac{1}{Q'(z_j)} = 0$, where $\{z_j\}$ is the set of zeroes of $Q$. Now, say that $Q(z) = \sum\limits_{k=0}^2 a_kz^k$, so that $Q'(z) = 2a_2 z + a_1$. Then we have:

\begin{align*}
\sum\limits_{z_j} \frac{1}{Q'(z_j)} &=\frac{1}{2a_2z_1+a_1} + \frac{1}{2a_2z_2+a_1}\\
&=\frac{(2a_2z_1+a_1)+(2a_2z_2+a_1)}{(2a_2z_1+a_1)(2a_2z_2+a_1)} \\
&=\frac{2(a_2(z_1+z_2) + a_1)}{(2a_2z_1+a_1)(2a_2z_2+a_1)} \\
\end{align*}

The numerator vanishes, because the zeroes of $Q$ are given by

\begin{align*}
z_j &= \frac{-a_1 \pm \sqrt{a_1^2 -4a_0a_2}}{2a_2}\\
\end{align*}

So that

\begin{align*}
a_2(z_1+z_2) + a_1 &= 0\\ 
\end{align*}

Next, let it be that $\sum\limits_{z_j} \frac{z_j^\al}{Q'(z_j)} = 0$ when the degree of $Q$ is less than $N$ (where $\{z_j\}$ is the set of zeroes of $Q$ and with $\al \leq \text{deg}(Q) - 2$, and where $Q$ only has simple zeroes). Let $Q$ be a polynomial of degree $N$ with only simple zeroes. Then there is a polynomial $q$ with $Q = q(z-z_N)$. Also, $Q' = q + q'(z-z_N)$ by the product rule. Also note that $q(z_j) = 0$ if $z_j$ is any one of $Q$'s zeroes except for $z_N$. So,

\begin{align*}
\sum\limits_{j=1}^N \frac{z_j^\al}{Q'(z_j)} &= \sum\limits_{j=1}^N \frac{z_j^\al}{q(z_j) + (z_j-z_N)q'(z_j)}\\
&= \sum\limits_{j=1}^{N-1} \frac{z_j^\al}{(z_j-z_N)q'(z_j)} + \frac{z_N^\al}{q(z_N)}\\
&= \sum\limits_{j=1}^{N-1} \frac{(z_j-z_N)^\al + P(z_j)}{(z_j-z_N)q'(z_j)} + \frac{z_N^\al}{q(z_N)}\\
&= \sum\limits_{j=1}^{N-1} \frac{(z_j-z_N)^{\al-1}}{q'(z_j)} + \sum\limits_{j=1}^{N-1} \frac{P(z_j)}{(z_j-z_N)q'(z_j)} + \frac{z_N^\al}{q(z_N)}\\
&=\sum\limits_{j=1}^{N-1} \frac{P(z_j)}{(z_j-z_N)q'(z_j)} + \frac{z_N^\al}{q(z_N)}\\
&=0 \text{ (I'm not sure how the last step works out, but this is the idea.) }
\end{align*} %There's some cancellation that works out nicely, I guess?

Where $P(z)$ is some polynomial depending on $z_N$.

So, by induction, $\sum\limits_{z_j} \frac{z_j^\al}{Q'(z_j)} = 0$ (where $\{z_j\}$ is the set of zeroes of $Q$ and with $\al \leq \text{deg}(Q) - 2$, and where $Q$ only has simple zeroes), which is equivalent to our result.

\shunt

\end{document}