
\documentclass[a4paper,12pt]{article}

\usepackage{fancyhdr}
\usepackage{amssymb}
%\usepackage{mathpazo}
\usepackage{mathtools}
\usepackage{amsmath}
\usepackage{slashed}
\usepackage{cancel}
\usepackage[mathscr]{euscript}
\usepackage{MaxPackage} %Note: You need MaxPackage installed or in the same folder as your .tex file or something.

\newcommand{\colorcomment}[2]{\textcolor{#1}{#2}} %First of these leaves in comments. Second one kills them.
%\newcommand{\colorcomment}[2]{}


\pagestyle{fancy}
\lhead{Max Jeter}
\chead{Class}
\rhead{Assignment, Page \thepage}

%Number of Problems		:4
%Clear					:1,2
%Begun					:4
%Not started			:3
%Can complete via book	:
%Needs Polish			:

%Pomodoroes				:6

\begin{document}

{\bf Problem 1:}

Let $f \in \scrO(\C)$ be such that $f(1/\nu) = (-1)^\nu/\nu$ for all $\nu \in \N$. 

Consider the sequences $\anbrack{a_n} = \frac{1}{2n}$ and $\anbrack{b_n} = \frac{1}{2n+1}$. We know $f(a_n)$ and $f(b_n)$, and both sequences converge to $0$. So by the uniqueness theorem, $f$ is uniquely determined by either one of these sequences. Yet, the holomorphic function $g(z) = z$ matches $f(a_n)$ at all points, and $h(z)= -z$ matches $f(b_n)$ at all points; this contradicts the uniqueness theorem.

\shunt

{\bf Problem 2:}

Consider $z^7 -2z^5+6z^3-z+1$.

First, on $\partial D_1(0)$, $\absval{z^7-2z^5+6z^3} \geq 3$ and $\absval{-z+1} \leq 2$. So by Rouche's Theorem, $z^7-2z^5+6z^3$ and $z^7 -2z^5+6z^3-z+1$ have the same number of zeroes on $D_1(0)$.

Now, $z^7-2z^5+6z^3$ has three zeroes (up to multiplicity) on $D_1(0)$; $z^7-2z^5+6z^3 = z^3(z^4-2z^2+6) = z^3(z^2-1+i\sqrt{20}) (z^2-1-i\sqrt{20})$. The zeroes are $0$ (Having multiplicity $3$) and four other complex numbers whose values have absolute value greater than $1$ (this is clear by inspection.)

So $z^7 -2z^5+6z^3-z+1$ has three zeroes on $D_1(0)$.

\shunt

{\bf Problem 3:}

Let $u$ be harmonic on $A_{r,R}$, the open annulus with inner radius $r$ and outer radius $R$.

\shunt

{\bf Problem 4:} %FUCK EVERYTHING YOU JUST WROTE, THERE IS A (STRONG) INDUCTIVE ARGUMENT. (Prove when deg Q = 2, apply product rule, win.)

We proceed by induction. (I have a feeling that there's a proof directly from some deep theorem of Algebra, but I don't know any Algebra. :( )

Let $Q$ be a polynomial of degree $2$. Note that in this case, we have the desired result if $\sum\limits_{z_j} \frac{1}{Q'(z_j)} = 0$, where $\{z_j\}$ is the set of zeroes of $Q$. Now, say that $Q(z) = \sum\limits_{k=0}^2 a_kz^k$, so that $Q'(z) = 2a_2 z + a_1$. Then we have:

\begin{align*}
\sum\limits_{z_j} \frac{1}{Q'(z_j)} &=\frac{1}{2a_2z_1+a_1} + \frac{1}{2a_2z_2+a_1}\\
&=\frac{(2a_2z_1+a_1)+(2a_2z_2+a_1)}{(2a_2z_1+a_1)(2a_2z_2+a_1)} \\
&=\frac{2(a_2(z_1+z_2) + a_1)}{(2a_2z_1+a_1)(2a_2z_2+a_1)} \\
\end{align*}

The numerator vanishes, because the zeroes of $Q$ are given by

\begin{align*}
z_j &= \frac{-a_1 \pm \sqrt{a_1^2 -4a_0a_2}}{2a_2}\\
\end{align*}

So that

\begin{align*}
a_2(z_1+z_2) + a_1 &= 0\\ 
\end{align*}

Next, 

%Let $P$, $Q$ be polynomials, and $Q$ have only simple zeroes, and the degree of $P$ being less than (or equal to) the degree of $Q$ minus $2$. Define $\{z_j\}$ to be the set of zeroes of $Q$.

%Note that we have our desired result if $\sum\limits_{z_j} \frac{z_j^\al}{Q'(z_j)} = 0$ for all $\al \in \N$ with $\al$ less than the degree of $Q$ minus $2$ by simply adding terms.

%Say that $Q(z) = \sum\limits_{k=0}^n a_kz^k$, so that $Q'(z) = \sum\limits_{k=1}^n a_kkz^{k-1}$. Then we have:

%\begin{align*}
%\sum\limits_{z_j} \frac{z_j^\al}{Q'(z_j)} &= \sum\limits_{z_j} \frac{z_j^\al}{\sum\limits_{k=1}^n a_kkz_j^{k-1}}\\
%&= \sum\limits_{z_j} \frac{z_j^{\al+1}}{\sum\limits_{k=1}^n a_kkz_j^{k}}\\
%&=
%\end{align*}

%This vanishes because $\sum\limits_{k=1}^n a_kz_j^k$ vanishes for all $j$. Thus, we have our result.

\shunt

\end{document}