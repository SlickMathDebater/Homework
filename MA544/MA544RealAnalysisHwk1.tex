
\documentclass[a4paper,12pt]{article}

\usepackage{fancyhdr}
\usepackage{amssymb}
%\usepackage{mathpazo}
\usepackage{mathtools}
\usepackage{amsmath}
\usepackage{slashed}
\usepackage[mathscr]{euscript}

\newcommand{\tab}{\hspace{4mm}} %Spacing aliases
\newcommand{\shunt}{\vspace{20mm}}

\newcommand{\sd}{\partial} %Squiggle d

\newcommand{\absval}[1]{\lvert #1 \rvert}
\newcommand{\norm}[1]{\|#1\|}
\newcommand{\anbrack}[1]{\left\langle #1 \right\rangle}



\newcommand{\al}{\alpha} %Steal ALL of Dr. Kable's Aliases! MWAHAHAHAHA!
\newcommand{\be}{\beta}
\newcommand{\ga}{\gamma}
\newcommand{\Ga}{\Gamma}
\newcommand{\de}{\delta}
\newcommand{\De}{\Delta}
\newcommand{\ep}{\epsilon}
\newcommand{\vep}{\varepsilon}
\newcommand{\ze}{\zeta}
\newcommand{\et}{\eta}
\newcommand{\tha}{\theta}
\newcommand{\vtha}{\vartheta}
\newcommand{\Tha}{\Theta}
\newcommand{\io}{\iota}
\newcommand{\ka}{\kappa}
\newcommand{\la}{\lambda}
\newcommand{\La}{\Lambda}
\newcommand{\rh}{\rho}
\newcommand{\si}{\sigma}
\newcommand{\Si}{\Sigma}
\newcommand{\ta}{\tau}
\newcommand{\ups}{\upsilon}
\newcommand{\Ups}{\Upsilon}
\newcommand{\ph}{\phi}
\newcommand{\Ph}{\Phi}
\newcommand{\vph}{\varphi}
\newcommand{\vpi}{\varpi}
\newcommand{\ch}{\chi}
\newcommand{\ps}{\psi}
\newcommand{\Ps}{\Psi}
\newcommand{\om}{\omega}
\newcommand{\Om}{\Omega}

\newcommand{\bbA}{\mathbb{A}}
\newcommand{\A}{\mathbb{A}}
\newcommand{\bbB}{\mathbb{B}}
\newcommand{\bbC}{\mathbb{C}}
\newcommand{\C}{\mathbb{C}}
\newcommand{\bbD}{\mathbb{D}}
\newcommand{\bbE}{\mathbb{E}}
\newcommand{\bbF}{\mathbb{F}}
\newcommand{\bbG}{\mathbb{G}}
\newcommand{\G}{\mathbb{G}}
\newcommand{\bbH}{\mathbb{H}}
\newcommand{\HH}{\mathbb{H}}
\newcommand{\bbI}{\mathbb{I}}
\newcommand{\I}{\mathbb{I}}
\newcommand{\bbJ}{\mathbb{J}}
\newcommand{\bbK}{\mathbb{K}}
\newcommand{\bbL}{\mathbb{L}}
\newcommand{\bbM}{\mathbb{M}}
\newcommand{\bbN}{\mathbb{N}}
\newcommand{\N}{\mathbb{N}}
\newcommand{\bbO}{\mathbb{O}}
\newcommand{\bbP}{\mathbb{P}}
\newcommand{\PP}{\mathbb{P}}
\newcommand{\bbQ}{\mathbb{Q}}
\newcommand{\Q}{\mathbb{Q}}
\newcommand{\bbR}{\mathbb{R}}
\newcommand{\R}{\mathbb{R}}
\newcommand{\bbS}{\mathbb{S}}
\newcommand{\bbT}{\mathbb{T}}
\newcommand{\bbU}{\mathbb{U}}
\newcommand{\bbV}{\mathbb{V}}
\newcommand{\bbW}{\mathbb{W}}
\newcommand{\bbX}{\mathbb{X}}
\newcommand{\bbY}{\mathbb{Y}}
\newcommand{\bbZ}{\mathbb{Z}}
\newcommand{\Z}{\mathbb{Z}}

\newcommand{\scrA}{\mathcal{A}}
\newcommand{\scrB}{\mathcal{B}}
\newcommand{\scrC}{\mathcal{C}}
\newcommand{\scrD}{\mathcal{D}}
\newcommand{\scrE}{\mathcal{E}}
\newcommand{\scrF}{\mathcal{F}}
\newcommand{\scrG}{\mathcal{G}}
\newcommand{\scrH}{\mathcal{H}}
\newcommand{\scrI}{\mathcal{I}}
\newcommand{\scrJ}{\mathcal{J}}
\newcommand{\scrK}{\mathcal{K}}
\newcommand{\scrL}{\mathcal{L}}
\newcommand{\scrM}{\mathcal{M}}
\newcommand{\scrN}{\mathcal{N}}
\newcommand{\scrO}{\mathcal{O}}
\newcommand{\scrP}{\mathcal{P}}
\newcommand{\scrQ}{\mathcal{Q}}
\newcommand{\scrR}{\mathcal{R}}
\newcommand{\scrS}{\mathcal{S}}
\newcommand{\scrT}{\mathcal{T}}
\newcommand{\scrU}{\mathcal{U}}
\newcommand{\scrV}{\mathcal{V}}
\newcommand{\scrW}{\mathcal{W}}
\newcommand{\scrX}{\mathcal{X}}
\newcommand{\scrY}{\mathcal{Y}}
\newcommand{\scrZ}{\mathcal{Z}}

\DeclarePairedDelimiter\ceil{\lceil}{\rceil}
\DeclarePairedDelimiter\floor{\lfloor}{\rfloor}

\newcommand{\colorcomment}[2]{\textcolor{#1}{#2}} %First of these leaves in comments. Second one kills them.
%\newcommand{\colorcomment}[2]{}


\pagestyle{fancy}
\lhead{Max Jeter}
\rhead{MA503, Assignment 1, Page \thepage}

\begin{document}

{\bf Problem 1:}

Let $f: X \to Y$ be a function such that there is a function $g: Y \to X$ such that $f \circ g$ is the identity on $Y$.

Then for all $y \in Y$, $f(g(y)) = y$.

So for all $y \in Y$, there is an $x \in X$ such that $f(x) = y$ ($x = f(y)$).

So $f$ is onto if there is a function $g: Y \to X$ such that $f \circ g$ is the identity on $Y$.

\shunt

{\bf Problem 2:}

Let $\scrC$ be an arbitrary collection of sets.

First, we show that $(\bigcup\limits_{C \in \scrC} C)^c = \bigcap\limits_{C \in \scrC} C^c$.

Let $x \in (\bigcup\limits_{C \in \scrC} C)^c$.

\tab Then $x \notin (\bigcup\limits_{C \in \scrC} C)$.

\tab So $x$ is not in $C$ for any $C \in \scrC$.

\tab So $x$ is in $C^c$ for all $C \in \scrC$.

\tab So $x \in \bigcap\limits_{C \in \scrC} C^c$.

This means that $(\bigcup\limits_{C \in \scrC} C)^c \subset \bigcap\limits_{C \in \scrC} C^c$.

Next, let $x \in \bigcap\limits_{C \in \scrC} C^c$.

\tab So $x$ is in $C^c$ for all $C \in \scrC$.

\tab So $x$ is not in $C$ for any $C \in \scrC$.

\tab Then $x \notin (\bigcup\limits_{C \in \scrC} C)$.

\tab Thus, $x \in (\bigcup\limits_{C \in \scrC} C)^c$.

This means that $(\bigcup\limits_{C \in \scrC} C)^c \supset \bigcap\limits_{C \in \scrC} C^c$.

So $(\bigcup\limits_{C \in \scrC} C)^c = \bigcap\limits_{C \in \scrC} C^c$.

Next, we show that $(\bigcap\limits_{C \in \scrC} C)^c = \bigcup\limits_{C \in \scrC} C^c$.

Let $x \in (\bigcap\limits_{C \in \scrC} C)^c$.

\tab Then $x \notin (\bigcap\limits_{C \in \scrC} C)$.

\tab So $x$ is not in $C$ for all $C \in \scrC$.

\tab So $x$ is in $C^c$ for some $C \in \scrC$.

\tab So $x \in \bigcup\limits_{C \in \scrC} C^c$.

This means that $(\bigcap\limits_{C \in \scrC} C)^c \subset \bigcup\limits_{C \in \scrC} C^c$.

Next, let $x \in \bigcup\limits_{C \in \scrC} C^c$.

\tab So $x$ is in $C^c$ for some $C \in \scrC$.

\tab So $x$ is not in $C$ for all $C \in \scrC$.

\tab Then $x \notin (\bigcap\limits_{C \in \scrC} C)$.

\tab Thus, $x \in (\bigcap\limits_{C \in \scrC} C)^c$.

This means that $(\bigcap\limits_{C \in \scrC} C)^c \supset \bigcup\limits_{C \in \scrC} C^c$.

Thus, $(\bigcap\limits_{C \in \scrC} C)^c = \bigcup\limits_{C \in \scrC} C^c$.

\shunt

{\bf Problem 3:}

Let $\scrC$ be a collection of sets, and $E$ be in the $\sigma$-algebra generated by $\scrC$.

Now, let $\scrA$ be the $\sigma$-algebra generated by $\scrC$ and let $\scrB = \bigcup\limits_{\scrC_0 \subset \scrC} \sigma(\scrC_0)$, where each $\scrC_0$ is countable and $\sigma(\scrC)$ is the $\sigma$-algebra generated by $\scrC$. We proceed by showing that $\scrA = \scrB$.

Now, let $A \in \scrA$.

\tab Then $A$ is in the smallest $\sigma$-algebra containing $\scrC$, meaning that $A$ is in every $\sigma$-algebra contianing $\scrC$.

\tab We show that $A \in \scrB$ by showing that $\scrB$ is a $\sigma$-algebra containing $\scrC$.

\tab \tab Let $B \in \scrB$.

\tab \tab Then $B$ is in the $\sigma$-algebra generated by some countable subcollection of $\scrC$.

\tab \tab So $B^c$ is in the $\sigma$-algebra generated by some countable subcollection of $\scrC$.

\tab \tab So $B^c \in \scrB$.

\tab \tab Next, let $\{B_i\}_{i \in \N} \subset \scrB$.

\tab \tab Then each $B_i$ is in the $\sigma$-algebra generated by some countable subcollection of $\scrC$.

\tab \tab So every $B_i$ is in the $\sigma$-algebra generated by some countable subcollection of $\scrC$. (The union of all of those countable subcollections in the previous lines is a countable union of countable sets...which is countable.)

\tab \tab So $\bigcup\limits_{i \in \N} \subset \scrB$.

\tab So $\scrB$ is a $\sigma$-algebra.

\tab Note that $\scrB$ contains $\scrC$ because each $C \in \scrC$ is in a countable subcollection of $\scrC$. (If this is not clear, try $\{C\}$.)

\tab So $\scrB$ is a $\sigma$-algebra containing $\scrC$.

\tab So $A \in \scrB$.

So, $\scrA \subset \scrB$.

Next, let $B \in \scrB$.

\tab Then $B$ is in the $\sigma$-algebra generated by some countable subcolection of $\scrC$.

\tab So $B$ is in the $\sigma$-algebra generated by $\scrC$.

\tab So $B \in \scrA$.

So, $\scrA \supset \scrB$.

Thus, $\scrA = \scrB$.

Now, this means that any element of the $\sigma$-algebra generated by $\scrC$ is in $\scrB$. In other words, it is in the $\sigma$-algebra generated by some countable subcollection of $\scrC$.

So for all $E$ there is a countable subcollection, $\scrC_0$ of $\scrC$ such that $E$ is in the $\sigma$-algebra generated by $\scrC_0$.

\shunt

{\bf Problem 4:}

Let $\anbrack{x_n}$ be a sequence of real numbers.

If $\lim\limits_{n \to \infty} x_n$ exists, then $\anbrack{x_n}$ is Cauchy.

\tab Let $\ep >0$. Then there is an $N \in \N$ such that for all $n \geq N$, $\absval{x_n - L} < \frac{\ep}{2}$, where $L = \lim\limits_{n \to \infty} x_n$.

\tab Now, if $n,m \geq N$, then $\absval{x_n - L} < \frac{\ep}{2}$ and $\absval{x_m - L} < \frac{\ep}{2}$.

\tab By the triangle inequality,
\begin{align*}
\absval{x_n - x_m} &\leq \absval{x_n - L} + \absval{x_m -L} \\
&< \frac{\ep}{2} + \frac{\ep}{2} \\
&= \ep.
\end{align*}

\tab So for all $\ep >0$ there is an $N \in \N$ such that $n,m \geq N$ implies that $\absval{x_n - x_m} < \ep$. That is, $\anbrack{x_n}$ is Cauchy.

Next, if $\anbrack{x_n}$ is Cauchy, then $\lim\limits_{n \to \infty} x_n$ exists.

\tab Let $\ep >0$. Then there is an $N \in \N$ such that for all $n,m \geq N$, $\absval{x_n - x_m} < \frac{\ep}{2}$.

\tab This means that for all $n \geq N$, $x_n$ is within the closed interval $[x_N - \frac{\ep}{2}, x_N +\frac{\ep}{2}]$.

\tab So $\anbrack{x_n}$ has a converging subsequence. Say that the converging subsequence converges to $L$.

\tab Now, $L \in [x_N - \frac{\ep}{2}, x_N +\frac{\ep}{2}]$. (Closed interval is closed.)

\tab This means that for all $n \geq N$, $\absval{x_n - L} < \ep$ (both $x_n$ and $L$ are in a closed ball of radius $\ep$.)

\tab So for all $\ep >0$ there is an $N \in \N$ such that for all $n \geq N$, $\absval{x_n - L} < \ep$ for some $L \in \R$. That is, $\anbrack{x_n}$ converges to $L$; $\lim\limits_{n \to \infty} x_n$ exists.

Therefore $\lim\limits_{n \to \infty} x_n$ exists if and only if $\anbrack{x_n}$ is Cauchy.

\shunt

{\bf Problem 5:}

Let $\anbrack{x_n}$ be a sequence of real numbers, with $\liminf(x_n)$ and $\limsup(x_n)$ both existing.

Then $\liminf(x_n)$ is the infimum of the set of limit points of $\anbrack{x_n}$.

Also, $\limsup(x_n)$ is the supremum of the set of limit points of $\anbrack{x_n}$.

Because $inf(A) \leq sup(A)$ for all nonempty $A \subset \R$, this means that $\liminf(x_n) \leq \limsup(x_n)$.

Moving on, let $\lim\limits_{n \to \infty} x_n = L$.

\tab Then all subsequences of $\anbrack{x_n}$ converge to $L$. So the infimum and the supremum of the set of limit points of $\anbrack{x_n}$ are both $L$ (the infimum and supremum of $\{L\}$ is $L$). So $\limsup(x_n) = \liminf(x_n) = L$.

Next, let $\liminf(x_n) = \limsup(x_n) = L$.

\tab Then if a subsequence of $\anbrack{x_n}$ converges, it converges to $L$; $L$ is the only limit point of $\anbrack{x_n}$.

\tab Moreover, $\anbrack{x_n}$ is bounded. Else, there is an unbounded subsequence of $\anbrack{x_n}$, and so either $\limsup(x_n)$ or $\liminf(x_n)$ is $\infty$ or $-\infty$.

\tab Now, let $\ep>0$. There is a last $N \in \N$ such that $\absval{x_N - L} \geq \ep$.

\tab Else, there are infinitely many $n \in \N$ such that $\absval{x_N - L} \geq \ep$. These create a subsequence of $\anbrack{x_n}$. This subsequence is a sequence in its own right, it is bounded, it has a converging subsequence. This converging subsequence cannot converge to $L$; each point of that subsequence is at least $\ep$ away from $L$. Thus, $\anbrack{x_n}$ has a subsequence converging to something other than $L$...which means that either $\liminf(x_n) \neq L$ or $\limsup(x_n) \neq L$. Either way, that contradicts our original assumption.

\tab So, for all $\ep >0$ there is an $N \in \N$ such that for all $n \geq N$, $\absval{x_n - L} < \ep$. That is, $\lim (x_n) = L$.

So $\liminf(x_n) = \limsup(x_n) = L$ if and only if $\lim (x_n) = L$. 

\shunt

{\bf Problem 6:}

It is a well-known fact that if there is an injection, $f: X \to Y$ and an injection $g: Y \to X$, then there is a bijection $h: X \to Y$. We exploit this fact.

Denote the Cantor set by $C$. A quick note: $C$ is closed, as it is the complement of an open set. (It is usually defined as the complement of an infinite union of open intervals. This union of open intervals is open...and thus, its complement is closed.)

There is an injection $f: C \to [0,1]$ given by $f(x) = x$.

Next, there is an injection $g: [0,1] \to C$ given as follows:

\tab Let $x \in [0,1]$. We know that $x$ has a binary expansion, $0.a_1a_2 \ldots$. (Even in the case of $x=1$, this expansion can be $0.11111\ldots$.)

\tab Note that this binary expansion need not be unique. We just choose one (which we can do, by the Axiom of Choice).

\tab We next define a sequence of nested closed subsets of $C$ as follows:

\tab \tab If $a_1= 0$, set $C_1(x) = [0,1/3]$. Else, set $C_1(x) = [2/3,1]$.

\tab \tab If $a_2 = 0$, set $C_2(x)$ equal to the lower third of $C_1(x)$. Else, set $C_2(x)$ equal to the upper third of $C_1(x)$.

\tab \tab For all $n > 1$, set $C_n(x)$ equal to the lower third of $C_{n-1}(x)$ if $a_n = 0$, else set $C_n(x)$ equal to the upper third of $C_{n-1}(x)$.

\tab The sequence $\anbrack{C_n}$ is a sequence of nested, closed subsets of $C$ whose diameter approaches $0$. So, there is a unique point,$y$, in the intersection of these closed subsets.

\tab We define $g(x) = y$.

\tab Now, if $a, b \in [0,1]$ and $a \neq b$, then $g(a) \neq g(b)$:

\tab \tab Because $a \neq b$, $a$ and $b$ have two different binary expansions, $0.a_1a_2 \ldots $ and $0.b_1b_2 \ldots$.

\tab \tab So there is an index, $i$, with $a_i \neq b_i$.

\tab \tab This means that $g(a)$ and $g(b)$ are contained in disjoint closed intervals, $C_i(a)$ and $C_i(b)$. So $g(a) \neq g(b)$.

We have an injection $f: C \to [0,1]$ and an injection $g: [0,1] \to C$, so there is a bijection $h: C \to [0,1]$.

\shunt

{\bf Problem 7:}

First, note that for any subset, $A$, of $\R$, the set of accumulation points, $A'$, is a subset of the closure of $A$. That is $A' \subset \overline{A}$.

The Cantor Set is closed. So, $C' \subset C$.

Now, let $x \in C$.

\tab Then $x \in [0,1/3]$ or $x \in [2/3, 1]$. Define $C_1$ to be the interval that $x$ is in.

\tab Similarly, $x$ is in either the lower third or the upper third of $C_1$. Define $C_2$ to be the third of $C_1$ that $x$ is in.

\tab For each $n \in \N$, define $C_n$ to be the third of $C_{n-1}$ that $x$ is in.

\tab Now, for each $n \in \N$, pick an $x_n \in C_n$ such that $x_n \neq x$.

\tab Note that the $C_n$s are nested, closed subsets of $\R$ whose diameter approaches $0$. This means that their intersection has a unique point. Because $x \in C_n$ for all $n \in \N$, this means that $x$ is that unique point.

\tab Moreover, it is clear that $\anbrack{x_n} \to x$. (Should I break out my epsilons, or is it OK to just state this?)

\tab So there's a sequence $\anbrack{x_n}$ in $C$ such that $\lim\limits{n \to \infty} x_n \to x$ and $x_n \neq x$. So $x \in C'$.

So $C \subset C'$.

So $C = C'$. That is, the set of accumulation points of the Cantor Set is the Cantor Set itself.

\shunt

\end{document}