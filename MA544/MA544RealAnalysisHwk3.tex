
\documentclass[a4paper,12pt]{article}

\usepackage{fancyhdr}
\usepackage{amssymb}
%\usepackage{mathpazo}
\usepackage{mathtools}
\usepackage{amsmath}
\usepackage{slashed}
\usepackage[mathscr]{euscript}

\newcommand{\tab}{\hspace{4mm}} %Spacing aliases
\newcommand{\shunt}{\vspace{20mm}}

\newcommand{\sd}{\partial} %Squiggle d

\newcommand{\absval}[1]{\lvert #1 \rvert}
\newcommand{\norm}[1]{\|#1\|}
\newcommand{\anbrack}[1]{\left\langle #1 \right\rangle}



\newcommand{\al}{\alpha} %Steal ALL of Dr. Kable's Aliases! MWAHAHAHAHA!
\newcommand{\be}{\beta}
\newcommand{\ga}{\gamma}
\newcommand{\Ga}{\Gamma}
\newcommand{\de}{\delta}
\newcommand{\De}{\Delta}
\newcommand{\ep}{\epsilon}
\newcommand{\vep}{\varepsilon}
\newcommand{\ze}{\zeta}
\newcommand{\et}{\eta}
\newcommand{\tha}{\theta}
\newcommand{\vtha}{\vartheta}
\newcommand{\Tha}{\Theta}
\newcommand{\io}{\iota}
\newcommand{\ka}{\kappa}
\newcommand{\la}{\lambda}
\newcommand{\La}{\Lambda}
\newcommand{\rh}{\rho}
\newcommand{\si}{\sigma}
\newcommand{\Si}{\Sigma}
\newcommand{\ta}{\tau}
\newcommand{\ups}{\upsilon}
\newcommand{\Ups}{\Upsilon}
\newcommand{\ph}{\phi}
\newcommand{\Ph}{\Phi}
\newcommand{\vph}{\varphi}
\newcommand{\vpi}{\varpi}
\newcommand{\ch}{\chi}
\newcommand{\ps}{\psi}
\newcommand{\Ps}{\Psi}
\newcommand{\om}{\omega}
\newcommand{\Om}{\Omega}

\newcommand{\bbA}{\mathbb{A}}
\newcommand{\A}{\mathbb{A}}
\newcommand{\bbB}{\mathbb{B}}
\newcommand{\bbC}{\mathbb{C}}
\newcommand{\C}{\mathbb{C}}
\newcommand{\bbD}{\mathbb{D}}
\newcommand{\bbE}{\mathbb{E}}
\newcommand{\bbF}{\mathbb{F}}
\newcommand{\bbG}{\mathbb{G}}
\newcommand{\G}{\mathbb{G}}
\newcommand{\bbH}{\mathbb{H}}
\newcommand{\HH}{\mathbb{H}}
\newcommand{\bbI}{\mathbb{I}}
\newcommand{\I}{\mathbb{I}}
\newcommand{\bbJ}{\mathbb{J}}
\newcommand{\bbK}{\mathbb{K}}
\newcommand{\bbL}{\mathbb{L}}
\newcommand{\bbM}{\mathbb{M}}
\newcommand{\bbN}{\mathbb{N}}
\newcommand{\N}{\mathbb{N}}
\newcommand{\bbO}{\mathbb{O}}
\newcommand{\bbP}{\mathbb{P}}
\newcommand{\PP}{\mathbb{P}}
\newcommand{\bbQ}{\mathbb{Q}}
\newcommand{\Q}{\mathbb{Q}}
\newcommand{\bbR}{\mathbb{R}}
\newcommand{\R}{\mathbb{R}}
\newcommand{\bbS}{\mathbb{S}}
\newcommand{\bbT}{\mathbb{T}}
\newcommand{\bbU}{\mathbb{U}}
\newcommand{\bbV}{\mathbb{V}}
\newcommand{\bbW}{\mathbb{W}}
\newcommand{\bbX}{\mathbb{X}}
\newcommand{\bbY}{\mathbb{Y}}
\newcommand{\bbZ}{\mathbb{Z}}
\newcommand{\Z}{\mathbb{Z}}

\newcommand{\scrA}{\mathcal{A}}
\newcommand{\scrB}{\mathcal{B}}
\newcommand{\scrC}{\mathcal{C}}
\newcommand{\scrD}{\mathcal{D}}
\newcommand{\scrE}{\mathcal{E}}
\newcommand{\scrF}{\mathcal{F}}
\newcommand{\scrG}{\mathcal{G}}
\newcommand{\scrH}{\mathcal{H}}
\newcommand{\scrI}{\mathcal{I}}
\newcommand{\scrJ}{\mathcal{J}}
\newcommand{\scrK}{\mathcal{K}}
\newcommand{\scrL}{\mathcal{L}}
\newcommand{\scrM}{\mathcal{M}}
\newcommand{\scrN}{\mathcal{N}}
\newcommand{\scrO}{\mathcal{O}}
\newcommand{\scrP}{\mathcal{P}}
\newcommand{\scrQ}{\mathcal{Q}}
\newcommand{\scrR}{\mathcal{R}}
\newcommand{\scrS}{\mathcal{S}}
\newcommand{\scrT}{\mathcal{T}}
\newcommand{\scrU}{\mathcal{U}}
\newcommand{\scrV}{\mathcal{V}}
\newcommand{\scrW}{\mathcal{W}}
\newcommand{\scrX}{\mathcal{X}}
\newcommand{\scrY}{\mathcal{Y}}
\newcommand{\scrZ}{\mathcal{Z}}

\DeclarePairedDelimiter\ceil{\lceil}{\rceil}
\DeclarePairedDelimiter\floor{\lfloor}{\rfloor}

\newcommand{\colorcomment}[2]{\textcolor{#1}{#2}} %First of these leaves in comments. Second one kills them.
%\newcommand{\colorcomment}[2]{}


\pagestyle{fancy}
\lhead{Max Jeter}
\rhead{MA544, Assignment 3, Page \thepage}

\begin{document}

I don't know if we covered the $\rh$ metric ($\rh(f,g) = \sup(\absval{f(x)-g(x)}: x \in X)$, where $X$ is the domain of $f$ and $g$), but I'm using it because it's nice and I like it.

{\bf Problem 1:}

Consider the sequence of functions $f_n:(-\pi/2,\pi/2) \to \R $ given by $ f_n(x) =\sec(x) + 1/n$.

It is rather clear that this sequence of functions converges uniformly (to $\sec(x)$).

However, the sequence of functions $\anbrack{f_n^2}$ fails to converve uniformly:

\tab For each $n \in \N$, $f_n^2(x) = \sec(x)^2 + 2\sec(x)/n + 1/n^2$. It is rather clear that $\anbrack{f_n^2}$ converges pointwise to $\sec(x)^2$. However, $\anbrack{f_n^2}$ does not converge uniformly:

\tab \tab %For every \ep >0 there is an x with |sec(x)^2 - f_n(x)^2| \geq \ep. So we can't have uniform convergence.

So $\anbrack{f_n}$ converges uniformly on $(-\pi/2, \pi/2)$, but $\anbrack{f_n^2}$ doesn't. This satisfies the problem.

\shunt

{\bf Problem 2:}

Note: After finishing this problem, I noticed that this follows immediately from a fragment of the proof of Arzela-Ascoli. I prefer this proof, as it is smoother, but it is important to note that such a thing is possible.

Let $\anbrack{f_n}$ be an equicontinuous sequence of functions on a compact set, $K$, with $\anbrack{f_n}$ converging pointwise to some function, say $f$.

By the Arzela-Ascoli theorem, we know that $\anbrack{f_n}$ has some subsequence that uniformly converges to some function. We know that this function must be $f$: if a subsequence of functions converges uniformly to $f$, it converges pointwise to $f$. If a sequence of functions converges pointwise to a function, $f$, then all of its subsequences converge to $f$. So if a sequence of functions converges pointwise to $f$, then any subsequence of functions that converges uniformly to a function must converge uniformly to $f$.

Now, consider such a converging subsequence, $\anbrack{f_{n_j}}$.

\tab Let $\ep >0$. There is a $J \in \N$ such that for all $j \geq J$, $\rh(f_{n_j}, f) < \ep/3$.

\tab In addition, by equicontinuity, there is a $\de >0$ such that for all $n \in \N$, $x,y \in K$, $d(x,y) < \de$ implies that $d(f_n(x),f_n(y)) < \ep/3$.

\tab We know that compact sets are totally bounded. (If this is not clear, consider a career in pastry making.)

\tab So, let $F$ be a finite collection of points of $K$ such that for all $x \in K$, $d(x,y) < \de$ for some $y \in F$.

\tab Now, for each $y \in F$, there is an $N_y \in \N$ such that for all $n \geq N_y$, $d(f_n(y),f(y)) < \ep/3$. 

\tab Define $N = \max(N_y,n_J)$. 

\tab Now, for all $n \geq N$, and for all $x \in K$, we have:

\begin{align*}
The proof
\end{align*}

So for all $\ep >0$ there is an $N \in \N$ such that for all $n \geq N$, for all $x \in K$, $d(f_n(x),f(x)) < \ep$. That is, $f_n$ converges uniformly to $f$.

To summarize, if $\anbrack{f_n}$ is an equicontinuous sequence of functions on a compact set, $K$, with $\anbrack{f_n}$ converging pointwise, then $\anbrack{f_n}$ converges uniformly.

\shunt

{\bf Problem 3:}

Let $\anbrack{f_n}$ be a uniformly bounded sequence of functions that are Riemann-integrable on $[a,b]$. Set

\begin{displaymath}
F_n(x) = \int\limits_a^x f_n(t)dt
\end{displaymath}

Now, the set of $F_n$s are equicontinuous: %Prove it. It comes from uniform boundedness.

In addition, the $F_n$s are defined on $[a,b]$, which is a compact space. By Arzela-Ascoli, there is a subsequence $\anbrack{F_{n_j}}$ that converges uniformly on $[a,b]$.
\shunt

{\bf Problem 4:}

\shunt

{\bf Problem 5:}

\shunt

{\bf Problem 6:}

\shunt

\end{document}