
\documentclass[a4paper,12pt]{article}

\usepackage{fancyhdr}
\usepackage{amssymb}
%\usepackage{mathpazo}
\usepackage{mathtools}
\usepackage{amsmath}
\usepackage{slashed}
\usepackage{cancel}
\usepackage[mathscr]{euscript}

\newcommand{\tab}{\hspace{4mm}} %Spacing aliases
\newcommand{\shunt}{\vspace{20mm}}

\newcommand{\sd}{\partial} %Squiggle d

\newcommand{\absval}[1]{\left\lvert #1 \right\rvert}
\newcommand{\norm}[1]{\|#1\|}
\newcommand{\anbrack}[1]{\left\langle #1 \right\rangle}



\newcommand{\al}{\alpha} %Steal ALL of Dr. Kable's Aliases! MWAHAHAHAHA!
\newcommand{\be}{\beta}
\newcommand{\ga}{\gamma}
\newcommand{\Ga}{\Gamma}
\newcommand{\de}{\delta}
\newcommand{\De}{\Delta}
\newcommand{\ep}{\epsilon}
\newcommand{\vep}{\varepsilon}
\newcommand{\ze}{\zeta}
\newcommand{\et}{\eta}
\newcommand{\tha}{\theta}
\newcommand{\vtha}{\vartheta}
\newcommand{\Tha}{\Theta}
\newcommand{\io}{\iota}
\newcommand{\ka}{\kappa}
\newcommand{\la}{\lambda}
\newcommand{\La}{\Lambda}
\newcommand{\rh}{\rho}
\newcommand{\si}{\sigma}
\newcommand{\Si}{\Sigma}
\newcommand{\ta}{\tau}
\newcommand{\ups}{\upsilon}
\newcommand{\Ups}{\Upsilon}
\newcommand{\ph}{\phi}
\newcommand{\Ph}{\Phi}
\newcommand{\vph}{\varphi}
\newcommand{\vpi}{\varpi}
\newcommand{\ch}{\chi}
\newcommand{\ps}{\psi}
\newcommand{\Ps}{\Psi}
\newcommand{\om}{\omega}
\newcommand{\Om}{\Omega}

\newcommand{\bbA}{\mathbb{A}}
\newcommand{\A}{\mathbb{A}}
\newcommand{\bbB}{\mathbb{B}}
\newcommand{\bbC}{\mathbb{C}}
\newcommand{\C}{\mathbb{C}}
\newcommand{\bbD}{\mathbb{D}}
\newcommand{\bbE}{\mathbb{E}}
\newcommand{\bbF}{\mathbb{F}}
\newcommand{\bbG}{\mathbb{G}}
\newcommand{\G}{\mathbb{G}}
\newcommand{\bbH}{\mathbb{H}}
\newcommand{\HH}{\mathbb{H}}
\newcommand{\bbI}{\mathbb{I}}
\newcommand{\I}{\mathbb{I}}
\newcommand{\bbJ}{\mathbb{J}}
\newcommand{\bbK}{\mathbb{K}}
\newcommand{\bbL}{\mathbb{L}}
\newcommand{\bbM}{\mathbb{M}}
\newcommand{\bbN}{\mathbb{N}}
\newcommand{\N}{\mathbb{N}}
\newcommand{\bbO}{\mathbb{O}}
\newcommand{\bbP}{\mathbb{P}}
\newcommand{\PP}{\mathbb{P}}
\newcommand{\bbQ}{\mathbb{Q}}
\newcommand{\Q}{\mathbb{Q}}
\newcommand{\bbR}{\mathbb{R}}
\newcommand{\R}{\mathbb{R}}
\newcommand{\bbS}{\mathbb{S}}
\newcommand{\bbT}{\mathbb{T}}
\newcommand{\bbU}{\mathbb{U}}
\newcommand{\bbV}{\mathbb{V}}
\newcommand{\bbW}{\mathbb{W}}
\newcommand{\bbX}{\mathbb{X}}
\newcommand{\bbY}{\mathbb{Y}}
\newcommand{\bbZ}{\mathbb{Z}}
\newcommand{\Z}{\mathbb{Z}}

\newcommand{\scrA}{\mathcal{A}}
\newcommand{\scrB}{\mathcal{B}}
\newcommand{\scrC}{\mathcal{C}}
\newcommand{\scrD}{\mathcal{D}}
\newcommand{\scrE}{\mathcal{E}}
\newcommand{\scrF}{\mathcal{F}}
\newcommand{\scrG}{\mathcal{G}}
\newcommand{\scrH}{\mathcal{H}}
\newcommand{\scrI}{\mathcal{I}}
\newcommand{\scrJ}{\mathcal{J}}
\newcommand{\scrK}{\mathcal{K}}
\newcommand{\scrL}{\mathcal{L}}
\newcommand{\scrM}{\mathcal{M}}
\newcommand{\scrN}{\mathcal{N}}
\newcommand{\scrO}{\mathcal{O}}
\newcommand{\scrP}{\mathcal{P}}
\newcommand{\scrQ}{\mathcal{Q}}
\newcommand{\scrR}{\mathcal{R}}
\newcommand{\scrS}{\mathcal{S}}
\newcommand{\scrT}{\mathcal{T}}
\newcommand{\scrU}{\mathcal{U}}
\newcommand{\scrV}{\mathcal{V}}
\newcommand{\scrW}{\mathcal{W}}
\newcommand{\scrX}{\mathcal{X}}
\newcommand{\scrY}{\mathcal{Y}}
\newcommand{\scrZ}{\mathcal{Z}}

\DeclarePairedDelimiter\ceil{\lceil}{\rceil}
\DeclarePairedDelimiter\floor{\lfloor}{\rfloor}

\newcommand{\colorcomment}[2]{\textcolor{#1}{#2}} %First of these leaves in comments. Second one kills them.
%\newcommand{\colorcomment}[2]{}


\pagestyle{fancy}
\lhead{Max Jeter}
\rhead{MA544, Assignment 4, Page \thepage}

\begin{document}

Note: for all of the following, $l(I)$ is the length of the interval, $I$.

{\bf Problem 1:} %Incomplete: find the correct functions.

Let $f$ be a measurable function defined on $[a,b]$ and assume that $f$ takes infinite values only on a set of measure zero. 

Let $\ep >0$. There is an $M >0$ and a set, $A$, of measure less than $\ep/3$ such that $\absval{f} < M$ everywhere except $M$. Otherwise, for all $M >0$ and all $A \subset [a,b]$ with measure less than $\ep/3$, $\absval{f} \geq M$. That means that $f$ takes infinite values everywhere on $[a,b]$, which is a blatant contradiction.

So, pick such an $M$ and such an $A$. Define $F = [a,b] \setminus A$. 

Now, we know that $f$ is measurable; for each $\al, \be \in \R$, $\{x: \al \leq f(x) \leq \be\}$ is measurable. We can consider the partition of $[-M,M]$ given by $[-M, -M + \ep)$, $[-M+\ep, -M+2\ep)$, $\ldots$ $[-M + n\ep, M]$. For each interval, $I$, in that partition, the set $\{x: f(x) \in I\}$ and define the simple function $\psi: [a,b] \to \R$ by

\begin{displaymath}
   \psi(x) = \left\{
     \begin{array}{lr}
       -M+i\ep & : f(x) \in [-M+i\ep, -M+(i+1)\ep) \text{ and } x \in F\\
       0 & : x \notin F
     \end{array}
   \right.
\end{displaymath} 

(I notice that there's a tiny little hitch on the last interval, and it doesn't impact the proof at all.) It is clear that $\psi$ is measurable, only takes finitely many values and has the properties $\psi(x) \in [-M,M]$ for all $x \in F$ and $\absval{f(x) - \psi(x)} < \ep$ if $x \in F$. 

Now, we can represent $\psi$ as the sum $\sum\limits_{i=0}^n (-M+i\ep)(m(F_i))$, where $F_i = \{x: x \in [-M,-M+i\ep)\}$. Now, for each $F_i$ there is a closed set, $F_i'$, of $F_i$ with $F_i \setminus F_i'$ having measure less than $\ep/(6(n+1))$. Moreover, there is a finite collection of intervals, $\anbrack{F_{i_m}}$, such that $F_i' \setminus \bigcup F_{i_m}$ has measure less than $\ep/(6(n+1))$.

This means that $F_i \setminus \bigcup F_{i_m}$ has measure less than $\ep/(3(n+1))$. There are $n+1$ such intervals, so we can deduce that $\bigcup F_i \setminus \bigcup \bigcup F_{i_m}$ has measure less than $\ep/3$.

Now, define $h: [a,b] \to \R$ by

\begin{displaymath}
   h(x) = \left\{
     \begin{array}{lr}
       -M+i\ep & : x \in F_{i_m} \text { for some $i, m$ ``that work''}\\
       0 & : \text{ else.}
     \end{array}
   \right.
\end{displaymath}

It is clear that $h$ is a step function (there were only finitely many values of $i$ and $n$ that worked), and that $h = \psi$ on each $F_{i_m}$. That is, $h = \psi$ except on a set of measure less than $\ep/3$.

By extension, we have $\absval{f-h} < \ep$ except on a set of measure less than $2\ep/3$, which satisfies that part of the problem.

Now, consider $F_{i_m}' = \{x \in F_{i_m} : x \in [c+\ep/3(k+1),d-\ep/3(k+1)]\}$ where $F_{i_m} = [c,d]$ for some $c,d \in \R$ and $k$ is the number of $F_{i_m}$s. (We are shaving off the ends).

Now, we can define $g: [a,b] \to \R$ by

\begin{displaymath}
   g(x) = \left\{
     \begin{array}{lr}
       -M+i\ep & : x \in F_{i_n}' \text { for some $i, n$ ``that work''}\\
       \text{(insert complex equation of a line here)} & : x \in F_{i_n} \setminus F_{i_n}' \text { for some $i, n$ ``that work''}\\
       0 & : \text{ else.}
     \end{array}
   \right.
\end{displaymath}

\shunt %Just have a picture

It is clear from the picture (I don't want to write the equation for that line, and you don't want to see it) that $g$ is continuous, and that $g = h$ except on a set of measure $\ep/3$. By extension, we have $\absval{f-g} < \ep$ except on a set of measure less than $\ep$, which satisfies the other part of the problem.

\shunt

{\bf Problem 2:}

Let $E_1, E_2$ be measurable. (In the below, I use the fact that certain sets are disjoint freely. It's important to point out that this fact is important.)

Then: 

\begin{align*}
m(E_1 \cup E_2) + m(E_1 \cap E_2) &= m((E_1 \triangle E_2) \cup (E_1 \cap E_2)) + m(E_1 \cap E_2)\\
&= m((E_1 \triangle E_2)) + 2m((E_1 \cap E_2)) \\
&= m((E_1 \setminus E_2) \cup (E_2 \setminus E_1)) + 2m((E_1 \cap E_2)) \\
&= m(E_1 \setminus E_2) + m((E_1 \cap E_2)) + m(E_2 \setminus E_1) + m((E_1 \cap E_2)) \\
&= m((E_1 \setminus E_2) \cup (E_1 \cap E_2)) + m((E_2 \setminus E_1) \cup (E_1 \cap E_2)) \text{ This is where I use the measurability.}\\
&= m(E_1) + m(E_2)
\end{align*} 

So we have our result.

\shunt

{\bf Problem 3:}

Let $f_n \to f$ almost everywhere on a set of finite measure, with each $f_n$ measurable. 

Then $f_n \to f$ except on a set of measure zero.

Let $\ep >0$. There is a covering of open intervals, $U$ with the sum of the lengths of those intervals less than $\ep/2$ that covers the set of points that $f_n$ does not converge to $f$ on.

Now, consider $F = E \setminus U$. Then $f_n \to f$ everywhere on $F$.

There is a set, $A \subset F$ with $m(A) < \ep/2$ and $f_n \to f$ uniformly on $F \setminus A$.

So, $\anbrack{f_n}$ converges to $f$ uniformly on $[a,b]$ except on $[a,b] \setminus F$ and $F \setminus A$, which both have measure less than $\ep/2$.

So, $\anbrack{f_n}$ converges to $f$ uniformly on $[a,b]$ except on a set of measure less than $\ep$. This satisfies the problem.

\shunt

{\bf Problem 4:}

Consider the ``Growing steeple'':

Define $f_n: [0,1] \to \R$ by:

 \begin{displaymath}
   f_n(x) = \left\{
     \begin{array}{lr}
       nx & : x \in [0,1/(2n)]\\
       n - nx & : x \in [1/(2n),1/n]\\
       0 & : else
     \end{array}
   \right.
\end{displaymath} 

Here is a picture.

\shunt

\shunt %Picture

Each $f_n$ is non-negative, measurable, and $f_n \to 0$ everywhere.

But $\int\limits_0^1 0 = 0$ and $\int\limits_0^1 f_n = 1$. So $\liminf\int\limits_0^1 f_n = 1$. That is, we have $\int\limits_0^1 f < \liminf\int\limits_0^1 f_n$, which is strict inequality in Fatou's Lemma.

Note: I saw the hint in Royden, but I felt like it was more interesting to give an example on a set of finite measure.

\shunt

{\bf Problem 5:}

Let $f$ be a non-negative integrable function. Consider $F(x) = \int\limits_{(-\infty,x]} f$.

Let $\ep >0$, and $p \in \R$. There is a bounded, non-negative function, $h$, vanishing outside of $E = [p-1,p+1]$, such that $h < f$ and $\int\limits_E f < \int\limits_E h+ \ep/2$, by (the proof of) Fatou's lemma. Let $M$ bound $h$.

Now, choose $\de = \min(1,\ep/2M)$, and let $x \in [p-\de,p+\de]$.

We know that if $x \geq p$, then:

\begin{align*}
\int\limits_{p-1}^{p+1} f &< \int\limits_{p-1}^{p+1} h + \ep/2\\
\int\limits_{p-1}^{p} f + \int\limits_{p}^{x} f + \int\limits_{x}^{p+1} f &< \int\limits_{p-1}^{p} h + \int\limits_{p}^{x} h + \int\limits_{x}^{p+1} h  + \ep/2\\
\int\limits_{p-1}^{p} f-h + \int\limits_{p}^{x} f-h + \int\limits_{x}^{p+1} f-h &< \ep/2
\end{align*}

So $\int\limits_p^x f-h < \ep/2$, because both of the other terms are positive. By splitting the integral and taking absolute values, we have $\absval{\int\limits_p^x f}<\absval{\int\limits_p^x h} + \ep/2$. By flipping the inequality of our assumption, we get that $\int\limits_x^p f-h < \ep/2$ if $x \leq p$, so by repeaitng the argument we have the result $\absval{\int\limits_p^x f}<\absval{\int\limits_p^x h} + \ep/2$ in any case.

Now, define $H(x) = \int\limits_{(-\infty,x]} h$. We also have that

\begin{align*}
\absval{H(x)-H(p)} &= \absval{\int\limits_{(-\infty,x]} h - \int\limits_{(-\infty,p]} h} \\
&=\absval{\int\limits_p^x h} \\
&< M \de \\
&< \ep/2
\end{align*}

So, we have

\begin{align*}
\absval{F(x)-F(p)} &= \absval{\int\limits_{(-\infty,x]} f - \int\limits_{(-\infty,p]} f} \\
&=\absval{\int\limits_p^x f} \\
&<\absval{\int\limits_p^x h} + \ep/2 \\
&=\absval{H(x)-H(p)} + \ep/2\\
&< \ep
\end{align*}

So for all $\ep>0$, $p \in \R$, there is a $\de >0$ such that $x \in [p-\de, p+\de]$ implies that $\absval{F(x)-F(p)} < \ep$; $F$ is continuous.

\shunt

{\bf Problem 6:}

Note: it seems that what you mean by ``Royden's Proposition 15'' may be ambiguous. I take this to mean the one that isn't proven in the book (the one in chapter 3). For the content covered, this could also mean one in chapter 4, but that is flat out given in the book, which would be too easy.

Note: This proof is incomplete; I ran out of time working on it. So, there's a lot of holes. However, it should be clear what my plan was, and how much of it I successfully executed (as of the time of this writing, about half.)

We wish to show that, given $E \subset \R$, the following are equivalent:

\begin{enumerate}
\item $E$ is measurable.
\item $\forall \ep >0 \exists U: U \supset E, m^*(U \setminus E) < \ep$, and $U$ is open.
\item $\forall \ep >0 \exists U: F \subset E, m^*(E \setminus F) < \ep$, and $F$ is closed.
\item $\exists G \in G_\de : E \subset G, m^*(G \setminus E) = 0$.
\item $\exists F \in F_\si : E \supset F, m^*(E \setminus F) = 0$.
\end{enumerate}

And, moreover, that if $m(E)$ is finite, the above are equivalent to

\begin{enumerate}
\setcounter{enumi}{5}
\item $\forall \ep >0$ there is a finite union, $U$, of open intervals such that $m^*(E \triangle U) < \ep$.
\end{enumerate}

First, consider the case where $m^*(E) < \infty$.

In this case, $1$ implies $2$;

\tab Let $E$ be measurable, and let $\ep >0$. Then $m(E) = \inf\limits_{\{I_n\}}(\sum l(I_n))$ where $\{I_n\}$ covers $E$.

\tab So, it is possible to pick a union of intervals, $U$ such that $\absval{m^*(U) - m^*(E)} < \ep$ and $U \supset E$. Arbitrary unions of intervals are open, so $U$ is open. Now, because $E$ is measurable, we know that $m^*(U) = m^*(U \cap E) + m^*(U \setminus E)$. In other words, $m^*(U) - m^*(U \cap E) = m^*(U \setminus E)$. But because $U \supset E$, this is just $m^*(U) - m^*(E) = m^*(U \setminus E)$.  The left hand side is bounded by $\ep$, and so we win.

Next, in this case $2$ is equivalent to $6$: %Use the fucking definition of outer measure, you prick.

\tab First, $6$ implies $2$: Let $\ep>0$, and $6$ hold. Then there is a finite union, $U$, of open intervals with $m^*(E \triangle U) < \ep/2$, which means that there is a collection of intervals, $\anbrack{I_n}$, with $\bigcup I_n \supset E \triangle U$ and $\sum l(I_n) < \ep/2$. It is clear that $V = U \cup \bigcup{I_n}$ is open, and contains $E$. Moreover, $m^*(V \setminus E) < \ep$, because of subadditivity of outer measure.

\tab Next, $2$ implies $6$: Let $\ep>0$, and $2$ hold. If $m^*(E)$ is finite, then $E$ is bounded. That means...%something

\tab By $2$, there's an open set, $U$, such that $U \supset E$ and $m^*(U \setminus E) < \ep/2$. Now, this means that there is a collection of intervals, $\anbrack{I_n}$, with $\bigcup I_n \supset U \setminus E$ and $\sum l(I_n) < \ep/4$.

Moving on, for any set $E$, we have $1$ implies $2$ implies $4$ implies $1$:

\tab It is clear that $1$ implies $2$ from the above; let $\ep >0$, and $1$ hold. From the above, this means that $2$ holds for each $E_n = [n,n+1] \cap E$ with $n \in \Z$. Apply $2$ to each of those sets; retrieve a collection of open sets, $U_n$, with $U_n \supset E_n$ and $m^*(U_n \setminus E_n) < \ep/(2^{n+1})$. It is clear that $U = \bigcup U_n$ contains $E$, and has $m^*(U \setminus E) \leq \ep$ (apply geometric series and subadditivity).

\tab It is also clear that $2$ implies $4$; we know that the standard topology on $\R$ has a countable basis, then we take an intersection of sets we get from $2$ with $\ep = 1/n$ for each $n \in \N$. (I'm obviously out of time.)

\tab Now, $4$ implies $1$: 

Last, we have $1$ implies $3$ implies $5$ implies $1$.

\tab First, $1$ implies $3$:

\tab It is clear that $3$ implies $5$; we know that the standard topology on $\R$ has a countable basis, then we take an union of sets we get from $2$ with $\ep = 1/n$ for each $n \in \N$. (I'm obviously out of time.)

\tab Finally, $5$ implies $1$: 

This yields the desired result.


\shunt

{\bf Bonus:}

When I was in second grade, my teacher would take 20 minutes of each day to read us a version of ``The Pilgrim's Progress'' aimed at children. A pair of children were born in a miserable town, and then they left for a much better place, far away, and they have to brave the wilderness and elements to get there. One of them was very devoted and faithful. The other was sort of mediocre in his devotion and faith. Soon after leaving, they meet a kid who is faithless; he joins them on their journey. They're put in a series of trials that tests their faith in some voice they all hear: the faithful kid succeeds at all of them, the mediocre kid succeeds sometimes, and the faithless kid fails all of them. They quest along the way until they reach a heavily gated town; the voice lets the devoted kid in, tells the mediocre one to go through the trials again, and it tells the rotten kid that he needs to just wallow in the terrible town.

That's sort of awful; they throw out two children into the wilderness, one of whom they believe is probably a reasonably good kid. The Incredible Hulk wouldn't have put up with that. That sort of thing would have made the Hulk incredibly angry, and he would have smashed that town to smithereens, beating up whoever was behind that voice. I don't know if that makes him my favorite superhero, but it does make him better than the voice that tells kids to endure all sorts of horrifying trials, otherwise they'll have to live in a miserable shack where they're abused by their drug-addled parents.
\shunt

\end{document}