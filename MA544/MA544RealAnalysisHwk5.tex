
\documentclass[a4paper,12pt]{article}

\usepackage{fancyhdr}
\usepackage{amssymb}
%\usepackage{mathpazo}
\usepackage{mathtools}
\usepackage{amsmath}
\usepackage{slashed}
\usepackage{cancel}
\usepackage[mathscr]{euscript}

\newcommand{\tab}{\hspace{4mm}} %Spacing aliases
\newcommand{\shunt}{\vspace{20mm}}

\newcommand{\sd}{\partial} %Squiggle d

\newcommand{\absval}[1]{\left\lvert #1 \right\rvert}
\newcommand{\norm}[1]{\|#1\|}
\newcommand{\anbrack}[1]{\left\langle #1 \right\rangle}



\newcommand{\al}{\alpha} %Steal ALL of Dr. Kable's Aliases! MWAHAHAHAHA!
\newcommand{\be}{\beta}
\newcommand{\ga}{\gamma}
\newcommand{\Ga}{\Gamma}
\newcommand{\de}{\delta}
\newcommand{\De}{\Delta}
\newcommand{\ep}{\epsilon}
\newcommand{\vep}{\varepsilon}
\newcommand{\ze}{\zeta}
\newcommand{\et}{\eta}
\newcommand{\tha}{\theta}
\newcommand{\vtha}{\vartheta}
\newcommand{\Tha}{\Theta}
\newcommand{\io}{\iota}
\newcommand{\ka}{\kappa}
\newcommand{\la}{\lambda}
\newcommand{\La}{\Lambda}
\newcommand{\rh}{\rho}
\newcommand{\si}{\sigma}
\newcommand{\Si}{\Sigma}
\newcommand{\ta}{\tau}
\newcommand{\ups}{\upsilon}
\newcommand{\Ups}{\Upsilon}
\newcommand{\ph}{\phi}
\newcommand{\Ph}{\Phi}
\newcommand{\vph}{\varphi}
\newcommand{\vpi}{\varpi}
\newcommand{\ch}{\chi}
\newcommand{\ps}{\psi}
\newcommand{\Ps}{\Psi}
\newcommand{\om}{\omega}
\newcommand{\Om}{\Omega}

\newcommand{\bbA}{\mathbb{A}}
\newcommand{\A}{\mathbb{A}}
\newcommand{\bbB}{\mathbb{B}}
\newcommand{\bbC}{\mathbb{C}}
\newcommand{\C}{\mathbb{C}}
\newcommand{\bbD}{\mathbb{D}}
\newcommand{\bbE}{\mathbb{E}}
\newcommand{\bbF}{\mathbb{F}}
\newcommand{\bbG}{\mathbb{G}}
\newcommand{\G}{\mathbb{G}}
\newcommand{\bbH}{\mathbb{H}}
\newcommand{\HH}{\mathbb{H}}
\newcommand{\bbI}{\mathbb{I}}
\newcommand{\I}{\mathbb{I}}
\newcommand{\bbJ}{\mathbb{J}}
\newcommand{\bbK}{\mathbb{K}}
\newcommand{\bbL}{\mathbb{L}}
\newcommand{\bbM}{\mathbb{M}}
\newcommand{\bbN}{\mathbb{N}}
\newcommand{\N}{\mathbb{N}}
\newcommand{\bbO}{\mathbb{O}}
\newcommand{\bbP}{\mathbb{P}}
\newcommand{\PP}{\mathbb{P}}
\newcommand{\bbQ}{\mathbb{Q}}
\newcommand{\Q}{\mathbb{Q}}
\newcommand{\bbR}{\mathbb{R}}
\newcommand{\R}{\mathbb{R}}
\newcommand{\bbS}{\mathbb{S}}
\newcommand{\bbT}{\mathbb{T}}
\newcommand{\bbU}{\mathbb{U}}
\newcommand{\bbV}{\mathbb{V}}
\newcommand{\bbW}{\mathbb{W}}
\newcommand{\bbX}{\mathbb{X}}
\newcommand{\bbY}{\mathbb{Y}}
\newcommand{\bbZ}{\mathbb{Z}}
\newcommand{\Z}{\mathbb{Z}}

\newcommand{\scrA}{\mathcal{A}}
\newcommand{\scrB}{\mathcal{B}}
\newcommand{\scrC}{\mathcal{C}}
\newcommand{\scrD}{\mathcal{D}}
\newcommand{\scrE}{\mathcal{E}}
\newcommand{\scrF}{\mathcal{F}}
\newcommand{\scrG}{\mathcal{G}}
\newcommand{\scrH}{\mathcal{H}}
\newcommand{\scrI}{\mathcal{I}}
\newcommand{\scrJ}{\mathcal{J}}
\newcommand{\scrK}{\mathcal{K}}
\newcommand{\scrL}{\mathcal{L}}
\newcommand{\scrM}{\mathcal{M}}
\newcommand{\scrN}{\mathcal{N}}
\newcommand{\scrO}{\mathcal{O}}
\newcommand{\scrP}{\mathcal{P}}
\newcommand{\scrQ}{\mathcal{Q}}
\newcommand{\scrR}{\mathcal{R}}
\newcommand{\scrS}{\mathcal{S}}
\newcommand{\scrT}{\mathcal{T}}
\newcommand{\scrU}{\mathcal{U}}
\newcommand{\scrV}{\mathcal{V}}
\newcommand{\scrW}{\mathcal{W}}
\newcommand{\scrX}{\mathcal{X}}
\newcommand{\scrY}{\mathcal{Y}}
\newcommand{\scrZ}{\mathcal{Z}}

\DeclarePairedDelimiter\ceil{\lceil}{\rceil}
\DeclarePairedDelimiter\floor{\lfloor}{\rfloor}

\newcommand{\colorcomment}[2]{\textcolor{#1}{#2}} %First of these leaves in comments. Second one kills them.
%\newcommand{\colorcomment}[2]{}


\pagestyle{fancy}
\lhead{Max Jeter}
\rhead{MA544, Assignment 5, Page \thepage}

\begin{document}

{\bf Problem 1:} 

Let $f_n \to f$ in measure, with an integrable function $g$ such that $\absval{f_n} \leq g$ for all $n$.

Because $f_n \to f$ in measure, there's a subsequence $\anbrack{f_{n_k}}$ such that $f_{n_k} \to f$ almost everywhere.

So the Lebesgue convergence theorem applies to that subsequence: $\absval{f_{n_k} - f} \to 0$ and there's an integrable function $g$ such that $\absval{f_{n_k}} \leq g$ for all $k$, so we have $\int \absval{f_{n_k} - f} \to 0$.

So if the sequence $\anbrack{\int \absval{f_n - f}}$ converges, then it converges to $0$.

Assume that the above sequence doesn't converge.

\tab That is, there is an $\ep>0$ such that for all $N \in \N$ there is an $n > N$ such that $\absval{\int \absval{f_n-f}} \geq \ep$.

\tab Thus, there is a sequence, $\anbrack{f_{n_k}}$, such that $\absval{\int \absval{f_{n_k}-f}} \geq \ep$ for all $k$.

\tab We know that $f_{n_k} \to f$ in measure. So there's a subsequence, $\anbrack{f_{n_{k_j}}}$ such that $f_{n_{k_j}} \to f$ almost everywhere. So the Lebesgue convergence theorem applies to that subsequence: $\int \absval{f_{n_{k_j}} - f} \to 0$. But this is a clear contradiction. 

So that sequence converges, and it converges to the right thing.

\shunt

{\bf Problem 2:}

Let $f$ be continuous on $[a,b]$, with one of its derivates everywhere nonnegative on $(a,b)$.

First, we will show this for a function $g$ with $D^+(g) \geq \ep >0$. If $g$ is such a function, then $\limsup\limits_{h \to 0^+} \frac{f(x+h)-f(x)}{h} \geq \ep> 0$. This means that $g$ is nondecreasing:

\tab We proceed by contradiction: let there be $x,y \in [a,b]$ (with $x<y$, without loss of generality) be such that $f(x) > f(y)$.

\tab Consider the set $A = \{\al \in [x,y): f(\al) > f(y)\}$. This set has a supremum, as it's nonempty. Define $\al =\sup(A)$. Either $\al = b$ (in which case, the derivate is negative at $b$, leading to our contradiction), or there is a $\de>0$ such that if $t \in [\al, \al+\de]$, then $f(t) < f(y)$. Moreover, $f(\al) = f(y)$: else, $\absval{f(\al) - f(y)} = \ep'$ for some $\ep' >0$, so by using continuity (specifically, the intermediate value theorem) we can find an $\al'$ between $\al$ and $y$ with $f(\al')$ between $f(\al)$ and $f(y)$, which causes a contradiction. So for any sequence $(t_n)$ decreasing to $\al$, we have $\frac{f(t_n) - f(\al)}{t_n - \al}$ negative. This means that $D^+(\al) \leq 0$. This contradicts our assumption on $D^+$.

We can mimic this proof to show that if $g$ has $D^-(g) \geq \ep >0$, then $g$ is nondecreasing.

Now, let $f$ have a derivate everywhere nonnegative on $(a,b)$. This means, in particular, that either $D^+$ or $D^-$ is everywhere nonnegative on $(a,b)$.

Then for every $\ep>0$, $g_\ep(x) = f(x) + \ep x$ has $D^+(g_\ep)$ (or $D^-(g_\ep)$) greater than $\ep$. So for all $\ep>0$, $g_\ep$ is nondecreasing. So for all $x,y \in [a,b]$ with $x<y$, $g_\ep(x) \leq g_\ep(y)$. That is, $f(x) + \ep x \leq f(y) + \ep y$. Taking limits as $\ep \to 0$, this means that $f(x) \leq f(y)$, for all $x,y \in [a,b]$ with $x<y$.

So $f$ is nondecreasing on $[a,b]$ if some derivate is everywhere nonnegative on $[a,b]$.

\shunt

{\bf Problem 3:}

Suppose that $f_n(x) \to f(x)$ at each $x \in [a,b]$.

Let $\De = \{x_1 < x_2 \ldots < x_k\}$ be a partition of $[a,b]$. Then $t(f) = \sum \absval{f(x_i)-f(x_{i-1}}$, and  $t(f_n) = \sum \absval{f_n(x_i)-f_n(x_{i-1}}$.

Now, $f_n \to f$ at all $x \in [a,b]$.

That is, for all $\ep >0$ and $x \in [a,b]$ there is an $N \in \N$ such that for all $n \geq N$, $\absval{f_n(x)-f(x)} < \ep$. So for all $\ep>0$ there's an $N \in \N$ such that for every $x_i$ in our partition and for all $n \geq N$, $\absval{f_n(x_i) - f(x)} < \ep/(2k)$. 

So we have:

\begin{align*}
t(f) - t(f_n) &= \sum \absval{f(x_i)-f(x_{i-1}} - \sum \absval{f_n(x_i)-f_n(x_{i-1}} \\
&\leq \sum\absval{f(x_i)-f_n(x_i)-(f(x_{i-1})-f_n(x_{i-1}))} \\
&\leq \sum \absval{f(x_i)-f_n(x_i)} + \sum \absval{f(x_{i-1})-f_n(x_{i-1})}\\
&< \ep
\end{align*}

To rephrase, $t(f) < t(f_n) + \ep$ for sufficiently large $n$.

So $T_a^b(f) \leq \liminf T_a^b(f_n)$, by taking limits as $n \to \infty$.

\shunt

{\bf Problem 4:} 

Suppose that $f \in BV([a,b])$. 
Then $f'$ exists almost everywhere, by a theorem in class. 
Moreover, $f$ is the difference of two monotone functions. 
That is, $f= f^+ -f^-$ for some monotone functions $f^+$ and $f^-$.

So, this means that we have

\begin{align*}
\int\limits_a^b \absval{f'} &= \int\limits_a^b \absval{(f^+)' - (f^-)'} \\
&\leq \int\limits_a^b \absval{(f^+)'} + \absval{(f^-)'}
\end{align*}

Now, we show that $\int\limits_a^b \absval{(f^+)'} \leq P_a^b(f)$.

\tab By one of the important theorems that is like the fundamental theorem of calculus, $\int\limits_a^b \absval{(f^+)'} \leq f^+(b)-f^+(a)$.

\tab We also know that $f^+(b) - f^+(a) \leq P_a^b$; 

\begin{align*}
P &= N +f(b)-f(a)\\
&=N+f^+(b)-f^-(b)-f^+(a)+f^-(a)\\
&=N+(f^+(b)-f^+(a))+(f^-(b)-f^-(a))\\
&\geq N+(f^+(b)-f^+(a))\\
&\geq (f^+(b)-f^+(a))
\end{align*}

Similarly, $\int\limits_a^b \absval{(f^-)'} \leq N_a^b(f)$. So, we have

\begin{align*}
\int\limits_a^b \absval{f'} &\leq \int\limits_a^b \absval{(f^+)'} + \absval{(f^-)'}\\
&\leq P_a^b + N_a^b \\
&\leq T_a^b
\end{align*}

as we desired.

\shunt

{\bf Problem 5:}

Let $g$ be an absolutely continuous monotone function on $[0,1]$, and $E$ be a set of measure $0$. Without loss of generality, we can take $g$ to be increasing. 

Let $\ep>0$.



\shunt

{\bf Problem 6:} 

Let $f$ be a nonnegative measurable function on $[0,1]$.

We know that $\ln$ is a concave function on $[0,1]$ (if this is not clear, it's the inverse of a convex function).

So $-\ln$ is a convex function on $[0,1]$.

So Jensen's inequality applies:

\begin{align*}
-\ln \int f &\leq -\int\ln f \\
\ln \int f & \geq \int \ln f
\end{align*}

This satisfies the problem.

\shunt

\end{document}