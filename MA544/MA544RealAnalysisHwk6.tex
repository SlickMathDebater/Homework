
\documentclass[a4paper,12pt]{article}

\usepackage{fancyhdr}
\usepackage{amssymb}
%\usepackage{mathpazo}
\usepackage{mathtools}
\usepackage{amsmath}
\usepackage{slashed}
\usepackage{cancel}
\usepackage[mathscr]{euscript}

\newcommand{\tab}{\hspace{4mm}} %Spacing aliases
\newcommand{\shunt}{\vspace{20mm}}

\newcommand{\sd}{\partial} %Squiggle d

\newcommand{\absval}[1]{\left\lvert #1 \right\rvert}
\newcommand{\norm}[1]{\|#1\|}
\newcommand{\anbrack}[1]{\left\langle #1 \right\rangle}



\newcommand{\al}{\alpha} %Steal ALL of Dr. Kable's Aliases! MWAHAHAHAHA!
\newcommand{\be}{\beta}
\newcommand{\ga}{\gamma}
\newcommand{\Ga}{\Gamma}
\newcommand{\de}{\delta}
\newcommand{\De}{\Delta}
\newcommand{\ep}{\epsilon}
\newcommand{\vep}{\varepsilon}
\newcommand{\ze}{\zeta}
\newcommand{\et}{\eta}
\newcommand{\tha}{\theta}
\newcommand{\vtha}{\vartheta}
\newcommand{\Tha}{\Theta}
\newcommand{\io}{\iota}
\newcommand{\ka}{\kappa}
\newcommand{\la}{\lambda}
\newcommand{\La}{\Lambda}
\newcommand{\rh}{\rho}
\newcommand{\si}{\sigma}
\newcommand{\Si}{\Sigma}
\newcommand{\ta}{\tau}
\newcommand{\ups}{\upsilon}
\newcommand{\Ups}{\Upsilon}
\newcommand{\ph}{\phi}
\newcommand{\Ph}{\Phi}
\newcommand{\vph}{\varphi}
\newcommand{\vpi}{\varpi}
\newcommand{\ch}{\chi}
\newcommand{\ps}{\psi}
\newcommand{\Ps}{\Psi}
\newcommand{\om}{\omega}
\newcommand{\Om}{\Omega}

\newcommand{\bbA}{\mathbb{A}}
\newcommand{\A}{\mathbb{A}}
\newcommand{\bbB}{\mathbb{B}}
\newcommand{\bbC}{\mathbb{C}}
\newcommand{\C}{\mathbb{C}}
\newcommand{\bbD}{\mathbb{D}}
\newcommand{\bbE}{\mathbb{E}}
\newcommand{\bbF}{\mathbb{F}}
\newcommand{\bbG}{\mathbb{G}}
\newcommand{\G}{\mathbb{G}}
\newcommand{\bbH}{\mathbb{H}}
\newcommand{\HH}{\mathbb{H}}
\newcommand{\bbI}{\mathbb{I}}
\newcommand{\I}{\mathbb{I}}
\newcommand{\bbJ}{\mathbb{J}}
\newcommand{\bbK}{\mathbb{K}}
\newcommand{\bbL}{\mathbb{L}}
\newcommand{\bbM}{\mathbb{M}}
\newcommand{\bbN}{\mathbb{N}}
\newcommand{\N}{\mathbb{N}}
\newcommand{\bbO}{\mathbb{O}}
\newcommand{\bbP}{\mathbb{P}}
\newcommand{\PP}{\mathbb{P}}
\newcommand{\bbQ}{\mathbb{Q}}
\newcommand{\Q}{\mathbb{Q}}
\newcommand{\bbR}{\mathbb{R}}
\newcommand{\R}{\mathbb{R}}
\newcommand{\bbS}{\mathbb{S}}
\newcommand{\bbT}{\mathbb{T}}
\newcommand{\bbU}{\mathbb{U}}
\newcommand{\bbV}{\mathbb{V}}
\newcommand{\bbW}{\mathbb{W}}
\newcommand{\bbX}{\mathbb{X}}
\newcommand{\bbY}{\mathbb{Y}}
\newcommand{\bbZ}{\mathbb{Z}}
\newcommand{\Z}{\mathbb{Z}}

\newcommand{\scrA}{\mathcal{A}}
\newcommand{\scrB}{\mathcal{B}}
\newcommand{\scrC}{\mathcal{C}}
\newcommand{\scrD}{\mathcal{D}}
\newcommand{\scrE}{\mathcal{E}}
\newcommand{\scrF}{\mathcal{F}}
\newcommand{\scrG}{\mathcal{G}}
\newcommand{\scrH}{\mathcal{H}}
\newcommand{\scrI}{\mathcal{I}}
\newcommand{\scrJ}{\mathcal{J}}
\newcommand{\scrK}{\mathcal{K}}
\newcommand{\scrL}{\mathcal{L}}
\newcommand{\scrM}{\mathcal{M}}
\newcommand{\scrN}{\mathcal{N}}
\newcommand{\scrO}{\mathcal{O}}
\newcommand{\scrP}{\mathcal{P}}
\newcommand{\scrQ}{\mathcal{Q}}
\newcommand{\scrR}{\mathcal{R}}
\newcommand{\scrS}{\mathcal{S}}
\newcommand{\scrT}{\mathcal{T}}
\newcommand{\scrU}{\mathcal{U}}
\newcommand{\scrV}{\mathcal{V}}
\newcommand{\scrW}{\mathcal{W}}
\newcommand{\scrX}{\mathcal{X}}
\newcommand{\scrY}{\mathcal{Y}}
\newcommand{\scrZ}{\mathcal{Z}}

\DeclarePairedDelimiter\ceil{\lceil}{\rceil}
\DeclarePairedDelimiter\floor{\lfloor}{\rfloor}

\newcommand{\colorcomment}[2]{\textcolor{#1}{#2}} %First of these leaves in comments. Second one kills them.
%\newcommand{\colorcomment}[2]{}


\pagestyle{fancy}
\lhead{Max Jeter}
\rhead{MA544, Assignment 6, Page \thepage}

\begin{document}

Note: The theme for a good chunk of these problems is ``Analyze the proof, change a component of it, re-execute the proof.'' In this spirit, a lot of this is ripped from Royden.

{\bf Problem 1:} %Strategy: mimic proof for L^p.

(If $1 < p < \infty$, find a representation for the bounded linear functionals on $\ell^p$, where $\ell^p$ consists of sequences $\anbrack{x_n}$ of real numbers such that

\begin{align*}
(\sum\absval{x_n}^p)^{1/p}<\infty ) 
\end{align*}

We proceed by mimicking the argument in Royden, because I'm uncreative.

Let $p \in (1,\infty)$.

First, each sequence, $\anbrack{y_n}$, in $\ell^q$, where $1/p + 1/q = 1$, defines a linear functional on $\ell^p$, by

\begin{align*}
F(x_n)= \sum\limits_{x=1}^\infty x_ny_n
\end{align*}

It's somewhat clear that this functional is linear. It is also bounded: we have an analogue for the H{\"o}lder inequality for sequences. (It's one of the exercises in the book.) So we have that $\norm{F} \leq \norm{y_n}_q$. (In fact, by setting $x_n = \text{sgn}(y_n) y_n^{q/p}$, we see that $\norm{F} = \norm{y_n}_q$.

So, we have that all sequences in $\ell^q$ define a linear functional on $\ell^p$. It suffices to show that all linear functionals on $\ell^p$ are given by a sequence in $\ell^q$.

We start by proving the following lemma: Fix $\anbrack{y_n}$. If there's an $M >0$ such that for all $\anbrack{x_n} \in \ell^p$,

\begin{align*}
\sum\limits_{n=0}^\infty \absval{x_ny_n} \leq M\norm{x_n}_p\\
\end{align*}

, then $\anbrack{y_n} \in \ell^q$ and $\norm{y_n} \leq M$.

\tab To show this, first %Mimic Lemma 12 proof. Note that Fatou's Lemma is used, and you need to sidestep that.

Moving on past the proof of the lemma, let $F$ be a bounded linear functional on $\ell^p$. Then, for brevity, let $\chi_n$ be the characteristic function on $\{1,2, \ldots n\}$. For each $n$, the value of $F(\chi_n)$ is a number, call it $\Phi(n)$.

Now, there is a sequence such that $\Phi(n) = \sum\limits_{i=1}^n y_i$. (In fact, it is given recursively by $y_1 = \Phi(1)$, and $y_n = \Phi(n)-y_{n-1}$ if $n >1$.) So, we have

\begin{align*}
F(\chi_n) = \sum\limits_{k=1}^\infty y_k \chi_n
\end{align*}

So, because each sequence is equal to a linear combination of indicator functions, we have

\begin{align*}
F(x_n) = \sum\limits_{k=1}^\infty y_k x_k
\end{align*}

for each sequence $x_n \in \ell^p$. 

%Something's a little off here...

\shunt

{\bf Problem 2:} 

(Let $f \in L^p$, and let $T_\De(f)$ denote the $\De$-approximant of $f$. Prove that

\begin{align*}
\norm{T_\De(f)}_p \leq \norm{f}_p)
\end{align*}

Let $\De = \{x_0,x_1 \ldots x_m\}$ be a partition of $[0,1]$. Then 

\begin{align*}
T_\De(f) = \sum\limits_{n=1}^{m} \frac{\chi_{[x_{n-1},x_n]}}{x_n-x_{n-1}}\int\limits_{x_{n-1}}^{x_n} f(t)dt
\end{align*} 

So we have %Well, do it then.

\begin{align*}
\norm{T_\De(f)}_p^p &= \int \absval{\sum\limits_{n=1}^{m} \frac{\chi_{[x_{n-1},x_n]}}{x_n-x_{n-1}}\int\limits_{x_{n-1}}^{x_n} f(t)dt}^pdx\\
&\leq \int \sum\limits_{n=1}^{m} \absval{ \frac{\chi_{[x_{n-1},x_n]}}{x_n-x_{n-1}}\int\limits_{x_{n-1}}^{x_n} f(t)dt}^pdx\\
&= \sum\limits_{n=1}^{m}\int  \absval{ \frac{\chi_{[x_{n-1},x_n]}}{x_n-x_{n-1}}\int\limits_{x_{n-1}}^{x_n} f(t)dt}^pdx\\
&= \sum\limits_{n=1}^{m}\int\limits_{x_{n-1}}^{x_n}  \absval{ \frac{1}{x_n-x_{n-1}}\int\limits_{x_{n-1}}^{x_n} f(t)dt}^pdx\\
&= \sum\limits_{n=1}^{m} (x_n-x_{n-1})\absval{ \int\limits_{x_{n-1}}^{x_n} \frac{f(t)}{(x_n-x_{n-1})}dt}^p\\
&\leq \sum\limits_{n=1}^{m} (x_n-x_{n-1})  \frac{\int\limits_{x_{n-1}}^{x_n}\absval{f(t)}^p dt}{(x_n-x_{n-1})} \text{ *See note}\\
&= \sum\limits_{n=1}^{m} \int\limits_{x_{n-1}}^{x_n}\absval{f(t)}^p \\
&= \int\limits_0^1 \absval{f(t)}^p dt \\
&= \norm{f}_p^p
\end{align*}

So we have that $\norm{T_\De(f)}_p \leq \norm{f}_p$. 

*Note: I'm convinced that I have written a true fact but can't seem to prove that this step works. A correct proof will likely appeal to the convexity of $x^p$.

\shunt

{\bf Problem 3:} 

(Prove that $\ell^p$, $1\leq p < \infty$, and $L^\infty$ are complete.)

(Note: I am completely convinced that there's a way to get this next part straight from Riesz-Fischer (without building up the machinery again), but I'm not quite good enough to see it. If there is a way, could you write it in the margins?)

First, $\ell^p$ is complete if $p \in [1, \infty)$; we show this by mimicking the proof of the Riesz-Fischer Theorem, as I am an uncreative n00b. To display the sheer laziness of this approach, I will be using the notation $a(n)$ to denote the $n$th term of a sequence in $\ell^p$ for this problem only.

Before doing this, we must prove the Minkowski Inequality for $\ell^p$ with $p \in [1,\infty)$:

\tab Let $f,g$ be two non-negative sequences in $\ell^p$ with $p \in [1,\infty)$.

\tab Note that this is trivial if $\norm{f}_p$ or $\norm{g}_p$ is zero, and we have equality. So, let us assert that this is not the case. Define $f_0=f/\norm{f}_p$ and $g_0=g/\norm{g}_p$. Also, define $\al = \norm{f}_p$, $\be=\norm{g}_p$, and $\la = \al/(\al+\be)$.

\tab Note that $\norm{f_0}_p=\norm{g_0}_p=1$.

\tab Then we have:

\begin{align*}
\absval{f+g}^p &= [\al f_0 + \be g_0]^p\\
&= (\al+\be)^p[\la f_0 + (1-\la) g_0]^p\\
&\leq (\al+\be)^p[\la f_0^p + (1-\la) g_0^p]
\end{align*}

\tab With the last step by the concavity of $x^p$ for $p \in (0,1)$.

\tab Summing up over both sides, we have:

\begin{align*}
\norm{f+g}_p^p &\geq (\al+\be)^p[\la \norm{f_0}_p^p+(1-\la)\norm{g_0}_p^p]\\
&=(\al+\be)^p\\
&=(\norm{f}_p+\norm{g}_p)^p
\end{align*}

\tab Raising both sides to the $1/p$th power yields the desired inequality. 

Now, we can proceed by ripping the proof out of the Royden almost verbatim.

\tab We need only show that each absolutely summable series in $\ell^p$ is summable in $\ell^p$ to some element of $\ell^p$.

\tab Let $\anbrack{f_n}$ be a sequence in $\ell^p$ with $\sum\limits_{n=1}^\infty \norm{f_n} = M < \infty$. Define sequences $g_n$ by setting $g_n(m) = \sum\limits_{k=1}^n \absval{f_n(m)}$. From the Minkowski inequality, we have

\begin{align*}
\norm{g_n} \leq \sum\limits_{k=1}^n \norm{f_k} \leq M
\end{align*}

So

\begin{align*}
\sum\limits_{x=0}^\infty g_n(x)^p \leq M^p
\end{align*}

For each $x \in \N$, $\anbrack{g_n(x)}$ is an increasing sequence of (extended) real numbers, and so must converge to an extended real number $g(x)$. By taking limits, we have

\begin{align*}
\sum\limits_{x=0}^\infty g(x)^p \leq M^p
\end{align*}

So $g(x)$ is finite for all $x \in \N$.

Now, for each $x$, $\sum\limits_{k=1}^\infty f_k(x)$ is absolutely summable; this is because $g(x)$ was finite. So $\sum\limits_{k=1}^\infty f_k(x)$ is summable; say it sums to $s(x)$. Say that $s$ is the limit of the partial sums $s_n = \sum\limits_{k=1}^n f_k$. Now since $\absval{s_n(x)} \leq g(x)$, we have $\absval{s(x)} \leq g(x)$. So $s \in \ell^p$.

Now, $\absval{s_n-s} \to 0$. So $\absval{s_n-s}^p \to 0$; this is because $p \geq 1$. Moreover, we know that $\absval{s_n-s}^p \leq 2^pg(x)^p$.

Recall that $g(x)^p$ had a finite sum. This means that $\absval{s_n-s}^p$ has a finite sum, for all $n \in \N$.

So we have that $\sum\limits_{x=1}^\infty \absval{s_n-s}^p \to 0$; %Bind the tail, then bind the head.

That is, we have that the series $\anbrack{f_n}$ converges to some sum, $s$. That is, every absolutely summable series is summable, so $\ell^p$ is complete.

Next, $L^\infty$ is complete;

\tab Let $\anbrack{f_n}$ be a Cauchy sequence in $L^\infty$. So at almost every $t \in [0,1]$, we have $f_n(t)$ Cauchy in $\R$; because $\norm{f_n}_\infty$ is Cauchy, the essential supremum of the $f_n$s as a sequence is Cauchy, so the supremum of the $f_n$s is Cauchy except on a countable union of sets of measure zero (that is, almost everywhere), so $f_n(t)$ is Cauchy at almost every $t \in [0,1]$. So $f_n$ converges pointwise almost everywhere to some $f$. In fact, it is readily checked (...it's an elementary epsilon-delta proof that you don't want to see) that $f_n$ converges uniformly to this $f$ except on a set of measure zero.

\tab Now, this $f$ is in $L^\infty$: There's an $f_n$ within $1$ of $f$ almost everywhere. The essential supremum of $f$ is at most $1$ away from the essential supremum of $f_n$. So the essential supremum of $f$ is finite. So $f \in L^\infty$.

\tab Next, $\norm{f_n - f}_\infty \to 0$: We know that $\absval{f_n-f} \to 0$ except on a set of measure zero. Take essential supremums of both sides, then you win. (Note: we really do need uniform convergence here.)

\tab So $\anbrack{f_n}$ converges in the mean to some $f \in L^\infty$ if $\anbrack{f_n}$ is Cauchy; we have our result.

\shunt

{\bf Problem 4:} 

(Let $\ell^\infty$ denote the set of all bounded sequences of real numbers. Set $\norm{(x_n)}_\infty = \text{sup}\absval{x_n}$. Prove that this is a norm, and $\ell^\infty$ is a Banach Space.)

First, this is a norm:

Let $\anbrack{x_n}$, $\anbrack{y_n}$ be bounded sequences of real numbers, and $\al \in \R$.

\tab First, $\norm{x_n}_\infty = 0$ if and only if $x_n$ is identically zero: $\sup(\absval{x_n}) = 0$ if each $x_n$ is zero. Further, if any $x_n$ is nonzero, then $\sup(\absval{x_n})$ is nonzero.

\tab Next, $\norm{\al x_n}_\infty = \sup(\absval{\al x_n}) = \sup(\absval{\al}\absval{x_n}) = \absval{\al}\sup(\absval{x_n}) = \absval{\al}\norm{x_n}_\infty$. (All of this follows from basic properties of the sup.)

\tab Last:

\begin{align*}
\norm{x_n+y_n} &= \sup(\absval{x_n+y_n})\\
&\leq \sup(\absval{x_n}+\absval{y_n})\\
&\leq \sup(\absval{x_n})+\sup(\absval{y_n})\\
&=\norm{x_n}+\norm{y_n}
\end{align*}

So this norm is a norm.

Next, $\ell^\infty$ is complete. We show this by mimicking the proof for $L^\infty$, as I am again an uncreative n00b:

%Proof: Needs adapted

\tab Let $\anbrack{f_n}$ be a Cauchy sequence in $\ell^\infty$. So at every $t \in [0,1]$, we have $f_n(t)$ Cauchy in $\R$; because $\norm{f_n}_\infty$ is Cauchy, the supremum of the $f_n$s as a sequence is Cauchy, so the supremum of the $f_n$s is Cauchy, so $f_n(t)$s must be Cauchy at each $t \in [0,1]$. So $f_n$ converges pointwise everywhere to some $f$. In fact, it is readily checked that $f_n$ converges uniformly to this $f$ except on a set of measure zero.

\tab Now, this $f$ is in $\ell^\infty$: There's an $f_n$ within $1$ of $f$ almost everywhere. The supremum of $f$ is at most $1$ away from the supremum of $f_n$. So the supremum of $f$ is finite. So $f \in \ell^\infty$.

\tab Next, $\norm{f_n - f}_\infty \to 0$:  We know that $\absval{f_n-f} \to 0$. Take essential supremums of both sides, then you win. (Note: we really do need uniform convergence here.)

\tab So $\anbrack{f_n}$ converges in the mean to some $f \in \ell^\infty$ if $\anbrack{f_n}$ is Cauchy; we have our result.

%End of what needs to be adapted

So, $\ell^\infty$ is a complete normed vector space. It's a Banach space.

\shunt

{\bf Problem 5:} 

(Prove the Minkowski inequality for $0 < p < 1$.)

We proceed by mimicking the proof of the Minkowski inequality for $p \geq 1$. 

Let $f,g$ be two non-negative functions in $L^p$ with $p \in (0,1)$.

Note that this is trivial if $\norm{f}_p$ or $\norm{g}_p$ is zero, and we have equality. So, let us assert that this is not the case. Define $f_0=f/\norm{f}_p$ and $g_0=g/\norm{g}_p$. Also, define $\al = \norm{f}_p$, $\be=\norm{g}_p$, and $\la = \al/(\al+\be)$.

Note that $\norm{f_0}_p=\norm{g_0}_p=1$.

Then we have:

\begin{align*}
\absval{f+g}^p &= [\al f_0 + \be g_0]^p\\
&= (\al+\be)^p[\la f_0 + (1-\la) g_0]^p\\
&\geq (\al+\be)^p[\la f_0^p + (1-\la) g_0^p]
\end{align*}

With the last step by the concavity of $x^p$ for $p \in (0,1)$.

Integrating both sides, we have:

\begin{align*}
\norm{f+g}_p^p &\geq (\al+\be)^p[\la \norm{f_0}_p^p+(1-\la)\norm{g_0}_p^p]\\
&=(\al+\be)^p\\
&=(\norm{f}_p+\norm{g}_p)^p
\end{align*}

Raising both sides to the $1/p$th power yields the result. 

\shunt

{\bf Problem 6:} 

(Young's inequality states that if $a,b \geq 0$, $1 < p < \infty$, and $1/p + 1/q =1$, then

\begin{align*}
ab \leq a^p/p + b^q/q
\end{align*}

Prove the H{\"o}lder inequality using this.)

First, we should perhaps establish Young's inequality; it was not discussed in class.

I acknowledge that the following is an unintuitive mess; I hope that there is a better means of doing this.

First, consider the right hand side of Young's inequality. Given all of the hypotheses, we have:

\begin{align*}
a^p/p + b^q/q &= a^p/p +\frac{b^{p/(p-1)}}{p/(p-1)}\\
&=a^p/p + (p-1)b^{p/(p-1)}/p\\
&=\frac{a^p+(p-1)b^{p/(p-1)}}{p}
\end{align*}

Thus, we have that

\begin{align*}
a^p/p+b^q/q -ab &= \frac{a^p-pab+(p-1)b^{p/(p-1)}}{p}
\end{align*}

We have Young's Inequality if the left hand side is greater than or equal to $0$. So we have Young's Inequality if the numerator is greater or equal to $0$, as we have that $p>1$, so that the denominator is positive.

Consider the function $f: \R \to \R$ given by 

\begin{align*}
f(x) = x^p-pxb+(p-1)b^{p/(p-1)}
\end{align*}

We know that $f(0) = (p-1)b^{p/(p-1)}$, so that $f(0)$ is positive. Also, it's clear that $\lim\limits_{x \to \infty} f(x) = \infty$. It's also clearly a differentiable function, with

\begin{align*}
f'(x) = px^{p-1} - pb
\end{align*}

So $f$ has a critical point at $b^{1/(p-1)}$. It is clear that $f'(0)$ is negative and that $f'(b)$ is positive, so $f$ takes its minimum at $b^{1/(p-1)}$.

However, $f(b^{1/(p-1)})=0$. So $f \geq 0$.

So $f(a) \geq 0$. So $a^p-pab+(p-1)b^{p/(p-1)} \geq 0$ which yields Young's Inequality, as stated above.

Moving on, we use this to prove the H{\"o}lder inequality. Assume that $p,q \in (1, \infty)$ with $1/p + 1/q = 1$, and let $f \in L^p$ and $g \in L^q$. 

Without loss of generality, we take $f$ and $g$ positive everywhere, as this has no effect on the norms. 

Define $\al = \norm{f}_p$, $\be = \norm{g}_q$, $f_0 = f/\al$, $g_0 = g/\be$. It's clear that $\norm{f_0}_p = \norm{g_0}_q = 1$. 

Now:

\begin{align*}
\frac{1}{\al \be} \int fg &= \int f_0 g_0 \\
&\leq \int \frac{f_0^p}{p} + \frac{g_0^q}{q}\\
&= \norm{f_0}_p^p/p + \norm{g_0}_q^q/q\\
&= 1/p + 1/q\\
&=1
\end{align*}

So $\int fg \leq \al \be$, that is, $\int fg \leq \norm{f}_p\norm{g}_q$, which is the desired result.

\shunt

{\bf Problem 7:}

(Pick up my dry cleaning.)

I cannot pick up your dry cleaning; I don't have a car, as I am too poor to afford one.

\shunt

\end{document}