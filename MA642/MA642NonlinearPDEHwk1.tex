
\documentclass[a4paper,12pt]{article}

\usepackage{fancyhdr}
\usepackage{amssymb}
%\usepackage{mathpazo}
\usepackage{mathtools}
\usepackage{amsmath}
\usepackage{slashed}
\usepackage{cancel}
\usepackage[mathscr]{euscript}
\usepackage{MaxPackage} %Note: You need MaxPackage installed or in the same folder as your .tex file or something.

\newcommand{\colorcomment}[2]{\textcolor{#1}{#2}} %First of these leaves in comments. Second one kills them.
%\newcommand{\colorcomment}[2]{}


\pagestyle{fancy}
\lhead{Max Jeter}
\chead{MA642}
\rhead{Assignment 1, Page \thepage}

%Number of Problems		: 
%Clear					:
%Begun					:7
%Not started			:
%Can complete via book	:
%Needs Polish			:9

%Pomodoros logged		:

\begin{document}

{\bf Problem 7:}

Let $u \in C^0(\Om)$.

Let $u$ be subharmonic in $\Om$. Then $u$ satisfies the mean value properties for any ball compactly contained in $\Om$; it is clear that this implies that $u$ satisfies the mean value properties locally.

Now, let $u$ satisfy the mean value properties locally. Fix $y \in \Om$. Without loss of generality, we can set $u(y) = 0$. Then there is a $\de >0$ such that $u(y) \leq \frac{1}{n\om_nR^{n-1}} \int\limits_{\partial B_{R}(y)} u ds$ for all $R \leq \de$. 

So, on $B_R(y)$, we have:

\begin{align*}
u(y) &\leq \frac{1}{n\om_nR^{n-1}} \int\limits_{\partial B_{R}(y)} u ds\\
0 &\leq \frac{1}{n\om_nR^{n-1}} \int\limits_{\partial B_{R}(y)} u ds\\
0 &\leq \int\limits_{\partial B_{R}(y)} u ds\\
\end{align*} %This isn't actually where you start. You know that you have to do something weird, but you're not sure *what*.

As the above holds for all $R < \de$, we get that $\int\limits_{B_R(y)} u dx \geq 0$ for all $R < \de$. Moreover, we get that $\int\limits_{B_R(y)} u dx \geq 0$ is increasing as a function of $R$.

That is,

\begin{align*}
\frac{d}{dR} \int\limits_{B_R(y)} u dx &\geq 0\\
\int\limits_{\partial B_R(y)} \frac{\partial u}{\partial \nu} dx &\geq 0\\
\int\limits_{B_R(y)} \De u dx &\geq 0\\
\end{align*} %This feels wrong and I don't know why.

So because the above holds for all $R < \de$, we have that $\De u(y) \geq 0$. That is, $u$ is subharmonic at $y$. 

\shunt

{\bf Problem 9:}

Let $u \in C^2(\Om)$.

First, $\De u \geq 0$ in $\Om$, if and only if $u$ is subharmonic in $\Om$, by definition. 

Next: let $u$ be subharmonic in $\Om$. Then $\De u \geq 0$ in $\Om$. Let $\phi \geq 0$ be a function in $C^2_c(\Om)$. So 

\begin{align*}
\int\limits_{\Om} u \De \phi dx &= \int\limits_{\Om} \phi \De u dx + \int\limits_{\partial \Om} u \partial_{\nu} \phi - \phi \partial_{\nu} u dS \\
&= \int\limits_{\Om} \phi \De u dx \text{ (Note: $\phi$ vanishes on $\partial \Om$, as $\phi$ has compact support.)} \\
&\geq 0 \\
\end{align*}

That is, $u$ is weakly subharmonic if $u$ is subharmonic.

Now, let $u$ be weakly subharmonic. Then for any $\phi \geq 0$ with $\phi \in C^2_c(\Om)$, we have

\begin{align*}
\int\limits_{\Om} u \De \phi dx &= \int\limits_{\Om} \phi \De u dx + \int\limits_{\partial \Om} u \partial_{\nu} \phi - \phi \partial_{\nu} u dS \\
&= \int\limits_{\Om} \phi \De u dx \\
&\geq 0
\end{align*}

Thus, $\De u \geq 0$ (Else, there's a point $y$ with $\De u(y) < 0$, so there's a neighborhood around $y$ with $\De u(y) < 0$, so picking $\phi >0$ on that neighborhood and $\phi = 0$ outside that neighborhood will yield a contradiction.). So, $u$ is subharmonic.

That is, $u$ is subharmonic if $u$ is weakly subharmonic.

That is, $u$ is subharmonic if and only if $u$ is weakly subharmonic.

Thus, the three conditions given are equivalent.

\shunt

\end{document}