
\documentclass[a4paper,12pt]{article}

\usepackage{fancyhdr}
\usepackage{amssymb}
%\usepackage{mathpazo}
\usepackage{mathtools}
\usepackage{amsmath}
\usepackage{slashed}
\usepackage{cancel}
\usepackage[mathscr]{euscript}
\usepackage{MaxPackage} %Note: You need MaxPackage installed or in the same folder as your .tex file or something.

\newcommand{\colorcomment}[2]{\textcolor{#1}{#2}} %First of these leaves in comments. Second one kills them.
%\newcommand{\colorcomment}[2]{}


\pagestyle{fancy}
\lhead{Max Jeter}
\chead{MA642}
\rhead{Assignment 2, Page \thepage}

%Number of Problems		: 
%Clear					:
%Begun					:2
%Not started			:3
%Can complete via book	:
%Needs Polish			:

%Pomodoros logged		:3

\begin{document}

{\bf Problem 2:}

Let $u$ be as described. %fill it in

Then there is a $B_r(x)$ (with $x \in \Om$) such that $B_r(x) \cap \partial \Om \subset A$ and $B_r(x) \setminus \overline{\Om} \neq \emptyset$.

Define $v: B_r(x) \cup \Om \to \R$ by $v(x) = u(x)$ for $x \in \Om$, else $x = 0$.

Then $v$ is harmonic on $B_r(x) \cup \Om$: it is clear that $v$ is harmonic on $\Om$ and on $B_r(x) \setminus \overline{\Om}$. Next, on $\Om\cap B_r(x)$, we have that $v$ is continuous (as $v(x) \to 0$ as $x \to \Om\cap B_r(x)$), that $v$ is differentiable (as $\frac{\partial v}{\partial \nu}(x) \to 0$ as $x \to \Om \cap B_r(x)$), and that $v$ has a continuous second derivative (...reasons). So, $\De v = 0$ on $\Om \cap B_r(x)$, because (limits). %All that's left is the intersection.

So, we have that $v$ is harmonic on $B_r(x) \cup \Om$ and that $v$ vanishes on some nonempty open subset of $B_r(x) \cup \Om$; thus, $v$ vanishes identically.

So $u$ vanishes identically. 

\shunt

{\bf Problem 3:}

%Possible idea: warp the fact we know from the ball to a general bounded domain. 

Let $G$ be the Green's function for $\Om$, a bounded domain.

Part a:

Consider $H(x,y) = G(x,y) - G(y,x)$. Then for all $u$ harmonic on $\Om$, we have

\begin{align*}
u(x) &= \int\limits_{\partial \Om} u(y) \frac{\partial G(x,y)}{\partial \nu} - G(x,y) \frac{\partial u(y)}{\partial \nu} dS_y\\
\end{align*}

%G is symmetric

\shunt

Part b:

%Probably do this by picking u sub/super harmonic with vanishing boundary data and using a property of those functions. 

%G is strictly negative if x \neq y

\shunt

Part c:

%\int\limits_{\Om} G(x,y)f(y) dy \to 0 as x \to \partial \Om if f is bounded/integrable on \Om

\shunt

\end{document}