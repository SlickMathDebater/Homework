
\documentclass[a4paper,12pt]{article}

\usepackage{fancyhdr}
\usepackage{amssymb}
%\usepackage{mathpazo}
\usepackage{mathtools}
\usepackage{amsmath}
\usepackage{slashed}
\usepackage[mathscr]{euscript}

\newcommand{\tab}{\hspace{4mm}} %Spacing aliases
\newcommand{\shunt}{\vspace{20mm}}

\newcommand{\sd}{\partial} %Squiggle d

\newcommand{\absval}[1]{\lvert #1 \rvert}
\newcommand{\anbrack}[1]{\left\langle #1 \right\rangle}
\newcommand{\norm}[1]{\|#1\|}


\newcommand{\al}{\alpha} %Steal ALL of Dr. Kable's Aliases! MWAHAHAHAHA!
\newcommand{\be}{\beta}
\newcommand{\ga}{\gamma}
\newcommand{\Ga}{\Gamma}
\newcommand{\de}{\delta}
\newcommand{\De}{\Delta}
\newcommand{\ep}{\epsilon}
\newcommand{\vep}{\varepsilon}
\newcommand{\ze}{\zeta}
\newcommand{\et}{\eta}
\newcommand{\tha}{\theta}
\newcommand{\vtha}{\vartheta}
\newcommand{\Tha}{\Theta}
\newcommand{\io}{\iota}
\newcommand{\ka}{\kappa}
\newcommand{\la}{\lambda}
\newcommand{\La}{\Lambda}
\newcommand{\rh}{\rho}
\newcommand{\si}{\sigma}
\newcommand{\Si}{\Sigma}
\newcommand{\ta}{\tau}
\newcommand{\ups}{\upsilon}
\newcommand{\Ups}{\Upsilon}
\newcommand{\ph}{\phi}
\newcommand{\Ph}{\Phi}
\newcommand{\vph}{\varphi}
\newcommand{\vpi}{\varpi}
\newcommand{\ch}{\chi}
\newcommand{\ps}{\psi}
\newcommand{\Ps}{\Psi}
\newcommand{\om}{\omega}
\newcommand{\Om}{\Omega}

\newcommand{\bbA}{\mathbb{A}}
\newcommand{\A}{\mathbb{A}}
\newcommand{\bbB}{\mathbb{B}}
\newcommand{\bbC}{\mathbb{C}}
\newcommand{\C}{\mathbb{C}}
\newcommand{\bbD}{\mathbb{D}}
\newcommand{\bbE}{\mathbb{E}}
\newcommand{\bbF}{\mathbb{F}}
\newcommand{\bbG}{\mathbb{G}}
\newcommand{\G}{\mathbb{G}}
\newcommand{\bbH}{\mathbb{H}}
\newcommand{\HH}{\mathbb{H}}
\newcommand{\bbI}{\mathbb{I}}
\newcommand{\I}{\mathbb{I}}
\newcommand{\bbJ}{\mathbb{J}}
\newcommand{\bbK}{\mathbb{K}}
\newcommand{\bbL}{\mathbb{L}}
\newcommand{\bbM}{\mathbb{M}}
\newcommand{\bbN}{\mathbb{N}}
\newcommand{\N}{\mathbb{N}}
\newcommand{\bbO}{\mathbb{O}}
\newcommand{\bbP}{\mathbb{P}}
\newcommand{\PP}{\mathbb{P}}
\newcommand{\bbQ}{\mathbb{Q}}
\newcommand{\Q}{\mathbb{Q}}
\newcommand{\bbR}{\mathbb{R}}
\newcommand{\R}{\mathbb{R}}
\newcommand{\bbS}{\mathbb{S}}
\newcommand{\bbT}{\mathbb{T}}
\newcommand{\bbU}{\mathbb{U}}
\newcommand{\bbV}{\mathbb{V}}
\newcommand{\bbW}{\mathbb{W}}
\newcommand{\bbX}{\mathbb{X}}
\newcommand{\bbY}{\mathbb{Y}}
\newcommand{\bbZ}{\mathbb{Z}}
\newcommand{\Z}{\mathbb{Z}}

\newcommand{\scrA}{\mathcal{A}}
\newcommand{\scrB}{\mathcal{B}}
\newcommand{\scrC}{\mathcal{C}}
\newcommand{\scrD}{\mathcal{D}}
\newcommand{\scrE}{\mathcal{E}}
\newcommand{\scrF}{\mathcal{F}}
\newcommand{\scrG}{\mathcal{G}}
\newcommand{\scrH}{\mathcal{H}}
\newcommand{\scrI}{\mathcal{I}}
\newcommand{\scrJ}{\mathcal{J}}
\newcommand{\scrK}{\mathcal{K}}
\newcommand{\scrL}{\mathcal{L}}
\newcommand{\scrM}{\mathcal{M}}
\newcommand{\scrN}{\mathcal{N}}
\newcommand{\scrO}{\mathcal{O}}
\newcommand{\scrP}{\mathcal{P}}
\newcommand{\scrQ}{\mathcal{Q}}
\newcommand{\scrR}{\mathcal{R}}
\newcommand{\scrS}{\mathcal{S}}
\newcommand{\scrT}{\mathcal{T}}
\newcommand{\scrU}{\mathcal{U}}
\newcommand{\scrV}{\mathcal{V}}
\newcommand{\scrW}{\mathcal{W}}
\newcommand{\scrX}{\mathcal{X}}
\newcommand{\scrY}{\mathcal{Y}}
\newcommand{\scrZ}{\mathcal{Z}}

\newcommand{\subgp}{\mathrel{\unlhd}}

\DeclarePairedDelimiter\ceil{\lceil}{\rceil}
\DeclarePairedDelimiter\floor{\lfloor}{\rfloor}

\newcommand{\colorcomment}[2]{\textcolor{#1}{#2}} %First of these leaves in comments. Second one kills them.
%\newcommand{\colorcomment}[2]{}


\pagestyle{fancy}
\lhead{Max Jeter}
\rhead{MA503, Assignment 10, Page \thepage}

\begin{document}

{\bf Problem 1:} 

Let $k \subset K$ and $k \subset L$ be finite field extensions contained in some field. 

In the following, I freely use the fact that $\text{dim}(U)\text{dim}(V) = \text{dim}(UV)/\text{dim}(U \cap V)$ for any vector spaces $U$ and $V$ over the same field. 

Part a:

First, $[KL:L][L:k] = [KL:k]$.

So this means that

\begin{align*}
\frac{[K:k]}{[KL:L]} &= \frac{\text{dim}(K)}{\frac{\text{dim}(KL)}{\text{dim}(L)}}\\
&= \frac{\text{dim}(K)\text{dim}(L)}{\text{dim}(KL)}
\end{align*}

That is, we have reduced this problem to the following one; if $[KL:k] \leq [K:k][L:k]$, then the right hand side is at least $1$.

So $[K:k] \geq [KL:L]$ if we manage to solve the following part.

\shunt 

Part b:

We know that both $K$ and $L$ are algebraic over $k$, because they're finite extensions. Let $K = k(\al_1,\al_2 \ldots \al_n)$ and $L = k(\be_1,\be_2 \ldots \be_m)$ (with all of the $al_i$s and $\be_i$s outside of $k$). Now, $KL$ is the smallest field containing both $K$ and $L$. So it must contain all of the $\al_i$s and $\be_i$s. So $KL = k(\al_1,\al_2 \ldots \al_n, \be_1, \be_2 \ldots \be_m)$. This means that $[KL:k] \leq [K:k][L:k]$ (it may have been ``less than'', as there may have been some redundancy in the $\al_i$s and the $\be_i$s.

\shunt 

Part c:

Let $K \cap L = k$. We know that both $K$ and $L$ are algebraic over $k$, because they're finite extensions. Let $K = k(\al_1,\al_2 \ldots \al_n)$ and $L = k(\be_1,\be_2 \ldots \be_m)$ (with all of the $al_i$s and $\be_i$s outside of $k$). Note that all of the $\al_i$s are linearly independent from the $\be_i$s, else $K$ and $L$ intersect outside of $k$. Now, $KL$ is the smallest field containing both $K$ and $L$. So it must contain all of the $\al_i$s and $\be_i$s. So $KL = k(\al_1,\al_2 \ldots \al_n, \be_1, \be_2 \ldots \be_m)$. This means that $[KL:k] = [K:k][L:k]$.

The converse is false. Consider $k = \Q$, $L = \Q(2^{1/3})$, $K = \Q(2^{1/3}e^{2\pi/3})$. It is clear that $[LK: k] = 3$ (both of those roots has the same minimal polynomial, $x^3-2$), but $[L:k][K:k] = 9$.

\shunt

{\bf Problem 2:} 

Let $K = \Q(\sqrt{2},\sqrt{3})$.

Then $[K:\Q] = [K:\Q(\sqrt{2})][\Q(\sqrt{2}):\Q] = 2*2=4$. (The minimal polynomial of $\sqrt{2}$ in $\Q[x]$ is $x^2-2$, a polynomial with $\sqrt{3}$ as a root in $\Q(\sqrt{2})[x]$ is $x^2-3$, and $\sqrt{3}$ isn't in $\Q(\sqrt{2})$.

Now, $K = \Q(\sqrt{2}+\sqrt{3})$;  first, $\Q(\sqrt{2}+\sqrt{3}) \subset K$, because $\sqrt{2}+\sqrt{3} \in K$; that is, all of the right hand side's generators are in $K$, so $\Q(\sqrt{2}+\sqrt{3}) \subset K$. Next, we have that

\begin{align*}
\frac{1}{\sqrt{2}+\sqrt{3}} &= \frac{-\sqrt{2}+\sqrt{3}}{5}\\
\end{align*}

So $\sqrt{2}-\sqrt{3} \in \Q(\sqrt{2}+\sqrt{3})$. Adding (or subtracting) $\sqrt{2}+\sqrt{3}$ to this and dividing by $2$ shows us that $\sqrt{2}$ and $\sqrt{3}$ are in $\Q(\sqrt{2}+\sqrt{3})$, so all of $K$'s generators are in $\Q(\sqrt{2}+\sqrt{3})$; $K \subset \Q(\sqrt{2}+\sqrt{3})$.

\shunt

{\bf Problem 3:} 

Let $k \subset K$ be an algebraic field extension.

Then every $k$-homomorphism $\de: K \to K$ is a monomorphism; this is immediate, as discussed in class.

Next, let $a \in K$. Then $a$ is the root of some $f \in k[x]$, because the extension is algebraic.

Now, for all $b \in K$, $\de(f(b)) = f(\de(b))$; this is because $\de$ fixes all of the coefficients of $f$.

So if $b$ is a root of $f$ that is in $K$, then 

\begin{align*}
\de(f(b)) &= f(\de(b)) \\
0 &= f(\de(b))
\end{align*}

Note that $\de(b)$ is also in $K$, so this means that $\de$ permutes the roots of $f$ that are in $K$. (There's finitely many of them, and $\de$ is a one-to-one map, so it's a permuation.)

This means that there's a $b \in K$ such that $\de(b) = a$, for all $a \in K$.

So $\de$ is onto. So $\de$ is a one-to-one and onto $k$-homomorphism, it is an isomorphism.

\shunt

{\bf Problem 4:} 

If $k$ is finite, then $k^*$ is cyclic: this is example 5.8 a.

If $k^*$ is cyclic, then it is a finitely generated abelian group; we can apply the fundamental theorem of finitely generated abelian groups to it. In particular, we have that $k^*$ is either group-isomorphic to $\Z/n$ for some $n \in \N$ or $k^*$ is group-isomorphic to $\Z$.

If $k^*$ is group-isomorphic to $\Z$, this means that $(-1_k)(-1_k) = 1_k$, but this means that $(-1_k)$ maps to an element of order $1$ or $2$ in $\Z$. But $\Z$ has no elements of order $2$; this means that $-1_k$ has order $1$ in $k^*$, which means that $-1_k=1_k$. 

So $k$ has characteristic $2$. That is, $\mathbb{F}_2 \subset k$ is a field extension. Because $k^*$ is cyclic, we have that $\mathbb{F}_2(a) = k$ for some $a \in k$. Moreover, $a$ is algebraic; (I don't know how to show this and I give up.)

So $k$ is an algebraic field extension of a finite field. So $k$ is a finite extension of a finite field, it is finite; but this contradicts the fact that $k^*$ is infinte, so $k^*$ cannot have been isomorphic to $\Z$.

So $k^*$ is group-isomorphic to $\Z/n$ for some $n \in \N$.

So $k^*$ is isomorphic to a finite group; $k^* = k \setminus \{0\}$ is finite, so $k$ is finite.

So $k$ is finite if and only if $k^*$ is cyclic.

\shunt

{\bf Problem 5:} 

Let $k$ be a field, and let $k(x)$ be the field of rational functions of $k$.

Consider $p(x)= x^{1/n}$. This has minimal polynomial $y^n - p(x)$.

That is, we can take an element of $\overline{k(x)}$ with arbitrarily large degree of minimal polynomial. That is, $[\overline{k(x)}: k(x)] = \infty]$.

\shunt

\end{document}