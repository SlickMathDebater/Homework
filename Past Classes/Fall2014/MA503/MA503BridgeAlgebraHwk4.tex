
\documentclass[a4paper,12pt]{article}

\usepackage{fancyhdr}
\usepackage{amssymb}
%\usepackage{mathpazo}
\usepackage{mathtools}
\usepackage{amsmath}
\usepackage{slashed}
\usepackage[mathscr]{euscript}

\newcommand{\tab}{\hspace{4mm}} %Spacing aliases
\newcommand{\shunt}{\vspace{20mm}}

\newcommand{\sd}{\partial} %Squiggle d

\newcommand{\absval}[1]{\lvert #1 \rvert}
\newcommand{\anbrack}[1]{\left\langle #1 \right\rangle}
\newcommand{\norm}[1]{\|#1\|}


\newcommand{\al}{\alpha} %Steal ALL of Dr. Kable's Aliases! MWAHAHAHAHA!
\newcommand{\be}{\beta}
\newcommand{\ga}{\gamma}
\newcommand{\Ga}{\Gamma}
\newcommand{\de}{\delta}
\newcommand{\De}{\Delta}
\newcommand{\ep}{\epsilon}
\newcommand{\vep}{\varepsilon}
\newcommand{\ze}{\zeta}
\newcommand{\et}{\eta}
\newcommand{\tha}{\theta}
\newcommand{\vtha}{\vartheta}
\newcommand{\Tha}{\Theta}
\newcommand{\io}{\iota}
\newcommand{\ka}{\kappa}
\newcommand{\la}{\lambda}
\newcommand{\La}{\Lambda}
\newcommand{\rh}{\rho}
\newcommand{\si}{\sigma}
\newcommand{\Si}{\Sigma}
\newcommand{\ta}{\tau}
\newcommand{\ups}{\upsilon}
\newcommand{\Ups}{\Upsilon}
\newcommand{\ph}{\phi}
\newcommand{\Ph}{\Phi}
\newcommand{\vph}{\varphi}
\newcommand{\vpi}{\varpi}
\newcommand{\ch}{\chi}
\newcommand{\ps}{\psi}
\newcommand{\Ps}{\Psi}
\newcommand{\om}{\omega}
\newcommand{\Om}{\Omega}

\newcommand{\bbA}{\mathbb{A}}
\newcommand{\A}{\mathbb{A}}
\newcommand{\bbB}{\mathbb{B}}
\newcommand{\bbC}{\mathbb{C}}
\newcommand{\C}{\mathbb{C}}
\newcommand{\bbD}{\mathbb{D}}
\newcommand{\bbE}{\mathbb{E}}
\newcommand{\bbF}{\mathbb{F}}
\newcommand{\bbG}{\mathbb{G}}
\newcommand{\G}{\mathbb{G}}
\newcommand{\bbH}{\mathbb{H}}
\newcommand{\HH}{\mathbb{H}}
\newcommand{\bbI}{\mathbb{I}}
\newcommand{\I}{\mathbb{I}}
\newcommand{\bbJ}{\mathbb{J}}
\newcommand{\bbK}{\mathbb{K}}
\newcommand{\bbL}{\mathbb{L}}
\newcommand{\bbM}{\mathbb{M}}
\newcommand{\bbN}{\mathbb{N}}
\newcommand{\N}{\mathbb{N}}
\newcommand{\bbO}{\mathbb{O}}
\newcommand{\bbP}{\mathbb{P}}
\newcommand{\PP}{\mathbb{P}}
\newcommand{\bbQ}{\mathbb{Q}}
\newcommand{\Q}{\mathbb{Q}}
\newcommand{\bbR}{\mathbb{R}}
\newcommand{\R}{\mathbb{R}}
\newcommand{\bbS}{\mathbb{S}}
\newcommand{\bbT}{\mathbb{T}}
\newcommand{\bbU}{\mathbb{U}}
\newcommand{\bbV}{\mathbb{V}}
\newcommand{\bbW}{\mathbb{W}}
\newcommand{\bbX}{\mathbb{X}}
\newcommand{\bbY}{\mathbb{Y}}
\newcommand{\bbZ}{\mathbb{Z}}
\newcommand{\Z}{\mathbb{Z}}

\newcommand{\scrA}{\mathcal{A}}
\newcommand{\scrB}{\mathcal{B}}
\newcommand{\scrC}{\mathcal{C}}
\newcommand{\scrD}{\mathcal{D}}
\newcommand{\scrE}{\mathcal{E}}
\newcommand{\scrF}{\mathcal{F}}
\newcommand{\scrG}{\mathcal{G}}
\newcommand{\scrH}{\mathcal{H}}
\newcommand{\scrI}{\mathcal{I}}
\newcommand{\scrJ}{\mathcal{J}}
\newcommand{\scrK}{\mathcal{K}}
\newcommand{\scrL}{\mathcal{L}}
\newcommand{\scrM}{\mathcal{M}}
\newcommand{\scrN}{\mathcal{N}}
\newcommand{\scrO}{\mathcal{O}}
\newcommand{\scrP}{\mathcal{P}}
\newcommand{\scrQ}{\mathcal{Q}}
\newcommand{\scrR}{\mathcal{R}}
\newcommand{\scrS}{\mathcal{S}}
\newcommand{\scrT}{\mathcal{T}}
\newcommand{\scrU}{\mathcal{U}}
\newcommand{\scrV}{\mathcal{V}}
\newcommand{\scrW}{\mathcal{W}}
\newcommand{\scrX}{\mathcal{X}}
\newcommand{\scrY}{\mathcal{Y}}
\newcommand{\scrZ}{\mathcal{Z}}

\newcommand{\subgp}{\mathrel{\unlhd}}

\DeclarePairedDelimiter\ceil{\lceil}{\rceil}
\DeclarePairedDelimiter\floor{\lfloor}{\rfloor}

\newcommand{\colorcomment}[2]{\textcolor{#1}{#2}} %First of these leaves in comments. Second one kills them.
%\newcommand{\colorcomment}[2]{}


\pagestyle{fancy}
\lhead{Max Jeter}
\rhead{MA503, Assignment 4, Page \thepage}

\begin{document}

{\bf Problem 1:}

Let $G$ be a group, and let $a, b \in G$ with $\absval{a} = m$, $\absval{b} = n$.

Part a:

First, if $m \mid k$, then $m = lk$ for some $l \in \Z$. So $a^k = a^{lm} = (a^{m})^l = e^l = e$.

Next, if $m \slashed{\mid} k$, then $k = lm + j$ for some $l \in \Z$, $j \in \N$ with $0 < j < m$. So $a^k = a^{lm+j} = a^{lm}a^j=a^j \neq e$. ($a^j \neq e$ for any $j$ between $0$ and $m$ (exclusive), because otherwise the order of $a$ would be less than $m$, which is against our assumptions.)

So $m \mid k$ if and only if $a^k = e$.

\shunt

Part b:

Let $ab=ba$, and $\anbrack{a} \cap \anbrack{b} = \{e\}$.

First, $\absval{ab} \leq \text{lcm}(m,n)$:

\tab Then $(ab)^{\text{lcm}(m,n)} = a^{\text{lcm}(m,n)}b^{\text{lcm}(m,n)} = ee=e$.

\tab So ${\text{lcm}(m,n)}$ is a positive number with the property $(ab)^{\text{lcm}(m,n)} =e$; $\text{lcm}(m,n)$ is greater than or equal to the order of $ab$. (So $\absval{ab} \leq \text{lcm}(m,n)$).

Next, $\absval{ab} \geq \text{lcm}(m,n)$:

\tab Let $r \in \N$ be such that $(ab)^r = e$.

\tab Then $(ab)^r = a^rb^r = e$. To rewrite this, we know that $a^r = b^{-r}$. Now, because $\anbrack{a} \cap \anbrack{b} = \{e\}$, we know that $a^s = b^t$ for any $s, t \in \Z$ implies that $a^s = b^t = e$. By the earlier problem, this means that $m \mid r$ and $n \mid -r$ (or equivalently, $n \mid r$).

\tab So by elementary number theory, this means that $\text{lcm}(m,n) \mid r$. So $r \geq \text{lcm}(m,n)$ if $(ab)^r = e$ and $r \geq 1$.

So by the squeeze theorem, $\absval{ab} = \text{lcm}(m,n)$.

\shunt

{\bf Problem 2:}

Consider $\de = (1 \; 2 \ldots n)$.

From theorem 4.9a, we know that the number of conjugacy classes of $\de$ is equal to $[G:C(x)]$.

From theorem 5.6, we know that every $n$-cycle is conjugate to $\de$. There are $(n-1)!$ $n$-cycles in $S_n$:

\tab We know that there are $n!$ elements of $S_n$. Pick an element of $S_n$...call it $\sigma$. Now, write the cycle $(\sigma(1) \; \sigma(2) \; \sigma(3) \; \ldots \sigma(n))$. This cycle is equivalent to $n$ other cycles, each given by

\begin{align*}
&(\sigma(2) \; \sigma(3) \; \sigma(4) \; \ldots \sigma(n) \; \sigma(1))\\
&(\sigma(3) \; \sigma(4) \; \sigma(5) \; \ldots \sigma(n) \; \sigma(1) \; \sigma(2))\\
&\ldots \\
&(\sigma(n) \; \sigma(1) \; \sigma(2) \; \ldots \sigma(n-1)).
\end{align*}

\tab So there are $n!/n = (n-1)!$ different $n$-cycles in $S_n$

So $[G:C(x)] = (n-1)!$. So $\absval{C(x)} = n$.

Now, there are $n$ elements of the form $\de^i$; we know from class that an $n$-cycle has order $n$, so $\absval{\{\de^i: i \in \Z\}}=\absval{\anbrack{\de}} = n$.

Each element of the form $\de^i$ commutes with $\de$ trivially.

So the only elements that commute with $\de$ are the elements of the form $\de^i$; there are $n$ of them, and there can only be $n$ different elements that commute with $\de$.

\shunt

{\bf Problem 3:}

The following proof is constructive; it mimicks a selection sort. I attempt to illustrate the proof using crayon, as the proof is nigh-illegible otherwise.

Define $s = (1 \; 2)$ and $r = (1 \; 2 \; 3 \ldots \; n)$. (``swap'' and ``rotation'').

\shunt %Show what s and r do. 

\shunt

Let $\sigma \in S_n$. Then for each $m \in \{1,2, \ldots ,n\}$:

\tab Define $\al_0 = (1)$. Determine $\sigma(1)$. Consider $r^{-(\sigma(1)-1)}$.

\shunt %Show what that rotation does.

\shunt

\tab Define $\al_1 = r^{-\sigma(1) -1}$. From the above diagram, it is clear that $\al_1(1) = \sigma(1)$. Now, determine $\al_1(\sigma(2))$. Define $\be_2 = r^{-(\al_1(\sigma(2))-1)}$. (Look at the picture)

\shunt %Show what beta does.

\shunt

\tab Define $\ga_2 = (rs)^{\be_2\al_1\sigma(2)-\be_2\al_1\sigma(1)) -1}$. (Seriously, just look at the pictures)

\shunt %Show what gamma does.

\shunt

\tab Now define $\al_2 = \ga_2\be_2\al_1$ Now, $\al_2(1) = \sigma(1)$ and $\al_2(2) = \sigma(2)$, as is clear from the illustrations.

\tab We iterate to completion: for each $n \in \N$ we can define $\al_n = \ga_n\be_n\al_{n-1}$ recursively, where $\be_n = r^{-(\al_{n-1}(\sigma(n))-(n-1))}$ and $\ga_n = (rs)^{\be_n\al_{n-1}\sigma(n)-\be_n\al_{n-1}\sigma(n-1)) -(n-1)}$. From the below illustrations, it should be clear that for each $n$, $\al_n(x) = \sigma(x)$ for all $x \leq n$.

\shunt %Show the iterations.

\shunt

\tab So we have constructed $\al_n = \sigma$, with $\al_n$ a product of $r$, $s$, and their inverses. Thus, $\sigma \in \anbrack{(1 \; 2) , (1 \; 2 \; 3 \ldots n)}$ for all $\sigma \in S_n$

In other words, $S_n \subset \anbrack{(1 \; 2) , (1 \; 2 \; 3 \ldots n)}$, which implies that $S_n = \anbrack{(1 \; 2) , (1 \; 2 \; 3 \ldots n)}$ (because subgroups generated by elements are still subgroups.)

\shunt

{\bf Problem 4:}

Let $p$ be a prime number and let $H<S_p$ contain a transposition and act transitively on $\{1,\ldots,p\}$.

From the earlier homework, $H$ has an element with no fixed point (it acts transitively on a finite set).

This means that $H$ contains a $p$-cycle:

\tab We know that $\absval{\overline{1}}\absval{H_1} = \absval{H}$, which implies that $\absval{H} = p \absval{H_{1}}$ by transitivity. So $p \mid \absval{H}$. Thus, $H$ has a subgroup of order $p$. So $H$ has an element of order $p$, by Cauchy's Theorem. We know that an element of $S_p$ of order $p$ is a $p$-cycle. So $H$ has a $p$-cycle.

This means that $H$ contains a transposition and a $p$-cycle; by a quick adaptation of the above problem (it is a bit more than a simple relabeling...but the proof can somewhat clearly be reworked to get the desired result), this means that $H = S_p$.

\shunt

{\bf Problem 5:}

This is given as an exercise in Hungerford: out of a sense of honesty, I must admit that I ran across this in the book, instead of coming up with it independently.

Consider $ H =\{(1), (1 \; 2)(3 \; 4), (1 \; 3) (2 \; 4), (1 \; 4) (2 \; 3)\}$.

First, $H \leq G$;

\tab We apply the subgroup criterion, and proceed by exhaustion. (In the below, I freely use the facts that $2$-cycles are their own inverse and that disjoint cycles commute).

\begin{align*}
(1) (1)^{-1} &= (1)\\
(1) ((1 \; 2)(3 \; 4))^{-1} &= (3 \; 4)(1 \; 2) = (1 \; 2)(3 \; 4)\\
(1) ((1 \; 3) (2 \; 4))^{-1} &= (2 \; 4) (1 \; 3)  =  (1 \; 3) (2 \; 4)\\
(1) ((1 \; 4) (2 \; 3))^{-1} &=  (2 \; 3) (1 \; 4)= (1 \; 4) (2 \; 3)\\
(1 \; 2)(3 \; 4) (1)^{-1} &= (1 \; 2)(3 \; 4)\\
(1 \; 2)(3 \; 4) ((1 \; 2)(3 \; 4))^{-1} &= (1)\\
(1 \; 2)(3 \; 4) ((1 \; 3) (2 \; 4))^{-1} &= (1 \; 2)(3 \; 4) (2 \; 4) (1 \; 3) = (1 \; 4) (2 \; 3)\\
(1 \; 2)(3 \; 4) ((1 \; 4) (2 \; 3))^{-1} &= (1 \; 2)(3 \; 4) (2 \; 3) (1 \; 4) = (1 \; 3) (2 \; 4)\\
(1 \; 3) (2 \; 4) (1)^{-1} &= (1 \; 3) (2 \; 4)\\
(1 \; 3) (2 \; 4) ((1 \; 2)(3 \; 4))^{-1} &= (1 \; 3) (2 \; 4) (3 \; 4) (1 \; 2) =(1 \; 4) (2 \; 3)\\
(1 \; 3) (2 \; 4) ((1 \; 3) (2 \; 4))^{-1} &= (1)\\
(1 \; 3) (2 \; 4) ((1 \; 4) (2 \; 3))^{-1} &= (1 \; 3) (2 \; 4) (2 \; 3) (1 \; 4) = (1 \; 2) (3 \; 4)\\
(1 \; 4) (2 \; 3) (1)^{-1} &= (1 \; 4) (2 \; 3)\\
(1 \; 4) (2 \; 3) ((1 \; 2)(3 \; 4))^{-1} &= (1 \; 4) (2 \; 3) (3 \; 4) (1 \; 2) = (1 \; 3) (2 \; 4)\\
(1 \; 4) (2 \; 3) ((1 \; 3) (2 \; 4))^{-1} &= (1 \; 4) (2 \; 3) (2 \; 4) (1 \; 3) = (1 \; 2) (3 \; 4)\\
(1 \; 4) (2 \; 3) ((1 \; 4) (2 \; 3))^{-1} &= (1)
\end{align*}

Next, $H \subgp S_4$;

\tab Recall that two elements of $S_4$ are conjugate if and only if their cycle decomposition has the same cycle type. Note that $H$ contains all of the elements of $S_4$ composed of a product of two disjoint $2$-cycles.

\tab So an element is conjugate to an element of $H$ if and only if it is in $H$. That is, $ghg^{-1} \in H$ for all $g \in G$, $h \in H$.

Thus, $H \subgp S_4$. (And so, $H \subgp A_4$).

\shunt

{\bf Problem 6:}

Note: I adapt the proof in Hungerford to fit my needs for this problem.

We know from class that for $n \geq 5$, $A_n$ is simple.

Let $N \subgp S_n$, with $n \geq 5$, with $N$ nontrivial.

Then either $N$ contains a $3$-cycle, $N$ contains an element $\sigma$ with its cycle decomposition having a cycle of length $r \geq 4$, $N$ contains an element $\sigma$ with its cycle decomposition having at least two cycles of length $3$, $N$ contains an element $\sigma$ that is a product of one $3$-cycle and some $2$-cycles, or every element of $N$ is a product of disjoint $2$-cycles.

\tab Case 1: If $N$ contains a $3$-cycle, then $N$ contains $A_n$; this implies that either $N=A_n$ or $N = S_n$.

\tab Case 2: If $N$ contains an element $\sigma$ with its cycle decomposition having a cycle of length $r \geq 4$, then $\sigma = (a_1a_2\ldots a_r)\tau$ for some $\tau$ disjoint from the first cycle. Let $\de = (a_1a_2a_3)$. Then $\sigma^{-1}(\de \sigma \de^{-1}) \in N$ by normality. But

\begin{displaymath}
\sigma^{-1}(\de \sigma \de^{-1}) = \tau^{-1}(a_1a_ra_{r-1} \ldots a_2)(a_1a_2a_3)(a_1a_2a_3 \ldots a_r)\tau (a_1a_3a_2) = (a_1a_3a_r) \in N.
\end{displaymath}


\tab So $N$ has a $3$-cycle, and we regress to the first case.

\tab Case 3: If $N$ contains an element $\sigma$ with its cycle decomposition having at least two cycles of length $3$, then $\sigma = (a_1a_2a_3)(a_4a_5a_6)\tau$ with $\tau$ disjoint from the first two cycles. Let $\de = (a_1a_2a_4)$. Then, as above, 

\begin{displaymath}
\sigma^{-1}(\de \sigma \de^{-1}) = \tau^{-1}(a_4a_6a_5)(a_1a_3a_2)(a_1a_2a_4)(a_1a_2a_3)(a_4a_5a_6)\tau (a_1a_4a_2) = (a_1a_4a_2a_6a_3) \in N.
\end{displaymath}

\tab And so we regress to the second case.

\tab Case 4: If $N$ contains an element $\sigma$ that is a product of one $3$-cycle and some $2$-cycles, then $\sigma = (a_1a_2a_3)\tau$ where $\tau$ is disjoint from the first cycle and is a product of disjoint $2$-cycles. Then $\sigma ^2 \in N$ and so,

\begin{displaymath}
\si^2 =  (a_1a_2a_3)\tau(a_1a_2a_3)\tau = (a_1a_3a_2)\tau^2 = (a_1a_3a_2)
\end{displaymath}

\tab So we regress to the first case.

\tab Case 5: If every element of $N$ is a product of disjoint $2$-cycles, then consider any $\si \in N$. If some $\si$ is a $2$-cycle, then $N = S_n$, by normality of $N$ and the fact that the $2$-cycles generate $S_n$. If not, then let $\si = (a_1a_2)(a_3a_4)\tau$ where $\tau$ is disjoint from the first two cycles. Then let $\de = (a_1a_2a_3)$. Then as above:

\begin{displaymath}
\sigma^{-1}(\de \sigma \de^{-1}) = \tau^{-1}(a_3a_4)(a_1a_2)(a_1a_2a_3)(a_1a_2)(a_3a_4)\tau(a_1a_3a_2)=(a_1a_3)(a_2a_4) \in \N
\end{displaymath}

\tab Define the above permutation to be $\ga$. There's an element, $a_5$, distinct from each of $a_1 \ldots a_4$. Consider $\la = (a_1a_3a_5)$. Now, we know that 

\begin{displaymath}
\ga^{-1}(\la \ga \la^{-1}) = (a_1a_3)(a_2a_4)(a_1a_3a_5)(a_1a_3)(a_2a_4)(a_1a_5a_3) = (a_1a_3a_5) \in \N
\end{displaymath}

\tab So we regress to the first case.

That is, in all cases, we have that a nontrivial subgroup of $S_n$ is either $A_n$ or $S_n$. This satisfies the problem.

\shunt

{\bf Problem 7:}

Define $\phi: \text{Aut}(A_4) \to S_4$ as follows:

\tab Define $a = (1 \; 2 \; 3)$, $b= (1 \; 2 \; 4)$, $c = (1 \; 3 \; 4)$, and $d = (2 \; 3 \; 4)$.

\tab Define $F: \{a,b,c,d\} \to \{1,2,3,4\}$ by $F(a) =1$, $F(b) = 2$, $F(c) = 3$, and $F(d) = 4$. (This is done primarily for convenience. It is rather clear that $F$ is bijective.)

\tab Then $\phi(\psi)$ is the transposition naturally given by the way that $\psi$ permutes these elements. That is,

\begin{align*}
\phi(\psi)(1) &= F(\psi(F^{-1}(1)) \\
\phi(\psi)(2) &= F(\psi(F^{-1}(2)) \\
\phi(\psi)(3) &= F(\psi(F^{-1}(3)) \\
\phi(\psi)(4) &= F(\psi(F^{-1}(4)) 
\end{align*}

\tab First: $\phi$ is well defined:

\tab \tab If $\psi \in \text{Aut}(A_4)$, then $\psi$ is one-to-one. So each of the expressions in the definition of $\phi$ is distinct from each other; that is, $\phi(\psi)$ is a permutation of $\{1,2,3,4\}$. That is, $\phi$ is well-defined.

\tab Next, $\phi$ is a homomorphism:

\tab \tab Let $\psi, \chi \in \text{Aut}(A_4)$. Then:

\begin{align*}
\phi(\psi\chi)(n) &= F(\psi\chi(F^{-1}(n))\\
&= F(\psi(\chi(F^{-1}(n)))\\
&= F(\psi(F^{-1}(F(\chi(F^{-1}(n)))))\\
&= \phi(\psi)\phi(\chi)(n)
\end{align*}

\tab Last, $\phi$ is bijective:

\tab \tab First, $\phi$ is injective:

\tab \tab \tab Let $\psi, \chi \in \text{Aut}(A_4)$, with $\phi(\psi) = \phi(\chi)$.

\tab \tab \tab Then $\psi(l) = \chi(l)$, for all $l \in \{a,b,c,d\}$.

\tab \tab \tab So $\psi = \chi$ on the generators of $A_4$. Thus, $\psi=\chi$ on $A_4$. This shows that $\phi$ is injective.

\tab \tab Next, $\phi$ is surjective:

\tab \tab \tab It is clear that $\text{Aut}(A_4)$ has $24$ elements: each permutation of $\{a,b,c,d\}$ defines an automorphism of $A_4$, and there are $24$ permutations of a $4$ element set.

Thus, we have an isomorphism from $A_4$ and $S_4$; $A_4 \cong S_4$.

\shunt

\end{document}