
\documentclass[a4paper,12pt]{article}

\usepackage{fancyhdr}
\usepackage{amssymb}
%\usepackage{mathpazo}
\usepackage{mathtools}
\usepackage{amsmath}
\usepackage{slashed}
\usepackage[mathscr]{euscript}

\newcommand{\tab}{\hspace{4mm}} %Spacing aliases
\newcommand{\shunt}{\vspace{20mm}}

\newcommand{\sd}{\partial} %Squiggle d

\newcommand{\absval}[1]{\lvert #1 \rvert}
\newcommand{\anbrack}[1]{\left\langle #1 \right\rangle}
\newcommand{\norm}[1]{\|#1\|}


\newcommand{\al}{\alpha} %Steal ALL of Dr. Kable's Aliases! MWAHAHAHAHA!
\newcommand{\be}{\beta}
\newcommand{\ga}{\gamma}
\newcommand{\Ga}{\Gamma}
\newcommand{\de}{\delta}
\newcommand{\De}{\Delta}
\newcommand{\ep}{\epsilon}
\newcommand{\vep}{\varepsilon}
\newcommand{\ze}{\zeta}
\newcommand{\et}{\eta}
\newcommand{\tha}{\theta}
\newcommand{\vtha}{\vartheta}
\newcommand{\Tha}{\Theta}
\newcommand{\io}{\iota}
\newcommand{\ka}{\kappa}
\newcommand{\la}{\lambda}
\newcommand{\La}{\Lambda}
\newcommand{\rh}{\rho}
\newcommand{\si}{\sigma}
\newcommand{\Si}{\Sigma}
\newcommand{\ta}{\tau}
\newcommand{\ups}{\upsilon}
\newcommand{\Ups}{\Upsilon}
\newcommand{\ph}{\phi}
\newcommand{\Ph}{\Phi}
\newcommand{\vph}{\varphi}
\newcommand{\vpi}{\varpi}
\newcommand{\ch}{\chi}
\newcommand{\ps}{\psi}
\newcommand{\Ps}{\Psi}
\newcommand{\om}{\omega}
\newcommand{\Om}{\Omega}

\newcommand{\bbA}{\mathbb{A}}
\newcommand{\A}{\mathbb{A}}
\newcommand{\bbB}{\mathbb{B}}
\newcommand{\bbC}{\mathbb{C}}
\newcommand{\C}{\mathbb{C}}
\newcommand{\bbD}{\mathbb{D}}
\newcommand{\bbE}{\mathbb{E}}
\newcommand{\bbF}{\mathbb{F}}
\newcommand{\bbG}{\mathbb{G}}
\newcommand{\G}{\mathbb{G}}
\newcommand{\bbH}{\mathbb{H}}
\newcommand{\HH}{\mathbb{H}}
\newcommand{\bbI}{\mathbb{I}}
\newcommand{\I}{\mathbb{I}}
\newcommand{\bbJ}{\mathbb{J}}
\newcommand{\bbK}{\mathbb{K}}
\newcommand{\bbL}{\mathbb{L}}
\newcommand{\bbM}{\mathbb{M}}
\newcommand{\bbN}{\mathbb{N}}
\newcommand{\N}{\mathbb{N}}
\newcommand{\bbO}{\mathbb{O}}
\newcommand{\bbP}{\mathbb{P}}
\newcommand{\PP}{\mathbb{P}}
\newcommand{\bbQ}{\mathbb{Q}}
\newcommand{\Q}{\mathbb{Q}}
\newcommand{\bbR}{\mathbb{R}}
\newcommand{\R}{\mathbb{R}}
\newcommand{\bbS}{\mathbb{S}}
\newcommand{\bbT}{\mathbb{T}}
\newcommand{\bbU}{\mathbb{U}}
\newcommand{\bbV}{\mathbb{V}}
\newcommand{\bbW}{\mathbb{W}}
\newcommand{\bbX}{\mathbb{X}}
\newcommand{\bbY}{\mathbb{Y}}
\newcommand{\bbZ}{\mathbb{Z}}
\newcommand{\Z}{\mathbb{Z}}

\newcommand{\scrA}{\mathcal{A}}
\newcommand{\scrB}{\mathcal{B}}
\newcommand{\scrC}{\mathcal{C}}
\newcommand{\scrD}{\mathcal{D}}
\newcommand{\scrE}{\mathcal{E}}
\newcommand{\scrF}{\mathcal{F}}
\newcommand{\scrG}{\mathcal{G}}
\newcommand{\scrH}{\mathcal{H}}
\newcommand{\scrI}{\mathcal{I}}
\newcommand{\scrJ}{\mathcal{J}}
\newcommand{\scrK}{\mathcal{K}}
\newcommand{\scrL}{\mathcal{L}}
\newcommand{\scrM}{\mathcal{M}}
\newcommand{\scrN}{\mathcal{N}}
\newcommand{\scrO}{\mathcal{O}}
\newcommand{\scrP}{\mathcal{P}}
\newcommand{\scrQ}{\mathcal{Q}}
\newcommand{\scrR}{\mathcal{R}}
\newcommand{\scrS}{\mathcal{S}}
\newcommand{\scrT}{\mathcal{T}}
\newcommand{\scrU}{\mathcal{U}}
\newcommand{\scrV}{\mathcal{V}}
\newcommand{\scrW}{\mathcal{W}}
\newcommand{\scrX}{\mathcal{X}}
\newcommand{\scrY}{\mathcal{Y}}
\newcommand{\scrZ}{\mathcal{Z}}

\newcommand{\subgp}{\mathrel{\unlhd}}

\DeclarePairedDelimiter\ceil{\lceil}{\rceil}
\DeclarePairedDelimiter\floor{\lfloor}{\rfloor}

\newcommand{\colorcomment}[2]{\textcolor{#1}{#2}} %First of these leaves in comments. Second one kills them.
%\newcommand{\colorcomment}[2]{}


\pagestyle{fancy}
\lhead{Max Jeter}
\rhead{MA503, Assignment 8, Page \thepage}

\begin{document}

I had only a rough idea how to do this problem set, so I couldn't really come close to finishing it. :/ Here's the scraps of what I sort of figured out.

{\bf Problem 1:} 

Let $R$ be a commutative ring with $p = a_nx^n + \ldots + a_0$ a zero divisor in $R[x]$.

Then there's a nonzero $q \in R[x]$ such that $pq=0$. We can choose $q$ to have minimal degree; let $q=b_mx^m+ \ldots +b_0$ be such that $pq=0$ and $q$ has minimal degree.

Define $b$ to be the product of all of the nonzero $b_i$s. Then $b$ is nonzero. Else, we can reduce the degree of $q$ (I'm not sure how the details go.)

Next, we show that $ba_i=0$ for all $i$;

\tab There is a first $i$ such that $b_i \neq 0$, call it $I$. Then $b_Ia_0=0$. So $ba_0=0$.

\tab Now, let $ba_i=0$ for all $i < N$. Then %I don't know what to do here.

\begin{align*}
\sum\limits_{i=0}^{N+I} ba_ib_{N+I-i} &=0\\
ba_Nb_I &=0\\
\end{align*}

So by induction, $ba_i=0$ for all $i$, and $b$ is nonzero, as desired.

\shunt

{\bf Problem 2:}

Part a:

First, let $p = a_kx^k+ \ldots +a_0 \in \text{rad}(R[x])$.

\tab Then $p^n = 0$ for some $n \in \N$. So $a_0^n=0$.

\tab Now, if for some $m\geq n$, we have $a_j^m=0$ for all $j<N$, then we have $p^m = 0$, and %stuff

Next, let $p \in \{a_nx^n + \ldots + a_0: a_i \in \text{rad}(R)\}$.

\tab Then for each $i$, there's an $n_i$ such that $a_i^{n_i}=0$. Define $n$ to be the product of all of the $n_i$s.

\tab Then we have

\begin{align*}
p^n&=stuff\\
\end{align*}

\shunt 

Part b:

Let $p =a_kx^k+ \ldots +a_0\in (R[x])^*$.

\tab Then there's a $q = b_lx^l+ \ldots +b_0 \in R[x]$ such that $pq=1$. So $a_0b_0=1$, so $a_0 \in R^*$.

\tab Further, stuff.

Next, let $p \in \{a_nx^n + \ldots + a_0: a_0 \in R^*, a_i \in \text{rad}(R) \text{ if and only if } i >0\}$.

\tab Then there's a $b$ such that $a_0b=1$.

\tab Consider $bp-1$. It is clear that $bp-1 \in \text{rad}(R[x])$, so $(bp-1)^m=0$ for some $m$. Without loss of generality, we can take $m$ to be odd. So $(bp-1)^m+1=1$. But $(bp-1)^m + 1$ is a multiple of $p$; every term of $(bp-1)^m$'s expansion except for $-1$ has a $p$ attached to it. So $p$ multiplied by something is $1$; $p$ is a unit.

\shunt

{\bf Problem 4:}

Let $n=4$ or $n=2^ip^j$ for some odd prime $p$, $i=0$ or $i=1$, and $j \geq 0$.

\tab If $n=4$, then $(\Z/n)^* \cong \Z/2$, which is cyclic.

\tab If $n=p^j$ for some odd prime $p$, then $(\Z/n)^*$ is generated, as a group, by $2$, and is thus cyclic. %Proof

\tab If $n=2p^j$ for some odd prime $p$, then $(\Z/n)^*$ is generated, as a group, by $3$, and is thus cyclic. %Proof

Next, let $(\Z/n)^*$ be cyclic.

\tab Assume that $n \neq 4$ and $n \neq 2^ip^j$ for any odd prime $p$, $i=0$ or $i=1$, and $j \geq 0$.

\tab This means that $n$ must be at least $8$. We have that $n$ has at least one odd prime in its prime factorization or it is a power of two. 

\tab If $n$ is a power of two greater than $8$, then $(\Z/n)^*$ is not cyclic; %Proof

\tab For the types of $n$ we have described, if $n$ has at least one odd prime in its prime factorization, then we know that either $n$ has two odd primes in its prime factorization or $n$ has a power of two greater than $8$ in its prime factorization.

\tab If $n$ has two odd primes in its prime factorization, then $(\Z/n)^*$ is not cyclic; it's isomorphic to a product of more than one nontrivial group, by the Chinese Remainder Theorem. (Note: we needed an odd prime for this because the Chinese Remainder Theorem burns anything that looks like $\Z/2$). 

\tab If $n$ has a power of two greater than $8$ in its prime factorization, then there's a homomorphism, $\phi$, from $\Z/n$ to $\Z/2^k$ for some $k$ at least $3$, having the property that $\phi$ preserves units. We know that $(\Z/2^k)^*$ is not cyclic, and so neither can $(\Z/n)^*$ be. (If $(\Z/n)^*$ was cyclic, we could take its generator, $a$, and have $\phi(a)$ generate $(\Z/2^k)^*$).

So $(\Z/n)^*$ is cyclic if and only if $n=4$ or $n=2^ip^j$ for some odd prime $p$, $i=0$ or $i=1$, and $j \geq 0$.

\shunt

\end{document}