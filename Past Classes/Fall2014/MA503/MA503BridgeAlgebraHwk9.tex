
\documentclass[a4paper,12pt]{article}

\usepackage{fancyhdr}
\usepackage{amssymb}
%\usepackage{mathpazo}
\usepackage{mathtools}
\usepackage{amsmath}
\usepackage{slashed}
\usepackage[mathscr]{euscript}

\newcommand{\tab}{\hspace{4mm}} %Spacing aliases
\newcommand{\shunt}{\vspace{20mm}}

\newcommand{\sd}{\partial} %Squiggle d

\newcommand{\absval}[1]{\lvert #1 \rvert}
\newcommand{\anbrack}[1]{\left\langle #1 \right\rangle}
\newcommand{\norm}[1]{\|#1\|}


\newcommand{\al}{\alpha} %Steal ALL of Dr. Kable's Aliases! MWAHAHAHAHA!
\newcommand{\be}{\beta}
\newcommand{\ga}{\gamma}
\newcommand{\Ga}{\Gamma}
\newcommand{\de}{\delta}
\newcommand{\De}{\Delta}
\newcommand{\ep}{\epsilon}
\newcommand{\vep}{\varepsilon}
\newcommand{\ze}{\zeta}
\newcommand{\et}{\eta}
\newcommand{\tha}{\theta}
\newcommand{\vtha}{\vartheta}
\newcommand{\Tha}{\Theta}
\newcommand{\io}{\iota}
\newcommand{\ka}{\kappa}
\newcommand{\la}{\lambda}
\newcommand{\La}{\Lambda}
\newcommand{\rh}{\rho}
\newcommand{\si}{\sigma}
\newcommand{\Si}{\Sigma}
\newcommand{\ta}{\tau}
\newcommand{\ups}{\upsilon}
\newcommand{\Ups}{\Upsilon}
\newcommand{\ph}{\phi}
\newcommand{\Ph}{\Phi}
\newcommand{\vph}{\varphi}
\newcommand{\vpi}{\varpi}
\newcommand{\ch}{\chi}
\newcommand{\ps}{\psi}
\newcommand{\Ps}{\Psi}
\newcommand{\om}{\omega}
\newcommand{\Om}{\Omega}

\newcommand{\bbA}{\mathbb{A}}
\newcommand{\A}{\mathbb{A}}
\newcommand{\bbB}{\mathbb{B}}
\newcommand{\bbC}{\mathbb{C}}
\newcommand{\C}{\mathbb{C}}
\newcommand{\bbD}{\mathbb{D}}
\newcommand{\bbE}{\mathbb{E}}
\newcommand{\bbF}{\mathbb{F}}
\newcommand{\bbG}{\mathbb{G}}
\newcommand{\G}{\mathbb{G}}
\newcommand{\bbH}{\mathbb{H}}
\newcommand{\HH}{\mathbb{H}}
\newcommand{\bbI}{\mathbb{I}}
\newcommand{\I}{\mathbb{I}}
\newcommand{\bbJ}{\mathbb{J}}
\newcommand{\bbK}{\mathbb{K}}
\newcommand{\bbL}{\mathbb{L}}
\newcommand{\bbM}{\mathbb{M}}
\newcommand{\bbN}{\mathbb{N}}
\newcommand{\N}{\mathbb{N}}
\newcommand{\bbO}{\mathbb{O}}
\newcommand{\bbP}{\mathbb{P}}
\newcommand{\PP}{\mathbb{P}}
\newcommand{\bbQ}{\mathbb{Q}}
\newcommand{\Q}{\mathbb{Q}}
\newcommand{\bbR}{\mathbb{R}}
\newcommand{\R}{\mathbb{R}}
\newcommand{\bbS}{\mathbb{S}}
\newcommand{\bbT}{\mathbb{T}}
\newcommand{\bbU}{\mathbb{U}}
\newcommand{\bbV}{\mathbb{V}}
\newcommand{\bbW}{\mathbb{W}}
\newcommand{\bbX}{\mathbb{X}}
\newcommand{\bbY}{\mathbb{Y}}
\newcommand{\bbZ}{\mathbb{Z}}
\newcommand{\Z}{\mathbb{Z}}

\newcommand{\scrA}{\mathcal{A}}
\newcommand{\scrB}{\mathcal{B}}
\newcommand{\scrC}{\mathcal{C}}
\newcommand{\scrD}{\mathcal{D}}
\newcommand{\scrE}{\mathcal{E}}
\newcommand{\scrF}{\mathcal{F}}
\newcommand{\scrG}{\mathcal{G}}
\newcommand{\scrH}{\mathcal{H}}
\newcommand{\scrI}{\mathcal{I}}
\newcommand{\scrJ}{\mathcal{J}}
\newcommand{\scrK}{\mathcal{K}}
\newcommand{\scrL}{\mathcal{L}}
\newcommand{\scrM}{\mathcal{M}}
\newcommand{\scrN}{\mathcal{N}}
\newcommand{\scrO}{\mathcal{O}}
\newcommand{\scrP}{\mathcal{P}}
\newcommand{\scrQ}{\mathcal{Q}}
\newcommand{\scrR}{\mathcal{R}}
\newcommand{\scrS}{\mathcal{S}}
\newcommand{\scrT}{\mathcal{T}}
\newcommand{\scrU}{\mathcal{U}}
\newcommand{\scrV}{\mathcal{V}}
\newcommand{\scrW}{\mathcal{W}}
\newcommand{\scrX}{\mathcal{X}}
\newcommand{\scrY}{\mathcal{Y}}
\newcommand{\scrZ}{\mathcal{Z}}

\newcommand{\subgp}{\mathrel{\unlhd}}

\DeclarePairedDelimiter\ceil{\lceil}{\rceil}
\DeclarePairedDelimiter\floor{\lfloor}{\rfloor}

\newcommand{\colorcomment}[2]{\textcolor{#1}{#2}} %First of these leaves in comments. Second one kills them.
%\newcommand{\colorcomment}[2]{}


\pagestyle{fancy}
\lhead{Max Jeter}
\rhead{MA503, Assignment 8, Page \thepage}

\begin{document}

{\bf Problem 1:} 

Let $R$ be a UFD and $P$ be a prime ideal.

Let $P$ fail to be principal. Let $a \in P$.

Now, $a$ has a prime factorization, $p_1^{\al_1}\ldots p_n^{\al_n}$.

Then one of the $p_i^{\al_i}$ is in $P$; $a \in P$, so $p_1^{\al_1} \in P$ or $p_2^{\al_2}\ldots p_n^{\al_n} \in P$. If $p_2^{\al_2}\ldots p_n^{\al_n} \in P$, then $p_2^{\al_2} \in P$ or $p_3^{\al_3}\ldots p_n^{\al_n} \in P$. We can iterate this process, so one of the $p_i^{\al_i}$ is in $P$.

So $p_i \in P$, by applying the same method.

So $(p_i) \subset P$. Because $p_i$ is prime, $(p_i)$ is prime (and nonzero). But it's not $P$, as $P$ is not principal.

So $P$ has a proper, nonzero prime ideal. 

\shunt

{\bf Problem 2:} 

Let $k$ be a field and $n \geq 2$.

If $\text{char}(k) = 2$, $x_1^2 + x_2^2 \ldots x_n^2 -1$ is equal to $(x_1+x_2+\ldots+x_n-1)^2$ (when you multiply it out,every term has a factor of $2$ except the $x_i^2$ and $-1$ terms) and so $x_1^2 + x_2^2 \ldots x_n^2 -1$ is reducible.

Now, if $\text{char}(k) \neq 2$, then $x_1^2 +x_2^2 -1$ is irreducible in $k[x_1,x_2]$; $x_1^2+x_2^2 -1$ is a is a monic polynomial of degree $2$ in $k[x_1][x_2] = k[x_1, x_2]$. So if it factors, it factors into a product of degree $1$ polynomials; so it factors into something of the form $(x_2+s)(x_2+r)$, with $r$ and $s$ both in $k[x_1]$. But the only way for this to happen is if $s=-r$. That is, $1-x_1^2$ must be a perfect square. However, its unique prime factorization is $(x_1+1)(x_1-1)$; it is not a perfect square.

We proceed by induction:

\tab Let $x_1^2 + x_2^2 \ldots x_{n-1}^2 -1$ is irreducible in $k[x_1, x_2 \ldots x_{n-1}]$, and set this equal to $p$. It is clear that $x_n^2+p$ is a monic polynomial of degree $2$ in $k[x_1, x_2 \ldots x_{n-1}][x_n] = k[x_1, x_2 \ldots x_{n}]$. So if it factors, it factors into a product of degree $1$ polynomials; so it factors into something of the form $(x_n+s)(x_n+r)$, with $r$ and $s$ both in $k[x_1, x_2 \ldots x_{n-1}]$. But this would mean that $p=rs$ for some $r,s \in k[x_1, x_2 \ldots x_{n-1}]$, so $p$ would be reducible.

\tab So if $x_1^2 + x_2^2 \ldots x_{n-1}^2 -1$ is irreducible in $k[x_1, x_2 \ldots x_{n-1}]$, then $x_1^2 + x_2^2 \ldots x_{n}^2 -1$ is irreducible in $k[x_1, x_2 \ldots x_{n}]$.

So we have our result. 


\shunt

{\bf Problem 3:}

By the reduction criterion, $x^4 + 3x^3 + 3x^2 -5$ is irreducible in $\Z[x]$ if it is irreducible in $\Z/(7)[x]$.

Now, if $x^4 + 3x^3 + 3x^2 -5 = x^4 + 3x^3 + 3x^2 +2$ is reducible, it either has a root or it can be written as a product of two monic order $2$ polynomials.

But $p(x) = x^4 + 3x^3 + 3x^2 +2$ has no roots; 

\begin{align*}
p(0)=2\\
p(1)=2\\
p(2)=5\\
p(3)=2\\
p(4)=1\\
p(5)=6\\
p(6)=3
\end{align*}

So if $p$ is reducible, then $p$ can be written as a product of two monic order $2$ polynomials. That is, 

\begin{align*}
x^4 + 3x^3 + 3x^2 +2 &= (x^2+ax+b)(x^2+cx+d)\\
&=x^4+(a+c)x^3+(b+d+ac)x^2+(ad+bc)x+bd
\end{align*}

This means that

\begin{align*}
a+c&=3\\
b+d+ac&=3\\
ad+bc&=0\\
bd&=2
\end{align*}

The first equation can be reduced to $c=3-a$, which yields

\begin{align*}
b+d+3a-a^2&=3 \text{ ($\al$)}\\
ad+3b-ba&=0 \text{ ($\be$)}\\
bd&=2 \text{ ($\ga$)}
\end{align*}

Now, ($\ga$) only $6$ solutions; we are working in a field, so for any given $b$ there is a unique solution of that equation for $d$. Also, $b=0$ fails.

So we have that the only six valid solutions for $b$ and $d$ are:

\begin{align*}
b=1,d=2\\
b=2,d=1\\
b=3,d=3\\
b=4,d=4\\
b=5,d=6\\
b=6,d=5\\
\end{align*}

The middle two fail, for any value of $a$; because $ad=ba$, ($\be$) gives us $3b=0$, which fails for any nonzero $b$. We are left with 

\begin{align*}
b=1,d=2\\
b=2,d=1\\
b=5,d=6\\
b=6,d=5\\
\end{align*}

A rearrangement of ($\al$) gives us $a(3-a)=3-b-d$.

For the first two cases, $b+d=3$, so we have that $a=3$ or $a=0$. If $a=0$, then ($\be$) reduces to $3b=0$, which fails for any nonzero $b$. If $a=3$, then ($\be$) reduces to $3d=0$ which fails for any nonzero $d$.

For the last two cases, $b+d=30=2$. This means that ($\al$) gives us that $a(3-a)=1$. But this is unsatisfiable; if $q = a(3-a)$, then:

\begin{align*}
q(0)=0\\
q(1)=2\\
q(2)=2\\
q(3)=0\\
q(4)=3\\
q(5)=4\\
q(6)=3\\
\end{align*}

That is, every possible solution for ($\ga$) modulo $7$ fails to satisfy the system of equations. That is, we cannot reduce $x^4 + 3x^3 + 3x^2 +2$ modulo $7$. 

So by the reduction criterion, $x^4 + 3x^3 + 3x^2 -5$ is irreducible in $\Z[x]$. So $x^4 + 3x^3 + 3x^2 -5$ is irreducible in $\Q[x]$.

\shunt

{\bf Problem 4:} 

Let $R = \Z[\sqrt{-5}]$, and $K = \text{Quot}(R)$. 

Consider $3x^2 + 4x + 3$. By the quadratic formula, if this polynomial has roots, they are $\frac{-2}{3} \pm \frac{\sqrt{-5}}{3}$. A factorization of $3x^2 + 4x + 3$ is given by $3(x+\frac{2}{3} + \frac{\sqrt{-5}}{3})(x+\frac{2}{3} - \frac{\sqrt{-5}}{3})$. So the polynomial is reducible in $K[x]$.

Now, in $R[x]$, $3x^2 + 4x + 3$ cannot have a constant factored out of it. As it is a degree $2$ polynomial, this means that it factors only as a product of two degree $1$ polynomials. So any factorization of that polynomial must be of the form $(rx+r'(2 +\sqrt{-5}))(sx+s'(2 -\sqrt{-5}))$, with $r',s' \in \Z[\sqrt{-5}]$ and $r=3r'$, $s=3s'$. Yet, this means that the leading coefficient of the polynomial is a multiple of $9$, which $3$ isn't. So the polynomial is irreducible in $R[x]$.

\shunt

{\bf Problem 5:} 

Note: I'm playing fast and loose with notation. I recognize this, but feel that it's still clear in context what is meant.

Let $R$ be a UFD and $P$ be a prime ideal of $R[x]$ with $P \cap R =0$. Define $K = \text{quot}(R)$.

We can view $P$ as a subset of $K[x]$. Consider $(P) \subset K[x]$. We see that $(P)$ is principal, as $K[x]$ is a principal ideal domain. Moreover, $(P)$ is not the entire ring, because $P \cap R = 0$. So $(P) = (p)$ for some $p \in K[x]$, with $p$ having degree at least $1$. %Note also that $(P)$ is prime.

We can impose that $p \in P$; if it isn't, then we can multiply $p$ by the least common multiple of the quotients of the coefficients of $p$ to get it in $R[x]$. Also, $R[x] \cap (P) = P$; if $x \in (P) \cap R[x]$ and, then there's a nonzero $r \in R$ such that $rx \in P$, but $r \notin P$, so $x \in P$. (Also, $P \subset R[x]$ and $P \subset (P)$).

Further, we can impose that the leading coefficient of $p$ divides the leading coefficient of any element of $P$.

This means that $p$ must be prime in $K[x]$; let $r,s \in K[x]$ with $rs = p$. One of $r$ or $s$ must have degree at least one, then. We can factor out a constant, $k$, with $p=kr's'$ and $r',s' \in R[x]$. That is, $r's' \in (P) \cap R[x]$, so $r's' \in P$. This means that $r'$ or $s'$ is in $P$. So $r'$ or $s'$ is in $(P)$. So $r$ or $s$ is a multiple of $p$ in $K[x]$. So $p$ is irreducible in $K[x]$, so $p$ is prime.

Thus, $p$ is prime in $R[x]$.

Now, we imposed that $p \in P$, so $(p) \subset P$.

Next, let $q \in P$. Then $q \in (P) \subset K[x]$, so $q \in (p) \subset K[x]$. That means that $p \mid q$ in $K[x]$; there is an $r \in K[x]$ such that $q=pr$. By multiplying by the least common multiple of the quotients of the coefficients of $r$, we can say that there is a $k \in R$ such that $kq=pr'$ for some $r' \in R[x]$. That is, $kq \in (p)$ for some $k \in R$. But $(p)$ is prime, and $(p) \subset P$. So $k \notin (p)$, so $q \in (p)$.

So $(p) = P$; $P$ is a principal ideal. 

\shunt

\end{document}