
\documentclass[a4paper,12pt]{article}

\usepackage{fancyhdr}
\usepackage{amssymb}
%\usepackage{mathpazo}
\usepackage{mathtools}
\usepackage{amsmath}
\usepackage{slashed}
\usepackage{cancel}
\usepackage[mathscr]{euscript}
\usepackage{MaxPackage} %Note: You need MaxPackage installed or in the same folder as your .tex file or something.

\newcommand{\colorcomment}[2]{\textcolor{#1}{#2}} %First of these leaves in comments. Second one kills them.
%\newcommand{\colorcomment}[2]{}


\pagestyle{fancy}
\lhead{Max Jeter}
\chead{CMS}
\rhead{Chapter 0, Page \thepage}

%Number of Problems		: 8
%Clear					:
%Begun					:
%Not started			:4,5,6,7,8
%Can complete via book	:
%Needs Polish			:1,2,3

%Pomodoros logged		:

\begin{document}

{\bf Problem 1:}

Let $f$ be defined in a neighborhood around $z$.

If $f$ is complex-differentiable at $z$, then $\frac{f(z+h) - f(z)}{h} \to a$ as $h \to 0$. Thus, $\absval{f(z+h) - f(z) - ah}\absval{\frac{1}{h-z}} \to 0$. Thus, $f(z+h) - f(z) - ah + o(h) = 0$ as $h \to 0$.

So $f(z+h) = f(z) + ah+ o(h)$ as $h \to 0$, which is our result.

Now, let there be $a$ such that $f(z+h) = f(z) + ah + o(h)$ as $h \to 0$.

Then $\frac{f(z+h) - f(z)}{h} = \frac{ah - o(h)}{h} = a$ as $h \to 0$. That is, $f$ is complex-differentiable at $z$. 

\shunt

{\bf Problem 2:}

Let $f$ be complex-differentiable at $z$. Then $f$ is real-differentiable at $z$. So $f$ is continuous at $z$.

\shunt

{\bf Problem 3:}

Let $T: \C \to \C$ be $\R$-linear.

If $T$ is $\C$-linear, then $T(iz) = iT(z)$ for all $z$, by definition.

If $T(iz) = iT(z)$ for all $z$, then let $x,z \in \C$, with $x = a+bi$. Then we have

\begin{align*}
T(xz) &= T((a+bi)z)\\
&= T(az) + T(biz)\\
&= T(az) + iT(bz)\
&= (a+ib)T(z)\\
&= xT(z)\\
\end{align*}

yielding the desired result; $T$ is $\C$-linear.

So $T$ is $\C$-linear if and only if $T$ is $\R$-linear and $iT(z) = T(iz)$ for all $z \in \C$. 

\shunt

{\bf Problem 4:} %You've done this problem before, and you don't want to do it again.

Suppose that $T: \R^2 \to \R^2$ is the $\R$-linear mapping given by the matrix $\left[\begin{smallmatrix} a&b\\ c&d \end{smallmatrix}\right]$.

If $a=d$ and $b=-c$, then for all $z = x+iy \in \C$, we have

\begin{align*}
T(iz) &= \left[\begin{smallmatrix} a&b\\ c&d \end{smallmatrix}\right] i\left[\begin{smallmatrix} x\\ y \end{smallmatrix}\right]\\
&= \left[\begin{smallmatrix} a&b\\ c&d \end{smallmatrix}\right] \left[\begin{smallmatrix} -y\\ x \end{smallmatrix}\right]\\
&= \left[\begin{smallmatrix} a&b\\ c&d \end{smallmatrix}\right] \left[\begin{smallmatrix} -y\\ x \end{smallmatrix}\right]\\
\end{align*}

\shunt

{\bf Problem 5,6:}

Somehow, the ``correct'' way to do these is to search for a basic calculus book and copy the proof contained within. The proof of these things isn't illuminating, and I see no reason to actually do them.

{\bf Problem 7:}

Define $f: \C \to \C$ by $f(z) = 0$ if $\text{Re}(z) = 0$ or $\text{Im}(z) = 0$, and $f(z) = 1$ elsewhere.

Then $f$ satisfies the Cauchy-Riemann equations at the origin ($u_x=u_y=v_x=v_y=0$), but $f$ is not complex-differentiable at the origin (as $f$ is not continuous at the origin (if this is not clear, consider the sequence of points $1/n + i/n$ as $n \to \infty$)...and complex-differentiable functions are continuous).

\shunt

{\bf Problem 8:}

Define $f: \C \to \C$ by $f(z) = \absval{z}^2 \sin(1/\absval{z})$ if $z \neq 0$ and $f(z) = 0$ if $z = 0$.

Then $f$ is complex differentiable at $0$: %Just mash through the definition.

Yet, $u_x$ isn't continuous at the origin; %Take derivatives. 

\shunt

\end{document}