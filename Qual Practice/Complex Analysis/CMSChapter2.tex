
\documentclass[a4paper,12pt]{article}

\usepackage{fancyhdr}
\usepackage{amssymb}
%\usepackage{mathpazo}
\usepackage{mathtools}
\usepackage{amsmath}
\usepackage{slashed}
\usepackage{cancel}
\usepackage[mathscr]{euscript}
\usepackage{MaxPackage} %Note: You need MaxPackage installed or in the same folder as your .tex file or something.

\newcommand{\colorcomment}[2]{\textcolor{#1}{#2}} %First of these leaves in comments. Second one kills them.
%\newcommand{\colorcomment}[2]{}


\pagestyle{fancy}
\lhead{Max Jeter}
\chead{Class}
\rhead{Assignment, Page \thepage}

%Number of Problems		: 6
%Clear					:
%Begun					:
%Not started			:
%Can complete via book	:
%Needs Polish			:

%Pomodoros logged		:

\begin{document}

{\bf Problem 1:}

Consider $f: \C \to \C$ given by $z \mapsto z^2 \sin(1/z)$. 

The ``proof'' that $f$ is differentiable at $0$ relies on this line:

\begin{displaymath}
\absval{\frac{f(h) - f(0)}{h-0}} = \absval{\frac{h^2 \sin(1/h)}{h}} = \absval{h \sin(1/h)} \leq \absval{h}
\end{displaymath}

The last step relies on the fact that $\absval{\sin(x)} \leq 1$, which is not true in $\C$; this is where the ``proof'' fails.

\shunt

{\bf Problem 2:}

Part i:

Consider $[b,a]$. Then for all continuous functions $f$, $\int\limits_{[b,a]} f(z) dz = \int\limits_0^1 f(\ga(t)) \ga'(t)dt =  -\int\limits_1^0 f(\ga(t))\ga'(t) dt = -\int\limits_{[a,b]} f(z) dz$, where $\ga: [0,1] \to \C$ is given by $t \mapsto tb + (1-t)a$. That is, $[b,a] = \dot{-}[a,b]$. 

\shunt

Part ii:

Let $\ga_1: [0,1] \to \C$ be a smooth curve in the plane. We can define $\ga_2$ so that $\ga_2 = \dot{-} \ga_1$ by choosing $\ga_2(t) = \ga_1(1-t)$.

\shunt

{\bf Problem 3:}

Let $p \in [a,b]$. Then there is a $T \in [0,1]$ such that $p = aT + (1-T)b$. Now, for all continuous functions $f$,:

\begin{align*}
\int\limits_{[a,b]} f(z) dz &= \int\limits_0^1 f(\ga(t)) \ga'(t) dt \\
&= \int\limits_0^T f(\ga(t)) \ga'(t) dt + \int\limits_T^1 f(\ga(t)) \ga'(t) dt \\
&= \int\limits_{[a,p]} f(z)dz + \int\limits_{[p,b]} f(z)dz \\
\end{align*}

That is, $[a,b] = [a,p] \dot{+} [p,b]$.

\shunt

{\bf Problem 4:}

Let $T$ be a triangle, with vertices $a,b,c$. Then 

\begin{align*}
\partial T &= [a,b] \dot{+} [b,c] \dot{+} [c,a] \\
&= [a,p] \dot{+} [p,b] \dot{+} [b,q] \dot{+} [q,c] \dot{+} [c,r] \dot{+} [r,a]\\
&= [a,p] \dot{+} [p,b] \dot{+} [b,q] \dot{+} [q,c] \dot{+} [c,r] \dot{+} [r,a] \dot{+}[p,q] \dot{-} [p,q]\dot{+}[q,r] \dot{-} [q,r]\dot{+}[r,p] \dot{-} [r,p]\\
&= [a,p] \dot{+} [p,b] \dot{+} [b,q] \dot{+} [q,c] \dot{+} [c,r] \dot{+} [r,a] \dot{+}[p,q] \dot{+} [q,p]\dot{+}[q,r] \dot{+} [r,q]\dot{+}[r,p] \dot{+} [p,r]\\
&= [a,p] \dot{+} [p,r]\dot{+} [r,a] \dot{+} [b,q] \dot{+} [q,p] \dot{+} [p,b]  \dot{+} [q,c] \dot{+} [c,r] \dot{+} [r,q] \dot{+}[p,q] \dot{+}[q,r] \dot{+}[r,p] \\
&= \partial T_1 \dot{+} \partial T_2 \dot{+} \partial T_3 \dot{+} \partial T_4
\end{align*}

The above isn't very illuminating. You really should be looking at the picture while doing this proof.

\shunt

{\bf Problem 5:}

Let $V = \C \setminus \{0\}$ and define $f: V \to \C$ by $f(z) = 1/z$.

Then $\int\limits_{\partial T} f(z) dz = 0$ for every triangle $T \subset V$ because $V$ is open and $f$ is differentiable at every point of $V$.

However, there is no $F: V \to \C$ with $F' = f$ in $V$, because $\int\limits_{\partial D_1(0)} f(z) dz = 2 \pi i$. If there were such an $F$, then this integral would have been zero, by Cauchy's Theorem for derivatives.

\shunt

{\bf Problem 6:}

This is really just a mixture of a definition chase and rewriting the previous few lines.

I may fill this in later, but it doesn't seem worth it now.

\shunt

\end{document}