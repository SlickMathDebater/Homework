
\documentclass[a4paper,12pt]{article}

\usepackage{fancyhdr}
\usepackage{amssymb}
%\usepackage{mathpazo}
\usepackage{mathtools}
\usepackage{amsmath}
\usepackage{slashed}
\usepackage{cancel}
\usepackage[mathscr]{euscript}
\usepackage{MaxPackage} %Note: You need MaxPackage installed or in the same folder as your .tex file or something.

\newcommand{\colorcomment}[2]{\textcolor{#1}{#2}} %First of these leaves in comments. Second one kills them.
%\newcommand{\colorcomment}[2]{}


\pagestyle{fancy}
\lhead{Max Jeter}
\chead{Class}
\rhead{Assignment, Page \thepage}

%Number of Problems		: 20
%Clear					:
%Begun					:
%Not started			:
%Can complete via book	:
%Needs Polish			:

%Pomodoros logged		:

\begin{document}

{\bf Problem 1:}

\shunt

{\bf Problem 2:}

\shunt

{\bf Problem 3:}

\shunt

{\bf Problem 4:}

Let $f \in H(\C)$ and $\absval{f(z)} \leq e^{\text{Re}(z)}$ for all $z$.

Then $\absval{\frac{f(z)}{e^{\text{Re}(z)}}} = \absval{\frac{f(z)}{e^z}} \leq 1$ for all $z$. That is, $\frac{f(z)}{e^z}$ is a bounded entire function: it's constant. So $\frac{f(z)}{e^z} = c$ for some $c \in \C$: $f(z) = ce^z$ for some $c \in \C$.

\shunt

{\bf Problem 5:}

\shunt

{\bf Problem 6:}

Let $f, g \in H(D_r(z))$ and $1 \leq m \leq n$ be such that $f$ has a zero of order $n$ at $z$ and $g$ has a zero of order $m$ at $z$.

Consider $f/g$. Then

\begin{align*}
f/g(w) &= \frac{f(w)}{g(w)}\\
&= \frac{(w-z)^nh(w)}{(w-z)^mj(w)}\\
&= \frac{(w-z)^{n-m}h(w)}{j(w)}\\
\end{align*}

where $h$ and $j$ holomorphic, and nonzero at $z$. It is clear that this has a limit as $w \to z$; the singularity at $z$ is removable.  

\shunt

{\bf Problem 7:}

%This *does* follow trivially from the magic formula for sin, but that's a bad, bad explanation. 

\shunt

{\bf Problem 8:}

\shunt

{\bf Problem 9:}

\shunt

{\bf Problem 10:}

Pick $V = \{z: \text{Im}(z) \neq 0\}$ and $f(z) = -1$ when $\text{Im}(z) >0$ and $f(z) = 1$ when $\text{Im}(z) < 0$.

We see that $f$ isn't constant, but $f$ is holomorphic and $f'(z) = 0$ for all $z \in V$.

(The point is that $V$ isn't connected.)

\shunt

{\bf Problem 11:}



\shunt

{\bf Problem 12:}

\shunt

{\bf Problem 13:}

\shunt

{\bf Problem 14:}

\shunt

{\bf Problem 15:}

\shunt

{\bf Problem 16:}

\shunt

{\bf Problem 17:}

\shunt

{\bf Problem 18:}

\shunt

{\bf Problem 19:}

\shunt

{\bf Problem 20:}

\shunt
\end{document}