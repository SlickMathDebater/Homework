
\documentclass[a4paper,12pt]{article}

\usepackage{fancyhdr}
\usepackage{amssymb}
%\usepackage{mathpazo}
\usepackage{mathtools}
\usepackage{amsmath}
\usepackage{slashed}
\usepackage{cancel}
\usepackage[mathscr]{euscript}
\usepackage{MaxPackage} %Note: You need MaxPackage installed or in the same folder as your .tex file or something.

\newcommand{\colorcomment}[2]{\textcolor{#1}{#2}} %First of these leaves in comments. Second one kills them.
%\newcommand{\colorcomment}[2]{}


\pagestyle{fancy}
\lhead{Max Jeter}
\chead{Class}
\rhead{Assignment, Page \thepage}

%Number of Problems		: 20
%Clear					:
%Begun					:
%Not started			:
%Can complete via book	:
%Needs Polish			:

%Pomodoros logged		:

\begin{document}

{\bf Problem 1:}

Consider $f: \C \to \C$ given by $f(z) = \frac{1}{1+z^2}$. Then $f$ is holomorphic except at $i$ and $-i$. Thus, $f$ is holomorphic on $\R$: $f$ is locally given by a convergent power series.

(It appears that there is also an argument through Real Analysis using part of the proof of one of the many things called ``Taylor's Theorem''. This yields an absolute mess, don't try this.)

Yet, the power series at the origin is given by $g(x) = \sum\limits_{n=0}^\infty (-1)^n x^{2n}$, which diverges when $\absval{x} \geq 1$.

\shunt

{\bf Problem 2:}

Let $A$ and $B$ be open, and $f \in \scrO(A)$ and $g \in \scrO(B)$, with $f=g$ on $A \cap B$. The function $F$ given by $F(z) = f(z)$ if $z \in A$ and $F(z) = g(z)$ if $z \in B$ is holomorphic; it is equal to a holomorphic function in $A$, it is equal to a holomorphic function in $B$.

The fact that $A$ and $B$ are open is used implicitly in the last line; holomorphy is a local property, and this is because we only speak of holomorphic functions on open sets.

\shunt

{\bf Problem 3:}

Consider $f: \overline{D_1(-1)} \to \overline{D_1(-1)}$ given by $f(z) = z$ and $g: \overline{D_1(1)} \to \overline{D_1(1)}$ given by $g(z) = -z$. Then $\overline{D_1(-1)} \cap \overline{D_1(1)} = \{0\}$, $f$ and $g$ agree on the intersection, but no analytic function exists with the desired properties (the function we get out of this isn't differentiable at $0$.)

\shunt

{\bf Problem 4:}

Let $f \in H(\C)$ and $\absval{f(z)} \leq e^{\text{Re}(z)}$ for all $z$.

Then $\absval{\frac{f(z)}{e^{\text{Re}(z)}}} = \absval{\frac{f(z)}{e^z}} \leq 1$ for all $z$. That is, $\frac{f(z)}{e^z}$ is a bounded entire function: it's constant. So $\frac{f(z)}{e^z} = c$ for some $c \in \C$: $f(z) = ce^z$ for some $c \in \C$.

\shunt

{\bf Problem 5:}

Let $f \in H(\C)$, $n \in \N$, and $\absval{f(z)} \leq (1+\absval{z})^n$ for all $z$. Then $f(z) = \sum\limits_{m=0}^\infty c_mz^m$ for some $\anbrack{c_m} \in \C$. Yet, $\absval{f(z)} \leq (1+\absval{z})^n$; if $c_m \neq 0$ for $m > n$, then the estimate breaks down. (Choosing $z \in \R$ incredibly large should get you this result.) Thus, $c_m = 0$ for sufficiently large $m$; $f$ is a polynomial.

\shunt

{\bf Problem 6:}

Let $f, g \in H(D_r(z))$ and $1 \leq m \leq n$ be such that $f$ has a zero of order $n$ at $z$ and $g$ has a zero of order $m$ at $z$.

Consider $f/g$. Then

\begin{align*}
f/g(w) &= \frac{f(w)}{g(w)}\\
&= \frac{(w-z)^nh(w)}{(w-z)^mj(w)}\\
&= \frac{(w-z)^{n-m}h(w)}{j(w)}\\
\end{align*}

where $h$ and $j$ holomorphic, and nonzero at $z$. It is clear that this has a limit as $w \to z$ (as $h/j$ has a limit as $w \to z$); the singularity at $z$ is removable.  

\shunt

{\bf Problem 7:}

Let $f \in H(\C)$, and $f(n) = 0$ for all $n \in \Z$. Consider $f(z)/\sin(\pi z)$. Note that $\sin(\pi z)$'s zeroes are all of order $1$; this follows readily from the power series expansion. So, by the prior exercise, $f(z)/\sin(\pi z)$ has only removable singularities.

%This *does* follow trivially from the magic formula for sin, but that's a bad, bad explanation. 

\shunt

{\bf Problem 8:}

Let $f \in H(\C)$, $f(z+1) = -f(z)$ for all $z$, $f(0) = 0$, and $\absval{f(z)} \leq e^{\pi \absval{\text{Im}(z)}}$ for all $z$.

We can focus on $z$ with $\text{Re}(z) \in [0,1)$, because $f(z+1) = -f(z)$; if the result holds in this region, it holds on all of $\C$. %I have no fucking clue how to proceed.

\shunt

{\bf Problem 9:}

Let $V$ be a connected open set, and $f \in H(V)$.

Pick $a \in V$. For any $b \in D_r(a)$, where $D_r(a) \subset V$, consider that $f(a)-f(b) = \int\limits_{[a,b]} f'(z) dz = 0$. That is, $f$ is constant on a disk around $a$; $f$ is constant, by the uniqueness theorem or the fact that the derivatives all vanish at a point. 

\shunt

{\bf Problem 10:}

Pick $V = \{z: \text{Im}(z) \neq 0\}$ and $f(z) = -1$ when $\text{Im}(z) >0$ and $f(z) = 1$ when $\text{Im}(z) < 0$.

We see that $f$ isn't constant, but $f$ is holomorphic and $f'(z) = 0$ for all $z \in V$.

(The point is that $V$ isn't connected.)

\shunt

{\bf Problem 11:}

Let $V = \C \setminus \{0\}$.

If there is $f \in H(V)$ with $e^{f(z)} = z$ for all $z \in V$, then $f'(z)e^{f(z)} =zf'(z)= 1$. That is, $f'(z) = 1/z$. Yet, by Cauchy's Theorem for derivatives, we must have that $\int\limits_{\partial D_1(0)} f'(z) dz = 0$, which fails.  

\shunt

{\bf Problem 12:}

Let $V$ be a bounded open subset of $\C$ and $f \in C(\overline{V}) \cap H(V)$. Let $M \geq 0$ and $\absval{f(z)} \leq M$ for all $z \in \partial V$. Consider that $f$ takes a maximum in absolute value; it's a continuous function on a bounded open subset of $\C$. If $f$ attains its maximum on the interior, it is constant. So $f$ attains its maximum on the boundary. Thus, a bound on the boundary also bounds the interior. 

\shunt

{\bf Problem 13:}

Let $V = D_1(0)$. Let $f \in C(\overline{V}) \cap H(V)$, $f(0) = 0$, and $\absval{f(z)} \leq 1$ for all $z \in \overline{V}$. Consider $g(z) = f(z)/z$. Note that the singularity at $0$ is removable; $g(z)$ is a holomorphic function. Moreover, $g(z) = f(z)$ on $\partial D_1(0)$; $\absval{g(z)} \leq 1$ for all $z \in \overline{V}$. So $\absval{f(z)} \leq \absval{z}$ for all $z \in \overline{V}$, as desired. 

\shunt

{\bf Problem 14:}

Let $V = D_1(0)$. Let $f \in H(V)$, $f(0) = 0$, and $\absval{f(z)} \leq 1$ for all $z \in V$. By applying the previous result, we get that $f(rz) \leq \absval{rz}$ for all $z \in V$, $r \in (0,1)$. Taking limits as $r \to 1$ yields the desired result.  

\shunt

{\bf Problem 15:}

\shunt

{\bf Problem 16:}

Read the words on the fucking paper.

\shunt

{\bf Problem 17:}

\shunt

{\bf Problem 18:}

\shunt

{\bf Problem 19:}

\shunt

{\bf Problem 20:}

\shunt
\end{document}