
\documentclass[a4paper,12pt]{article}

\usepackage{fancyhdr}
\usepackage{amssymb}
%\usepackage{mathpazo}
\usepackage{mathtools}
\usepackage{amsmath}
\usepackage{slashed}
\usepackage{cancel}
\usepackage[mathscr]{euscript}
\usepackage{MaxPackage} %Note: You need MaxPackage installed or in the same folder as your .tex file or something.

\newcommand{\colorcomment}[2]{\textcolor{#1}{#2}} %First of these leaves in comments. Second one kills them.
%\newcommand{\colorcomment}[2]{}


\pagestyle{fancy}
\lhead{Max Jeter}
\chead{Class}
\rhead{Assignment, Page \thepage}

%Number of Problems		: 23
%Clear					:
%Begun					:
%Not started			:
%Can complete via book	:
%Needs Polish			:

%Pomodoros logged		:

\begin{document}

{\bf Problem 1:}

Consider $\text{Ind}(\partial D(a,r),a)$. Define $\tha: [0,1] \to \R$ by $x \to 2\pi x$. Note that $\tha$ satisfies $\ga(t) - a = \absval{\ga(t) - a} e^{i \tha (t)}$, where $\ga(t) = re^{2\pi t}+a$. From the definitions, we have

\begin{align*}
\text{Ind}(\partial D(a,r),a) &= \frac{1}{2\pi} (\tha (1) - \tha(0))\\
&= \frac{1}{2\pi} (\tha (1) - \tha(0))\\
&= 1
\end{align*}

\shunt

{\bf Problem 2:}

Consider $\text{Ind}(\partial D(a,r),z)$. Because $\partial D(a,r)$ is a smooth, closed curve, we have that $\text{Ind}(\partial D(a,r),z) = \frac{1}{2\pi i} \int\limits_{\partial D(a,r)} \frac{dw}{w-z}$.

So,

\begin{align*}
\text{Ind}(\partial D(a,r),z) &= \frac{1}{2\pi i} \int\limits_{\partial D(a,r)} \frac{dw}{w-z}\\
&= \frac{1}{2\pi i} \int\limits_{\partial D(z,r)} \frac{dw}{w-z}\\
&=\frac{1}{2\pi i} 2 \pi i\\
=1
\end{align*}

if $z \in D(a,r)$.

And

\begin{align*}
\text{Ind}(\partial D(a,r),z) &= \frac{1}{2\pi i} \int\limits_{\partial D(a,r)} \frac{dw}{w-z}\\
&= \frac{1}{2\pi i} \int\limits_{\partial D(z,r)} \frac{dw}{w-z}\\
&=0
\end{align*}

if $z \notin D(a,r)$ (as the function is holomorphic on that disk.)

Note that in our definitions, the index isn't defined when $a \in \ga^*$. That is, $\text{Ind}(\partial D(a,r),z)$ isn't defined when $z \in \partial D(a,r)$.

\shunt

{\bf Problem 3:}

Let $V$ be an open subset of the plane and $\Ga$ be a cycle in $V$. Also, let $\Ga$ have the property that $\text{Ind}(\Ga, a) =0$ for all $a \in \C \setminus V$.

Let $f \in H(V)$.

Then define $g(w) = \frac{f(w)}{w-z}$. We know that $g$ is differentiable on $V \setminus \{z\}$. Referring to a figure much like Figure 2.1 and using Theorem 4.10 to show that the ``outer'' chunks of the integral vanish, we get $\int\limits_{\Ga} g(w) dw = \int\limits_{\partial D(z,r)} g(w)dw$ for arbitrarily small $r>0$.

\shunt

{\bf Problem 4:}

\shunt

{\bf Problem 5:}

\shunt

{\bf Problem 6:}

\shunt

{\bf Problem 7:}

\shunt

{\bf Problem 8:}

\shunt

{\bf Problem 9:}

Consider $f$, holomorphic on some disk, $\Om$, centered at $z$. Consider $g(w) = \frac{f(w)}{w-z}$; then we have that $\int\limits_{\partial Om} g(w) dw= 2\pi i \text{Res}_z g$. (Note that $g$'s only singularity is at $z$.) Moreover, note that $g(w) = \frac{\sum\limits_{n=0}^\infty a_n(w-z)^n}{w-z}$. Thus, by the residue theorem, $\int\limits_{\partial \Om} \frac{f(w)}{w-z} dw = f(z)$. 

\shunt

{\bf Problem 10:}

\shunt

{\bf Problem 11:}

\shunt

{\bf Problem 12:}

\shunt

{\bf Problem 13:}

\shunt

{\bf Problem 14:}

\shunt

{\bf Problem 15:}

\shunt

{\bf Problem 16:}

\shunt

{\bf Problem 17:}

\shunt

{\bf Problem 18:}

\shunt

{\bf Problem 19:}

\shunt

{\bf Problem 20:}

\shunt

{\bf Problem 21:}

\shunt

{\bf Problem 22:}

\shunt

{\bf Problem 23:}

\shunt

\end{document}