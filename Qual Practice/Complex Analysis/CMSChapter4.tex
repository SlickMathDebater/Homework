
\documentclass[a4paper,12pt]{article}

\usepackage{fancyhdr}
\usepackage{amssymb}
%\usepackage{mathpazo}
\usepackage{mathtools}
\usepackage{amsmath}
\usepackage{slashed}
\usepackage{cancel}
\usepackage[mathscr]{euscript}
\usepackage{MaxPackage} %Note: You need MaxPackage installed or in the same folder as your .tex file or something.

\newcommand{\colorcomment}[2]{\textcolor{#1}{#2}} %First of these leaves in comments. Second one kills them.
%\newcommand{\colorcomment}[2]{}


\pagestyle{fancy}
\lhead{Max Jeter}
\chead{Class}
\rhead{Assignment, Page \thepage}

%Number of Problems		: 23
%Clear					:
%Begun					:
%Not started			:
%Can complete via book	:
%Needs Polish			:

%Pomodoros logged		:

\begin{document}

{\bf Problem 1:}

\shunt

{\bf Problem 2:}

\shunt

{\bf Problem 3:}

\shunt

{\bf Problem 4:}

\shunt

{\bf Problem 5:}

\shunt

{\bf Problem 6:}

\shunt

{\bf Problem 7:}

\shunt

{\bf Problem 8:}

\shunt

{\bf Problem 9:}

Consider $f$, holomorphic on some disk, $\Om$, centered at $z$. Consider $g(w) = \frac{f(w)}{w-z}$; then we have that $\int\limits_{\partial Om} g(w) dw= 2\pi i \text{Res}_z g$. (Note that $g$'s only singularity is at $z$.) Moreover, note that $g(w) = \frac{\sum\limits_{n=0}^\infty a_n(w-z)^n}{w-z}$. Thus, by the residue theorem, $\int\limits_{\partial \Om} \frac{f(w)}{w-z} dw = f(z)$. 

\shunt

{\bf Problem 10:}

\shunt

{\bf Problem 11:}

\shunt

{\bf Problem 12:}

\shunt

{\bf Problem 13:}

\shunt

{\bf Problem 14:}

\shunt

{\bf Problem 15:}

\shunt

{\bf Problem 16:}

\shunt

{\bf Problem 17:}

\shunt

{\bf Problem 18:}

\shunt

{\bf Problem 19:}

\shunt

{\bf Problem 20:}

\shunt

{\bf Problem 21:}

\shunt

{\bf Problem 22:}

\shunt

{\bf Problem 23:}

\shunt

\end{document}