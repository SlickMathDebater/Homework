
\documentclass[a4paper,12pt]{article}

\usepackage{fancyhdr}
\usepackage{amssymb}
%\usepackage{mathpazo}
\usepackage{mathtools}
\usepackage{amsmath}
\usepackage{slashed}
\usepackage{cancel}
\usepackage[mathscr]{euscript}
\usepackage{MaxPackage} %Note: You need MaxPackage installed or in the same folder as your .tex file or something.

\newcommand{\colorcomment}[2]{\textcolor{#1}{#2}} %First of these leaves in comments. Second one kills them.
%\newcommand{\colorcomment}[2]{}


\pagestyle{fancy}
\lhead{Max Jeter}
\chead{MA544}
\rhead{Practice Set, Page \thepage}

\begin{document}

%This is an attempt at a rough working-through of some problems from Royden's Real Analysis (3rd ed.) textbook.
%Ch. 3: 4 7 8 11 12 14 17 22 23 28 31
%Ch. 4: 2 8 14 15 16 22 25
%Ch. 5: 4 5 8 10 14 16 20 23 27
%Ch. 6: 2 4 8 10 11 15 18 23 24
%Ch. 11: 2 5 7 10 11 13 21 24 25 28 30 33 34 39 41 44 47
%Ch. 12: 1 4 8

%Currently skipped: Ch 3, 28/31, Ch 4 25

{\Huge{\textit{\textbf{Chapter 3:}}}}

\shunt

{\bf Problem 4:}

Let $m^*(E) = \infty$ for an infinite set and $m^*(E) = \absval{E}$ for a finite set.

It's clear that $m^*$ is defined for all sets of real numbers, is translation invariant, and countably additive. So $m^*$ is a measure; we call it the counting measure.

\shunt

{\bf Problem 7:}

If $m^*(E)$ is the Lebesgue Outer Measure, it's somewhat clear that it's translation invariant; we can do this by making an open cover and shifting it.

\shunt

{\bf Problem 8:}

If $m^*(A) = 0$, then $m^*(A \cup B) \geq m^*(B)$, by monotonicity.

But also, $m^*(A \cup B) \leq m^*(A) + m^*(B) = m^*(B)$ by countable subadditivity. 

So $m^*(A \cup B) = m^*(B)$.

\shunt

{\bf Problem 11:}

Each $(a, \infty)$ is measurable.

We have $\bigcap\limits^\infty (n, \infty) = \emptyset$ which has measure $0$, but $m((n, \infty)) \to \infty$. So $m(\bigcap\limits^n E_i) \slashed{\to} m(\bigcap\limits^\infty E_n)$

\shunt

{\bf Problem 12:}

Let $\anbrack{E_i}$ be a sequence of disjoint measurable sets, and $A$ be a set.

Then $m^*(A \cap \bigcup\limits^n E_i) = \sum\limits^n m^*(A \cap E_i)$.

So $m^*(A \cap \bigcup\limits^\infty E_i) \geq \sum\limits^n m^*(A \cap E_i)$.

But $n$ is arbitrary, so $m^*(A \cap \bigcup\limits^\infty E_i) \geq \sum\limits^\infty m^*(A \cap E_i)$.

Either by employing a similar argument or appealing to countable subadditivity, we get $m^*(A \cap \bigcup\limits^\infty E_i) = \sum\limits^\infty m^*(A \cap E_i)$.

\shunt

{\bf Problem 14:}

Part a:

The Cantor set has measure zero; it's usually defined as

\begin{displaymath}
[0,1] \setminus ((1/3,2/3) \cup ((1/9,2/9) \cup (7/9,8/9)) \cup \ldots)
\end{displaymath}

Now, $[0,1]$ has measure $1$, and is measurable.

Also, $((1/3,2/3) \cup ((1/9,2/9) \cup (7/9,8/9)) \cup \ldots)$ has measure $1$ (consider the geometric series $1/3, 2(1/3)^2 \ldots$. It sums up to $1$).

So the measure of the cantor set is $1-1=0$.

\shunt

Part b:

If we only remove $\al 3^-n$ at each step when we define the cantor set, then we can show that it would still be closed (as a complement in $[0,1]$ of an open set) and by employing the same geometric series argument, it would have measure $1-\al$.

\shunt

{\bf Problem 17:}

Part a:

Consider the $P_i$s as defined in this section. We're given that $m[0,1) = \sum m^*P_i = \sum m^*P$, so that the right hand side is either zero or infinite. But if it was zero, then we break countable subadditivity; it must be infinite. So we have an example where $m(\bigcup E_i) \leq \sum m^*(E_i)$.

\shunt

Part b:

Define $E_0 = [0,1) \setminus P_0$ and $E_n = [0,1) \setminus P_n$ to get the desired result.

\shunt

{\bf Problem 22:}

Part a:

If $f$ is measurable, then the restriction of $f$ to any measurable set is measurable. If $D_1$ isn't measurable, then the interesection of all of the $\{x: f(x) \geq n\}$ isn't measurable, which is bad. Similarly, $D_2$ must be measurable.

Now, if all of $D_1$, $D_2$ and the restriction of $f$ to $D \setminus D_1 \cup D_2$ are measurable, then for each $\al$ we get $\{x: f(x) \geq \al\}$ the union of $D_1$ and a measurable set, so we win.

\shunt

Part b:

Apply the same trick as used earlier this chapter; prove that if $f$ and $g$  measurable, then so is $f^2$ and $f+g$, and win using $fg= 1/2[(f+g)^2 -f^2-g^2]$. 

\shunt

Parts c and d are painfully trivial.

{\bf Problem 23:}

This was a homework problem; just go there.

\shunt

{\bf Problem 28:}

I'm not sure how to do this one.

\shunt

{\bf Problem 31:}

Not sure how to do this one either. It looks like a very likely qual problem, too...:/

\pagebreak

{\Huge{\textit{\textbf{Chapter 4:}}}}

\shunt

{\bf Problem 2:}

Part a: Let $f$ be a bounded function on $[a,b]$ and let $h$ be the upper envelope of $f$ (that is, $h(x) = \inf\limits_{\de>0} \sup\limits_{\absval{x-y}< \de}(f(y))$)

Then $U-\int\limits_a^b f \geq \int\limits_a^b h$; let $\phi$ be a step function with $\phi \geq f$. Then $\phi \geq h$ except at a finite number of points, because step functions are discontinuous on only finitely many points and the upper envelope is lower than any continuous function above $f$.

Also, $U-\int\limits_a^b f \leq \int\limits_a^b h$; there's a sequence of step functions converging downwards to $h$, so by bounded convergence, we have our result.

So $U-\int\limits_a^b f = \int\limits_a^b h$.

\shunt

Part b:

We get a similar result for the lower envelope. So a bounded function on $[a,b]$ is Riemann-integrable if and only if the integrals of its upper and lower envelopes are equal.

If the upper and lower envelopes are unequal on a set of greater than measure zero, this fails, as the lower envelope is always lower than the upper envelope.

If the upper and lower envelopes are equal except on a set of measure zero, this succeeds, rather obviously.

So a bounded function on $[a,b]$ is Riemann-integrable if and only if the upper and lower envelopes are equal except on a set of measure zero. That is, a bounded function on $[a,b]$ is Riemann-integrable if and only if the function is continuous except on a set of measure zero.

\shunt

{\bf Problem 8:}

Let $\anbrack{f_n}$ be a sequence of nonnegative functions on a domain, $E$. Define $f(x) = \liminf f_n(x)$.

Let $h \leq f$ be any non-negative, simple function with finite measure support on the domain (say it has finite measure support on $F$. 

Then define $h_n = \min(h,f_n)$.

Now, $\int\limits_E h \leq \int\limits_F h= \lim \int\limits_F h_n \leq \liminf \int\limits_E f_n$.

By taking supremums over $h$, we have our result.

\shunt

{\bf Problem 14:}

Part a:

Let $\anbrack{g_n} \to g$ almost everywhere, $\anbrack{f_n} \to f$ almost everywhere, and $\absval{f_n} \leq g_n$, with all of the above functions being measurable, and $\int g = \lim \int g_n$.

Then $\int \absval{f_n-f} \leq \absval{\int f_n -f} = \absval{\int f_n - \int f} \to 0$.

\shunt

Part b: NOTE: a similar problem was an exam problem. This problem can be generalized, and should be done in the context of $L^p$ spaces.

Let $\anbrack{f_n}$ be a sequence of integrable functions in $L^p$ with $f_n \to f$ almost everywhere.

If $\norm{f_n} \slashed{\to} \norm{f}$, then there's an $\ep >0$ and a subsequence $f_{n_k}$ with $\absval{\norm{f_{n_k}} - \norm{f}} \geq \ep$. But

\begin{align*}
2\norm{f_n - f} &\geq \absval{\norm{f_n} - \norm{f}}\\
&\slashed{\to} 0
\end{align*}

If $\norm{f_n} \to \norm{f}$, then: 

\begin{align*}
\norm{f_n - f} &\leq \absval{\norm{f_n} - \norm{f}} \text{ (By reverse triangle inequality.)} \\
&\to 0
\end{align*} 

So $\norm{f_n - f} \to 0$ if and only if $\norm{f_n} \to \norm{f}$.

\shunt

{\bf Problem 15:}

The entire problem is ``Apply Littlewood's Three Principles'' and the ``$2^{-n}\ep$ trick''. (On $[-1,1]$ there is a (property) function such that $\absval{f-\phi_1} < 2^{-1}\ep/2$... similarly, there is such a function on $[-2,-1)$ and $(1,2]$ such that $\absval{f-\phi_2} < 2^{-2}\ep/2$...induct, paste everything together, integrate, geometric series, win.)

\shunt

{\bf Problem 16:} NOTE: this was an exam problem.

First, note that if we have that this is true for all step functions vanishing except on a vinite interval, then we have our result; if $\lim\limits_{n \to \infty} cos(nx)\phi(x) dx = 0$ for all such step functions $\phi$, then because there's such a step function with $\int\absval{f - \phi} < \ep$ for all $\ep>0$, we have our result.

So, let $\phi$ be a step function on $[a,b]$, and let $\ep>0$. Partition $[a,b]$ by $a = x_0 < x_1 \ldots x_l = b$ so that $\phi$ is constant on each $(x_i,x_i+1)$. Let $M$ be the maximum of $\absval{\phi}$ (which exists, as $\phi$ takes only finitely many values). Pick $n$ large enough so that $2\pi / n < \ep /(lM)$. Integrate over each chunk of the partition; we end up with everything cancelling out except on sets of length less than $2\pi/n$. There's at most $l$ of them, having magnitude at most $M$; we've won.

\shunt

{\bf Problem 22:}

Note: This problem is lol.

Let there be a sequence, $\anbrack{f_n}$, of measurable functions on a set, $E$, of finte measure, with $f_n \to f$ in measure.

Then every subsequence of $f_n$ converges to $f$ in measure, so every subsequence has a subsequence converging to $f$ in measure.

Now, let there be a sequence, $\anbrack{f_n}$, of measurable functions on a set, $E$, of finte measure, with every subsequence of $f_n$ having a subsequence converging to $f$ in measure.  Then every subsequence of $f_n$ has a  subsequence which has every subsequence have a subsequence that converges almost everywhere to $f$. Thus, every subsequence of $f_n$ has a subsequence that converges almost everywhere to $f$. So $f_n$ converges to $f$ in measure.

\shunt

{\bf Problem 25:}

...Seriously, the hint gives this entire question away. Pretty lame stuff, bro.

\pagebreak

{\Huge{\textit{\textbf{Chapter 5:}}}}

\shunt

{\bf Problem 4:}

\shunt

{\bf Problem 5:}

\shunt

{\bf Problem 8:}

\shunt

{\bf Problem 10:}

\shunt

{\bf Problem 14:}

\shunt

{\bf Problem 16:}

\shunt

{\bf Problem 20:}

\shunt

{\bf Problem 23:}

\shunt

{\bf Problem 27:}

\pagebreak

{\Huge{\textit{\textbf{Chapter 6:}}}}

\shunt

{\bf Problem 2:}

\shunt

{\bf Problem 4:}

\shunt

{\bf Problem 8:}

\shunt

{\bf Problem 10:}

\shunt

{\bf Problem 11:}

\shunt

{\bf Problem 15:}

\shunt

{\bf Problem 18:}

\shunt

{\bf Problem 23:}

\shunt

{\bf Problem 24:}

\shunt

\end{document}